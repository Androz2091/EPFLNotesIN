% !TeX program = lualatex

\documentclass[a4paper]{article}

% Expanded on 2021-10-20 at 15:04:30.

\usepackage{../../style}

\title{AICC 1}
\author{Joachim Favre}
\date{Mercredi 20 octobre 2021}

\begin{document}
\maketitle

\lecture{10}{2021-10-20}{Sequences and countable infinities}{
\begin{itemize}[left=0pt]
    \item Definition of least upper, greatest lower bound and lattices.
    \item Definition of lexicographic ordering.
    \item Definition of sequences, and explanation of the most important ones (arithmetic progression, geometric progression and recurrence relations).
    \item Explanation on how to compute sums and products.
    \item Definition of set cardinality, and of countable sets.
    \item Proof that $\mathbb{N}$ and $\mathbb{Q}$ are countable, but $\mathbb{R}$ is not.
\end{itemize}

}

\begin{parag}{Example}
    Let's draw the following Hasse diagram:
    \imagehere[0.25]{HasseDiagramUpperLowerBounds.png}

    Then, we can see that $h$ is an upper bound for $\left\{a, e, d\right\}$; $f$ is a least upper bound for $\left\{a, e, d\right\}$ and $\left\{j, h\right\}$ has no upper bound.
\end{parag}

\begin{parag}{Definition: Lattices}
    A partially ordered set in which every pair of elements has both a least upper bound and a greatest lower bound is called a lattice.

    \begin{subparag}{Example}
        $\left(\mathcal{P}\left(S\right), \subseteq\right)$ is a lattice. Indeed, the least upper bound of two subsets is $A \cup B$ and the greatest lower bound is $A \cap B$.
    \end{subparag}
\end{parag}

\begin{parag}{Definition: Lexicographic ordering}
    Given two posets $\left(A_1, \preccurlyeq_1\right)$ and $\left(A_2, \preccurlyeq_2\right)$, the \important{lexicographic ordering} on $A_1 \times A_2$ is defined by saying that $\left(a_1, a_2\right)\preccurlyeq \left(b_1, b_2\right)$ if:
    \[a_1 \preccurlyeq_1 b_1 \lor \left(a_1 = b_1 \land a_2 \preccurlyeq_2 b_2\right)\]

    This definition can easily be extended to a lexicographic ordering on a $n$-ary Cartesian product.

    \begin{subparag}{Motivation}
        This is basically the generalised idea of alphabetical order.
    \end{subparag}

    \begin{subparag}{Example}
        Let's consider $\left(\mathbb{Z} \times \mathbb{Z}, \leq\right)$. All ordered pairs less than $\left(3, 4\right)$ are:
        \imagehere[0.7]{LexicographicOrdering.png}
    \end{subparag}
\end{parag}

\subsection{Sequences}
\begin{parag}{Definition: Sequences}
    A \important{sequence} is a function from a subset of the integers --- usually $\mathbb{Z}_+$ or $\mathbb{N}$ --- to a set $S$.

    Let $f : \mathbb{Z}_+ \mapsto S$ be the function that defines a sequence. We write $a_n$ to denote the image $f\left(n\right)$ of integer $n$. We call $a_n$ a \important{term} of the sequence.

    \begin{subparag}{Intuition}
        Sequences are basically ordered lists of elements of a set.
    \end{subparag}

    \begin{subparag}{Example}
        Let $\left(a_n\right)$ denote the sequence that is defined by $a_n = \frac{1}{n}, n \geq 1$. Then:
        \[a_1 = 1, \mathspace a_2 = \frac{1}{2}, \mathspace a_3 = \frac{1}{3}, \mathspace, \ldots\]
    \end{subparag}
\end{parag}

\begin{parag}{Types of sequences}
    We can give name to the following sequences:
    \begin{description}
        \item[Arithmetic Progression:] $a_n = a + n\cdot d$, where $a$ and $d$ are given constants. For example, a car going at constant speed.
        \item[Geometric Progression:] $a_n = ar^{n}$. For example, interest rates.
        \item[Recurrence Relations:] $a_n = g\left(a_{n-1}, a_{n-2}, \ldots, a_{n-k}\right)$ with $k$ initial conditions.
    \end{description}

    \begin{subparag}{Remark}
        Note that the arithmetic progression can be defined recursively as $a_n = a_{n-1} + d$ and $a_0 = a$. Similarly, the geometric progression can be defined recursively as $a_n = ra_{n-1}$ and $a_0 = a$.
    \end{subparag}
\end{parag}

\begin{parag}{Sum}
    We define
    \[s_n = \sum_{j=1}^{n} a_j = a_1 + \ldots + a_n\]

    \begin{subparag}{Important sums}
        If we have $a_j = d$, then:
        \[s_n = \sum_{j=1}^{n} d = nd\]

        Else, if we have $a_j = j$, then (Gauss is too strong, he found it when was in Elementary School when his teacher asked his class to sum numbers from 1 to 100, in order to have some peace):
        \[s_n = \sum_{j=1}^{n} j = \frac{n\left(n+1\right)}{2} \]
    \end{subparag}
\end{parag}

\begin{parag}{Product}
    Concerning products, we define:
    \[p_n = \prod_{j=1}^{n} a_j = a_1 a_2 \cdots a_n\]

    \begin{subparag}{Important products}
        If we have $a_j = r$, then:
        \[p_n = \prod_{j=1}^{n} r = r^{n}\]

        If $a_j = j$, then we have:
        \[p_n = \prod_{j=1}^{n} j = n\left(n-1\right)\ldots1 = n!\]
    \end{subparag}
\end{parag}

\begin{parag}{Telescoping series}
    Given $a_0, \ldots, a_n$, then:
    \[\sum_{j=0}^{n} \left(a_j - a_{j-1}\right) = a_n - a_0\]

    \begin{subparag}{Proof}
        \begin{multiequality}
            \sum_{j=1}^{n} \left(a_j - a_{j-1}\right) & = \sum_{j=1}^{n} a_j - \sum_{j=1}^{n} a_{j-1} \\
            & = \sum_{j=1}^{n-1} a_j + a_n - \sum_{j=0}^{n-1} a_j \\
            & = \sum_{j=1}^{n-1} a_j + a_n - \sum_{j=1}^{n-1} a_j - a_0 \\
            & = a_n - a_0
        \end{multiequality}
    \end{subparag}

    \begin{subparag}{Example}
        We see that, if we pick $a_j = j^2$, then:
        \[a_j - a_{j-1} = j^2 - \left(j-1\right)^2 = 2j - 1\]

        So:
        \[\sum_{j=0}^{n} \left(2j - 1\right) = n^2 - 0^2 = n^2\]
    \end{subparag}
\end{parag}

\begin{parag}{Definition: String}
    A \important{string} is a finite sequence of characters from a finite set $A$ (an alphabet).

    \begin{subparag}{Example}
        The \important{empty string} is represented by $\lambda$. The string $abcde$ has length 5.
    \end{subparag}
\end{parag}

\begin{parag}{Lexicographic ordering on strings}
    A lexicographic ordering of strings can be defined using the ordering letters in the alphabet. This is the same ordering as that used in dictionaries.

    We can note that strings with lexicographic ordering are well-ordered sets.
\end{parag}

\subsection{Countable sets}
\begin{parag}{Cardinality comparison}
    The cardinality of a set $A$ is equal to the cardinality of a set $B$, denoted by $\left|A\right| = \left|B\right|$ if and only if there is a bijection from $A$ to $B$ (we could draw a line going from each element of $a$ to each element of $b$, exactly one).

    If there is an injection of from $A$ to $B$, the cardinality of $A$ is less than or equal to the cardinality of $B$, denoted $\left|A\right| \leq \left|B\right|$.

    When $\left|A\right| \leq \left|B\right|$ and $A$ and $B$ have different cardinality, we say that the cardinality of $A$ is less than the cardinality of $B$, denoted $\left|A\right| < \left|B\right|$.
\end{parag}

\begin{parag}{Definition}
    A set that is either finite or has the same cardinality as the set of positive integers, $\mathbb{Z}_+$, is called \important{countable}. Else, it is \important{uncountable}.

    When a set is finitely countable, then its cardinality is its number of of elements. If it is countably infinite, its cardinality is $\aleph_0$.

    \begin{subparag}{Remark}
        Note that $\aleph$ is read ``aleph''; it is the first letter of the Hebrew alphabet.
    \end{subparag}
\end{parag}

\begin{parag}{Example 1}
    Let's show that the set of positive integers $E$ is a countable set.

    Indeed, let $f: \mathbb{Z}_+ \mapsto E$, $f\left(x\right) = 2x$. This want to show that $f$ is a bijection.

    \begin{subparag}{Injective}
        Suppose that $f\left(n\right) = f\left(m\right)$. Then:
        \[2n = 2m \implies n = m\]

        Therefore, $f$ is injective.
    \end{subparag}

    \begin{subparag}{Surjective}
        Let $t$ be an arbitrary even positive integer. We know that $t = 2k$ or some positive integer $k$. Thus, $f\left(k\right) = t$, so $f$ is surjective.
    \end{subparag}

    Since $f$ is bijective, we deduce that the $E$ is a countable set.

    \begin{subparag}{Generalisation}
        More generally, an infinite subset of a countable subset is countable.
    \end{subparag}
\end{parag}

\begin{parag}{Example 2: cardinality of power sets}
    Let $S$ be a set.

    Then there exists no surjective function $f : S \mapsto \mathcal{P}\left(S\right)$. In other words, $\left|S\right| < \left|\mathcal{P}\left(S\right)\right|$.

    \begin{subparag}{Proof}
        Let's assume for contradiction that such an $f$ exists. Let's consider the following set:
        \[T = \left\{s \in S | s \not\in f\left(s\right)\right\}\]

        Since $T \subseteq S$, we know that $T \in \mathcal{P}\left(S\right)$, and thus, since $f$ is surjective, there exists $s_0 \in S$ such that $f\left(s_0\right) = T$.

        If $s_0 \in T$, then $s_0 \not\in f\left(s_0\right) = T$, which is a contradiction.

        If $s_0 \not\in T = f\left(s_0\right)$, then $s_0 \in T$ by definition of $T$, which is also a contradiction.

        \qed
    \end{subparag}
\end{parag}

\begin{parag}{Theorem: Showing countability}
    An infinite set $S$ is countable if and only if it is possible to list the elements of the set in a sequence indexed by positive integers.

    \begin{subparag}{Proof $\implies$}
        Since $S$ is countable, there exists a bijection $f : \mathbb{Z}_+ \mapsto T$. Therefore, we can form the sequence $a_1, \ldots, a_n, \ldots$ where:
        \[a_1 = f\left(1\right), \ldots, a_n = f\left(n\right), \mathspace, \ldots\]
    \end{subparag}

    \begin{subparag}{Proof $\impliedby$}
        Since we can list the set in a sequence $\left(a_n\right)$ indexed by the positive integers, then we can define the function $f\left(n\right) = a_n$, which is a bijection.
    \end{subparag}
\end{parag}

\begin{parag}{Example}
    Let's show that the set of integers $\mathbb{Z}$ is countable. We can list its elements in a sequence:
    \[0, 1, -1, 2, -2, 3, -3, \mathspace\]

    Thus, $\mathbb{Z}$ is countable.
\end{parag}

\begin{parag}{Hilbert's Grand Hotel}
    Hilbert's Grand Hotel has an infinite number of rooms, each occupied by a guest. However, we can always accommodate a new guest at this hotel.

    Indeed, since the Grand Hotel is countable, we can list rooms: Room 1, Room 2, and so on. When a new guest arrives, we move the guest in Room $n$ to Room $n+1$ for all positive integers $n$. This frees up Room 1, which we assign to the new guest, and all the current guests still have rooms.
\end{parag}

\begin{parag}{Cardinality of rational numbers}
    We want to show that rational numbers, $\mathbb{Q}$, are countable.

    We can arrange them in a table, as $\frac{x}{y}$ in the $x$\Th row and $y$\Th column. We can follow a path that goes through all of them, and skip the ones we get multiple times ($\frac{1}{2}$ and $\frac{2}{4}$, for example).

    \begin{center}
        \imagehere[0.3]{CountabilityRationalNumbers.png}
    \end{center}
\end{parag}

\begin{parag}{Theorem}
    The union of a countable number of countable sets is countable.

    \begin{subparag}{Proof}
        We can again put them in a table and follow a path that goes through all of them.
    \end{subparag}
\end{parag}

\begin{parag}{Set of finite strings}
    The set of finite strings $s$ over a finite alphabet $A$ is countably infinite.

    Indeed, we can show that by listing all the strings. First, we list all the strings of length 0 in lexicographic order. Then, we list all the string of length 1 in lexicographic order. And we continue similarly.
\end{parag}

\begin{parag}{Theorem: Real numbers}
    The set of real numbers $\mathbb{R}$ is uncountable.

    \begin{subparag}{Proof}
        Let us take $\left[0,1\right]  \subseteq R$. Let's assume for contradiction that it is countable. Thus, we have a first irrational number that can be written as
        \[r_1 = 0.d_{11} d_{12} d_{13} d_{14} \ldots\]

        Using its decimal expansion. We can do the same for $r_2, \ldots, r_i$. So, generally,
        \[r_i = 0.d_{i1} d_{i2} \ldots d_{ii} \ldots\]

        Let us now take the number
        \[r = 0. d_1 d_2 d_3 \ldots\]
        where $d_i$ is defined as followed:
        \begin{functionbypart}{d_i}
        3 \text{ if } d_{ii} \neq 3 \\
        4 \text{ if } d_{ii} = 3
        \end{functionbypart}

        If $r$ were in the list, then $r = r_i$. However, it would then mean that if $d_{ii} \neq 3$ then $d_i = 3$, and if $d_{ii} = 3$, then $d_i \neq 3$, which is a contradiction. Thus, $r$ is not in the list and we have our contradiction.

        \qed
    \end{subparag}
\end{parag}

\end{document}
