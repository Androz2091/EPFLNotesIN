% !TeX program = lualatex

\documentclass{article}

% Expanded on 2021-09-28 at 08:13:59.

\usepackage{../../style}

\title{Lecture 3}
\author{Joachim Favre}
\date{Mardi 28 septembre 2021}

\begin{document}
\maketitle

\lecture{3}{2021-09-28}{Introduction to predicate logic}{
\begin{itemize}[left=0pt]
    \item Introduction and history of predicate logic.
    \item Definition of the concept of variables, predicates and propositional functions.
    \item Definition of quantifiers (universal, existential, and uniqueness).
    \item Definition of the validity of a statement (valid, satisfiable, or unsatisfiable).
\end{itemize}
}

\section{Predicate logic}
\subsection{Introduction}

\begin{parag}{Introduction}
    We realise that propositional logic is not enough.

    For example, if we know that ``all men are mortal'' and ``Socrates is a man'', we cannot say that ``Socrates is mortal'' using propositional logic. Indeed, we would have to ``look inside'' the propositions, but we cannot since sentences are either always true or always false.

    We will need to go further than propositional logic, with predicate logic. This kind of logic introduces a concept of quantifiers, and they allow us to overcome the limitations of the Aristotelian logic.
\end{parag}

\begin{parag}{Small history}
    Aristotle developed a (limited form of) predicate logic. It was then independently developed by Frege and Pierce between the \nth{19} and the \nth{20} century.

    Thus, it took also around 2000 years to develop.
\end{parag}

\begin{parag}{Impact in mathematics}
    Predicate logic (also called first-order logic) is the standard language to represent mathematical statements.

    Some mathematicians such as Hilbert hoped that this would be a complete system of logic to describe mathematics. However, in the twentieth century, Gödel proved that it is not possible not prove everything using predicate logic: this is Gödel's incompleteness theorem.
\end{parag}

\begin{parag}{Impact in computer science}
    Predicate logic is very important in computer science. Indeed, we find it when formulation general search queries in databases, in logic programming (such as Prolog), automated theorem proving, software verification, symbolic AI systems, and so on.
\end{parag}

\begin{parag}{Variables}
    As explained above, we want to characterise an object by its properties. Let's call that object $x$; this is a \important{variable}.

    We could then write $\text{man}\left(x\right)$ to say ``$x$ is a man'', or $x > 3$ to say ``$x$ is a number larger than 3''.
\end{parag}

\begin{parag}{Definitions}
    We define \important{predicates} to be statements that contains a variable.

    For example, the following statements are predicates: 
    \[x > 3, \mathspace x = y + 3, \mathspace x + y = z\]
    
    Its variables can be replaced by a value from a domain $U$, the \important{universe of discourse}, for example the integers. Depending on the value that the variable takes, the predicate becomes a proposition (from propositional logic) which is either true or false. Thus, connectives from propositional logic can be applied to predicate logic.
\end{parag}

\begin{parag}{Example 1}
    Let $R\left(x, y, z\right) := x + y = z$ and the domain be integers.

    $R\left(2, -1, 5\right)$ is false since $2 + \left(-1\right) \neq 5$; $R\left(3, 4, 7\right)$ is true since $3 + 4 = 7$; and we don't know for $R\left(x, 3, z\right)$ since $x + 3 = z$ can be true or false. 
\end{parag}

\begin{parag}{Example 2}
    Let $P\left(x\right) := x > 0$. Then, the following propositions are true: 
    \[P\left(3\right) \lor P\left(-1\right), \mathspace P\left(3\right) \to \lnot P\left(-1\right)\]
    
    And, the following propositions are false: 
    \[P\left(3\right) \land P\left(-1\right), \mathspace P\left(3\right) \to P\left(-1\right)\]
\end{parag}

\begin{parag}{Propositional functions}
    We call \important{propositional functions} expressions constructed from predicates and logical connectives containing variables.

    For example, the following expressions are propositional functions: 
    \[R\left(x, y\right) := P\left(x\right) \to P\left(y\right), \mathspace R\left(y\right) = P\left(3\right) \land P\left(y\right)\]
\end{parag}

\subsection{Quantifiers}
\begin{parag}{Quantifiers}
    We need \important{quantifiers} to express to which extent a propositional functions is true.

    Indeed, depending on what the value of the variables are, the propositional function can have different values. This is captured by the quantifiers.

    For example, $x > 0$ is true for 1, 2 and so on, but not for 0, -1, -2, \ldots. However, $x < x - 1$ is never true and $x < x + 1$ is always true.

    Such quantifiers can be used as logical connective.
\end{parag}

\begin{parag}{Universal quantifier}
    The universal quantification of a propositional function $P\left(x\right)$ is the statement ``$P\left(x\right)$ is true for all values $x$ from its domain $U$''. We write this $\forall x\ P\left(x\right)$, and read ``for all $x$, $P\left(x\right)$''. We call $\forall$ the \important{universal quantifier}.

    In other words, $\forall x\ P\left(x\right)$ means that for any $x$ in the universe, the proposition evaluates to true.
\end{parag}

\begin{parag}{Example}
    If $P\left(x\right) := x > 0$ and $U$ is the integers, then $\forall x\ P\left(x\right)$ is false, because $P\left(0\right)$ is false.

    Now, if we take the same predicate but with $U$ being the positive integers (note that \textit{\fullbf{0 is neither positive nor negative}} in this course), then $\forall x\ P\left(x\right)$ is true.

    Thus, it is important to notice that the result of a statement can depend on the universe we choose.
\end{parag}

\begin{parag}{Existential quantifier}
    The existential quantification of a propositional function $P\left(x\right)$ is the statement ``there exists an element $x$ from the domain $U$ such that $P\left(x\right)$ is true''. We write $\exists x\ P\left(x\right)$, and it reads ``for some $x$, $P\left(x\right)$'', or ``for at least one $x$, $P\left(x\right)$''. We call $\exists$ the \important{existential quantifier}.

    \begin{subparag}{Link with universal quantifier}
        We see that if $\forall x\ P\left(x\right)$ is true, then $\exists x\ P\left(x\right)$ must also be true (if the domain is not empty). Indeed, if a propositional function is true for every element, then there definitely exists an element for which the propositional function is true.

    \end{subparag}
\end{parag}

\begin{parag}{Empty domains}
    Note that when the domain is empty, weirder things appear. Indeed, $\forall x\ P\left(x\right)$ is true (every elements from the empty domain make $P\left(x\right)$ true) but $\exists x\ P\left(x\right)$ is false (there is no element from the domain (since it is empty) that makes $P\left(x\right)$ true).
    
    Thus, by convention, we never consider empty domains.
\end{parag}


\begin{parag}{Example}
    This time, if we have $P\left(x\right) := x > 0$, then $\exists x\ P\left(x\right)$ is true both if $U$ is the integers or if $U$ is the positive integers. However, it is false if $U$ is the negative integers.
\end{parag}

\begin{parag}{Summary}
    We can draw the following table:
    \begin{center}
    \begin{tabular}{|c|c|c|}
        \hline
        \textbf{Statement} & \textbf{Is true when} & \textbf{Is false when} \\
        \hline
        $\forall x\ P\left(x\right)$ & $P\left(x\right)$ is true for every $x$. & There is an $x$ for which $P\left(x\right)$ is false. \\
        $\exists x\ P\left(x\right)$ & There is an $x$ for which $P\left(x\right)$ is true. & $P\left(x\right)$ is false for every $x$. \\
        \hline
    \end{tabular}
    \end{center}

    We call a \important{counterexample} for $\forall x\ P\left(x\right)$, a value $x$ for which $P\left(x\right)$ is false; and a \important{witness} for $\exists x\ P\left(x\right)$, a value $x$ for which $P\left(x\right)$ is true.
\end{parag}

\begin{parag}{Quantifiers with finite domains}
    We can see that, in fact, if the domain $U$ is finite we do not need such quantifiers.

    Indeed, for example, if $U$ consists of the integers 1, 2 and 3, then $\forall x\ P\left(x\right)$ is equivalent to: 
    \[P\left(1\right) \land P\left(2\right) \land P\left(3\right)\]
    
    Moreover, $\exists x\ P\left(x\right)$ is equivalent to: 
    \[P\left(1\right) \lor P\left(2\right) \lor P\left(3\right)\]
    
\end{parag}

\begin{parag}{Uniqueness quantifier}
    We use $\exists!x\ P\left(x\right)$ to express the fact that $P\left(x\right)$ is true for one and only one $x$ in the domain $U$. We usually say ``there is a unique $x$ such that $P(x)$'', or ``there is one and only one $x$ such that $P\left(x\right)$''.

    \begin{subparag}{Expression using other quantifiers}
        This is skipping a bit ahead, but we can express the uniqueness quantifier using other quantifiers:
        \[\exists!x\ P\left(x\right) \equiv \exists x \left(P\left(x\right) \land \forall y \left(y \neq x \to \lnot P\left(y\right)\right)\right) \]
    \end{subparag}
\end{parag}

\begin{parag}{Example}
    For instance, if $P\left(x\right) := x + 1 = 0$ and $U$ is the integers, then $\exists!x\ P\left(x\right)$ is true. However, if $P\left(x\right) := x > 0$ with the same domain, then $\exists!\ P\left(x\right)$ is false.
\end{parag}

\begin{parag}{Composite statements involving quantifiers}
    We can apply connectives from propositional logic to predicates. For example: 
    \[\left(\forall x\ P\left(x\right)\right) \lor Q\left(x\right)\]
    
    Note that this is not a proposition since there is a $x$ in $Q\left(x\right)$, which is not bound to a quantifier. Moreover, we have to be careful about the fact that the two $x$'s are different variables: another way of writing this statement would have been $\left(\forall y\ P\left(y\right)\right)Q\left(x\right)$. 

    The quantifiers have higher precedence than all the logical connectives from propositional logic. For example, $\forall x\ P\left(x\right) \lor Q\left(x\right)$ means $\left(\forall x\ P\left(x\right)\right) \lor Q\left(x\right)$; $\forall x\left(P\left(x\right) \lor Q\left(x\right)\right)$ means something different.
\end{parag}

\begin{parag}{Variable binding}
    We say that a quantifier binds the variable of a propositional function.

    $P\left(x\right)$ is a propositional function with a \important{free variable} $x$, and $\forall x\ P\left(x\right)$ is a proposition with \important{bound variable} $x$.

    Using precedence rule, we see that $\forall x\ P\left(x\right) \lor Q\left(x\right)$ has a free variable, but $\forall x\ \left(P\left(x\right) \lor Q\left(x\right)\right)$ has no free variable.
\end{parag}

\begin{parag}{Validity and satisfiability}
    A statement involving predicates and quantifiers \textit{with all variables bound} is \important{valid} if it is true for all domains, and it is true for every propositional function substituted for the predicates in the assertion (in propositional logic we called this a tautology). 

    Similarly, such a statement is \important{satisfiable} if it is true for some domains, and if some propositional function can be substituted for the predicates in the assertion.

    Otherwise, it is \important{unsatisfiable}.

    In other words, valid means true whatever the variable we choose, satisfiable means that there exists some case in which it is true, unsatisfiable means that it is always false.
\end{parag}

\begin{parag}{Example}
    For instance, the following statement is valid: 
    \[\forall x\ \lnot S\left(x\right) \leftrightarrow \lnot \exists x\ S\left(x\right)\]
    
    The following statement is satisfiable since we can pick $F\left(x\right) \equiv x > 0$ and $T\left(x\right) \equiv x > 0$: 
    \[\forall x\left(F\left(x\right) \leftrightarrow T\left(x\right)\right)\]
    
    The following statement is unsatisfiable: 
    \[\forall x\left(F\left(x\right) \land \lnot F\left(x\right)\right)\]
\end{parag}

\begin{parag}{Translating from natural language}
    Let's say we want to translate ``every student in this class has taken a course in Java'' to predicate logic.

    Let's first define $J\left(x\right) := $ ``$x$ has taken a course in Java'' and $S\left(x\right) := $ ``$x$ is a student in this class''. We must then decide on the domain $U$. If it is all the students in the class, it is very easy, we can say $\forall x\ J\left(x\right)$. If $U$ is all people in the world, then the sentence can be translated to $\forall x\left(S\left(x\right) \to J\left(x\right)\right)$. Note that $\forall x\left(S\left(x\right) \land J\left(x\right)\right)$ is incorrect since it would mean that everyone on the Earth is a student and has taken a Java course.

    \vspace{1em}

    Let's do the same exercise but with ``some student in this class has taken a course in Java''. Again, if $U$ is all students in the class it is very easy since the sentence translates to $\exists x\ J\left(x\right)$. If $U$ is all people in the world, then it translates to $\exists x\left(S\left(x\right) \land J\left(x\right)\right)$. Note that $\exists x\left(S\left(x\right) \to J\left(x\right)\right)$ is incorrect since all students could have taken no Java course: there exists somebody on this Earth who is not a student and thus makes $S\left(x\right) \to J\left(x\right)$ true.
\end{parag}

\begin{parag}{Similar example}
    Let $P\left(x\right)$, $Q\left(x\right)$ and $R\left(x\right)$ be the statements ``$x$ is a clear explanation'', ``$x$ is satisfactory'', and ``$x$ is an excuse'', respectively. Let's pick all English texts as the domain for $x$.

    ``All clear explanations are satisfactory'' is translated to: 
    \[\forall x\left(P\left(x\right) \to Q\left(x\right)\right)\]
    
    ``Some excuses are unsatisfactory'' is translated to: 
    \[\exists x\left(R\left(x\right) \land \lnot Q\left(x\right)\right)\]
    
    ``Some excuses are not clear explanations'' is translated to: 
    \[\exists x\left(R\left(x\right) \land \lnot P\left(x\right)\right)\]
\end{parag}


\begin{parag}{Shorthand notation}
    We can define the following shorthand notations:
    \[\forall x \in S\ P\left(x\right) \equiv \forall x \left(x \in S \to P\left(x\right)\right)\]
    \[\exists x \in S\ P\left(x\right) \equiv \exists x\left(x \in S \land P\left(x\right)\right)\]

    This works too for equalities:
    \[\forall x \neq 0\ P\left(x\right) \equiv \forall x \left(x \neq 0 \to P\left(x\right)\right)\]

    \begin{subparag}{Negations}
        We can notice that this notation is coherent with negations. Indeed, we do have that:
        \[\lnot \forall x \in S\ P\left(x\right) \equiv \lnot \forall x\left(x \in S \to P\left(x\right)\right) \equiv \lnot \forall x \left(\lnot x \in S \lor P\left(x\right)\right)\]

        Which is equal to: 
        \[\exists x \left(x \in S \land \lnot P\left(x\right)\right) \equiv \exists x \in S\ \lnot P\left(x\right)\]
        

        We also find that:
        \[\lnot \exists x \in S\ P\left(x\right) \equiv \lnot \left(\lnot \forall x \in S\ \lnot P\left(x\right)\right) \equiv \forall x \in S\ \lnot P\left(x\right)\]

        This makes sense, since we have to keep the same hypothesis when negating a statement.
    \end{subparag}
\end{parag}

\begin{parag}{Sudoku}
    Propositional logic and predicate logic are powerful enough to describe games, such as Sudoku.

    Let's define $P\left(i, j, n\right) := \text{cell}\left(i, j\right)$ contains the number $n$. We then want to encode that ``every row contains every number''. For example, in the row 1, with the number 1:
    \[P\left(1, 1, 1\right) \lor P\left(1, 2, 1\right) \lor \ldots \lor P\left(1, 9, 1\right) = \lor_{j=1}^{9} p\left(1, j, 1\right)\]

    This is true for every row and for every number, thus it is translated to
    \[\land_{i=1}^{9} \land_{n=1}^{9} \lor_{j=1}^{9} P\left(i, j, n\right)\]

    Using predicate logic, we  can simplify the notation:
    \[\land_{i=1}^{9} \land_{n=1}^{9} \lor_{j=1}^{9} P\left(i, j, n\right) \equiv \forall i \forall n \exists j P\left(i, j, n\right)\]

    We can do the same reasoning for columns and squares. It would be incredibly hard to translate all our results to a truth table (there would be more rows than atoms in the universe, and if we created one universe for each atoms there would still not be enough atoms), but the point is that it can be translated.
\end{parag}

\subsection{Logical equivalences in predicate logic}
\begin{parag}{Equivalences}
    Two statements $S$ and $T$ involving predicates and quantifiers are logically equivalent if and only if they have the same truth values no matter which predicates are substituted and which domain of discourse is used.

    As mentioned earlier, by convention we do not consider empty domain of discourse.
\end{parag}

\begin{parag}{Example 1}
    We can draw the following equivalence: 
    \[\forall x \lnot \lnot S\left(x\right) \equiv \forall x\ S\left(x\right)\]
    
    Indeed, $\lnot \lnot S\left(x\right) \equiv S\left(x\right)$, and this is independent of the choice of $S$ and of $x$.
\end{parag}

\begin{parag}{Example 2}
    Let's say we want to show: 
    \[\forall x \left(P\left(x\right) \land Q\left(x\right)\right) \equiv \forall x P\left(x\right) \land \forall x Q\left(x\right)\]

    Note that to prove that $a \leftrightarrow b$, it is often easier to show $a \to b$ and $b \to a$.
    
    \begin{subparag}{Proof of $\to$}
        We suppose that $\forall x\left(P\left(x\right) \land Q\left(x\right)\right)$ is true.

        If $a$ is in the domain, then both $P\left(a\right)$ and $Q\left(a\right)$ are true. Since they are both true for every element $a$ in the domain, we know that both $\forall x P\left(x\right)$ and $\forall x Q\left(x\right)$ are true. Thus, $\forall x P\left(x\right) \land \forall x Q\left(x\right)$ is true.
    \end{subparag}

    \begin{subparag}{Proof of \textleftarrow}
        We suppose that $\forall x P\left(x\right) \land \forall x Q\left(x\right)$ is true. 

        If $a$ is in the domain, then both $P\left(a\right)$ and $Q\left(a\right)$ are true. Thus, $P\left(a\right) \land Q\left(a\right)$ is true, and we can deduce that $\forall x\left(P\left(x\right) \land Q\left(x\right)\right)$ is true. 
    \end{subparag}
\end{parag}

\begin{parag}{Distribution of quantifiers of connectives}
    Hereinabove we saw that the universal quantifier is distributive over the and connective. Note that the distribution of quantifiers over connectives is a bit special. Indeed, the following statements are true: 
    \[\forall x\left(P\left(x\right) \land Q\left(x\right)\right) \equiv \forall x P\left(x\right) \land \forall x Q\left(x\right)\]
    \[\exists x \left(P\left(x\right) \lor Q\left(x\right)\right) \equiv \exists x P\left(x\right) \lor \exists x Q\left(x\right)\]

    However:
    \[\exists x\left(P\left(x\right) \land Q\left(x\right)\right) \not\equiv \exists x P\left(x\right) \land \exists x Q\left(x\right)\]
    \[\forall x \left(P\left(x\right) \lor Q\left(x\right)\right) \not\equiv \forall x P\left(x\right) \lor \forall x Q\left(x\right)\]

    Indeed, for both we can use the following counter example: $P\left(x\right) := $ ``$x$ is even'', $Q\left(x\right) := $ ``$x$ is odd'' and $\mathbb{Z}$ as the domain.

\end{parag}

\end{document}
