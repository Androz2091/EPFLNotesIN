% !TeX program = lualatex

\documentclass{article}

% Expanded on 2021-09-29 at 15:13:32.

\usepackage{../../style}

\title{AICC-1}
\author{Joachim Favre}
\date{Mercredi 29 septembre 2021}

\begin{document}
\maketitle

\lecture{4}{2021-09-29}{Nested quantifiers, and negating them}{
\begin{itemize}[left=0pt]
    \item Explanation on how to negate quantifiers, using the De Morgan's Laws for quantifiers.
    \item Explanation on how to have nested quantifiers.
    \item Explanation of the argument form.
    \item Summary of the different methods we have to build knowledge.
\end{itemize}
}

\begin{parag}{De Morgan's Laws for Quantifiers}
    We have the following equivalences: 
    \[\lnot \forall P\left(x\right) = \exists x \lnot P\left(x\right)\]
    \[\lnot \exists P\left(x\right) = \forall x \lnot P\left(x\right)\]
    
    \begin{subparag}{Proof of the first one}
        We know that $\lnot \forall x P\left(x\right)$ is true if and only if $\forall x P\left(x\right)$ is false. This is equivalent to the existence of an element $a$ such that $P\left(a\right)$ is false, and thus $\lnot P\left(a\right)$ is true. From that, we can deduce that $\exists x \lnot P\left(x\right)$ is true.
    \end{subparag}
    
    \begin{subparag}{Justification of the name}
        Those laws are in fact the equivalents of De Morgan's Laws for infinite set. Indeed, if we have $U = \left\{1, 2, 3\right\}$, then: 
        \[\exists x\ P\left(x\right) = P\left(1\right) \lor P\left(2\right) \lor P\left(3\right) \]

        Thus, negating both side and applying de Morgan's Laws: 
        \[\lnot \left(\exists x\ P\left(x\right)\right) = \lnot P\left(1\right) \land \lnot P\left(2\right) \land \lnot P\left(3\right) = \forall x \lnot P\left(x\right)\]
    \end{subparag}
    
\end{parag}

\begin{parag}{Exercise}
    Let the following statements: 
    \begin{enumerate}
        \item $\exists!x P\left(x\right) \to \exists x P\left(x\right)$
        \item $\forall x P\left(x\right) \to \exists!x P\left(x\right)$
        \item $\exists!x \lnot P\left(x\right) \to \lnot \forall x P\left(x\right)$
    \end{enumerate}
    
    We wonder what are their truth values.

    \begin{enumerate}
        \item It is valid, since if there exists exactly one, then there exists at least one.
        \item It is satisfiable, when there is exactly one element in the domain.
        \item Using De Morgan's law, we see that it is equivalent to
        \[\exists! x \lnot P\left(x\right) \to \exists x \lnot P\left(x\right)\]
        which is equivalent to the first one, so it is valid.
    \end{enumerate}
    
    Now, let's consider what happens when we have an empty domain:
    \begin{enumerate}
        \item It is also valid.
        \item It is unsatisfiable since we have $T \to F$.
        \item It is also valid, since it is equivalent to the first one.
    \end{enumerate}
\end{parag}

\subsection{Nested quantifiers}
\begin{parag}{Nested quantifiers}
    \important{Nested quantifiers} are often necessary to express the meaning of sentences in natural language as well as important concepts in computer science and mathematics.

    \begin{subparag}{Example}
        We can translate ``every real number has an inverse'' as: 
        \[\forall x \exists y\left(x + y = 0\right)\]
        where the domains for both $x$ and $y$ are real numbers.
    \end{subparag}

    We can draw the following diagram, by linking two points if and only if $P\left(x, y\right)$ is true:
    \imagehere[0.7]{NestedQuantifiersDiagram.png}
\end{parag}

\begin{parag}{Nested propositional functions}
    We can also take \important{nested propositional functions}.

    For example, $\forall x \exists y \left(x + y = 0\right)$ can be viewed as $\forall x Q\left(x\right)$, where: 
    \[Q\left(x\right) := \exists y P\left(x, y\right), \mathspace P\left(x, y\right) := \left(x + y = 0\right)\]
    
    Note that $\exists y P\left(x, y\right)$ has an unbound variable.
\end{parag}

\begin{parag}{Order of quantifiers}
    The order of quantifiers is really important. 

    For example, let's take $U$ to be the real numbers, and $P\left(x, y\right) := \left(x + y = 0\right)$. Then, $\forall x \exists y P\left(x, y\right)$ is true (for all number, there does exist an inverse), but $\exists y \forall x P\left(x, y\right)$ is false (there does not exist any real number which is an inverse for every number).

    We see that it is not always the case. Indeed, if we have $Q\left(x\right) := \left(x + y = y + x\right)$, then both $\forall x \forall y P\left(x, y\right)$ and $\forall y \forall x P\left(x, y\right)$ are true (addition is indeed commutative).

    In fact, we can always switch the order of the quantifiers whenever we only have universal quantifiers or only existential quantifiers: 
    \[\forall x \forall y P\left(x, y\right) \equiv \forall y \forall x P\left(x, y\right)\]
    \[\exists x \exists y P\left(x, y\right) \equiv \exists y \exists x P\left(x, y\right)\]
\end{parag}

\begin{parag}{Translation to natural language}
    Let $F\left(x, y\right) := $ ``$x$ and $y$ are friends'' and students as the domain. Let's say we want to translate: 
    \[\exists x \forall y \forall z\left(F\left(x, y\right) \land F\left(x, z\right) \land y \neq z \to \lnot F\left(y, z\right)\right)\]
    When translated into natural language, we find: ``there exists a student, such that for all of his friends that are different, it is the case that they not friends.''

    It shows that it is not always evident to convert back and forth, and that our language has a huge flexibility and shortcuts.
\end{parag}


\begin{parag}{Translation from maths}
    Let's say we want to translate ``the sum of two positive integers is always positive'' into a logical expression.

    First, let's rewrite this statement to make the implied quantifiers and domains and domains explicit: ``for every two integers, if these integers are both positive, then the sum of these integers is positive''.

    Second, let's introduce the variables: ``for every two integers $x$ and $y$, if $x$ is positive and $y$ is positive, then $x+y$ is positive''.

    Third, we can translate it to predicate logic: 
    \[\forall x \forall y\left(x > 0 \land y > 0 \to x + y > 0\right)\]
\end{parag}


\begin{parag}{Translation from natural language}
    Let $L\left(x, y\right) := $ ``$x$ loves $y$''.

    Let's translate the following sentences:
    \begin{enumerate}
        \item ``Everybody loves somebody.''
        \item ``There is someone how is loved by everyone.''
        \item ``There is someone who loves someone.''
        \item ``Everyone loves himself.''
    \end{enumerate}

    We have:
    \begin{enumerate}
        \item $\forall x \exists y L\left(x, y\right)$
        \item $\exists x \forall y L\left(y, x\right)$
        \item $\exists x \exists y L\left(x, y\right)$
        \item $\forall x L\left(x, x\right)$
    \end{enumerate}
   
    \begin{subparag}{Personal remark}
        I truly hope that $\forall x \exists y \left(L\left(x, y\right) \land L\left(y, x\right)\right)$.
    \end{subparag}
\end{parag}

\begin{parag}{Example}
    Let's determine the truth values of the following statements, where the domain is the real numbers:
    \begin{enumerate}
        \item $\forall x \exists y\left(x^2 = y\right)$
        \item $\forall x \exists y \left(x = y^2\right)$
        \item $\exists x \forall y\left(xy = 0\right)$
        \item $\exists x \exists y\left(x + y \neq y + x\right)$
    \end{enumerate}
    
    We have:
    \begin{enumerate}
        \item It is true, we can choose $y = x^2$.
        \item It is false, we can pick $x = -1$.
        \item It is true, we can choose $x = 0$.
        \item It is false, since we know that $\forall x \forall y\left(x + y = y + x\right)$ (the addition is commutative), and this is the exact negation.
    \end{enumerate}
    
\end{parag}

\begin{parag}{Prenex Normal Form (not in exams I think)}
    We say that a statement is in Prenex Normal Form (PNF) if and only if it is of the form: 
    \[Q_1 x_1 Q_2 x_2 \ldots Q_k x_k P\left(x_1, x_2, \ldots, x_k\right)\]
    where each $Q_i$ is either the existential quantification or the universal quantifier, and $P\left(x_1, \ldots, x_k\right)$ is a predicate involving no quantifier.

    \begin{subparag}{Example}
        Let's say we have the following proposition: 
        \[\exists x P\left(x\right) \to \exists x Q\left(x\right)\]
        
        By doing some tricks (such as introducing another variable for clarity), we see that it is equivalent to: 
        \[\forall x \lnot P\left(x\right) \lor \exists y Q\left(y\right) \equiv \forall x \exists y\left(\lnot P\left(x\right) \lor Q\left(y\right)\right)\]

        The last one is the PNF.
    \end{subparag}
    
\end{parag}

\section{Proofs (deriving knowledge)}
\subsection{Valid arguments}

\begin{parag}{Deriving knowledge}
    Let's say we know something, let's say $p$ ($p$ is true). Then, we may want to extend this knowledge to other facts. We have already seen equivalence proofs, that show that: 
    \[p \leftrightarrow p'\]
    
    However, we may also want to derive knowledge, which only goes in one way (we are thus going to have something less general). If $p \to q$, then we can deduce that $q$ is true. However, if $q$ is true we cannot deduce that $p$ is true. Those are arguments.
\end{parag}

\begin{parag}{Definitions}
    An \important{argument} in propositional logic is a sequence of propositions. All but the final proposition are called \important{premises}, and the last statement is the \important{conclusion}. The argument is valid if the premises imply the conclusion.

    An \important{argument form} is an argument that is valid no matter what propositions are substituted into its propositional variables. \important{Inference rules} are simple arguments forms that will be used to construct more complex argument forms.
\end{parag}

\begin{parag}{Modus Ponens}
    We can note:
    \begin{center}
    \begin{tabular}{rl}
        & $p \to q$ \\
        & $p$ \\
        \hline
        $\therefore$ & $q$
    \end{tabular}
    \end{center}
    
    In other words, if we know that $p\to q$ and $p$ are true, then we can deduce $q$. Note that $\therefore$ means ``therefore'' (it must not be confused with $\because$ which means ``because''). 
    
    We call this the \important{Modus Ponens}, it is a valid argument form.

    \begin{subparag}{Example}
        Let's say that we know that ``if I have passed AICC, then I can advance to year 2 of the studies'' ($p \to q$) and ``I have passed AICC'' ($p$). Then we can conclude that ``I can advance to year 2 of the studies'' ($q$). 

        This is a valid argument.
    \end{subparag}
    
    \begin{subparag}{Validity}
        We know that the following proposition is a tautology:
        \[\left(p \land p \to q\right) \to q\]

        So, if we know that $p$ and $p \to q$ are true, then $q$ must necessarily be true (else, the proposition would not be a tautology). Thus, the Modus Ponens is indeed a valid argument form.
    \end{subparag}
    
\end{parag}

\begin{parag}{Valid argument}
    As mentioned earlier, a valid argument is a sequence of statements, where each statement is either a premise or follows from previous statements by inference rules. It takes the following form:
    \begin{center}
    \begin{tabular}{rl}
        & $p_1$ \\
        & $\vdots$  \\
        & $p_n$ \\
        \hline
        $\therefore$ & $q$
    \end{tabular}
    \end{center}

    We can use the same reasoning as for the Modus Ponens to derive inference rules out of any tautology of the form: 
    \[\left(p_1 \land p_2 \land \ldots \land p_n\right) \to q\]
\end{parag}

\begin{parag}{Example 1}
    We know the following inference rule (which is Modus Ponens):
    \begin{center}
    \begin{tabular}{rl}
        & $p$ \\
        & $p \to q$ \\
        \hline
        $\therefore$ & $q$
    \end{tabular}
    \end{center}

    So, applying it to some specific propositions, we get the following argument form: 
    \begin{center}
    \begin{tabular}{rl}
        & $r \land s$ \\
        & $\left(r \land s\right) \to q$ \\
        \hline
        $\therefore$ & $q$
    \end{tabular}
    \end{center}

    Thus, we can derive the following argument:
    \begin{center}
    \begin{tabular}{rl}
        & $p$ \\
        & $p \to \left(r \land s\right)$ \\
        & $\left(r \land s\right) \to q$ \\
        & $r \land s$ \\
        \hline
        $\therefore$ & $q$
    \end{tabular}
    \end{center}

    Where the 3 first propositions are premises, and $r \land s$ is found using Modus Ponens.
\end{parag}

\begin{parag}{Usefulness}
    Let's say we know that ``If I have passed AICC, Analysis 1, Linear Algebra, etc., then I can advance to year 2 of the studies'', and ``I have passed AICC'', ``I have passed Analysis 1'', ``I have passed Linear Algebra'', ... 

    Then, it seems clear we can deduce that ``I can advance to year 2 of the studies''. However, we need to show that this argument is valid.

    \begin{subparag}{Truth table}
        Let's say we wanted to prove this using a truth table. If we had 20 school subjects, then we would have $2^{20} = 1\,048\,576$ rows, which is not very practical. 
    \end{subparag}

    \begin{subparag}{Arguments}
        To show this using arguments, we first need to introduce another inference rule, conjunction: 
        \begin{center}
        \begin{tabular}{rl}
            & $p$  \\
            & $q$  \\
            \hline
            $\therefore$ & $p \land q$
        \end{tabular}
        \end{center}
        
        This is based on the following tautology: 
        \[p \land q \to p \land q\]

        We can now use this to find the following argument form:
        \begin{center}
        \begin{tabular}{rl}
            & $\left(p_1 \land \ldots \land p_n\right) \to q$ \\
            & $p_1$   \\
            & $\vdots$ \\
            & $p_n$  \\
            & $p_1 \land \ldots \land p_n$ \\
            \hline
            $\therefore$ & $q$
        \end{tabular}
        \end{center}
        
        Thus, our argument was indeed valid!
    \end{subparag}
\end{parag}

\begin{parag}{Example: logic programming}
    There are some langages based on logic, for example \textit{Prolog}.

    We have predicates, such as \texttt{lecturer(L, C)} (at a course) and \texttt{student(S, C)}.

    Then we can add facts: \texttt{lecturer(Karl, CS101)}, \texttt{student(Frank, CS101)}, \texttt{student(Cléa, CS101)}, \ldots

    We can also add rules: \texttt{teaches(L, S) :- lecturer(L, C), student(S, C)}. Such a rule will be true if the lecturer and the student have the same class, \texttt{C}.

    We can then make queries: \texttt{teaches(Karl, Frank)} which will say True. Or, we can also get a list using \texttt{teaches(Karl, X)}.
\end{parag}


\end{document}
