\documentclass[a4paper]{article}

% Expanded on 2021-10-13 at 15:14:34.

\usepackage{../../style}

\title{AICC-1}
\author{Joachim Favre}
\date{Mercredi 13 octobre 2021}

\begin{document}
\maketitle

\lecture{8}{2021-10-13}{End of functions and beginning of relations}{
\begin{itemize}[left=0pt]
    \item Definition of inverse, composed and partial functions.
    \item Definition of relations, and their compositions.
    \item Explanation of relations on a set, and their possible properties (reflexive, symmetric, antisymmetric and transitive).
\end{itemize}
}

\parag{Proofs}{
    Let $f : A \mapsto B$ be a function.

    To prove that $f$ is injective, we select arbitrary $x, y \in A$, we show that if $f\left(x\right) = f\left(y\right)$ then $x = y$. To show that $f$ is not injective, we find $x, y \in A$ such that $x \neq y$ and $f\left(x\right) = f\left(y\right).$

    To prove that $f$ is surjective, we select an arbitrary $y \in B$, and we show the existence of an element $x \in A$ such that $f\left(x\right) = y$. To show that $f$ is not surjective, we must find a $y \in B$ such that $f\left(x\right) \neq y$ for all $x \in A$.
}

\parag{Examples}{
    \begin{center}
        \begin{tabularx}{\linewidth}{>{\hsize=0.35\hsize}X>{\hsize=0.65\hsize}X}
        $f : \mathbb{Z} \mapsto \mathbb{Z}, f\left(x\right) = x + 1$ & Injective and surjective. \\
        $f : \mathbb{N} \mapsto \mathbb{N}, f\left(x\right) = x + 1$ & Injective but not surjective since $f\left(x\right) = 0$ has no solution.  \\
        $f : \mathbb{Z} \mapsto \mathbb{Z}, f\left(x\right) = x^2$ & Not injective since $f\left(-1\right) = 1 = f\left(1\right)$ and not surjective since $f\left(x\right) = -1$ has no solution.
    \end{tabularx} 
    \end{center}
}

\parag{Inverse functions}{
    Let $f$ be a bijection from $A$ to $B$. The \important{inverse} of $f$, denoted $f^{-1}$, is the function from $B$ to $A$ defined as: 
    \[f^{-1}\left(y\right) = x \iff f\left(x\right) = y\]
    
    \subparag{Importance of bijectivity}{
        If $f$ was not injective, then $f^{-1}$ would have multiple values for some $x$.

        If $f$ was not surjective, the $f^{-1}$ would not be defined for some $x$.
    }
    
    \subparag{Example}{
        The inverse of $f: \mathbb{Z} \mapsto \mathbb{Z}, f\left(x\right) = x+1$ is: 
        \[\begin{split}
        f^{-1}: \mathbb{Z} &\longmapsto \mathbb{Z} \\
        x &\longmapsto x - 1
        \end{split}\]

        However, the function $g: \mathbb{R} \mapsto \mathbb{R}, g\left(x\right) = x^2$ has no inverse since it is neither surjective nor injective.
    }
}

\parag{Composition of functions}{
    Let $f: B \mapsto C$ and $g: A \mapsto B$ be functions. The \important{composition} of $f$ with $g$, denoted $f \circ g$ is the function from $A$ to $C$ defined by: 
    \[\left(f \circ g\right)\left(x\right) = f\left(g\left(x\right)\right)\]

    \subparag{Example}{
        \imagehere{FunctionComposition.png}

        Similarly, if $f\left(x\right) = x^2$ and $g\left(x\right) = x + 1$, then: 
        \[f\left(g\left(x\right)\right) = f\left(x + 1\right) = \left(x + 1\right)^2 = x^2 + 2x + 1\]
        \[g\left(f\left(x\right)\right) = g\left(x^2\right) = x^2 + 1\]

        We can note that, in general, composition is not commutative.
    }
}

\parag{Partial functions}{
    A \important{partial function} $f$ from a set $A$ to a set $B$ is an assignment to each element $a$ in a \textit{subset} of $A$, called the \important{domain of definition} of $f$, of a unique element $b$ in $B$.

    The sets $A$ and $B$ are called the \important{domain} and \important{codomain} of $f$ respectively. We say that $f$ is \important{undefined} for elements in $A$ that are not in the domain of definition of $f$. When the domain of definition of $f$ equals $A$, we say that $f$ is a \important{total function}.

    \subparag{Example}{
        Let the following function:
        \[\begin{split}
        f: \mathbb{Z} &\longmapsto \mathbb{R} \\
        n &\longmapsto \sqrt{n}
        \end{split}\]
        
        This is a partial function, since its $-1$ is in the domain but the function is undefined for it. The domain of definition of the function is $\mathbb{N}$ ($f$ is undefined for negative integers).
    }
}

\parag{Proposition}{
    Let $f, g$ be functions. If $g$ and $g \circ f$ are both injective, then $f$ is injective.

    \subparag{Proof}{
        Let's assume for contradiction that there exists $x_1, x_2$ such that $f\left(x_1\right) = f\left(x_2\right) = c$ and $x_1 \neq x_2$. So, 
        \[\left(g \circ f\right)\left(x_1\right) = g\left(f\left(x_1\right)\right) = g\left(c\right) = g\left(f\left(x_2\right)\right) = \left(g \circ f\right)\left(x_2\right)\]
        
        Since $g \circ f$ is injective,this implies that $x_1 = x_2$. This is a contradiction.

        \qed
    }
}

\parag{Properties}{
    Let $f: A \mapsto B$. Then for any sets $S, T \subseteq A$:
    \begin{enumerate}
        \item $f\left(S \cup T\right) = f\left(S\right) \cup f\left(T\right)$
        \item $f\left(S \cap T\right) \subseteq f\left(S\right) \cap f\left(T\right)$
    \end{enumerate}
    
    \subparag{Justification for (2)}{
        Let's have an example of $f\left(S \cap T\right) \neq f\left(S\right) \cap f\left(T\right)$. Let $f: \left\{0, 1\right\}\mapsto \left\{0, 1\right\}$, defined such that $f\left(0\right) = 1$ and $f\left(1\right) = 1$. So, on the left hand side: 
        \[f\left(\left\{0\right\} \cap \left\{1\right\}\right) = f\left(\o\right) = \o\]

        On the right hand side: 
        \[f\left(\left\{0\right\}\right) \cap f\left(\left\{1\right\}\right) = \left\{1\right\} \cap \left\{1\right\} = \left\{1\right\}\]
        
        Indeed, we can write intersections the following way: 
        \[f\left(S \cap T\right) = \left\{y | \exists x\left(x \in S \cap T \land y = f\left(x\right)\right)\right\}\]

        And: 
        \begin{multiequality}\label{eq:label}
            f\left(S\right) \cap f\left(T\right) & = \left\{y |\exists x\left(x \in S \land y = f\left(x\right)\right)\right\} \cap \left\{y | \exists x\left(x \in T \land y = f\left(x\right)\right)\right\}  \\
            & = \left\{y | \exists x\left(x \in S \land y = f\left(x\right)\right) \land \exists x \left(x \in T \land y = f\left(x\right)\right)\right\}
        \end{multiequality}

        
        But to prove that both sides are equal, we would need to have the distributivity of the existence quantifier over $\land$, which is not a true law.
    }
}

\parag{Example}{
    Let $f : \left]0, 1\right[ \mapsto \mathbb{R}$ be the function defined as
    \begin{functionbypart}{f\left(x\right)}
        2 - \frac{1}{x}, \mathspace x \in \left]0, \frac{1}{2}\right[  \\
        \frac{1}{1-x} - 2, \mathspace x \in \left[\frac{1}{2}, 1\right[ 
    \end{functionbypart}
   
    We can convince ourselves that it is monotonic by studying the different cases, and thus injective. This should be enough for a MCQ, but let's prove it formally.
    
    \subparag{Proof}{
        Let's prove it by contraposition, we want to show that if $x_1 \neq x_2$, then $f\left(x_1\right) \neq f\left(x_2\right)$.

        \important{Case 1:} Let's pick $x_1 < \frac{1}{2}$ and $x_2 \geq \frac{1}{2}$. We then have $f\left(x_1\right) < 0$ and $f\left(x_2\right) \geq 0$. So, necessarily: 
        \[f\left(x_1\right) \neq f\left(x_2\right)\]

        \important{Case 2:} Let's pick $x_1, x_2 < \frac{1}{2}$, with $x_1 \neq x_2$. Then we have $f\left(x_1\right) = 2 - \frac{1}{x_1}$ and $f\left(x_2\right) = 2 - \frac{1}{x_2}$. However, we know that:
        \[x_2 \neq x_1 \iff \frac{1}{x_2} \neq \frac{1}{x_1} \iff 2 - \frac{1}{x_2} \neq 2 - \frac{1}{x_1}\]
        
        \important{Case 3:} Let's pick $x_1, x_2 \geq \frac{1}{2}$, with $x_1 \neq x_2$. This is left as an exercise for the reader.
    }
    
}

\section{Relations, sequences and summations}
\subsection{Relations}
\parag{Definition: binary relations}{
    A \important{binary relation} $R$ from a set $A$ to a set $B$ is a subset of their Cartesian product: 
    \[R \subseteq A \times B\]

    We notice that $\o$ is a relation from any two sets $A$ and $B$. 
    
    \subparag{Example}{
        Let $A = \left\{0, 1, 2\right\}$ and $B = \left\{a, b\right\}$. Then, the following set is a relation from $A$ to $B$: 
        \[R = \left\{\left(0, a\right), \left(0, b\right), \left(1, a\right), \left(2, b\right)\right\}\]
    }

    \subparag{Link with functions}{
        Relations are more general concepts than functions. Indeed, we can always construct a relation from a function:
        \[\left\{(a_1, f(a_1)), (a_2, f(a_2)), \ldots\right\}\]

        In general, the inverse cannot be done (there might be a number which would get multiple images).
    }
}

\parag{Representations}{
    We can use directed graphs and tables to express relations:
    \imagehere[0.5]{RelationRepresentation.png}

    We see that this could not be a function, since there are multiple arrows leaving the element $0$.
}


\parag{Combining relations}{
    Let $R_1$ and $R_2$ be two relations. We can combine them using basic set operations to form new relations, such as: 
    \[R_1 \cup R_2, \mathspace R_1 \cap R_2, \mathspace R_1 \setminus R_2, \mathspace R_2 \setminus R_1\]
}

\parag{Compositions}{
    Let $A, B, C$ be sets, $R$ be a relation from $A$ to $B$, and $S$ be a relation from $B$ to $C$. The \important{composite} of $R$ and $S$, denoted $S \circ R$, is the relation consisting of ordered pairs $\left(a, c\right)$, where $a \in A$, $c \in C$ and for which there exists an element $b \in B$ such that $\left(a, b\right) \in R$ and $\left(b, c\right) \in S$.

    Note that the order is important, $S \circ R \neq R \circ S$ in general (one may even be defined, whereas the other not).

    \subparag{Example}{
        Let's look at the following diagram:
        \imagehere[0.7]{RelationComposition.png}

        In this case, we have: 
        \[S \circ R = \left\{\left(b, x\right), \left(b, z\right)\right\}\]
    }
    
}

\parag{Definition: $N$-ary relations}{
    Let $A_1, \ldots, A_n$ be sets. An \important{$n$-ary relation} on these sets is a subset of the Cartesian product $A_1 \times \ldots \times A_n$. The sets are called the \important{domains} of the relation, and $n$ is called its \important{degree}.

    \subparag{Example}{
        Database tables are $n$-ary relations:
        \imagehere{DatabaseRelation.png}

        \textit{Personal remark: Note that Livai (or Mikasa) is a student in this school.}
    }
    
}

\subsection{Relations on a set}
\parag{Definition: Relation on a set}{
    A \important{binary relation $R$ on a set $A$} is a subset of the Cartesian product $A \times A$. In other words, it is a relation from $A$ to $A$.

    \subparag{Example}{
        Let $A = \left\{a, b, c\right\}$. The following set is a relation on $A$: 
        \[R = \left\{\left(a, a\right), \left(a, b\right), \left(a, c\right)\right\}\]
    }
}

\parag{Definition: Reflexive relation}{
    A relation $R$ on a set $A$ is \important{reflexive} if and only if $\left(a, a\right) \in R$ for every element $a \in A$.
    
    In other words, $R$ is reflexive if and only if: 
    \[\forall x\left(x \in A \implies \left(x, x\right) \in R\right)\]
}

\parag{Definition: Symmetric relations}{
    A relation $R$ on a set $A$ is \important{symmetric} if and only if $\left(b, a\right) \in R$ whenever $\left(a, b\right) \in R$ for all $a, b \in A$.

    In other words, $R$ is symmetric if and only if: 
    \[\forall x \forall y \left(\left(x, y\right) \in R \to \left(y, x\right)\right)\]
}

\parag{Definition: Antisymmetric relations}{
    A relation $R$ on a set $A$ is \important{antisymmetric} if and only if $\left(a, b\right) \in R$ and $\left(b, a\right) \in R$ then $a = b$ for all $a, b \in A$.

    In other words, $R$ is antisymmetric if and only if: 
    \[\forall x \forall y \left(\left(x, y\right) \in R \land \left(y, x\right) \in R \to x = y\right)\]
    
    \subparag{Remark}{
        Symmetric and antisymmetric are not opposite of each other. The following relation is both symmetric and antisymmetric: 
        \[\left\{\left(a, a\right)\right\}\]
        
        Similarly, the following relation is neither symmetric nor antisymmetric: 
        \[\left\{\left(a, b\right), \left(b, a\right), \left(a, c\right)\right\}\]
    }
}

\parag{Definition: Transitive relations}{
    A relation $R$ on a set $A$ \important{transitive} if and only if, whenever $\left(a, b\right) \in R$ and $\left(b, c\right) \in R$, then $\left(a, c\right) \in R$, for all $a, b, c \in A$.

    In other words, $R$ is transitive if and only if: 
    \[\forall x \forall y \forall z\left(\left(x, y\right) \in R \land \left(y, z\right) \in R \to \left(x, z\right) \in R\right)\]
}

\parag{Summary}{
    We can draw the following table:
    \begin{center}
    \begin{tabular}{|ll|}
        \hline
        \fullbf{Name} & \fullbf{Definition} \\
        \hline
        \textit{Reflexive} & $\forall x \left(x \in A \to \left(x, x\right) \in R\right)$ \\
        \textit{Symmetric} & $\forall x \forall y \left(\left(x, y\right) \in R \to \left(y, x\right) \in R\right)$ \\
        \textit{Antisymmetric} & $\forall x \forall y \left(\left(x, y\right) \in R \land \left(y, x\right) \in R \to x = y\right)$ \\
        \textit{Transitive} & $\forall x \forall y \forall z \left(\left(x, y\right) \in R \land \left(y, z\right) \in R \to \left(x, z\right) \in R\right)$ \\
        \hline
    \end{tabular}
    \end{center}
}

\parag{Examples}{
    \begin{center}
    \begin{tabular}{|lllll|}
        \hline
        \fullbf{Relation} & \fullbf{Reflexive} & \fullbf{Symmetric} & \fullbf{Antisymmetric} & \fullbf{Transitive} \\
        \hline
        $\left\{\left(a, b\right) | a \leq b\right\}$          & Yes & No  & Yes & Yes \\
        $\left\{\left(a, b\right) | a > b\right\}$          & No  & No  & Yes & Yes \\
        $\left\{\left(a, b\right) | a = b \lor a = -b\right\}$ & Yes & Yes & No  & Yes \\
        $\left\{\left(a, b\right) | a = b\right\}$          & Yes & Yes & Yes & Yes \\
        $\left\{\left(a, b\right) | a = b + 1\right\}$      & No  & No  & Yes & No  \\
        $\left\{\left(a, b\right) | a + b \leq 3\right\}$      & No  & Yes & No  & No  \\
        \hline
    \end{tabular}
    \end{center}
}

\parag{Number of relations on a set}{
    We wonder how many relations can be made on a set $A$. We know that: 
    \[\left|A \times A\right| = \left|A\right|\cdot \left|A\right| = \left|A\right|^2\]
     
    Then, every subset of $A \times A$ can be a relation, so we are looking at the power set: 
    \[\left|\mathcal{P}\left(A \times A\right)\right| = 2^{\left|A \times A\right|} = 2^{\left|A\right|^2}\]
    
    So, there are often many possible relations, but it is finite on a finite set.
}

\end{document}
