% !TeX program = lualatex

\documentclass{article}

% Expanded on 2021-09-30 at 08:03:30.

\usepackage{../../style}

\title{Algèbre linéaire}
\author{Joachim Favre}
\date{Jeudi 30 septembre 2021}

\begin{document}
\maketitle

\lecture{3}{2021-09-30}{Équations vectorielles et matricielles}{
\begin{itemize}[left=0pt]
    \item Définition des équations vectorielles.
    \item Définition du span, ou vect en français.
    \item Définition du produit matrice-vecteur.
    \item Définition des équations matricielle.
    \item Observation que les systèmes d'équations, les matrices augmentées, les équations vectorielles et les équations matricielles sont des notations équivalentes.
    \item Définition des systèmes homogènes et de leur solution triviale.
\end{itemize}

}

\begin{parag}{Exemple 2}
    Dessinons les combinaisons linéaires possibles de $\bvec{v}_1$ et $\bvec{v}_2$ s'ils sont définis tels que
    \[\bvec{v}_1 = \begin{bmatrix} -1 \\ 1 \end{bmatrix}, \mathspace \bvec{v}_2 = \begin{bmatrix} 2 \\ 1 \end{bmatrix}  \]


    \imagehere{GrilleCombinaisonLineaire.png}

    Les combinaisons linéaires des vecteurs forment une grille, on se rend compte qu'on peut atteindre n'importe quelle vecteur.

    On peut maintenant se poser la question suivante: existe-t-il $x_1, x_2 \in \mathbb{R}$ tels que $x_1 \bvec{v}_1 + x_2 \bvec{v}_2 = \bvec{b}$ avec
    \[\bvec{v}_1 = \begin{bmatrix} -1 \\ 1 \end{bmatrix}, \begin{bmatrix} 2 \\ 1 \end{bmatrix} \text{ et } \bvec{b} = \begin{bmatrix} 5 \\ 2 \end{bmatrix} \]

    Si oui quelles valeurs de $x_1$ et $x_2$ conviennent ? On appelle ceci une \important{équation vectorielle}.
\end{parag}

\subsubsection{Équation vectorielle}

\begin{parag}{Équation vectorielle}
    Nous cherchons une solution à l'équation mentionnée ci-dessus. Ainsi, on a
    \[c_1 \begin{bmatrix} -1 \\ 1 \end{bmatrix} + c_2 \begin{bmatrix} 2 \\ 1 \end{bmatrix} = \begin{bmatrix} 5 \\ 2 \end{bmatrix} \iff \begin{bmatrix} -c_1 \\ c_1 \end{bmatrix} + \begin{bmatrix} 2c_2 \\ c_2 \end{bmatrix} = \begin{bmatrix} -c_1 + 2c_2 \\ c_1 + c_2 \end{bmatrix} = \begin{bmatrix} 5 \\ 2 \end{bmatrix} \]

    On se retrouve donc avec le système d'équations suivantes
    \[\begin{systemofequations}
    -c_1 + 2c_2 = 5 \\
    c_1 + c_2 = 2
    \end{systemofequations}\]
\end{parag}

\begin{parag}{Exemple en 3D avec deux vecteurs de base}
    On a
    \[\bvec{a}_1 = \begin{bmatrix} 1 \\ -2 \\ -5 \end{bmatrix}, \bvec{a}_2 = \begin{bmatrix} 2 \\ 5 \\ 6 \end{bmatrix} \text{ et } \bvec{b} = \begin{bmatrix} 7 \\ 4 \\ -3 \end{bmatrix} \]

    On veut savoir si $\bvec{b}$ est une combinaison linéaire des deux autres vecteurs, donc on a l'équation suivante:
    \[x_1 \bvec{a}_1 + x_2 \bvec{a}_2 = \bvec{b}\]

    En faisant les mêmes étapes que toute à l'heure on se retrouve avec le système
    \[\begin{systemofequations}
    x_1 + 2x_2 = 7 \\
    -2x_1 + 5x_2 = 4 \\
    -5x_1 + 6x_2 = -3
    \end{systemofequations}\]

    La matrice augmentée est donnée par
    \[\begin{bmatrix} 1 & 2 & 7 \\ -2 & 5 & 4 \\ -5 & 6 & -3 \end{bmatrix} = \begin{bmatrix}  &  &  \\ \bvec{a}_1 & \bvec{a}_2 & \bvec{b} \\  &  &  \end{bmatrix} \]

    On peut donc clairement écrire la matrice directement.
\end{parag}

\begin{parag}{Conclusion}
    Une équation vectorielle avec $\bvec{a}_1, \ldots, \bvec{a}_n$ et $\bvec{b}$ dans $\mathbb{R}^m$
    \[x_1 \bvec{a}_1 + \ldots + x_n \bvec{a}_n = \bvec{b}\]

    a le même ensemble solution que le système linéaire dont la matrice augmentée est
    \[\begin{bmatrix}  &  &  &  \\ &  &  &  \\ \bvec{a}_1 & \ldots & \bvec{a}_n & \bvec{b} \\  &  &  &  \\  &  &  &  \end{bmatrix} \]

    En particulier, $\bvec{b}$ peut être généré comme combinaison linéaire de $\bvec{a}_1, \ldots, \bvec{a}_n$ si et seulement si ce système linéaire a (au moins) une solution.
\end{parag}

\begin{parag}{Définition}
    Étant donnés des vecteurs $\bvec{v}_1, \ldots, \bvec{v}_n \in \mathbb{R}^n$, on écrit $\vect\left(\bvec{v}_1, \ldots, \bvec{v}_p\right)$ ou $\Span\left(\bvec{v}_1, \ldots, \bvec{v}_p\right)$ (en anglais) pour désigner \important{le sous-ensemble de $\mathbb{R}^n$ engendré par $\bvec{v}_1, \ldots, \bvec{v}_n$}, c'est-à-dire l'ensemble des vecteurs de la forme
    \[c_1 \bvec{v}_1 + \ldots + c_n \bvec{v}_n\]

    où les $c_i$ peuvent être n'importe quel scalaire.

    \begin{subparag}{Remarque}
        Il est intéressant de noter que le vecteur nul fait toujours parti du span de n'importe quel ensemble de vecteurs.
    \end{subparag}


\end{parag}

\subsection{Équation matricielle}

\begin{parag}{Équation matricielle}
    On peut rajouter une troisième manière équivalente de dire les choses.

    Pour $n$ vecteurs $\bvec{a}_1, \ldots, \bvec{a}_n \in \mathbb{R}^m$, on a défini la combinaison linaire. Utilisons les matrices pour simplifier la notation:

    \begin{subparag}{Définition}
        Soit $A$ une matrice de taille $m \times n$ (on écrit $A \in \mathbb{R}^{m \times n}$). Soient $\bvec{a}_1, \ldots, \bvec{a}_n \in \mathbb{R}^m$ les colonnes de $A$, de sorte que
        \[A = \begin{bmatrix}  &  &  \\ \bvec{a}_1 & \ldots & \bvec{a}_n \\  &  &  \end{bmatrix} \]

        On définit \important{le produit de $A$ avec tout vecteur $\bvec{x}$}
        \[\bvec{x} = \begin{bmatrix} x_1 \\ \vdots \\ x_n \end{bmatrix} \in \mathbb{R}^n\]

        comme suit:
        \[A \bvec{x} = x_1 \bvec{a}_1 + \ldots x_n \bvec{a}_n\]

        Il est important de noter, que cette définition ne marche que si le nombre de colonnes de $A$ est égal au nombre de coefficients de $\bvec{x}$.
    \end{subparag}

\end{parag}

\begin{parag}{Exemple}
    Si on a
    \[A = \begin{bmatrix} 3 & -2 & 1 \\ 2 & 9 & 8 \end{bmatrix} \text{ et } \bvec{x} = \begin{bmatrix} 2 \\ -1 \\ 0 \end{bmatrix} \]

    Alors,
    \[A \bvec{x} = 2\begin{bmatrix} 3 \\ 2 \end{bmatrix} - 1 \begin{bmatrix} -2 \\ 9 \end{bmatrix} + 0 \begin{bmatrix} 1 \\ 8 \end{bmatrix} = \begin{bmatrix} 8 \\ -5 \end{bmatrix} \]

    Si on avait un $\bvec{x}$ avec des composantes quelconque (évêque):
    \[A \bvec{x} = x_1 \begin{bmatrix} 3 \\ 2 \end{bmatrix} + x_2 \begin{bmatrix} -2 \\ 9 \end{bmatrix} + x_3 \begin{bmatrix} 1 \\ 8 \end{bmatrix} = \begin{bmatrix} 3x_1 -2x_2 + 3x_3 \\ 2x_1 + 9x_2 + 8x_3 \end{bmatrix} \]

    On observe de manière plus générale que le produit $A \bvec{x}$ peut être interprété en termes des lignes de $A$ également. En d'autres mots, si on a
    \[A = \begin{bmatrix} a_{11} & \cdots & a_{1n} \\ \vdots &  & \vdots \\ a_{m1} & \ldots & a_{mn} \end{bmatrix} \text{ et } \bvec{x} = \begin{bmatrix} x_1 \\ \vdots \\ x_n \end{bmatrix} \]

    Alors,
    \[A \bvec{x} = x_1 \begin{bmatrix} a_{11} \\ \vdots \\ a_{m1} \end{bmatrix} + \ldots + x_n \begin{bmatrix} a_{1n} \\ \vdots \\ a_{mn} \end{bmatrix}  = \begin{bmatrix} a_{11} x_1 + \ldots + a_{1n} x_n \\ \vdots \\ a_{m1} x_1 + \ldots + a_{mn} x_n \end{bmatrix} \]

\end{parag}

\begin{parag}{Observation}
    Considérons le système de deux équations en trois inconnues suivant:
    \[\begin{systemofequations}
    3x_1 - 2x_2 + x_3 = b_1 \\
    2x_1 + 9x_2 + 8x_3 = b_2
    \end{systemofequations}\]

    En le transformation en équation de vecteur, on a:
    \[x_1 \begin{bmatrix} 3 \\ 2 \end{bmatrix} + x_2 \begin{bmatrix} -2 \\ 9 \end{bmatrix} + x_3 \begin{bmatrix} 1 \\ 8 \end{bmatrix} = \begin{bmatrix} b_1 \\ b_2 \end{bmatrix} \]

    Puis en équation matricielle, on a
    \[\begin{bmatrix} 3 & -2 & 1 \\ 2 & 9 & 8 \end{bmatrix} \begin{bmatrix} x_1 \\ x_2 \\ x_3 \end{bmatrix} = A \bvec{x} = \begin{bmatrix} b_1 \\ b_2 \end{bmatrix} \]

    Par ailleurs, la matrice augmentée de ce système est
    \[\begin{bmatrix} 3 & -2 & 1 & b_1 \\ 2 & 9 & 8 & b_2 \end{bmatrix}  \]

    Toutes ces notations sont équivalentes.
\end{parag}

\begin{parag}{En général}
    Soit $A \in \mathbb{R}^{m \times n}$ une matrice avec colonnes $\bvec{a}_1, \ldots \bvec{a}_n \in \mathbb{R}^{m}$, et soit $\bvec{b} \in \mathbb{R}^m$, un vecteur.

    L'équation matricielle
    \[A \bvec{x} = \bvec{b}\]
    est équivalent à (= a le même ensemble solution que) l'équation vectorielles
    \[x_1 \bvec{a}_1 + \ldots + x_n \bvec{a}_n = \bvec{b}\]

    Les deux sont de plus équivalentes au système d'équation dont la matrice augmentée est
    \[\begin{bmatrix}  &  &  &  \\ \bvec{a}_1 & \ldots & \bvec{a}_n & \bvec{b} \\  &  &  &  \end{bmatrix} \in \mathbb{R}^{m \times \left(n+1\right)}
        \]

    Finalement, si on écrit
    \[A = \begin{bmatrix} a_{11} & \ldots & a_{1n} \\ \vdots &  & \vdots \\ a_{m1} & \ldots & a_{mn} \end{bmatrix} \]

    ce système s'écrit encore comme:
    \[\begin{systemofequations}
    a_{11} x_1 + \ldots + a_{1n} x_n = b_1  \\
    \vdots \\
    a_{m1} x_1 + \ldots + a_{mn} x_n = b_m
    \end{systemofequations}\]
\end{parag}

\begin{parag}{Interprétation géométrique}
    Un système peut s'écrire $A \bvec{x} = \bvec{b}$, ce qui est équivalent à l'équation
    \[x_1 \bvec{a}_1 + \ldots + x_n \bvec{a}_n = \bvec{b}\]

    Donc, il existe (au moins) une solution pour $A \bvec{x} =  \bvec{b}$ si et seulement si $\bvec{b}$ est une combinaison linéaire des colonnes de $A$, c'est à dire, que $\bvec{b} \in \vect\left(\bvec{a}_1, \ldots, \bvec{a}_n\right)$
\end{parag}

\begin{parag}{Théorème}
    Pour $A \in \mathbb{R}^{m \times n}$, les quatre propriétés suivantes sont équivalentes (elles sont soit toutes vraies, soit toutes fausses):
    \begin{enumerate}
        \item Pour tout $\bvec{b} \in \mathbb{R}^m$, l'équation $A \bvec{x} = \bvec{b}$ a (au moins) une solution.
        \item Tout vecteur de $\mathbb{R}^m$ est une combinaison linéaire des colonnes de $A$.
        \item Les colonnes de $A$ engendrent $\mathbb{R}^{m}$.
        \item Il existe dans chaque ligne de $A$ une position pivot (attention, $A$ est la matrice des coefficients, pas la matrice augmentée).
    \end{enumerate}


    Note pour (4): on ne peut pas avoir que des zéros dans une ligne, car ça voudrait dire que la matrice échelonnée aurait un pivot dans la colonne de $\bvec{x}$ (puisque, de toutes façons, il a des composantes dans chaque ligne). Donc on serait dans le cas où on aurait $0 = c$, qui est une contradiction.
\end{parag}

\begin{parag}{Théorème}
    Soit $A \in \mathbb{R}^{m\times n}$, $\bvec{u}, \bvec{v} \in \mathbb{R}^{n}$ et $c \in \mathbb{R}$. Alors,
    \[A \left( \bvec{u} + \bvec{v}\right) = A \bvec{u} + A \bvec{v}\]

    De manière similaire,
    \[A\left(c \bvec{u}\right) = c A \bvec{u}\]

    \begin{subparag}{Preuve}
        La preuve est considérée comme trivial et laissée au lecteur comme exercice.

        Cependant, voici un ca particulier pour que les notation soient correctes: Supposons pour simplifier que $n = 3$:
        \[A = \begin{bmatrix}  &  &  \\ \bvec{a}_1 & \bvec{a}_2 & \bvec{a}_3 \\  &  &  \end{bmatrix} , \bvec{u} = \begin{bmatrix} u_1 \\ u_2 \\ u_3 \end{bmatrix} \text{ et } \bvec{v} = \begin{bmatrix} v_1 \\ v_2 \\ v_3 \end{bmatrix} \]

        Alors,
        \[A\left(\bvec{u} + \bvec{v}\right) = A \left( \begin{bmatrix} u_1 \\ u_2 \\ u_3 \end{bmatrix} + \begin{bmatrix} v_1 \\ v_2 \\ v_3 \end{bmatrix}\right) = A \begin{bmatrix} u_1 + v_1 \\ u_2 + v_2 \\ u_3 + v_3 \end{bmatrix} \]

        Or, en utilisant la définition du produit matrice-vecteur, on obtient:
        \[\left(u_1 + v_1\right) \bvec{a}_1 + \left(u_2 + v_2\right) \bvec{a}_2 + \left(u_3 + v_3\right) \bvec{a}_3\]

        En distribuant et en réorganisant les termes, on a
        \[u_1 \bvec{a}_1 + u_2 \bvec{a}_2 + u_3 \bvec{a}_3 + v_1 \bvec{a}_1 + v_2 \bvec{a}_2 + v_3 \bvec{a}_3 = A \bvec{u} + A \bvec{v}\]

    \end{subparag}

\end{parag}

\subsection{Système homogène}


\begin{parag}{Système homogène}
    On appelle le cas particulier suivant
    \[A \bvec{x}= \bvec{0}\]
    un système \important{homogène}.

    Cela correspond à un système de la forme
    \[\begin{systemofequations}
    a_{11} x_1 + \ldots + a_{1n} x_n = 0 \\
    \vdots \\
    a_{m1} x_1 + \ldots + a_{mn} x_n = 0
    \end{systemofequations}\]

    Ce système a toujours une solution --- appelée \important{solution triviale}:
    \[\bvec{x} = \bvec{0}\]

    Cela nous montre donc que les systèmes de cette forme sont toujours compatibles.

    Maintenant, on peut se demander quand il a d'autres solutions (donc une infinité par notre théorème).
\end{parag}

\begin{parag}{Théorème}
    L'équation homogène $A \bvec{x} = \bvec{0}$ admet une solution non-triviale si et seulement si elle possède au moins une variable libre (c'est-à-dire que le système a une infinité de solutions, puisqu'on sait qu'il est compatible).


\end{parag}


\begin{parag}{Exemple}
    Si on a un système
    \[\begin{systemofequations}
    -3x_1 + 5x_2 - 4x_3 = 0 \\
    -3x_1 - 2x_2 + 4x_3 = 0 \\
    6x_1 + x_2 - 8x_3
    \end{systemofequations}\]

    On remarque que peu importe les opérations élémentaires qu'on applique sur la matrice augmentée, les zéros vont toujours rester dans lea dernière colonne. Pour notre système, on se rend compte qu'il est équivalent au système suivant:
    \[\begin{systemofequations}
    x_1 = \frac{4}{3}x_3 \\
    x_2 = 0 \\
    0 = 0
    \end{systemofequations}\]

    On remarque donc que $x_3$ est libre. Ainsi, on peut écrire les solutions sous la forme d'un vecteur:
    \[\bvec{x} = \begin{bmatrix} x_1 \\ x_2 \\ x_3 \end{bmatrix} = \begin{bmatrix} \frac{4}{3}x_3 \\  0\\  x_3 \end{bmatrix} = x_3 \begin{bmatrix} \frac{4}{3} \\ 0 \\ 1 \end{bmatrix} \]

    On voit donc clairement que l'ensemble des solutions représente une droite qui passe par l'origine (puisque $\bvec{x} = \bvec{0}$ est une solution).

    L'ensemble solution est donc de la forme $\bvec{x} = t \bvec{v}$ avec $t \in \mathbb{R}$, un paramètre. Autrement dit, l'ensemble solution est la droite $\vect\left(\bvec{v}\right)$, qui passe par l'origine.
\end{parag}

\begin{parag}{Exemple 2}
    Si on a le système
    \[\begin{systemofequations}
    10x_1 - 3x_2 - 2x_3 = 0
    \end{systemofequations}\]

    En passant par une matrice augmentée, on peut la réduire (simplement en divisant par 10), puis on peut revenir en système
    \[\begin{systemofequations}
    x_1 = \frac{3}{10}x_2 + \frac{2}{10}x_3 \\
    x_2, x_3 \text{ libres}
    \end{systemofequations}\]

    Donc, les solutions sont:
    \[\bvec{x} = \begin{bmatrix} x_1 \\ x_2 \\ x_3 \end{bmatrix} = \begin{bmatrix} \frac{3}{10}x_2 + \frac{2}{10}x_3 \\ x_2 \\ x_3 \end{bmatrix} = x_2 \begin{bmatrix} \frac{3}{10} \\  1\\  0\end{bmatrix} + x_3 \begin{bmatrix} \frac{2}{10} \\ 0 \\ 1 \end{bmatrix} = s \bvec{u} + t \bvec{v} \]

    Où $s, t \in \mathbb{R}$ sont des paramètres. Une écriture équivalente des solution est un plan (puisque $\bvec{u}$ et $\bvec{v}$ ne sont pas colinéaires), le plan $\vect\left(\bvec{u}, \bvec{v}\right)$.
\end{parag}


\end{document}
