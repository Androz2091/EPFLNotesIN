\documentclass{article}

% Expanded on 2021-09-21 at 23:02:17.

\usepackage{../../style}

\title{Analyse 1}
\author{Joachim Favre}
\date{Mercredi 22 septembre 2021}

\begin{document}
\maketitle

\lecture{1}{2021-09-22}{Organisation et prérequis}{
\begin{itemize}[left=0pt]
    \item Explication de l'organisation du cours.
    \item Révision des identités algébriques polynomiales, trigonométriques, exponentielles et logarithmiques.
    \item Révision de la trigonométrie.
    \item Révision des types de fonctions (polynômiales, rationnelles, algébriques, élémentaires transcendantes, et réciproques), du concept de bijectivité, des fonctions composées et des transformations des graphiques.
\end{itemize}

}

\section{Organisation}
\parag{Informations générales}{
    La Professeure est Anna Lachowska. Les informations liées au cours seront sur Moodle. Voir \url{annalachowska.github.io}.
}

\parag{Cours}{
    Les notes écrites seront disponibles en PDF après le cours. De plus, on peut trouver le cours pré enregistré de l'année passée sur SwitchTube. Il y a cependant les suites définies par récurrence et les limites supérieures/inférieures qu'on verra cette année, alors qu'ils ne l'ont pas vu l'année passée.

    Les cours sont naturellement retransmis sur Zoom, où on peut poser des questions. 
}

\parag{Exercices}{
    Les étudiants sont répartis dans différentes classes, voir sur Moodle. La série d'exercice prend plus de 2h, donc il vaut mieux la faire avant (en général la série sera mise le dimanche).

    Il n'est pas obligatoire d'aller au séances d'exercices le samedi. De manière générale, une seule fois par semaine est suffisante (que ce soit le jeudi ou le dimanche), mais il est plutôt recommandé de venir le jeudi.
}


\parag{Questions}{
    Nous pouvons poser nos questions aux assistants, ou sur le Forum Piazza, où les étudiants et les assistants pourront nous répondre.
}

\parag{Sujets}{
    \begin{itemize}[left=0pt]
        \item Axiomes des nombres réels et complexes
        \item Suites et séries numériques (limites d'une suite)
        \item Fonctions réelles d'une variable réelle, limite et continuité, dérivée, étude
        \item Série entière, polynômes de Taylor
        \item Intégrales définies, indéfinies et impropres.
    \end{itemize}
}

\parag{Littérature}{
    L'ouvrage de référence est Jacques Douchet et Bruno Zwahlen, \textit{Calcul différentiel et intégrale}. Il n'est pas nécessaire de lire pour le cours, tout est dans les lectures, mais cela peut être pratique pour réviser les définitions.
}

\parag{Examen}{
    Un seul écrit de 3h, principalement composé de QCM et de vrai-faux. On n'aura pas le droit d'avoir de document ou de calculatrice.
}

\section{Prérequis}
\subsection{Identités algébriques}
\parag{Polynomiales}{
    \begin{itemize}[left=0pt]
        \item $\left(x + y\right)^2 = x^2 + 2xy + y^2$
        \item $x^2 - y^2 = (x + y)(x - y)$
        \item $x^3 - y^3 = (x - y)(x^2 + xy + y^2)$
        \item $x^3 + y^3 = \left(x + y\right)\left(x^2 -xy + y^2\right)$
    \end{itemize}
 
}

\parag{Exponentielles}{
    Soit $a, b$ nombre réels positifs; $x, y$ nombres réels, $n$ naturel positif:
    \begin{itemize}
        \item $a^x a^y = a^{x+y}$
        \item $\frac{a^x}{a^y} = a^{x-y}$
        \item $\left(ab\right)^x = a^x b^x$
        \item $a^0$ = 1 
        \item $\left(a^x\right)^y = a^{x\cdot y}$
        \item $\sqrt[n]{a} = a^{\frac{1}{n}}$
        \item $\left(\frac{a}{b}\right)^x = \frac{a^x}{b^x}$
        \item $a^1 = a$
    \end{itemize}
    
}

\parag{Logarithmes}{
    Dans ce cours, $\log\left(x\right) = \log_e\left(x\right)$. Avec $x, y$ réels positifs:
    \begin{itemize}
        \item $\log\left(xy\right) = \log\left(x\right) + \log\left(y\right)$
        \item $\log\left(\frac{x}{y}\right) = \log\left(x\right) - \log\left(y\right)$
        \item $\log\left(x^c\right) = c\log\left(x\right) \mathspace c\in\mathbb{R}$
        \item $\log_a\left(1\right) = 0 \mathspace a \in \mathbb{R} \setminus \left\{1\right\}$
        \item $\log_a\left(a\right)=1$
    \end{itemize}
    
}

\parag{Trigonométrie}{
    \imagehere{DefinitionSinxCosx.png}
    On définit $\sin\left(x\right)$ et $\cos\left(x\right)$ comme projection pour tout $x$ réel. On définit 
    \[\tg\left(x\right) = \frac{\sin\left(x\right)}{\cos\left(x\right)}, \cos\left(x\right) \neq 0 \mathspace \text{ et } \mathspace \ctg\left(x\right) = \frac{\cos\left(x\right)}{\sin\left(x\right)}, \sin\left(x\right) \neq 0\]
    
    Il est important de connaître la propriété suivante: 
    \[\sin\left(x \pm y\right) = \sin\left(x\right)\cos\left(y\right) \pm \cos\left(x\right)\sin\left(y\right) \mathspace \text{ et } \mathspace \cos\left(x \pm y\right) = \cos\left(x\right)\cos\left(y\right) \mp \sin\left(x\right)\sin\left(y\right)\]

   De là, on peut trouver toutes les égalités, type que:
   \[1 = \cos\left(x - x\right) = \cos^2\left(x\right) + \sin^2\left(x\right)\]

   Pour trouver les valeurs de $\cos\left(x\right), \sin\left(x\right)$ pour certaines valeurs:

   \imagehere{ValsSinxCosx.png}
}

\subsection{Types de fonctions élémentaires}
\begin{enumerate}[left=0pt]
    \item Polynômes: $f\left(x\right) = 3x^3 + 5x + 4$. 
        \begin{itemize}
            \item Linéaire: $f\left(x\right) = ax + b \mathspace a,b$ nombres réels
            \item Quadratique: $f\left(x\right) = ax^2 + bx + c \mathspace a,b,c$ réels, mais $a\neq 0$
        \end{itemize}
    \item Fonctions rationnelles: $f\left(x\right) = \frac{P\left(x\right)}{Q\left(x\right)}$ où $P\left(x\right)$ et $Q\left(x\right)$ sont des polynômes, mais $Q\left(x\right) \neq 0$
    \item Fonctions algébriques: Toute fonction obtenue obtenue à partir des polynômes par application des opérations $+, -, \div, \cdot, \sqrt{}$.
    Example: $f\left(x\right) = \sqrt{x}, x > 0$ 
\item Fonctions (élémentaires, toujours, parce que sinon y'a les intégrales qui arrivent) transcendantes: les fonctions qui ne sont pas algébriques
        \begin{itemize}
            \item Fonctions trigonométriques (et leur réciproques): $f\left(x\right) = \sin\left(x\right)$ et $f\left(x\right) = \cos\left(x\right)$. Il existe une relation entre les deux: $\cos\left(x\right) = \sin\left(x + \frac{\pi}{2}\right)$. Il faut savoir les dessiner:

            \item Fonctions exponentielles et logarithmiques: $f\left(x\right) = e^x$ ; $g\left(x\right) = \log\left(x\right), x > 0$ ($\log$ en base $e$). 

                Nous avons que pour tout x réel $\log\left(e^x\right) = x$ et pour tout réel positif: $e^{\log\left(x\right)} = x$. Cela veut donc dire que $e^x$ et $\log\left(x\right)$ sont des fonctions réciproques. 
        \end{itemize}
\end{enumerate}

\parag{}{
    \imagehere{GraphSinxCosx.png} 
    \imagehere{GraphExpxLnx.png}
}
 
\parag{Fonctions réciproques}{
    En général, si on a 
    \[y = f\left(x\right) \over{\iff}{\text{déf}} x = f^{-1}\left(y\right)\]
    on dit que $f\left(x\right)$ et $f^{-1}\left(x\right)$ sont réciproques (il faut préciser les valeurs admissibles). 

    Les graphiques de $f$ et $f^{-1}$ sont symétriques par rapport à la droite $x = y$, car par définition, on lit le même graphique mais en échangeant les axes $x$ et $y$ ; c'est le même point qu'on peut lire de manière différente.

    \subparag{Exemple}{
        Si on a 
        \[f\left(x\right) = a^x \mathspace a > 0, a \neq 1, x \text{ réel}\]
        
        Alors, la fonction réciproque de $f$ est $f^{-1}\left(x\right) = \log_a\left(x\right), x > 0$.
    }
    
}

\subsection{Fonctions bijectives et réciproques}
\parag{Définition du domain de définition et de l'ensemble image}{
    Soient $E, F$ deux ensembles des nombres réels. 

    $f: E \mapsto F$ est une règle qui donne une seule valeur $f\left(x\right)$ pour $x \in D_f \subset E$. Avec 
    \[D_f = D\left(f\right) \over{=}{\text{déf}} \left\{x \in E : f\left(x\right) \text{ est bien définie}\right\} = \text{\important{le domaine de définition}}\]
    \[f\left(D\right) \over{=}{\text{déf}} \left\{y \in F: \exists x \in D_f \telque f\left(x\right) = y\right\} = \text{\important{l'ensemble image}}\]
}

\parag{Définition de surjectivité}{
    $f: E \mapsto F$ est \important{surjective} si $\forall y \in F$, $\exists x \in D_f \telque f\left(x\right) = y$.
}

\parag{Définition d'injectivité}{
    $f: E \mapsto F$ est \important{injective} si $\forall x_1, x_2 \in D_f \telque f\left(x_1\right) = f\left(x_2\right) \implies x_1 = x_2$
}

\parag{Définition de bijectivité}{
    Si $f: E \mapsto F$ est injective et surjective, elle est \important{bijective}.
}

\parag{Exemple}{
    Si on prend $f: \mathbb{R} \mapsto \mathbb{R}$, $f\left(x\right) = x^2$ n'est pas surjective. 

    Cependant, si on considère $f: \mathbb{R} \mapsto \mathbb{R}_+$, $f\left(x\right) = x^2$ est surjective, mais elle n'est pas injective car $f\left(-2\right) = f\left(2\right)$

    Mais, $f: \mathbb{R}_+ \mapsto \mathbb{R}_+$, $f\left(x\right) = x^2$ est injective. Elle est donc aussi bijective. On peut donc voir qu'on peut couper l'ensemble de définition pour rendre une fonction injective, et couper l'ensemble d'image pour rendre une fonction surjective. En faisant les deux, nous pouvons rendre une fonction bijective et donc pouvoir définir la fonction réciproque.
}

\parag{Définition de la fonction réciproque}{
    Pour $f: E \mapsto F$ bijective, on définit la fonction réciproque par l'équation $f\left(x\right) = y \iff x = f^{-1}\left(y\right)$, avec $x\in E$ et $y \in F$. 

    La fonction doit être injective, car sinon on aurait plusieurs solutions de $f\left(x\right) = y$ par rapport à $y$, et si elle n'était pas surjective $x = f^{-1}\left(y\right)$ pourrait ne pas avoir qu'une seule valeur à certains point. 

    Si la fonction est bijective, on est donc certains de l'existence de la réciproque. Cependant, cela ne veut pas dire qu'il est facile de trouver cette fonction.
}

\parag{Exemple 1}{
    $f\left(x\right) = x^2: \mathbb{R}_+ \mapsto \mathbb{R}_+$ est bijective, donc $f^{-1}\left(x\right) = \sqrt{x}: \mathbb{R}_+ \mapsto \mathbb{R}_+$ est sa fonction réciproque.

    \imagehere{GraphSqxSqrtx.png}

    À nouveau, les graphiques sont symétriques par rapport à la droite $y = x$.
}

\parag{Exemple 2}{
    Par convention, on choisit $\left[-\frac{\pi}{2};\frac{\pi}{2}\right]$ comme domaine de définition pour rendre $\sin$ bijectif (on pourrait en choisir un autre, mais c'est une convention; et c'est important pour $\Arcsin$).

    $\sin: \left[-\frac{\pi}{2}; \frac{\pi}{2}\right] \mapsto \left[-1;1\right] $ est bijective. 
    \[y = \sin\left(x\right) \iff x \over{=}{\text{déf}} \Arcsin\left(y\right)\]
   
    On a $\Arcsin: \left[-1; 1\right] \mapsto \left[-\frac{\pi}{2}; \frac{\pi}{2}\right]$
}

\parag{Exemple 3}{
    $\cos: \left[0; \pi\right] \mapsto \left[-1;1\right]$ est bijective.
    \[y = \cos\left(x\right) \iff x = \Arccos\left(y\right)\]

    On a $\Arccos: \left[-1;1\right] \mapsto \left[0;\pi\right]$
}

\parag{}{
    \imagehere{GraphSinCosArcsinArccos.png}
}


\subsection{Fonctions composées}
\parag{Définition de fonctions composées}{
    Soit $f: D_f \mapsto \mathbb{R}$, $g: D_g \mapsto \mathbb{R}$. Supposons que $f\left(D_f\right) \subset D_g$. Alors on peut définir la \important{fonction composée}: $g \circ f: D_f \mapsto \mathbb{R}$ par la formule $\left(g \circ f\right)\left(x\right) = g\left(f\left(x\right)\right)$. La composition se lit donc de droite à gauche. 

    En général, $g \circ f \neq f \circ g$
}

\parag{Exemple 1}{
    $f\left(x\right) = 2x + 3$ et $g\left(x\right) = \sin\left(x\right)$ donne 
    \[g \circ f\left(x\right) = g\left(2x + 3\right) = \sin\left(2x + 3\right)\]
    
    De la même manière 
    \[f \circ g\left(x\right) = f\left(\sin\left(x\right)\right) = 2\sin\left(x\right) + 3\]
}

\parag{Exemple 2}{
    Essayons de combiner les fonctions réciproques et les fonctions composées. On sait que $f^{-1} \circ f\left(x\right) = f^{-1}\left(f\left(x\right)\right) = f^{-1}\left(y\right) = x$ par définition d'une fonction réciproque, et que $f \circ f^{-1}\left(y\right) = f\left(f^{-1}\left(y\right)\right) = f\left(x\right) = y$. 
}

\parag{Exemple 3}{
    Exercice au lecteur: $f\left(x\right) = x^3$, calculer $f\circ f \circ f \circ f\left(x\right)$
}

\subsection{Transformation des graphiques}
\parag{Exemple 1}{
    \imagehere{GraphSqxDeplacement.png}

    On fait monter le graphique de $f\left(x\right)$ en faisant $f\left(x\right) + c$, ou on déplace le graphique \textbf{vers la gauche} en faisant $f\left(x + c\right)$.
}

\parag{Exemple 2}{
    \imagehere{GraphSinxDeplacement.png}

    En prenant $cf\left(x\right)$ il faut étendre le graphique de $f\left(x\right)$ en direction verticale. 

    En prenant $f\left(cx\right)$, alors il faut serrer le graphique dans la direction horizontale (si $\left|c\right| > 1$, sinon (si on divise par une valeur, en gros) il faut l'étendre). 
}





\end{document}
