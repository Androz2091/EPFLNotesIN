% !TeX program = lualatex

\documentclass[a4paper]{article}

% Expanded on 2021-10-27 at 10:07:19.

\usepackage{../../style}

\title{Analyse 1}
\author{Joachim Favre}
\date{Mercredi 27 octobre 2021}

\begin{document}
\maketitle

\lecture{11}{2021-10-27}{Cratères de convergence lunaires}{
    \begin{itemize}[left=0pt]
        \item Commentaire hors cours sur la fonction zêta de Riemann.
        \item Définition de la convergence absolue, et preuve qu'une série absolument convergente est convergente.
        \item Explication et preuve de la condition nécessaire (``test for divergence''), du critère de Leibnitz pour les séries alternées, du critère de comparaison pour les séries à termes non-négatifs, du critère de d'Alembert et du critère de Cauchy.
    \end{itemize}

}

\begin{parag}{Exemple 3}
    La série suivante est convergente:
    \[\sum_{n=1}^{\infty} \frac{1}{n^2} \]

    \begin{subparag}{Preuve}
        Soit la suites des séries partielles:
        \[S_n = \sum_{k=1}^{n} \frac{1}{k^2} = 1 + \underbrace{\frac{1}{2^2} + \frac{1}{3^2}}_{< 2\cdot \frac{1}{2^2}} + \underbrace{\frac{1}{4^2} + \frac{1}{5^2}}_{< 2\cdot\frac{1}{4^2}} + \underbrace{\frac{1}{6^2} + \frac{1}{7^2}}_{< 2\cdot \frac{1}{6^2}} + \ldots + \frac{1}{n^2}\]

        Donc:
        \[S_n < 1 + 2 \sum_{k = 1}^{n} \frac{1}{\left(2k\right)^2} = 1 + 2 \sum_{k = 1}^{n} \frac{1}{4} \frac{1}{k^2} = 1 + \frac{1}{2} \sum_{k = 1}^{n} \frac{1}{k^2}\]

        En d'autres mots:
        \[S_n < 1 + \frac{1}{2}S_n \implies \frac{1}{2}S_n < 1 \implies S_n < 2, \mathspace \forall n \geq 2\]

        Donc $\left(S_n\right)$ est majorée par 2. De plus,
        \[S_{n+1} - S_n = \frac{1}{\left(n+1\right)^2} > 0\]
        ce qui veut dire que $\left(S_n\right)\uparrow$. Puisqu'elle est aussi majorée, elle converge donc.

        On peut en conclure que la série $\sum_{n=1}^{\infty} \frac{1}{n^2}$ est convergente.

    \end{subparag}

    \begin{subparag}{Remarque 1}
        La série
        \[\sum_{n=1}^{\infty} \frac{1}{n^{p}}\]
        est convergente pour tout $p > 1$. (La preuve est dans la série 7.)
    \end{subparag}

    \begin{subparag}{Remarque 2}
        On a que
        \[\sum_{n=1}^{\infty} \frac{1}{n^2} = \zeta\left(2\right) = \frac{\pi^2}{6}\]

        La preuve de cette égalité est dans un fichier sur Moodle.
    \end{subparag}
\end{parag}

\begin{parag}{La fonction zêta de Riemann}
    La fonction zêta de Riemann est définie telle que
    \[\zeta\left(s\right) = \sum_{n=1}^{\infty} \frac{1}{n^{s}}, \mathspace s > 1\]

    \begin{subparag}{Valeurs connues}
        \begin{itemize}[left=0pt]
            \item $\zeta\left(2\right) = \frac{\pi^2}{6}$
            \item $\zeta\left(2k\right) = C_k \pi^{2k}$, où $C_k \in \mathbb{Q}$.

                On sait que $\zeta\left(2k\right)$ est transcendant pour tout $k \in \mathbb{N}^*$

            \item $\zeta\left(3\right)$ est transcendant.
            \item Au moins une valeur entre $\zeta\left(5\right)$, $\zeta\left(7\right)$, $\zeta\left(9\right)$ et $\zeta\left(11\right)$ est irrationnelle.
        \end{itemize}

        Une hypothèse dit que $\zeta\left(n\right)$ est transcendant pour tous $n \geq 2$ naturels.
    \end{subparag}

    \begin{subparag}{Hypothèse de Riemann}
        Riemann a permis d'étendre cette fonction pour tous le nombres complexes $z \neq 1$. Son hypothèse dit que tous les zéros non-triviaux de cette fonction ont comme partie réelle $\frac{1}{2}$.
    \end{subparag}

    \begin{subparag}{Exercice (hors du cours)}
        Démontrer que pour tout $s \geq 2$ naturel:
        \[\left(\prod_{p \text{ premier}}^{} \left(1 - \frac{1}{p^{s}}\right)\right)\zeta\left(s\right) = 1\]

        Astuce, calculer:
        \[\frac{1}{2^{s}}\zeta\left(s\right)\]

        Pour voir une preuve complète, il y a un fichier sur Moodle pour voir cette preuve.

    \end{subparag}

    \begin{subparag}{Remarque}
        La fonction zêta de Riemann n'est pas au champ pour l'examen.
    \end{subparag}
\end{parag}


\begin{parag}{Définition (Convergence absolue)}
    Une série $\sum_{n = 1}^{\infty} a_n$ est dite \important{absolument convergente} si la série
    \[\sum_{n = 1}^{\infty} \left|a_n\right| \]
    est convergente.
\end{parag}

\begin{parag}{Proposition}
    Une série absolument convergente est convergente.

    \begin{subparag}{Preuve}
        Soit les suite des sommes partielles suivantes:
        \[S_n = \sum_{k=1}^{n} a_k, \mathspace P_n = \sum_{k=1}^{n} \left|a_k\right|\]

        Par hypothèse, $\left(P_n\right)$ converge. C'est donc une suite de Cauchy. Pour tout $m, n \geq n_0$ (on prend $m > n$ sans perte de généralité (spdg)), on en déduit que:
        \[\left|S_m - S_n\right| = \left|\sum_{k=n+1}^{m} a_k\right| \leq \sum_{k=n+1}^{m} \left|a_k\right| = \left|P_m  - P_n\right| \leq \epsilon\]
        puisque $\left(P_n\right)$ est une suite de Cauchy. On sait donc que $\left(S_n\right)$ est aussi une suite de Cauchy (puisque pour tout $\epsilon > 0$, il existe $n_0$ tel que pour tout $m \geq n \geq n_0$, alors $\left|S_m - S_n < \epsilon\right|$).

        On en déduit donc que $\left(S_n\right)$ converge.

        \qed

    \end{subparag}

\end{parag}

\begin{parag}{Proposition (condition nécessaire)}
    Si la série $\sum_{n=1}^{\infty} a_n$ converge, alors
    \[\lim_{n \to \infty} a_n = 0\]

    \begin{subparag}{Preuve}
        On sait que $S_n = \sum_{k=1}^{n} a_k$ est une suite de Cauchy. Donc, pour tout $\epsilon > 0$, il existe $n_0 \in \mathbb{N}$ tel que pour tout $m, n \geq n_0$:
        \[\left|S_m - S_n\right| \leq \epsilon\]

        En particulier,
        \[\left|a_{n+1}\right| = \left|S_{n+1} - S_n\right| \leq \epsilon\]

        On sait donc que, par la définition de la limite :
        \[\lim_{n \to \infty} a_n = 0\]

    \end{subparag}

    \begin{subparag}{Remarque}
        Attention, la contraposée n'est pas vraie ; si la limite est 0, alors la série ne converge pas forcément.

        Par exemple, la série harmonique, i.e:
        \[\sum_{n = 1}^{\infty} \frac{1}{n}\]

    \end{subparag}

\end{parag}


\begin{parag}{Exemple}
    Si on a
    \[\sum_{n=0}^{\infty} \frac{\left(-1\right)^{n}}{2} \implies \lim_{n \to \infty} \frac{\left(-1\right)^{n}}{2} \text{ n'existe pas}\]

    On sait donc que cette série ne converge pas, par la contraposée du théorème ci-dessus.

\end{parag}

\subsection{Critères de convergence}

\begin{parag}{Proposition (Critère de Leibnitz pour les séries alternées)}
    Soit $\left(a_n\right)$ une suite telle que:
    \begin{enumerate}
        \item Il existe $p \in \mathbb{N}$ tel que pour tout $n \geq p$ on a:
            \[\left|a_{n+1}\right| \leq \left|a_n\right|\]

            En d'autres mots, la suite est décroissante en valeur absolue.
        \item Il existe $p \in \mathbb{N}$ tel que pour tout $n \geq p$ on a:
            \[a_{n+1}\cdot a_n \leq 0\]

            En d'autres mots, la série est alternée.
        \item On a :
            \[\lim_{n \to \infty} a_n = 0\]

    \end{enumerate}

    Alors, la série $\sum_{n=1}^{\infty} a_n$ est convergente.


    \begin{subparag}{Preuve}
        Pour tout $n \geq p$ et pour tout $k \in \mathbb{N}$, avec \important{$k > 1$ pair}, on a :
        \[S_{n+k} - S_{n-1} = a_n + a_{n+1} + \ldots + a_{n+k}\]

        Puisque $k$ est pair, on a un nombre impair de termes. On peut les regrouper par deux de deux manières différentes:
        \[a_n + \left(a_{n+1} + a_{n+2}\right) + \ldots + \left(a_{n + k -1} + a_{n + k}\right) = \left(a_{n} + a_{n+1}\right) + \ldots + a_{n+k}\]


        \important{Si $a_n > 0$:} Puisque les termes deviennent de plus en plus petit en valeur absolue, et qu'ils sont alternés:
        \[\overbrace{\underbrace{a_n}_{> 0} + \underbrace{\left(a_{n+1} + a_{n+2}\right)}_{\leq 0} + \ldots + \underbrace{\left(a_{n + k -1} + a_{n + k}\right)}_{\leq 0}}^{\leq a_n} = \underbrace{\left(a_{n} + a_{n+1}\right)}_{\geq 0} + \ldots + \underbrace{a_{n+k}}_{> 0} \geq 0\]

        On a donc que
        \[0 \leq a_n + a_{n+1} + \ldots + a_{n+k} \leq a_n\]

        \important{Si $a_n < 0$} On trouve avec le même raisonnement que:
        \[a_n \leq a_n + \ldots + a_{n+k} \leq 0\]

        Donc, \important{de manière générale}, on a que:
        \[\underbrace{\left|a_n + \ldots + a_{n+k}\right|}_{\left|S_{n+k} - S_{n-1}\right|} \leq \left|a_n\right|\]

        Cependant, la $\lim_{n \to \infty} \left|a_n\right| = 0$. Donc, par la définition de la limites, on sait que pour tout $\epsilon > 0$, il existe $n_0 \in \mathbb{N}$ tel que pour tout $n \geq n_0$, alors $\left|a_k\right| \leq \epsilon$. On a donc que:
        \[\left|S_{n + k} - S_{n-1}\right| = \left|a_n\right| \leq \epsilon\]


        \important{Si $k$ est impair}, alors, par le même argument, on a que :
        \[\left|S_{n+k} - S_{n-1}\right| \leq \epsilon\]

        Dans les deux cas, $\left(S_n\right)$ est une suite de Cauchy. Donc,
        \[\sum_{n=0}^{\infty} a_n\]
        est convergente.


    \end{subparag}

\end{parag}

\begin{parag}{Série harmonique alternée}
    Prenons la série
    \[\sum_{n = 1}^{\infty} \left(-1\right)^{n} \frac{1}{n}\]

    Elles est convergente par le critère de Leibninz.

    \begin{subparag}{Remarque}
        On verra plus tard que
        \[\sum_{n = 1}^{\infty} \left(-1\right)^{n} \frac{1}{n} = \ln\left(2\right)\]
    \end{subparag}
\end{parag}

\begin{parag}{Proposition (critère de comparaison pour les séries à termes non-négatifs)}
    Soit $\left(a_n\right)$ et $\left(b_n\right)$ deux suites telles que $\exists k \in \mathbb{N}$ tel que $0 \leq a_n \leq b_n$ pour tout $n \geq k$.

    Si $\sum_{n=0}^{\infty} b_n$ converge, alors $\sum_{n=0}^{\infty} a_n$ converge.

    Si $\sum_{n=0}^{\infty} a_n$ diverge, alors $\sum_{n=0}^{\infty} b_n$ diverge.

    \begin{subparag}{Preuve}
        Soit les suites de sommes partielles suivantes:
        \[S_n = \sum_{i=0}^{n} a_i, \mathspace P_n = \sum_{i=0}^{n} b_i\]

        Puisque pour tout $i \geq k$ on a $0 \leq a_i \leq b_i$, on peut appliquer une somme sur cette inégalité :
        \[\sum_{i=n+1}^{m} a_i \leq \sum_{i=n+1}^{m} b_i \implies 0 \leq S_m - S_n \leq P_m - P_n, \mathspace \forall m > n \geq k\]

        Si $\left(P_n\right)$ est une suite de Cauchy, alors $\left(S_n\right)$ l'est aussi. Donc, si $\sum_{n=0}^{\infty} b_n$ converge, alors $\sum_{n=0}^{\infty} a_n$ converge.


        Si $\left(S_n\right)$ est divergente, alors $\lim_{n \to \infty} S_n = \infty$. Donc, puisque $a_n \leq b_n$ pour tout $n \geq k$, alors $\lim_{n \to \infty} P_n = \infty$ et donc $\sum_{n=0}^{\infty} b_n$ est divergente.

        \qed
    \end{subparag}
\end{parag}


\begin{parag}{Exemple}
    Prenons la série suivante :
    \[\sum_{n=0}^{\infty} \frac{\cos\left(n!\right)}{\left(n+1\right)^2}\]

    Considérons la série des valeurs absolues:
    \[\sum_{n=0}^{\infty} \left|\frac{\cos\left(n!\right)}{\left(n+1\right)^2}\right|\]

    Or,
    \[\left|\frac{\cos\left(n!\right)}{\left(n+1\right)^2}\right| \leq \frac{1}{\left(n+1\right)^2}\]

    Mais, cette série est convergente:
    \[\sum_{k=0}^{\infty} \frac{1}{\left(k+1\right)^2} = \sum_{k=1}^{\infty} \frac{1}{k^2}\]

    Donc, $\left|\frac{\cos\left(n!\right)}{\left(n+1\right)^2}\right|$ converge par le critère de comparaison. Ainsi, $\sum_{n=0}^{\infty} \frac{\cos\left(n!\right)}{\left(n+1\right)^2}$ est absolument convergente et donc convergente.

\end{parag}

\begin{parag}{Remarque}
    Si $\sum_{n=0}^{\infty} a_n$ ne possède que des termes positifs, et la suite des sommes partielles est majorée, alors la série est convergente par définition.

    De la même manière, si $\sum_{n=0}^{\infty} a_n$ ne possède que des termes négatifs, et la suite des sommes partielles est minorée, alors la série est convergente par définition.
\end{parag}

\begin{parag}{Proposition (Critère de d'Alembert)}
    Soit $\left(a_n\right)$ une suite telle que $a_n \neq 0$ pour tout $n \in \mathbb{N}$ et telle que
    \[\lim_{n \to \infty} \left|\frac{a_{n+1}}{a_n}\right| = \rho \in \mathbb{R}\]

    Alors, si $\rho < 1$, la série $\sum_{n=0}^{\infty} a_n$ est absolument convergente.

    Si $\rho > 1$, alors la série $\sum_{n=0}^{\infty} \left|a_n\right|$ diverge.

    \begin{subparag}{Idée de preuve}
        Faire une comparaison avec une série géométrique. La preuve est faite en exercice dans la série 7.
    \end{subparag}

\end{parag}

\begin{parag}{Proposition (Critère de Cauchy (de la racine))}
    Soit $\left(a_n\right)$ une suite telle que la limite existe, et:
    \[\lim_{n \to \infty} \left|a_n\right|^{\frac{1}{n}} = \rho \in \mathbb{R}\]

    Alors, si $\rho < 1$, la série $\sum_{n=0}^{\infty} a_n$ est absolument convergente.

    Si $\rho > 1$, alors la série $\sum_{n=0}^{\infty} a_n$ diverge.

    \begin{subparag}{Idée de preuve}
        Faire une comparaison avec une série géométrique. La preuve est faite en exercice dans la série 7.
    \end{subparag}

\end{parag}

\begin{parag}{Remarques}
    \begin{enumerate}[left=0pt]
        \item Si $\lim_{n \to \infty} \left|\frac{a_{n+1}}{a_n}\right| = r$ et $\lim_{n \to \infty} \left|a_n\right|^{\frac{1}{n}} = \ell$. Alors, nécessairement $ r = \ell$.
        \item Parfois, $\lim_{n \to \infty} \sqrt[n]{\left|a_n\right|}$ existe, mais $\lim_{n \to \infty} \left|\frac{a_{n+1}}{a_n}\right|$ n'existe pas.

            Donc, le critère de Cauchy est plus fort que le critère de d'Alembert.
        \item Si $\lim_{n \to \infty} \left|\frac{a_{n+1}}{a_n}\right| = 1$ ou $\lim_{n \to \infty} \sqrt[n]{\left|a_n\right|} = 1$, alors on ne peut pas faire de conclusion sur la convergence de la série.
    \end{enumerate}

    \begin{subparag}{Exemple du point (3)}
        La série suivante diverge:
        \[\sum_{n = 1}^{\infty} \frac{1}{k} \implies \lim_{n \to \infty} \left|\frac{\frac{1}{n+1}}{\frac{1}{n}}\right| = \lim_{n \to \infty} \frac{n}{n+1} = 1\]

        Alors que la série suivante converge:
        \[\sum_{n = 1}^{\infty} \frac{1}{n^2} \implies \lim_{n \to \infty} \left|\frac{\frac{1}{\left(n+1\right)^2}}{\frac{1}{n^2}}\right| = \lim_{n \to \infty} \frac{n^2}{\left(n+1\right)^2} = 1\]
    \end{subparag}
\end{parag}

\begin{parag}{Exemple 1}
    Prenons la série
    \[\sum_{n=1}^{\infty} \frac{5^{n}}{n!}\]

    Le critère de d'Alembert a l'air le plus simple à utiliser:
    \[\lim_{n \to \infty} \left|\frac{a_{n+1}}{a_n}\right| = \lim_{n \to \infty} \frac{5^{n+1}}{\left(n+1\right)!} \frac{n!}{5^{n}} = \lim_{n \to \infty} \frac{5}{n+1} = 0 < 1\]

    Donc, la série est convergente.

    Puisque cette série converge, alors, par la condition nécessaire :
    \[\lim_{n \to \infty} \frac{5^{n}}{n!} = 0\]
\end{parag}

\begin{parag}{Exemple 2}
    Prenons la série avec paramètre suivante:
    \[\sum_{k=0}^{\infty} \frac{x^{k}}{k!}, \mathspace x \in \mathbb{R}\]

    Par le critère de d'Alembert:
    \[\lim_{k \to \infty} \frac{\left|x\right|^{k+1}}{\left(k+1\right)!} \frac{k!}{\left|x\right|^{k}} = \lim_{k \to \infty} \frac{\left|x\right|}{k+1} = 0 < 1\]

    Donc $\sum_{k=0}^{\infty} \frac{x^{k}}{k!}$ converge absolument pour tout $x \in \mathbb{R}$.

    \begin{subparag}{Remarque}
        On verra plus tard que
        \[\sum_{k=0}^{\infty} \frac{x^{k}}{k!} = e^x, \mathspace x \in \mathbb{R}\]
    \end{subparag}
\end{parag}

\begin{parag}{Exemple 3}
    Prenons la série
    \[\sum_{n = 1}^{\infty} \left(\frac{nx}{3n - 1}\right)^{2n - 1}, \mathspace x \in \mathbb{R}\]

    Par le critère de Cauchy:
    \[\lim_{n \to \infty} \left|\frac{nx}{3n - 1}\right|^{\frac{2n - 1}{n}} = \lim_{n \to \infty} \left|\frac{x}{3 - \frac{1}{n}}\right|^{2 - \frac{1}{n}} = \left|\frac{x}{3}\right|^2 = \rho\]

    Si $\left|x\right| < 3$, alors la série converge absolument. Si $\left|x\right|> 3$ la série diverge.

    Considérons maintenant le cas où $x = 3$, on peut regarder la condition nécessaire:
    \[\lim_{n \to \infty} \left(\frac{3n}{3n - 1}\right)^{2n - 1} = \lim_{n \to \infty} \frac{1}{\left(\frac{3n - 1}{3n}\right)^{2n - 1}} = \lim_{n \to \infty} \frac{\left(1 - \frac{1}{3n}\right)}{\left(1 - \frac{1}{3n}\right)^{2n}}\]

    On peut manipuler cette égalité pour trouver la limite vers $\frac{1}{e}$:
    \[\lim_{n \to \infty} \frac{\left(1 - \frac{1}{3n}\right)}{\left(1 - \frac{1}{3n}\right)^{3n \cdot \frac{2}{3}}} = \frac{1}{e^{-\frac{2}{3}}} \neq 0\]
    Donc ça diverge.


    Maintenant, prenons $x = -3$. On a donc:
    \[\sum_{n = 1}^{\infty} \left(\frac{-3n}{3n -1}\right)^{2n - 1} = \sum_{n = 1}^{\infty} \underbrace{\left(-1\right)^{2n - 1}}_{= -1} \left(\frac{3n}{3n-1}\right)^{2n - 1} = -\sum_{n = 1}^{\infty} \underbrace{\left(\frac{3n}{3n - 1}\right)^{2n - 1}}_{\to e^{\frac{2}{3}} \neq 0}\]



    Pour conclusion la série $\sum_{n = 1}^{\infty} \left(\frac{nx}{3n - 1}\right)^{2n - 1}$ converge absolument si $\left|x\right| < 3$ et diverge si $\left|x\right| \geq 3$.

\end{parag}




\end{document}
