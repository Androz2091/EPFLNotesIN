\documentclass[a4paper]{article}

% Expanded on 2021-10-13 at 10:13:44.

\usepackage{../../style}

\title{Analyse I}
\author{Joachim Favre}
\date{Mercredi 13 octobre 2021}

\begin{document}
\maketitle

\lecture{7}{2021-10-13}{Suite des limites de suites}{
\begin{itemize}[left=0pt]
    \item Démonstration de l'unicité des limites.
    \item Démonstration que toute suite convergente est bornée.
    \item Explication du comportement des limites lorsque nous additionnons, ou multiplions des suites convergentes.
    \item Étude de la convergence des limites lorsqu'elles sont combinées par des opérations.
    \item Démonstration de la limite d'un quotient entre deux polynômes.
    \item Démonstration que s'il existe une relation d'ordre entre deux suites convergentes, alors elle existe aussi pour leur limites.
    \item Démonstration du théorème des deux gendarmes pour les suites.
\end{itemize}

}

\parag{Exemple}{
    Soit $p \in \mathbb{Q}, p > 0$. Considérons la suite 
    \begin{systemofequations}{}
    &\ a_0 = 1 \\
    &\ a_n = \frac{1}{n^p}, \mathspace n \geq 1
    \end{systemofequations}

    On veut démontrer que 
    \[\lim_{n \to \infty} a_n = \lim_{n \to \infty} \frac{1}{n^p} = 0\]
    
    \subparag{Démonstration}{
    Soit $\epsilon > 0$. On cherche $n_0 \in \mathbb{N}$ tel que $\forall n \geq n_0$, on a:
    \[\left|\frac{1}{n^p} - 0\right| \leq \epsilon \iff \frac{1}{n^p} \leq \epsilon \iff \frac{1}{\epsilon} \leq n^p \over{\iff}{$p > 0$} \frac{1}{\epsilon^{\frac{1}{p}}} \leq n\]
        
    Il est possible de démontrer cette dernière implication, que si $x \leq y$ alors $x^p \leq y^p$ et $x^{\frac{1}{p}} \leq y^{\frac{1}{p}}$, en utilisant les axiomes des nombres réels. Voir le paragraphe suivant.

    Si on choisit 
    \[n_0 = \left\lfloor \frac{1}{\epsilon^{\frac{1}{p}}}\right\rfloor + 1 > \frac{1}{\epsilon^{\frac{1}{p}}}\]

    Alors, $\forall n \geq n_0$: 
    \[n > \frac{1}{\epsilon^{\frac{1}{p}}} \iff \frac{1}{n} < \epsilon^{\frac{1}{p}} \over{\iff}{$p>0$} \frac{1}{n^p} < \epsilon\]
    

    On a bien démontré que 
    \[\lim_{n \to \infty} \frac{1}{n^p} = 0 \mathspace \forall p > 0\]
    en utilisant la définition de la limite (on l'a démontré pour $p \in \mathbb{Q}$, mais cela marche aussi pour $p \in \mathbb{R}$, mais on définira les puissances réelles plus tard). 
    }
}

\parag{Exercice}{
    On veut démontrer que 
    \[\forall x, y \in \mathbb{R} \telque 0 < x \leq y \implies x^m \leq y^m \text{ et } x^{\frac{1}{m}} \leq y^{\frac{1}{m}} \mathspace \forall m \in \mathbb{N}^*\]
    
    Astuce: démontrer la première implication en faisant une preuve par récurrence, puis utiliser ce résultat pour la puissance de réciproque.

    Il y a une correction dans les slides de la professeure, mais je ne considère pas que ça vaut la peine de la mettre ici.
}

\parag{Proposition (Unicité de la limite)}{
    Soit $\left(a_n\right)_{n\in\mathbb{N}}$ une suite de nombres réels et supposons que $a \in \mathbb{R}$ et $b \in \mathbb{R}$ sont des limites de $\left(a_n\right)$. Alors, $a = b$. 

    \subparag{Démonstration}{
        Soit $\epsilon > 0$. Puisque $\lim_{n \to \infty} a_n = a$, on sait que:
        \[\exists n_0 \in \mathbb{N} \telque \forall n \geq n_0, \ \left|a_n - a\right|\leq \frac{\epsilon}{2}\]
        
        Le $\frac{\epsilon}{2}$ vient du fait que c'est vrai pour tout $\epsilon$ (on peut donc prendre celui qu'on veut). De la même manière, puisque $\lim_{n \to \infty} a_n = b$: 
        \[\exists m_0 \in \mathbb{N} \telque \forall n \geq m_0, \ \left|a_n - b\right| \leq \frac{\epsilon}{2}\]

        Donc, $\forall n \geq \max\left(n_0, m_0\right)$, on sait que: 
        \[\left|a - b\right| = \left|a - a_n + a_n - b\right| \over{\leq}{$\dagger$} \left|a - a_n\right| + \left|a_n - b\right| \leq \frac{\epsilon}{2} + \frac{\epsilon}{2} = \epsilon \]
        où $\dagger$ est l'inégalité triangulaire. Donc, puisque le résultat qu'on a trouvé est vrai pour tout $\epsilon > 0$, on a: 
        \[\forall \epsilon > 0 \ \left|a - b\right| \leq \epsilon \implies a - b = 0 \implies a = b\]

        Puisque le seul nombre positif qui est plus petit que tous les nombres positifs est 0.

        \qed
    }
}

\parag{Inégalité triangulaire}{
    Pour tout $x, y \in \mathbb{R}$, on a que: 
    \[\left|x + y\right| \leq \left|x\right| + \left|y\right|\]
    
    \subparag{Démonstration}{
        $\forall x \in \mathbb{R}$, on a $x \leq \left|x\right|$ et $-x \leq \left|x\right|$, et de la même manière pour $\forall y \in \mathbb{R}$. 

        Alors, si $x + y \geq 0$, on a: 
        \[\left|x + y\right| = x + y \leq \left|x\right| + \left|y\right|\]
        
        Sinon, si $x + y$ < 0, on a: 
        \[\left|x + y\right| = -x -y \leq \left|x\right| + \left|y\right|\]
        
        \qed
    }
    
}

\parag{Exemple divergent}{
    On veut démontrer que $a_n = \left(-1\right)^n$ est divergente.

    \subparag{Démonstration}{
        Supposons par l'absurde qu'il existe la limite $\ell \in \mathbb{R}$ de $\left(-1\right)^n$. 

        Donc, si on prend $\epsilon = \frac{1}{4}$, on sait que: 
        \[\exists n_0 \in \mathbb{N} \telque \forall n \geq n_0 \implies \left|a_n - \ell\right| \leq \frac{1}{4}\]

        On remarque que $a_{2k} = 1$ et $a_{2k + 1} = -1$ pour tout $k \in \mathbb{N}$. Donc, pour tout $k$  tels que $2k \geq n_0$ et $2k + 1 \geq n_0$, on a: 
        \[\left|a_{2k} - \ell\right| = \left|1 - \ell\right| \leq \frac{1}{4} \mathspace \text{et} \mathspace \left|a_{2k + 1} - \ell\right| = \left|-1 - \ell\right| \leq \frac{1}{4}\]

        De là, en utilisant l'inégalité triangulaire, on trouve: 
        \[2 = \left|\ell - 1 + \left(-1 - \ell\right)\right| \leq \left|\ell - 1\right| + \left|-1 - \ell\right| \leq \frac{1}{4} + \frac{1}{4} = \frac{1}{2}\]
        
        Ce qui est une contradiction, on ne peut pas dire que $2 \leq \frac{1}{2}$. On en déduit donc que la série n'a pas de limite et donc qu'elle est divergente.

        \qed
    }
}

\parag{Proposition}{
    Toute suite convergente est bornée.

    \subparag{Démonstration}{
        Soit $\left(a_n\right)$ une série convergente. On sait donc que 
        \[\lim_{n \to \infty} a_n = \ell \in \mathbb{R}\]
        
        Prenons $\epsilon = 1$. Donc, il existe $n_0 \in \mathbb{N}$ tel que pour tout $n \geq n_0$ on a: 
        \[\left|a_n - \ell\right| \leq 1 \iff \ell - 1 \leq a_n \leq \ell + 1\]

        Soit $S = \left\{a_0, a_1, \ldots, a_{n_0 - 1}\right\}$. C'est un ensemble fini, donc il a un minimum et un maximum. On peut donc dire que la suite $\left(a_n\right)$ est bornée par 
        \[\min\left(\min S, \ell - 1\right) \mathspace \text{ et } \mathspace \max\left(\max S, \ell + 1\right)\]
        
        \qed
    }
    

    \subparag{Réciproque}{
        La réciproque de la proposition est fausse. En effet, on a démontré un contre-exemple ci-dessus. $a_n = \left(-1\right)^n$ est bornée par $-1$ et $1$, cependant elle est divergente.  
    }
}

\subsection{Opération algébriques sur les limites}
\parag{Proposition}{
    Soient $\left(a_n\right)$ et $\left(b_n\right)$ deux suites convergentes. En d'autres mots:
    \[\lim_{n \to \infty} a_n = a \mathspace \text{ et } \lim_{n \to \infty} b_n = b\]
    
    Alors:
    \begin{enumerate}
        \item $\lim\limits_{n \to \infty} \left(a_n \pm b_n\right) = a \pm b$
        \item $\lim\limits_{n \to \infty} pa_n = pa$ pour tout $p \in \mathbb{R}$.
        \item $\lim\limits_{n \to \infty} \left(a_n\cdot b_n\right) = a\cdot b$
        \item $\lim\limits_{n \to \infty} \left(\frac{a_n}{b_n}\right) = \frac{a}{b}$, si $b \neq 0$.
    \end{enumerate}
}

\parag{Convergence des opérations}{
    Soit $\left(a_n\right)$ et $\left(b_n\right)$ des suites. On peut étudier la convergence de leurs opérations:
    \begin{enumerate}
        \item\label{enum:convegenceSommeSuites} Si $\left(a_n + b_n\right)$ converge, alors soit $\left(a_n\right)$ et $\left(b_n\right)$ convergent, soit elles divergent toutes les deux.
        \item Si $\left(a_n b_n\right)$ converge, on ne sait rien de la convergence de $\left(a_n\right)$ et $\left(b_n\right)$.
        \item Si $\lim_{n \to \infty} a_n = 0$, alors la suite $\left(\frac{1}{a_n}\right)$ est divergente, si elle existe (s'il n'y a pas de division par zéro).
    \end{enumerate}

    De plus:
    \begin{enumerate}
        \setcounter{enumi}{3}
        \item Si $\left(a_n + b_n\right)$ converge et que $\left(b_n\right)$ converge, alors $\left(a_n\right)$ converge.
        \item Si $\left(a_n b_n\right)$ converge et que $\left(b_n\right)$ converge vers quelque chose d'autre que 0, alors $\left(a_n\right)$ est convergente.
    \end{enumerate}
    
    \subparag{Remarque}{
        On peut tirer un cas particulier de la propriété (\ref{enum:convegenceSommeSuites}). Si on a 
        \[\lim_{n \to \infty} \left(a_n - b_n\right) = 0\]
        alors soit $\left(a_n\right)$ et $\left(b_n\right)$ divergent, soit leur limite sont égales.
    }
    
    \subparag{Preuve de la propriété (3)}{
        Soit $\left(a_n\right)$, une suite qui converge vers 0. On sait donc que $\forall \epsilon > 0$, $\exists n_0 \in \mathbb{N}$ tel que $\forall n \geq n_0$ on a: 
        \[\left|a_n - 0\right| \leq \epsilon \iff \left|a_n\right| \leq \epsilon \iff \left|\frac{1}{a_n}\right| \geq \frac{1}{\epsilon}\]

        Soit $M > 0$, un nombre quelconque. Prenons $\epsilon = \frac{1}{M}$. On a donc que $\forall n \geq n_0$: 
        \[\left|\frac{1}{a_n}\right| \geq \frac{1}{\epsilon} = M\]

        Ainsi, on voit que $\left(\frac{1}{a_n}\right)$ n'est pas bornée et donc divergente.

        \qed
    }
    

    \subparag{Preuve de la propriété (5)}{
        Soit $\left(a_n\right)$ et $\left(b_n\right)$ deux suites telles que $\left(a_n b_n\right)$ converge, et 
        \[\lim_{n \to \infty} b_n = b \neq 0\]
        
        On a donc:
        \[a_n = \frac{a_n b_n}{b_n} \implies \lim_{n \to \infty} a_n = \lim_{n \to \infty} \frac{a_n b_n}{b_n} = \frac{\lim\limits_{n \to \infty} a_n b_n}{\lim\limits_{n \to \infty} b_n}\]
        
        On peut séparer la limite sur le numérateur et le dénominateur, puisque les deux limites existent. On en déduit donc que $\left(a_n\right)$ est convergente.

        \qed
    }
}

\parag{Proposition (quotient de deux suites polynomiales)}{
    Pour $p, q \in \mathbb{N}^*$, si on a:
    \[x_n = a_p n^p + \ldots + a_1 n + a_0 \mathspace a_i \in \mathbb{R},\ a_p \neq 0\]
    et: 
    \[y_n = b_q n^q + \ldots + b_1 n + b_0 \mathspace b_i \in \mathbb{R},\ b_q \neq 0\]
    
    Alors,
    \begin{functionbypart}{\lim_{n \to \infty} \left(\frac{x_n}{y_n}\right)}
    0, \mathspace \text{si } p < q \\
    \frac{a_p}{b_q}, \mathspace \text{si } p = q \\
    \text{diverge}, \mathspace \text{si } p < q
    \end{functionbypart}
    
    \subparag{Démonstration}{
        On a: 
        \[\frac{x_n}{y_n} = \frac{a_p n^p + \ldots + a_0}{b_q n^q + \ldots + b_0} = \frac{n^p \overbrace{\left(a_p + a_{p-1} \frac{1}{n} + \ldots + \frac{a_0}{n^p}\right)}^{u_{n}}}{n^q \underbrace{\left(b_q + \ldots + \frac{b_0}{n^q}\right)}_{v_n}} = \frac{n^p u_n}{n^q v_n}\]
        
        Regardons la limite du quotient entre $u_n$ et $v_n$: 
        \[\lim_{n \to \infty} \frac{u_n}{v_n} = \lim_{n \to \infty} \frac{a_p + \overbrace{a_{p-1} \frac{1}{n} }^{\to 0}+ \ldots + \overbrace{a_0 \frac{1}{n_p}}^{\to 0}}{b_q + \ldots + \underbrace{b_0 \frac{1}{n^q}}_{\to 0}} = \frac{a_p}{b_q}\]
        qui n'est pas égal à 0, puisque $a_p \neq 0$. 

        \important{Si $p < q$:} 
    \[\lim_{n \to \infty} \frac{n^p}{n^q} = \lim_{n \to \infty} \frac{1}{n^{q - p}} = 0\]
        puisque $q - p > 0$. Donc: 
        \[\lim_{n \to \infty} \frac{x_n}{y_n} = \lim_{n \to \infty} \frac{n^p}{n^q} \frac{u_n}{v_n} = 0\cdot \frac{a_p}{b_q} = 0\]
        
        \important{Si $p = q$:}
        \[\lim_{n \to \infty} \frac{n^p}{n^p} = \lim_{n \to \infty} 1 = 1 \]
        
        Donc:
        \[\lim_{n \to \infty} \frac{x_n}{y_n} = \lim_{n \to \infty} \frac{n^p}{n^p} \frac{u_n}{v_n} = 1 \cdot \frac{a_p}{b_q} = \frac{a_p}{b_q}\]
        
        \important{Si $p > q$:} On sait, puisque 
        \[\lim_{n \to \infty} \frac{1}{n^{p-q}} = 0\]
        alors $\left(n^{p - q}\right)$ est divergente. Or, puisque 
        \[\frac{x_n}{y_n} = n^{p - q} \frac{u_n}{v_n}\]
        est le produit entre une suite divergente et une suite qui convergent vers autre chose que 0, on sait qu'elle est divergente.
    }
    
}

\parag{Exemple}{
    On peut utiliser notre théorème pour trouver que
    \[\lim_{n \to \infty} \frac{85n^2 + 5n + 2}{2n^3 + n + 7} = 0\]
}

\subsection{Relation d'ordre}
\parag{Proposition}{
    Soient $\left(a_n\right)$ et $\left(b_n\right)$ deux suites convergentes. En d'autres mots: 
    \[\lim_{n \to \infty} a_n = a \mathspace \text{ et } \mathspace \lim_{n \to \infty} b_n = b\]
    
    Supposons que $\exists m_0 \in \mathbb{N}$ tel que $\forall n \geq m_0$ on a que 
    \[a_n \geq b_n\]
    
    Alors, on sait que 
    \[a \geq b\]
    
    \subparag{Démonstration}{
        Supposons par l'absurde que $b > a$.

        Soit $\epsilon = \frac{b-a}{4}$. Par la définition de la limite, on sait qu'il existe $n_0 \in \mathbb{N}$ tel que 
        \[a - \epsilon \leq a_n \leq a + \epsilon \mathspace \text{ et } \mathspace b - \epsilon \leq b_n \leq b + \epsilon\]

        De là, on sait que pour tout $n \geq n_0$ on a: 
        \[a_n \leq a + \epsilon = a + \frac{b-a}{4} < a + \frac{b-a}{2} = b - \frac{b-a}{2} < b - \frac{b-a}{4} = b - \epsilon \leq b_n\]

        En d'autres mots, $\forall n \geq n_0$: 
        \[a_n < b_n\]
        
        Ce qui est une contradiction.

        \qed
    }
}

\parag{Théorème des deux gendarmes pour les suites}{
    Soient $\left(a_n\right)$, $\left(b_n\right)$ et $\left(c_n\right)$ telles que 
    \begin{enumerate}
        \item $\lim\limits_{n \to \infty} a_n = \lim\limits_{n \to \infty} c_n = \ell$
        \item $\exists k \in \mathbb{N}$ tel que $\forall n \geq k$ on a $a_n \leq b_n \leq c_n$
    \end{enumerate}

    Alors 
    \[\lim_{n \to \infty} b_n = \ell\]
    
    \subparag{Démonstration}{
        Soit $\epsilon > 0$. On sait qu'il existe $n_0 \in \mathbb{N}$ tel que $\forall n \geq n_0$, on a: 
        \[-\epsilon \leq a_n - \ell \leq \epsilon \mathspace \text{et} \mathspace -\epsilon \leq c_n - \ell \leq \epsilon\]

        De plus, pour tout $n \geq k$, on sait que 
        \[a_n - \ell \leq b_n - \ell \leq c_n - \ell\]

        Ainsi, $\forall n \geq \max\left(n_0, k\right)$, alors les deux inégalités sont satisfaites. Donc: 
        \[-\epsilon \leq a_n - \ell \leq b_n - \ell \leq c_n - \ell \leq c_n - \ell \leq \epsilon\]

        En d'autres mots: 
        \[-\epsilon \leq b_n - \ell \leq \epsilon \implies \left|b_n - \ell\right| \leq \epsilon\]
        
        Ainsi, par la définition de la limite, on a bien trouvé que 
        \[\lim_{n \to \infty} b_n = \ell\]
        
        \qed
    }
}

\end{document}
