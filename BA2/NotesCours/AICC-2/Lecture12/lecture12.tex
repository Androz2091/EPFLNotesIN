% !TeX program = lualatex

\documentclass[a4paper]{article}

% Expanded on 2022-03-31 at 17:56:02.

\usepackage{../../style}

\title{AICC-2}
\author{Joachim Favre}
\date{Jeudi 31 mars 2022}

\begin{document}
\maketitle

\lecture{12}{2022-03-31}{Prime time}{
\begin{itemize}[left=0pt]
    \item Definition of prime numbers, and explanation of the uniqueness of the prime factorisation theorem.
    \item Definition of GCD and coprime numbers.
    \item Proof that there are infinitely many primes.
\end{itemize}

}


\begin{parag}{Equations modulo 12}
    Let's consider the $\Mod 12$ world.

    We wonder if $\congruent{2+x}{2 + y}{12}$ implies $\congruent{x}{y}{12}$. We know $\congruent{-2}{-2}{12}$, thus: 
    \[\congruent{2+x}{2 + y}{12} \implies \congruent{x}{y}{12}\]
    
    More generally, we can add, subtract, and multiply by integers. Let's now consider divisions: we wonder if $\congruent{2x}{2y}{12}$ implies $\congruent{x}{y}{12}$. In fact this does not hold, we can see $x = 1$ and $y = 7$ is a counterexample. Indeed, $\congruent{2}{14}{12}$, but $\notcongruent{1}{7}{12}$. Note that we could not use the same argument as for additions, saying $\congruent{\frac{1}{2}}{\frac{1}{2}}{12}$, since $\frac{1}{2}$ is not an integer, and thus does not exist in our world.

    More strangely, $\congruent{5x}{5y}{12}$ implies $\congruent{x}{y}{12}$. Indeed, $\congruent{5}{5}{12}$ and $\congruent{25}{1}{12}$, thus; 
    \[\congruent{5x}{5y}{12} \implies \congruent{25x}{25y}{12} \implies \congruent{x}{y}{12}\]

    Thus, we needed a multiplicative inverse. We can draw its table modulo 12:
    \begin{center}
    \begin{tabular}{c|ccccccccccc}
        $a$ & 1 & 2 & 3 & 4 & 5 & 6 & 7 & 8 & 9 & 10 & 11 \\
        \hline
        Inverse & 1 & --- & --- & --- & 5 & --- & 7 & --- & --- & --- & 11
    \end{tabular}
    \end{center}
    
    We observe that the world mod 12 is not a nice place to be.
\end{parag}

\begin{parag}{Definition: Prime number}
    A \important{prime number} (or a \important{prime}) is an integer $> 1$ that has no positive divisors other than 1 and itself. 

    Numbers that are not prime are called \important{composite.}

    \begin{subparag}{Remark}
        1 is not prime since it allows us to get the following nice theorem.
    \end{subparag}
\end{parag}

\begin{parag}{Theorem: Prime factorisation}
    Every integer greater than 1 has a unique prime factorisation (except for order).

    \begin{subparag}{Example}
        We can draw this as a tree. We can see that $100 = 2\cdot 2\cdot 5\cdot 5$:
        \imagehere[0.4]{PrimeFactorisationTree.png}

        There is no other way of writing 100 using prime factors.
    \end{subparag}

    \begin{subparag}{Remark}
        Note that, in general, it is very hard to compute a prime factorisation. A 1000 bits number cannot be factored with today technology. 

        There is no proof to show that this problem is hard, we only see it (and hope that it holds). 
    \end{subparag}
\end{parag}


\begin{parag}{Definition: Divide}
    We write $a \divides b$ if $a$ divides $b$, i.e. if $\exists c \in \mathbb{Z}$ such that: 
    \[b = ac\]
    
    Equivalently in terms of prime factors: 
    \[a = p_1^{\gamma_1} \cdots p_n^{\gamma_n}\]
    \[b = p_1^{\widetilde{\gamma}_1} \cdots p_n^{\widetilde{\gamma}_n} \cdot p_{n+1}^{\widetilde{\gamma}_{n+1}} \cdots p_m^{\widetilde{\gamma}_m}\]
    where $\widetilde{\gamma}_i \geq \gamma_i$, for $i = 1, \ldots, n$.
\end{parag}

\begin{parag}{Example}
    Let the two following numbers: 
    \[168 = 2^3 \cdot 3 \cdot 7, \mathspace 12 = 2^2 \cdot 3\]
    
    Thus, $12 \divides 168$. Let's now consider the two following numbers: 
    \[30 = 2\cdot 3\cdot 5, \mathspace 12 = 2^2 \cdot 3\]
    
    We can see that $12 \ndivides 30$, since $12$ has 2 to the second power whereas 30 only has it to the first power.
\end{parag}

\begin{parag}{Definition: GCD}
    Let $a$ and $b$ be integers, not both zero. The largest integer that divides both is called the \important{greatest common divisor} (GCD) of $a$ and $b$. It is denoted by $\gcd\left(a, b\right)$.
\end{parag}

\begin{parag}{Theorem}
    Let $a$ and $b$ be positive integers, not both zero, and let $p_1, \ldots, p_k$ be the sequence of prime numbers that divide $a$ or $b$. We already know we can write: 
    \[a = p_1^{\alpha_1} \cdots p_k^{\alpha_k}\]
    \[b = p_1^{\beta_1} \cdots p_k^{\beta_k}\]
    where $\alpha_i \geq 0$ and $\beta_i \geq 0$.

    Then, letting $\gamma_i = \min\left(\alpha_i, \beta_i\right)$, we have:
    \[\gcd\left(a, b\right) = p_1^{\gamma_1} \cdots p_k^{\gamma_k}\]
\end{parag}

\begin{parag}{Example}
    Let's take the two following numbers: 
    \[12 = 2^2 \cdot 3 = 2^2 \cdot 3^1 \cdot 5^0\]
    \[30 = 2\cdot 3\cdot 5 = 2^1 \cdot 3^1 \cdot 5^1\]

    Thus: 
    \[\gcd\left(12, 30\right) = 2^{1} \cdot 3^1 \cdot 5^0 = 6\]
\end{parag}

\begin{parag}{Definition: Coprime numbers}
    Two numbers $a$ and $b$ are called \important{coprime} (or relatively prime, or mutually prime) when: 
    \[\gcd\left(a, b\right) = 1\]
    
    This is equivalent to saying that they have no common prime factor.

    \begin{subparag}{Example}
        We notice that $\gcd\left(9, 100\right) = 1$, thus they are coprime.
    \end{subparag}
\end{parag}

\begin{parag}{Theorem}
    Let $p$ be a prime number, and let $a$ be an integer such that $0 < a < p$. Then: 
    \[\gcd\left(p, a\right) = 1\]
    
    \begin{subparag}{Proof}
        $p$ only contains $p$ in its prime factorisation, and the one of $a$ cannot contain $p$, since $a < p$.

        \qed
    \end{subparag}
\end{parag}

\begin{parag}{Properties}
    We wonder if $ab \divides c$ implies $a \divides c$ and $b \divides c$. We can use prime factorisation: 
    \[a = p_1^{\alpha_1} \cdots p_n^{\alpha_n}, \mathspace b = p_1^{\beta_1} \cdots p_n^{\beta_n}, \mathspace c = p_1^{\gamma_1} \cdots p_n^{\gamma_n}\]
    
    Since we know $ab \divides c$, we know that $\gamma_i \geq \alpha_i + \beta_i$ for all $i$. This definitely implies that $\gamma_i \geq \alpha_i$ and $\gamma_i \geq \beta_i$, thus $a \divides c$ and $b \divides c$.

    \vspace{1em}

    We now wonder if $a \divides c$ and $b \divides c$ imply that $ab \divides c$. Using the same prime factorisations, we see that $\gamma_i \geq \alpha_i$ and $\gamma_i \geq \beta_i$ do not imply in general that $\gamma_i \geq \alpha_i + \beta_i$. 

    Modifying a bit our property, we want to know, if $a \divides c$, $b \divides c$ and $\gcd\left(a, b\right) = 1$ imply $ab \divides c$.\label{ref:divisibilityPropertiesForChineseRemaindersTheorem} The hypothesis imply that $\gamma_i \geq \alpha_i$, $\gamma_i \geq \beta_i$, and $\alpha_i > 0 \implies \beta_i = 0$ and $\beta_i > 0 \implies \alpha_i = 0$. This clearly implies that $\gamma_i \geq \alpha_i + \beta_i$.
\end{parag}

\begin{parag}{Theorem: Infinite number of primes}
    There are an infinite number of primes.

    \begin{subparag}{Proof}
        This proof was made by Euclid in 300 BC. It is usually done by contradiction, but since we are in AICC, let's instead make a procedure that allows us to get an infinite number of primes.
    
        Let's say we have $n$ primes, we will construct one more prime. Let's consider the number $q = p_1 \cdots p_n + 1$. There are two possibilities: either $q$ is a prime, either it is not. In the first case, we have found our new prime number. If $q$ is not prime, then there exists a prime number $p$ and a number $c \geq 1$ such that $q = pc$. If $p$ is not in our list, then we have found a new prime. If $p$ is already in our list, then it means that $p \divides q$ and $p \divides p_1 \cdots p_n = q - 1$, which implies that $p = 1$ (the only numbers that can divide two consecutive numbers is 1). This is impossible since it was a prime number. 

        In any case, we have shown that having $n$ primes allows us to construct an $\left(n+1\right)$th prime. Since we know 2 is a prime, this procedure allows us to construct infinitely many primes (note that the procedure does not necessarily generate all possible primes, it juste generates an infinite number of them).
        
        \qed
    \end{subparag}
    
\end{parag}



\end{document}
