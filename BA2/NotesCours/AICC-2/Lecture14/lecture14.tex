% !TeX program = lualatex

\documentclass[a4paper]{article}

% Expanded on 2022-04-07 at 13:44:22.

\usepackage{../../style}

\title{AICC-2}
\author{Joachim Favre}
\date{Jeudi 07 avril 2022}

\begin{document}
\maketitle

\lecture{14}{2022-04-07}{Groups}{
\begin{itemize}[left=0pt]
    \item Explanation of Euclid's algorithm.
    \item Proof of Bézout's identity.
    \item Proof that $\left[a\right]_m$ has an inverse if and only if $a$ and $m$ are coprime.
    \item Explanation of check digits, the procedure mod 97--10, and example of how IBANs are made.
    \item Definition of commutative groups.
    \item Definition of $\mathbb{Z}/m\mathbb{Z}^*$, and proof that it is a commutative group.
    \item Definition of Euler's totient function, and proof of some of its properties.
    \item Definition of the Cartesian product for groups.
\end{itemize}
}

\begin{parag}{Lemma}
    The following identity holds: 
    \[\gcd\left(a, b\right) = \gcd\left(a - kb, b\right)\]
    
    \begin{subparag}{Proof}
        Knowing that $d \divides a$ and $d \divides b$ is equivalent to saying that $d \divides a-kb$ and $d \divides b$.

        Indeed, if $d$ divides $a$ and $b$, then it divides $b$ and $a - kb$. Also, if $d$ divides $b$ and $a - kb$, then it divides $a - kb + kb = a$. We can write those as fractions, it may help to visualise: 
        \[\frac{a}{d} \in \mathbb{Z} \text{ and } \frac{b}{d} \in \mathbb{Z} \implies \frac{a - kb}{d} = \frac{a}{d} - k \frac{b}{d} \in \mathbb{Z}\] 
        \[\frac{b}{d} \in \mathbb{Z} \text{ and } \frac{a - kb}{d} \in \mathbb{Z} \implies \frac{a}{d} = \frac{a - kb}{d} + k \frac{b}{d} \in \mathbb{Z}\] 
        
        Since the set of divisors of $a$ and $b$ is the same as the set of divisors of $b$ and $a - kb$, then clearly the greatest divisor is the same in both cases.

        \qed
    \end{subparag}
\end{parag}

\begin{parag}{Euclid algorithm}
    We know we can write $a = bq + r$, where $0 \leq r < b$. Thus, from our lemma: 
    \[\gcd\left(a, b\right) = \gcd\left(b, r\right) = \gcd\left(b, a \Mod b\right)\]
    
    We can continue that way recursively.

    \begin{subparag}{Example}
        Let us compute the gcd between 122 and 22: 
        \[\gcd\left(122, 22\right) = \gcd\left(22, 12\right) = \gcd\left(12, 10\right) = \gcd\left(10, 2\right) = \gcd\left(2, 0\right) = 2\]
        
        Since: 
        \[122 = 22\cdot 5 + 12, \mathspace 22 = 12\cdot 1 + 10, \mathspace 12 = 10\cdot 1 + 2, \mathspace 10 = 2\cdot 5 + 0\]
        
    \end{subparag}
\end{parag}

\begin{parag}{Theorem: Bézout's identity}
    Let $a$ and $b$ be integers, not both zero. 

    Then, there exist integers $u$ and $v$ (possibly negative), such that: 
    \[\gcd\left(a, b\right) = au + bv\]
    
    \begin{subparag}{Proof}
        We find the $u$ and the $v$ through an algorithm. 

        We know that, since $a = bq + r$, then: 
        \[\gcd\left(a, b\right) = \gcd\left(b, r\right)\]
        
        Let's suppose there exist numbers $\widetilde{u}$ and $\widetilde{v}$ such that: 
        \[\gcd\left(b, r\right) = b \widetilde{u} + r\widetilde{v} = b\widetilde{u} + \left(a - bq\right)\widetilde{v} = a\widetilde{v} + b\left(\widetilde{u} - q\widetilde{v}\right) = au + bv\]
        
        Let us also see the following initial step: 
        \[\gcd\left(a, 0\right) = a = 1\cdot a + 0\cdot 0 \implies u^* = 1, v^* = 0\]
        
        Note that, in the initial step, $v^*$ is not unique (but taking 0 is usually simpler).

        This algorithm gives us a way to construct our $u$ and $v$, proving that they exist.

        \qed
    \end{subparag}

    \begin{subparag}{Example}
        We can draw the following table:
        \begin{center}
        \begin{tabular}{|c||c|c|c|c|c|c|c}
            \hline
            $\gcd\left(a, b\right)$ & $a = bq + r$ & $q$ & $u = \widetilde{v}$ & $v = \widetilde{u} - q\widetilde{v}$ \\
            \hhline{=#=|=|=|=|}
            $\gcd\left(150, 33\right)$ & $33 \cdot 4 + 18$ & 4 & $2$ & $-1 -8 = -9$\\
            \hline
            $\gcd\left(33, 18\right)$ & $18\cdot 1 + 15$ & 1 & $-1$ & $1 - \left(-1\right)1 = 2$\\
            \hline
            $\gcd\left(18, 15\right)$ & $15\cdot 1 + 3$ & $1$ & $1$ & $0 - 1 = -1$\\
            \hline
            $\gcd\left(15, 3\right)$ & $3\cdot 5 + 0$ & 5 & 0 & 1 \\
            \hline
            $\gcd\left(3, 0\right) = 3$ & & & 1 & 0 \\
            \hline
        \end{tabular}
        \end{center}

        We complete the three first columns first, and then complete the last two columns by going from bottom to top. Note that a good idea is to verify that our current $u$ and $v$ are correct, in every row. For instance, for the third row from bottom:
        \[3 = \gcd\left(18, 15\right) = 18\left(1\right) + 15\left(-1\right) = 3\]
        which is correct.

        In the end, we get that, indeed: 
        \[2\cdot 150 - 9\cdot 33 = 3 = \gcd\left(150, 33\right)\]
    \end{subparag}
\end{parag}

\begin{parag}{Theorem}
    $\left[a\right]_m$ has a multiplicative inverse \textit{if and only if} $a$ and $m$ are coprime (meaning that $\gcd\left(a, m\right) = 1$).

    \begin{subparag}{Remark}
        The proof $\impliedby$ is important since it gives us a procedure for finding an inverse.
    \end{subparag}
    

    \begin{subparag}{Proof $\impliedby$}
        We suppose by hypothesis that $\gcd\left(a, m\right) = 1$.

        By Bézout identity, we know $\exists u,v$ such that: 
        \[1 = au + mv \implies \left[1\right]_m = \left[au\right]_m + \underbrace{\left[mv\right]_m}_{= 0} = \left[a\right]_m \left[u\right]_m\]
        since $mv$ is a multiple of $m$.

        We have found our inverse, $\left[u\right]_m$.
    \end{subparag}

    \begin{subparag}{Proof $\implies$}
        Let us suppose that $\left[a\right]_m$ has an inverse, $\left[u\right]_m$. Then, for some integer $v$ (the first part is true for any $v$, but then, after the implication, we only have one possible $v$): 
        \[\left[au + mv\right]_m = \left[a\right]_m \left[u\right]_m \over{=}{hyp} \left[1\right]_m \implies au + mv = 1\]
        
        A number dividing both $a$ and $m$ definitely divides the left hand side, thus it needs to divide the right hand side. Since only $-1$ and $1$ divide the right hand side, this tells us that no number apart from those divide both $a$ and $b$, and thus: 
        \[\gcd\left(a, m\right) = 1\]
        
        \qed
    \end{subparag}

    \begin{subparag}{Example}
        Let us draw the multiplication table for $\mathbb{Z} / 4 \mathbb{Z}$:
        \begin{center}
        \begin{tabular}{c|cccc}
            $\cdot$ & 0 & 1 & 2 & 3 \\
            \hline
            0 & 0 & 0 & 0 & 0 \\
            1 & 0 & 1 & 2 & 3 \\
            2 & 0 & 2 & 0 & 2 \\
            3 & 0 & 3 & 2 & 1
        \end{tabular}
        \end{center}

        We see that our theorem is correct: the even rows (and columns) do not have a multiplicative inverse, whereas the odd ones do.
    \end{subparag}
\end{parag}

\begin{parag}{Corollary 1}
    $\gcd\left(a, m\right) = 1$ if and only if there exist integers $u$ and $v$ such that $1 = au + mv$.

    \begin{subparag}{Proof $\implies$}
        We suppose that $\gcd\left(a, m\right) = 1$.

        By Bézout's identity, we know that there exists integers $u$ and $v$ such that: 
        \[1 = au + mv\]
    \end{subparag}

    \begin{subparag}{Proof $\impliedby$}
        We know by hypothesis that $1 = au + mv$, where $u$ and $v$ are integers.

        Then, we can compute: 
        \[\left[1\right]_m = \left[au + mv\right]_m = \left[a\right]_m \left[u\right]_m + \underbrace{\left[m\right]_m}_{= 0} \left[v\right]_m = \left[a\right]_m \left[u\right]_m\]
        
        Which shows that $\left[u\right]_m$ is the inverse of $\left[a\right]_m$. Thus, since $\left[a\right]_m$ has an inverse, we know that $a$ and $m$ are coprime, i.e: 
        \[\gcd\left(a, m\right) = 1\]
        
        \qed
    \end{subparag}
\end{parag}

\begin{parag}{Corollary 2}
    If $p$ is a prime, all elements of $\mathbb{Z} / p\mathbb{Z}$, except for $\left[0\right]_p$, have a multiplicative inverse.

    \begin{subparag}{Proof}
        We know that, for any $0 < a < p$, then $\gcd\left(a, p\right) = 1$, so we can apply the theorem we have just seen. If $a = 0$, then:
        \[\gcd\left(a, p\right) = \gcd\left(0, p\right) = p \neq 1\]

        \qed
    \end{subparag}
\end{parag}

\begin{parag}{Example: Check digits}
    Let's consider the number $0212351234$. We can compute its remainder $\Mod 97$, which gives us 95. Instead of publishing this number, we can publish the same number with the modulo appended at the end: 
    \[021235123495\]
    
    Now, if we swapped two numbers, getting $0212{\color{red}53}123495$, we would know it is wrong since $0212531234 \Mod 97 = 63 \neq 95$. This algorithm allows us to know there is a problem.
\end{parag}

\begin{parag}{Procedure mod 97--10}
    This algorithm is a variant of the previous one.

    We start with a number $n$. We multiply our number by 100, getting $100n$, compute $r = 100n \Mod 98$, then add a check $c = 98 - r$ to our number multiplied by 100. In other words, we compute:
    \[m = 100n + \left(98 - 100n \Mod 97\right)\]

    Then, to check if our number was modified, we can verify that the result modulo 97 is 1. In other words, the test passes if:
    \[\left[\text{number with check digits}\right]_{97} = \left[1\right]_{97}\]

    \begin{subparag}{Proof: Test passes if no mistake}
        We want know to verify that the following is equal to 1: 
        \begin{multiequality}
        \left(100n + \left(98 - 100n \Mod 97\right)\right) \Mod 97 =\ & \left(100n + 98 - 100n\right) \Mod 97  \\
        =\ & 98 \Mod 97  \\
        =\ & 1 
        \end{multiequality}
        
        We still need to prove that if we swap any two consecutive numbers, we will not get 1 at the end.
    \end{subparag}

    \begin{subparag}{Proof: Swapping numbers}
        Let's now suppose that we swap two consecutive digits in our number $n$, giving us $\widetilde{n}$. For example: 
        \[n = 212ba1234 \implies \widetilde{n} = 212ab1234\]
        
        We want to express $\widetilde{n}$ as a function of $n$ ($k = 4$ in our example): 
        \[\widetilde{n} = n - 10^k \left(a + 10b\right) + 10^k \left(b + 10a\right) = n + 10^k \left(9a - 9b\right)\]
        
        The test fails to detect only if $10^k \left(9a - 9b\right) \Mod 97 = 0$, meaning:
        \[\left(\left[10\right]_{97}\right)^{k} \left[9\right]_{97} \left[a - b\right]_{97} = \left[0\right]_{97}\]
        
        However, since we know that all non-zero element of $\mathbb{Z} / 97 \mathbb{Z}$ has an inverse (since 97 is prime), our equation is equivalent to: 
        \[\left[a - b\right]_{97} = \left[0\right]_{97}\]
        
        To conclude, the transposition is not detected if and only if $a = b$ (since $0 \leq a, b \leq 9$), which would mean that there is no transposition.

        \qed
    \end{subparag}
    
    \begin{subparag}{Remark}
        We know that $97$ is the largest prime under 100, this is why we picked it.      
    \end{subparag}
\end{parag}

\begin{parag}{IBAN}
    An IBAN is constructed the following way. The first 5 numbers represent the bank, the next 12 the account, the next 4 the country (we convert the two letters of the country, CH for Switzerland for instance, using $A \mapsto 10, \ldots, Z \mapsto 35$, leading to 1217 for CH), and then do a MOD97--10 procedure for the last two digits. We finally replace the four digits of the country by its letters, and reposition the two letters and the two digits of the MOD97--10 procedure to the front, leading to, for instance: 
    \[\text{CH}{\color{red}54}\,00243\,0001\,2345\,6789\]

    This means that, we can verify the user did not make a mistake writing his or her IBAN thanks to the MOD97--10 procedure. With our example, we can verify that: 
    \[00243\,0001\,2345\,6789\,{\color{red}1217\,54} \Mod 97 = 1\]
\end{parag}

\subsection{Commutative groups}
\begin{parag}{Introduction}
    After $\mathbb{Z} / m\mathbb{Z}$, we could go into two directions. Focus on finite groups, which have one operation, or focus on finite fields, which have two operations. 

    We will begin with groups for now, and move onto fields later.
\end{parag}

\begin{parag}{Definition: Commutative group}
    Let $G$ be a finite set, and $\star$ be an operation on pairs of elements of $G$, such that:
    \begin{enumerate}
        \item Closure: $\forall a, b \in G$, $a \star b \in G$
        \item Associativity: $a \star \left(b \star c\right) = \left(a \star b\right) \star c$
        \item Commutativity: $a\star b = b \star a$
        \item Existence of identity: $\exists e \in G$ such that $e \star a = a\star e = a$
        \item Existence of inverse: $\forall a \in G, \exists b \in G$ such that $a \star b = b\star a = e$
    \end{enumerate}

    Then, a \important{commutative group} (or Abelian group) is $\left(G, \star\right)$. Note that we can also define the concept of group, which does not have the commutativity property, but we are not interested in them in this course.

    \begin{subparag}{Remark}
        We are generalising an idea, in order to be able to apply it to many stuff. If we can prove something is a group, then we will have a lot of properties for free.
    \end{subparag}
\end{parag}

\begin{parag}{Example}
    The following are groups: 
    \[\left(\mathbb{R}, +\right), \mathspace \left(\mathbb{R} \setminus \left\{0\right\}, \cdot\right), \mathspace \left(\mathbb{C}, +\right), \mathspace \left(\mathbb{Z} / m\mathbb{Z}, +\right), \mathspace \left(\mathbb{Z} / p\mathbb{Z} \setminus \left\{\left[0\right]_p\right\}, \cdot\right)\]
    where $p$ is a prime number.
    
    However, the following are not groups: 
    \[\left(\mathbb{R}, \cdot\right), \mathspace \left(\mathbb{Z} / m\mathbb{Z}, \cdot\right), \mathspace \left(\mathbb{Z} / q \mathbb{Z} \setminus \left\{\left[0\right]_q\right\}, \cdot\right),\mathspace \left(\mathbb{N}, +\right), \mathspace \left(\mathbb{Z} \setminus \left\{0\right\}, \cdot\right)\]
    where $q$ is not a prime number. For instance, $\left(\mathbb{R}, \cdot\right)$ is not a group since $0$ does not have an inverse.
\end{parag}

\begin{parag}{Definition}
    We define $\mathbb{Z} / m\mathbb{Z}^*$ to be the set of elements of $\mathbb{Z} / m\mathbb{Z}$ that have a multiplicative inverse.

    \begin{subparag}{Example}
        For instance: 
        \[\mathbb{Z} / 4\mathbb{Z} = \left\{0, 1, 2, 3, 4\right\} \implies \mathbb{Z} / 4\mathbb{Z}^* = \left\{1, 3\right\}\]
    \end{subparag}
    
\end{parag}

\begin{parag}{Theorem}
    For every integer $m > 1$, $\left(\mathbb{Z} / m\mathbb{Z}^*, \cdot\right)$ is a commutative group.

    \begin{subparag}{Proof}
        It seems clear that all the axioms for the operator hold, except for closure which does not seem so clear from first sight. Let us prove that it holds.

        Let $a, b \in \mathbb{Z} / m\mathbb{Z}^*$, meaning that they have a multiplicative inverse. Let us show that the multiplicative inverse of $\left[ab\right]_m$ is $\left(\left[b\right]_m\right)^{-1} \left(\left[a\right]_m\right)^{-1}$:
        \begin{multiequality}
        \left(\left[b\right]_m\right)^{-1} \left(\left[a\right]_m\right)^{-1} \left[ab\right]_m =\ & \left(\left[b\right]_m\right)^{-1} \underbrace{\left(\left[a\right]_m\right)^{-1} \left[a\right]_m}_{=1} \left[b\right]_m  \\
        =\ & \left(\left[b\right]_m\right)^{-1} \left[b\right]_m  \\
        =\ & 1 
        \end{multiequality}
        
        Since $\left[ab\right]_m$ has a multiplicative inverse, we know $\left[ab\right]_m \in \mathbb{Z} / m\mathbb{Z}^*$.

        \qed
    \end{subparag}
\end{parag}

\begin{parag}{Example}
    Let us draw the multiplication table of $\left(\mathbb{Z} / 5\mathbb{Z}^*, \cdot\right)$:
    \begin{center}
    \begin{tabular}{c|cccc}
        $\cdot$ & 1 & 2 & 3 & 4 \\
        \hline
        1 & 1 & 2 & 3 & 4 \\
        2 & 2 & 4 & 1 & 3 \\
        3 & 3 & 1 & 4 & 2 \\
        4 & 4 & 3 & 2 & 1 \\
    \end{tabular}
    \end{center}

    We can realise that every number appears exactly once in every row and in every column.

    We have seen that in $\mathbb{Z} / m\mathbb{Z}$, the map $x \mapsto ax$ is a bijection when $a^{-1}$ exists. Each row (and each column) of the above table is such a map, which explains our observation.
\end{parag}

\begin{parag}{Number of elements}
    We wonder how many elements there are in $\mathbb{Z} / m\mathbb{Z}^*$. 

    We have the following candidates: 
    \[\mathbb{Z} / m\mathbb{Z} = \left\{0, 1, \ldots, m -1\right\}\]
    
    But then, $a$ is in $\mathbb{Z} / m\mathbb{Z}^*$ if and only if $\gcd\left(a, m\right) = 1$. This makes us come to our following definition.
\end{parag}

\begin{parag}{Definition: Euler's totient function}
    We define \important{Euler's totient function}, $\phi\left(n\right)$, to be the number of $i \in\left\{1, 2, \ldots, n\right\}$ such that $i$ and $n$ are coprime (meaning $\gcd\left(i, n\right) = 1$).

    \begin{subparag}{Remark}
        Note that, when we work modulo $n$: 
        \[\left\{1, 2, \ldots, n\right\} = \left\{0, 1, \ldots, n-1\right\}\]

        Thus, our function is exactly what we have seen in our last paragraph.
    \end{subparag}

    \begin{subparag}{Properties}
        As we have seen, $\phi\left(m\right)$ is the cardinality of $\mathbb{Z} / m\mathbb{Z}^*$. Also, we can see that, if $p$ is prime, then $\phi\left(p\right) = p-1$.
    \end{subparag}
\end{parag}

\begin{parag}{Examples}
    Let us compute the following values of Euler's totient function: 
    \[\phi\left(1\right) = 1, \mathspace \phi\left(2\right) = 1, \mathspace \phi\left(3\right) = 2, \mathspace \phi\left(4\right) = 2, \mathspace \phi\left(5\right) = 4, \mathspace \phi\left(6\right) = 2\]
    
    Indeed, for example, for $n = 4$, we have $\mathbb{Z} / 4\mathbb{Z}^* = \left\{1, 3\right\}$.

    We can observe that this is not a monotonic function.
\end{parag}

\begin{parag}{Theorem}
    Let $p$ be a prime number, and $k \in \mathbb{Z}_+$. Then: 
    \[\phi\left(p^k\right) = p^k - p^{k-1}\]
    
    \begin{subparag}{Proof}
        Our candidates are: 
        \[\left\{1, 2, \ldots, p, p+1, \ldots, 2p, \ldots, 3p, \ldots, \underbrace{p^{k-1} p}_{= p^k}\right\}\]
        
        We have $p^k$ candidates. Also, we see that $p$, $2p$, and so on, are not comprime with $p^k$ (since they are multiples of $p$), and we have $p^{k-1}$ of them. All others are coprime with $p^k$, since $p$ is a prime. Thus, we have a total number of: 
        \[p^k - p^{k-1}\]
        
        \qed
    \end{subparag}
\end{parag}

\begin{parag}{Theorem}
    Let $p$ and $q$ be two distinct primes. Then: 
    \[\phi\left(pq\right) = \left(p-1\right)\left(q -1\right)\]
    
    \begin{subparag}{Proof}
        Let's say without loss of generality that $p < q$. Then, our candidates are: 
        \[\left\{1, 2, 3, \ldots, p, \ldots, q, \ldots, 2p, \ldots, 2q, \ldots, pq\right\}\]
        
        We have $pq$ candidates, from which we need to remove our $q$ multiples of $p$ and our $p$ multiples of $q$. Moreover, we must be sure not to double-count $pq$, so we get: 
        \[pq - q - p + 1 = \left(p-1\right)\left(q - 1\right)\]
        
        \qed
    \end{subparag}
\end{parag}

\begin{parag}{Definition: Regular Cartesian product}
    Let $\mathcal{A}_1$ and $\mathcal{A}_2$ be two sets. The \important{Cartesian product} $\mathcal{A}_1 \times \mathcal{A}_2$ is the set: 
    \[\mathcal{A} = \mathcal{A}_1 \times \mathcal{A}_2 = \left\{\left(a_1, a_2\right) \text{ such that } a_1 \in \mathcal{A}_1, a_2 \in \mathcal{A}_2\right\}\]

    \begin{subparag}{Example}
        For example:
        \[\left\{0, 1\right\} \times \left\{0, 1, 2\right\} = \left\{\left(0, 0\right), \left(0, 1\right), \left(0, 2\right), \left(1, 0\right), \left(1, 1\right), \left(1, 2\right)\right\}\]
    \end{subparag}
\end{parag}

\begin{parag}{Definition: Cartesian product for groups}
    Similarly, the \important{Cartesian product of two groups}, shorten the \important{product group}, $\left(G, \star\right) = \left(G_1, \star_1\right)\times \left(G_2, \star_2\right)$ is the set $G_1 \times G_2$ endowed with the binary operation $\star$, defined by: 
    \[\left(a_1, a_2\right) \star \left(b_1, b_2\right) = \left(a_1 \star_1 b_1, a_2 \star_2 b_2\right)\]

    \begin{subparag}{Example}
        For instance, let us draw the operation table for $\left(\mathbb{Z} / 2\mathbb{Z}, +\right) \times \left(\mathbb{Z} / 3\mathbb{Z}, \cdot\right)$, denoting $\left(a, b\right)= ab$:

        \begin{center}
        \begin{tabular}{c|cccccc}
            $\star = \left(+, \cdot\right)$ & 00 & 01 & 02 & 10 & 11 & 12 \\
            \hline
            00 & 00 & 00 & 00 & 10 & 10 & 10 \\
            01 & 00 & 01 & 02 & 10 & 11 & 12 \\
            02 & 00 & 02 & 01 & 10 & 12 & 11 \\
            10 & 10 & 10 & 10 & 00 & 00 & 00 \\
            11 & 10 & 11 & 12 & 00 & 01 & 02 \\
            12 & 10 & 12 & 11 & 00 & 02 & 01
        \end{tabular}
        \end{center}
    \end{subparag}
\end{parag}

\end{document}
