% !TeX program = lualatex

\documentclass[a4paper]{article}

% Expanded on 2022-04-15 at 16:02:48.

\usepackage{../../style}

\title{AICC-2}
\author{Joachim Favre}
\date{Jeudi 14 avril 2022}

\begin{document}
\maketitle

\lecture{16}{2022-04-14}{Rivest-Shamir-Aldeman's cryptosystem}{
\begin{itemize}[left=0pt]
    \item Explanation and proof of Lagrange's theorem, and of three of its corollary (including Euler's theorem and Fermat's little theorem).
    \item Explanation and proof of the Chinese Remainders Theorem.
    \item Explanation of how the RSA cryptosystem works.
\end{itemize}
}

\begin{parag}{Observation}
    Let us consider the group $\left(\mathbb{Z} / 10\mathbb{Z}^*, \cdot\right)$. We have $e = \left[1\right]_{10}$. Also, let us take $a = \left[3\right]_{10}$:
    \begin{center}
    \begin{tabular}{c|ccccccccc}
        $k$ & 0 & 1 & 2 & 3 & 4 & 5 & 6 & 7 & 8 \\
        \hline
        $a^k$ & 1 & 3 & 9 & 7 &  1 & 3 & 9 & 7 & 1
    \end{tabular}
    \end{center}

    We see a pattern emerging, telling us $3^k = 1$ if and only if $k$ is a multiple of 4, which is exactly the order of $a = 3$.
\end{parag}

\begin{parag}{Theorem}
    Let us consider a group $\left(G, \star\right)$, with identity element $e$.

    $a^k = e$ if and only if $k$ is a multiple of the order of $a$.


    \begin{subparag}{Proof}
        Let $p$ be the order of $a$, and $k \in \mathbb{N}$ be any integer. We know we can write $k = mp + r$ where $0 \leq r < p$. Then we have:
        \[a^k = a^{mp + r} = a^{mp} \star a^{r} = \left(a^p\right)^m \star a^r = e^m \star a^r = e \star a^r = a^r\]

        We now see that $a^r = e$ if and only if $r = 0$, since $r < p$ and $p$ is the smallest non-zero number such that $a^p = e$. In other words, $a^k = a^r = e$ if and only if $k = mp$ (meaning that $k$ is a multiple of $p$).

        \qed
    \end{subparag}
\end{parag}

\begin{parag}{Recall: Equivalence relation}
    We need a small recall on equivalence relation for the proof of Lagrange's theorem, coming hereinafter.

    \begin{subparag}{Binary relation}
        A binary relation $R$, is a subset $R \subset A \times B$. $a$ being in relation with $b$, written $a \sim b$, is represented by having $\left(a, b\right) \in R$.

        If we have $A = B$, then we speak of a relation on a set $A$.
    \end{subparag}

    \begin{subparag}{Equivalence relation}
        A relation on a set $A$ is called an equivalence relation if it follows the following properties:
        \begin{enumerate}
            \item Reflexive: $a \sim a$, $\forall a \in A$
            \item Symmetric: $a \sim b \implies b \sim a$, $\forall a, b \in A$
            \item Transitive: $a \sim b \land b \sim c \implies a \sim c$, $\forall a, b, c \in A$
        \end{enumerate}
    \end{subparag}

    \begin{subparag}{Equivalence classes}
        In an equivalence relation, we define equivalence classes:
        \[\left[a\right] = \left\{b \in A : b \sim a\right\}\]

        We can use any element of an equivalence class to represent it.

        It is important to notice that an equivalence relation on $A$ partitions $A$ into equivalence classes: all of them are disjoint subsets of $A$ and their union is $A$.
    \end{subparag}


    \begin{subparag}{Example}
        For instance, let us take $A$ being the set of students in AICC2. The following set is an equivalence relation:
        \[R = \left\{\left(a, b\right) \in A^2 : \text{$a$ and $b$ graduated from the same high school}\right\}\]

        We can partition $A$ into sets of student that graduated from the same high school, and each student is in exactly one such subset.

        However, the following is not an equivalence relation since it is not symmetric:
        \[R = \left\{\left(a, b\right) \in A^2 : \text{$a$ trusts $b$}\right\}\]

        Similarly, the following set is also not one, since it is not transitive:
        \[R = \left\{\left(a, b\right) \in A^2 : \text{$a$ and $b$ got the same score in AICC1 or AICC2}\right\}\]
    \end{subparag}
\end{parag}

\begin{parag}{Lagrange's theorem}
    Let $\left(G, \star\right)$ be a finite commutative group of cardinality $n$.

    The order of each of its element divides $n$.

    \begin{subparag}{Proof}
        Let $G$ be a finite commutative group of cardinality $n$, and let $p$ be the order of $a \in G$. Then, let us define the following subgroup of $G$, which has cardinality $p$:
        \[H = \left\{a, a^2, \ldots, \underbrace{a^p}_{= e}\right\}\]

        We can now define a relation on $G$, as:
        \[x \sim y \iff \exists h \in H \text{ such that } x \star h = y\]

        We notice that it is an equivalence relation:
        \begin{enumerate}
            \item Reflexive: We know $x \sim x$ since we can pick $h = a^p = e \in H$.
            \item Symmetric: Any $h = a^i$ has an inverse $h^{-1} = a^{p - i} \in H$, thus we do have that:
            \[x \star h = y \implies x = y \star h^{-1}\]
            \item Transitive: Let's suppose that we know that $x \star h_1 = y$ and $y \star h_2 = z$. Then, we see that:
                \[z = y \star h_2 = \left(x \star h_1\right) \star h_2 = x \star \left(h_1 \star h_2\right)\]

                Clearly, $h_1 \star h_2 \in H$, so we are good.
        \end{enumerate}

        Since it is an equivalence relation, it partitions $G$ into equivalence classes. Proving that equivalence classes all have cardinality $p$ would finish the proof, since then, necessarily $n$ would be a multiple of $p$ and thus $p$ would need to divide $n$.

        Let us consider the equivalence class of an arbitrary element $b \in G$:
        \[\left[b\right] = \left\{ba, ba^2, \ldots, ba^p\right\}\]

        We want to compute its cardinality. We can see that $f: H \mapsto \left[b\right]$ which sends $a^i \mapsto ba^i$ is one-to-one, since:
        \[f\left(a^i\right) = f\left(a^j\right) \implies ba^i = ba^j \implies b^{-1} b a^i = b^{-1} b a^j \implies a^i = a^j\]

        We are using the fact that $b$ has an inverse since we are working in a group.

        Since this function is one-to-one, it means that the cardinality of $\left[b\right]$ is at least $p$. But since it is also clearly at most $p$ by construction, we have found that the cardinality of $\left[b\right]$ is exactly $p$.

        Since $b$ was an arbitrary element of $G$, we have found that $G$ can be partitioned in some number of partitions of cardinality $p$, and this definitely means that $p$ divides the cardinality of $G$.

        \qed
    \end{subparag}

    \begin{subparag}{Example}
        If we have a group with 10 elements, then its elements' orders are in the set $\left\{1, 2, 5, 10\right\}$.
    \end{subparag}
\end{parag}

\begin{parag}{Corollary}
    Let $\left(G, \star\right)$ be a finite commutative group of order $n$ and identity $e$. Then, for all $a \in G$:
    \[a^n = e\]

    \begin{subparag}{Proof}
        We know that the order of $a$ divides $n$, meaning that $n = kp$ and thus that $n$ is a multiple of $p$. However, by a theorem we have seen earlier, this means that:
        \[a^n = e\]

        \qed
    \end{subparag}
\end{parag}


\begin{parag}{Corollary: Euler's theorem}
    Let $m \leq 2$ be an integer. Then, for all $a \in \left(\mathbb{Z} / m\mathbb{Z}^*, \cdot\right)$, we have:
    \[a^{\phi\left(m\right)} = \left[1\right]_m\]

    Equivalently, for all integers $a$ relatively prime to $m$, we have:
    \[\congruent{a^{\phi\left(m\right)}}{1}{m}\]
\end{parag}

\begin{parag}{Corollary: Fermat's little theorem}
    Let $p$ be a prime number. Then, for all $a \in \mathbb{Z} / p\mathbb{Z}$, we have:
    \[a^p = a\]

    Equivalently, for all integers $a$:
    \[\congruent{a^p}{a}{p}\]

    \begin{subparag}{Proof}
        Let us first consider the case where $a \neq \left[0\right]_p$. This case comes directly from the fact that, for a prime number $p$, we have $\phi\left(p\right) = p - 1$. Thus:
        \[\congruent{a^{\phi\left(p\right)}}{1}{p} \implies \congruent{a^{p-1}}{1}{p} \implies \congruent{a^p}{a}{p}\]

        Also, when $a = \left[0\right]_p$, then the following equality definitely holds:
        \[a^p = a\]

        \qed
    \end{subparag}

    \begin{subparag}{Example}
        We have:
        \[\congruent{2^3}{2}{3}, \mathspace \congruent{3^3}{3}{3}, \mathspace \congruent{4^3}{4}{3}\]
    \end{subparag}
\end{parag}

\begin{parag}{Chinese Remainders Theorem}
    Let's draw a table by starting on the main diagonal, and when we reach the end, we reenter at the opposite end, like if the table was a torus (the bottom side is connected to the top side, and the left side is connected to the right side).

    For instance, we can fill the following table:
    \begin{center}
    \begin{tabular}{|c|c|c|}
        \hline
        0, 12 & 4, 16 & 8, 20  \\
        \hline
        9, 21 & 1, 13 & 5, 17 \\
        \hline
        6, 18 & 10, 22 & 2, 14 \\
        \hline
        3, 15 & 7, 19 & 11, 23 \\
        \hline
    \end{tabular}
    \end{center}

    We could continue filling this table with numbers to infinity. However, we can notice an interesting property. First, two numbers are in the same box if and only if they have the same rest modulo 12. Also, this gives us the rest of any number modulo 3 and 4. Indeed, let us add numbers on the side:
    \begin{center}
    \begin{tabular}{c||c|c|c|}
        & 0 & 1 & 2  \\
        \hhline{=#=|=|=|}
        0 & 0, 12 & 4, 16 & 8, 20  \\
        \hline
        1 & 9, 21 & 1, 13 & 5, 17 \\
        \hline
        2 & 6, 18 & 10, 22 & 2, 14 \\
        \hline
        3 & 3, 15 & 7, 19 & 11, 23 \\
        \hline
    \end{tabular}
    \end{center}

    For instance, we see that 22 is in the column with the one, so $22 \Mod 3 = 1$, and it is in the row with the 2, so $22 \Mod 4 = 2$.

    Moreover, this allows us do computations through coordinates. For instance, $\left[8\right]_{12}$ has coordinates $\left(\left[2\right]_3, \left[0\right]_4\right)$ and $\left[7\right]_{12}$ has coordinates $\left(\left[1\right]_3, \left[3\right]_4\right)$. Let us multiply coordinate by coordinate:
    \[\left[2\right]_3 \cdot \left[1\right]_3 = \left[2\right]_3\]
    \[\left[0\right]_4 \cdot \left[3\right]_4 = \left[0\right]_4\]

    And the element at coordinates $\left(\left[2\right]_3, \left[0\right]_4\right)$ is $\left[8\right]_{12}$, which is exactly $\left[8\cdot 7\right]_{12}$. We can do the exact same trick for addition.

    Let us now try another box with other dimensions:
    \begin{center}
    \begin{tabular}{|c|c|c|c|}
        \hline
        0, 4 & & 2, 6 & \\
        \hline
             & 1, 5  & & 3, 7 \\
        \hline

    \end{tabular}
    \end{center}

    It will never get filled, so we cannot use our techniques.

    \begin{subparag}{Mathematical formulation}
        We have $m_1 \cdot m_2$ numbers placed in a grid of $m_1 \times m_2$ drawers ($m_1$ rows and $m_2$ columns). We can see the numbers as elements of $\mathbb{Z} / m_1 m_2 \mathbb{Z}$ and index the drawers with the elements of $\left(\mathbb{Z} / m_1 \mathbb{Z}\right) \times \left(\mathbb{Z} / m_2 \mathbb{Z}\right)$.

        We can define the mapping as a function:
        \[\begin{split}
            \psi: \mathbb{Z} / m_1 m_2 \mathbb{Z} &\longmapsto \mathbb{Z} / \left(m_1 \mathbb{Z}\right) \times \left(\mathbb{Z} / m_2\mathbb{Z}\right) \\
            \left[k\right]_{m_1 \cdot m_2} &\longmapsto \left(\left[k\right]_{m_1}, \left[k\right]_{m_2}\right)
        \end{split}\]

        If this mapping is surjective, then it is also injective (we have $m_1 m_2$ elements and $m_1 m_2$ boxes, if all of them are filled, then none has two elements) and thus bijective.
    \end{subparag}
\end{parag}

\begin{parag}{Theorem: Chinese remainders}
    If $m_1$ and $m_2$ are relatively prime, then the map defined by:
    \[\begin{split}
        \psi: \mathbb{Z} / m_1 m_2 \mathbb{Z} &\longmapsto \left(\mathbb{Z} / m_1 \mathbb{Z}\right) \times \left(\mathbb{Z} / m_2\mathbb{Z}\right) \\
        \left[k\right]_{m_1 \cdot m_2} &\longmapsto \left(\left[k\right]_{m_1}, \left[k\right]_{m_2}\right)
    \end{split}\]
    is bijective and an isomorphism with respect to $+$ and $\cdot$.

    Note that, if $m_1$ and $m_2$ are not relatively prime, then $\psi$ is neither onto nor one-to-one.

    \begin{subparag}{Remark}
        Note that, for both addition and multiplication, we are working over $\mathbb{Z} / m_1 m_2 \mathbb{Z}$, not $\mathbb{Z} / m_1 m_2 \mathbb{Z}^*$. This is really important, since we developed this part of theory for exactly this property.
    \end{subparag}

    \begin{subparag}{Example}
        In our first table, we indeed had $\gcd\left(4, 3\right) = 1$, thus it is bijective. However, in our second table, we had $\gcd\left(2, 4\right) = 2 \neq 1$, so it was not bijective (which meant it was not filled).
    \end{subparag}

    \begin{subparag}{Intuition}
        If $\gcd\left(m_1, m_2\right) = 1$, then this theorem states that we can compute sums and products in $\mathbb{Z} / m_1 m_2 \mathbb{Z}$ or in $\mathbb{Z} / m_1 \mathbb{Z} \times \mathbb{Z} / m_2 \mathbb{Z}$ (whichever is more convenient).
    \end{subparag}

    \begin{subparag}{Proof: Implies bijection}
        Let's assume that $m_1$ and $m_2$ are coprime. As explained earlier, if we prove that this mapping is one-to-one, then we will have proven that the mapping is bijective. Thus, let us have two numbers $k$ and $k'$ that fall in the same box:
        \[\psi\left(\left[k\right]_{m_1 m_2}\right) = \psi\left(\left[k'\right]_{m_1 m_2}\right) \implies \left(\left[k\right]_{m_1}, \left[k\right]_{m_2}\right) = \left(\left[k'\right]_{m_1}, \left[k'\right]_{m_2}\right)\]

        We want to show that this implies $\left[k\right]_{m_1m_2} = \left[k'\right]_{m_1m_2}$. Our assumption tells us:
        \[\left[k\right]_{m_1} = \left[k'\right]_{m_1}, \mathspace \left[k\right]_{m_2} = \left[k'\right]_{m_2}\]

        This means that $m_1$ divides $k - k'$ and $m_2$ divides $k - k'$. But, since $m_1$ and $m_2$ are coprime, then it necessarily means that $m_1 m_2$ divides $k - k'$ (we have proven such a fact on page \pageref{ref:divisibilityPropertiesForChineseRemaindersTheorem}, when showing properties of divisibility), and thus:
        \[\left[k\right]_{m_1 m_2} = \left[k'\right]_{m_1 m_2}\]
    \end{subparag}

    \begin{subparag}{Proof: Isomorphism}
        For two groups $\left(G, \star\right)$ and $\left(H, \otimes\right)$ to be isomorphic, there must exist a mapping $f : G \mapsto H$ such that:
        \[f\left(a \star b\right) = f\left(a\right) \otimes f\left(b\right)\]

        In our case, we have:
        \[G = \mathbb{Z} / m_1 m_2 \mathbb{Z}, \mathspace H = \left(\mathbb{Z}/ m_1 \mathbb{Z}\right) \times \left(\mathbb{Z} / m_2 \mathbb{Z}\right)\]
        and we are working over addition.

        We can take $\psi$ as our mapping. Indeed, for any $k, \ell \in \mathbb{Z} / m_1 m_2 \mathbb{Z}$, we have:
        \[\psi\left(k + \ell\right) = \left(\left[k + \ell\right]_{m_1}, \left[k + \ell \right]_{m_2}\right)\]
        \[\psi\left(k\right) + \psi\left(\ell \right) = \left(\left[k\right]_{m_1}, \left[k\right]_{m_2}\right) + \left(\left[\ell \right]_{m_1}, \left[\ell \right]_{m_2}\right)\]

        We see that they are equal, i.e:
        \[\psi\left(k + \ell\right) = \psi\left(k\right) + \psi\left(\ell\right)\]

        Thus, our groups are isomorphic. The proof is similar for multiplication.

        \qed
    \end{subparag}
\end{parag}

\begin{parag}{Example 1}
    Let's consider the equation $x^3 = \left[7\right]_{12}$ for $x \in\mathbb{Z}/12 \mathbb{Z}$. We notice $12 = 3\cdot 4$, so we can write $x = \left(x_1, x_2\right)$, and we want:
    \[x_1^3 = \left[7\right]_3 = \left[1\right]_3, \mathspace x_2^3 = \left[7\right]_4 = \left[3\right]_4\]

    By trying all numbers, we notice that the only solutions are:
    \[x_1 = \left[1\right]_3, \mathspace x_2 = \left[3\right]_4\]

    Thus, drawing the Chinese remainder theorem table, we find that the only solution is:
    \[x = \left[7\right]_{12}\]
\end{parag}

\begin{parag}{Example 2}
    We wonder if we can affirm that:
    \[\left(\left[x\right]_{77}\right)^2 = \left[0\right]_{77} \implies \left[x\right]_{77} = \left[0\right]_{77}\]

    In other words, we want to know if there are $x \neq \left[0\right]_{77}$ such that $\left(\left[x\right]_{77}\right)^2 = \left[0\right]_{77}$.

    We notice that $77 = 7\cdot 11$, so we can pick:
    \[x \mapsto \left(\left[x_1\right]_7, \left[x_2\right]_{11}\right)\]

    We know that $\left(\mathbb{Z} / 7\mathbb{Z}^*, \cdot\right)$ is a group, so there is no number in $\mathbb{Z} / \mathbb{Z}^*$ with a power that yields 0. Thus we can affirm that:
    \[\left(\left[x_1\right]_{7}\right)^2 = \left[0\right]_7 \implies x_1 = \left[0\right]_7\]

    We can do the exact same reasoning, leading to:
    \[\left(\left[x_2\right]_{11}\right)^2 = \left[0\right]_{11} \implies x_2 = \left[0\right]_{11}\]

    Which implies by the Chinese remainder theorem that:
    \[\left(\left[x\right]_{77}\right)^2 = \left[0\right]_{77} \implies \left[x\right]_{77} = \left[0\right]_{77}\]
\end{parag}

\begin{parag}{Backward mapping}
    We want to know how to get $\left[x\right]_{pq}$ when having $\left(\left[x\right]_p, \left[x\right]_q\right)$.

    First, we use extended Euclid's algorithm to find integers $u$ and $v$ such that:
    \[1 = \gcd\left(p, q\right) = pu + qv\]

    We let $a = qv$ and $b = pu$, and we see that:
    \[\left[a\right]_q = \left[qv\right]_q = 0, \mathspace \left[a\right]_p = \left[qv\right]_p = \left[1 - pu\right]_p = \left[1\right]_p\]

    The same thing holds for $b$:
    \[\left[b\right]_p = \left[pu\right]_p = 0, \mathspace \left[b\right]_q = \left[pu\right]_q = \left[1 - qv\right]_q = \left[1\right]_q\]

    Thus, for any integers $k_1$ and $k_2$, the Chinese remainder theorem gives us the mapping:
    \[\psi\left(\left[a k_1 + b k_2\right]_{pq}\right) = \left(\left[ak_1 + bk_2\right]_p, \left[ak_1 + bk_2\right]_q\right) = \left(\left[k_1\right]_{p}, \left[k_2\right]_{q}\right)\]

    But since our function is bijective, we find that its inverse is:
    \[\psi^{-1}\left(\left[k_1\right]_p, \left[k_2\right]_q\right) = \left[a k_1 + b k_2\right]_{pq}\]
\end{parag}


\subsection{RSA cryptosystem}
\begin{parag}{Theorem: Mix of Chinese remainders and Fermat}
    Let $p$ and $q$ be distinct prime numbers, and let $k$ be a multiple of both $\left(p - 1\right)$ and $\left(q - 1\right)$.

    Then, for every integer $a$ and every non-negative integer $\ell$, we have:
    \[\left(\left[a\right]_{pq}\right)^{\ell k + 1} = \left[a\right]_{pq}\]

    \begin{subparag}{Proof}
        Let $p$ and $q$ be distinct primes. Also, let $k$ be a multiple of both $\phi\left(p\right) = p - 1$ and $\phi\left(q\right) = q - 1$, meaning that, for some $m \in \mathbb{Z}$:
        \[k = m\left(p-1\right)\left(q-1\right)\]

        We then see that, for any non-negative integer $\ell$, using Fermat's little theorem:
        \[\left(\left[a\right]_p\right)^{\ell k} = \left(\left(\left[a\right]_p\right)^{p - 1}\right)^{\ell m\left(q - 1\right)} = \left(\left[1\right]_p\right)^{\ell m\left(q - 1\right)} = \left[1\right]_p\]

        We can do a similar trick for $q$, which allows us to find that:
        \[\left(\left[a\right]_p\right)^{\ell k + 1} = \left[a\right]_p, \mathspace \left(\left[a\right]_q\right)^{\ell k + 1} = \left[a\right]_q\]

        And thus, by the Chinese remainders theorem, this combines to:
        \[\left(\left[a\right]_{pq}\right)^{\ell k + 1} = \left[a\right]_{pq}\]

        \qed
    \end{subparag}
\end{parag}

\begin{parag}{RSA encryption}
    Let's suppose that we can find three integers $m$ (modulus), $e$ (encoding exponent) and $d$ (decoding exponent) such that, for all integers $t \in \mathbb{Z} / m\mathbb{Z}$ (our plaintext), we have:
    \[\left[\left(t^e\right)^d\right]_m = \left[t\right]_m\]

    Then, the receiver can generate those $m, e, d$. The $\left(m, e\right)$ is the public encoding key, announced in a public directory, and $\left(m, d\right)$ is the private decoding key ($d$ never leaves the receiver). To send the plaintext $t \in \mathbb{Z} / m\mathbb{Z}$, the encoder forms the cryptogram by doing:
    \[c = t^e \Mod m\]

    This is not a problem since doing exponents with big numbers is easy thanks to the fast exponentiation algorithm. Then, the intended decoder performs $c^d \Mod m$ and obtains the plaintext $t$.

    \begin{subparag}{Example}
        Let us take $m = 33$, $e = 7$ and $d = 3$. Also, let's suppose that the plaintext is $t = 2$.

        Then, the encryption is:
        \[c = t^e \Mod m = 2^7 \Mod 33 = 29\]

        And, for the decryption:
        \[c^d \Mod m = 29^3 \Mod 33 = 2\]
        as expected.

        This works similarly for any $t \in \left\{1, \ldots, 32\right\}$ (we exclude $t = 0$, because $c = 0$ necessarily means that $t = 0$).
    \end{subparag}

    \begin{subparag}{Remark}
        Combining the fact that we want $\left[a\right]_m^{ed} = \left[a\right]_m$ and the theorem we just saw, we can set the equation:
        \[ed = \ell k + 1 \iff ed + \widetilde{\ell} k = 1\]

        We can exactly use Bézout to generate such $d$ and $\widetilde{\ell }$, when we know $k$ and $e$ if they are such that $\gcd\left(k, e\right) = 1$.
    \end{subparag}

\end{parag}

\begin{parag}{Key generation}
    We use the following algorithm to generate our keys:
    \begin{itemize}
        \item We generate large primes $p$ and $q$ at random.
        \item We compute the modulus as $m = pq$.
        \item We compute $k$, a number we keep secret, to be a multiple of $\left(p-1\right)$ and $\left(q-1\right)$. For instance, we can take $k = \phi\left(pq\right)$ or $k = \lcm\left(p-1, q-1\right)$.
        \item We produce the public encoding exponent by taking a number $e$ such that $\gcd\left(e, k\right) = 1$. A common choice is the prime number $e = 2^{16} + 1$. Note that there is no need for $e$ to be distinct for each recipient.
        \item We use Bézout's construction (the extended Euclid's algorithm) to produce the positive decoding exponent $d$ such that $de + k \ell = 1 \iff de = \hat{k} \ell + 1$.
    \end{itemize}

    As explained before, the public key is $\left(m, e\right)$ and the private key is $\left(m, d\right)$.

    \begin{subparag}{Decoding}
        For the decoding, we have that:
        \[\left(\left[t\right]^e\right)^d = \left[t\right]_m^{ed} = \left[t\right]_{pq}^{1 + \hat{k} \ell } = \left[t\right]_{pq} = \left[t\right]_m\]
        as expected.
    \end{subparag}

    \begin{subparag}{Example}
        Let us take $p = 3$ and $q = 11$.

        Then, we have $m = pq = 33$, and we can pick:
        \[k = \lcm\left(p-1, q-1\right) = \lcm\left(2, 10\right) =  10\]

        We then pick $e = 7$, which is relatively prime with $k$. We can use Bézout's construction to find that $d = 3$ (we can check it worked by verifying that $ed \Mod k = 1$, which is correct).

        Then, the public key is $\left(33, 7\right)$ and the private key is $\left(33, 3\right)$.
    \end{subparag}

    \begin{subparag}{Remark 1}
        The power of this cryptosystem comes from the difficulty to find the prime factorisation of big numbers. Indeed, else, it would be very easy to extract $p$ and $q$ from $m$, and we could use this easily to find the private key.
    \end{subparag}

    \begin{subparag}{Remark 2}
        In this cryptosystem, all the efforts for the key generation comes from the person receiving the message.
    \end{subparag}


    \begin{subparag}{Personal note}
        The best known classical algorithm to find the prime factorisation of a number, the General Number Field Sieve, is $O\left(\exp\left(d^{\frac{1}{3}}\right)\right)$, where $d$ is the number of digits of the number we want to factor. However, the best quantum algorithm, Shor's algorithm, is $O\left(d^3\right)$.

        This is one of the big applications that make quantum computers really important in research nowadays.
    \end{subparag}
\end{parag}


\end{document}
