% !TeX program = lualatex

\documentclass[a4paper]{article}

% Expanded on 2022-05-05 at 14:21:39.

\usepackage{../../style}

\title{AICC2}
\author{Joachim Favre}
\date{Jeudi 05 mai 2022}

\begin{document}
\maketitle

\lecture{20}{2022-05-05}{Minimum distance}{
\begin{itemize}[left=0pt]
    \item Definition of minimum distance decoding.
    \item Definition of the minimum distance of a code.
    \item Proof of some theorems linking the maximum erasure and error weight so that that minimum distance decoding finds the good answer.
    \item Proof of Singleton's Bound.
    \item Example using the Mod 97--10 procedure.
\end{itemize}

}

\begin{parag}{Definition: Minimum distance decoding}
    For any code $\mathcal{C}$ and for any channel yielding $y$, the Minimum Distance (MD)-Decoder outputs:
    \[\hat{c} \in \arg \min_{x \in \mathcal{C}} d_H\left(y, x\right)\]

    The arg maps back to the codeword. We are having a set since there might be multiple minimums.

    In other words, the decoder makes a table, measuring the distance between what it gets and all the possible codewords. Then, it just picks the one with the smallest Hamming distance.

    \begin{subparag}{Intuition}
        This decoder uses the intuition we had previously: it makes sense to map what we get to the closest codeword.
    \end{subparag}
\end{parag}

\begin{parag}{Example}
    Let's consider again our encoder doing repetition encoding $ab \mapsto aaabbb$ with $a, b \in \left\{0, 1\right\}$. If a MD-decoder receives $0??111$ which got out of an erasure channel, then it can draw:
    \begin{center}
    \begin{tabular}{c|c}
        $x \in \mathcal{C}$ & $d\left(x, 0??111\right)$ \\
        \hline
        000000 & 5 \\
        000111 & 2 \\
        111000 & 6 \\
        111111 & 3
    \end{tabular}
    \end{center}

    Thus the decoder would translate $0??111 \mapsto 000111 \mapsto 01$.

    Let's now consider a decoder getting 010111 after an error channel:
    \begin{center}
    \begin{tabular}{c|c}
        $x \in \mathcal{C}$ & $d\left(x, 010111\right)$ \\
        \hline
        000000 & 4 \\
        000111 & 1 \\
        111000 & 5 \\
        111111 & 2
    \end{tabular}
    \end{center}

    Thus, the decoder would translate $010111 \mapsto 000111 \mapsto 01$.
\end{parag}


\begin{parag}{Definition: Minimum distance of a code}
    Let's say we have a code $\mathcal{C}$. Then, the \important{minimum distance} of this code, is defined as:
    \[d_{min}\left(\mathcal{C}\right) \over{=}{def} \min_{\substack{x, y \in \mathcal{C} \\ x \neq y}} d_H\left(x, y\right)\]

    \begin{subparag}{Example}
        Let's consider the following code:
        \[\mathcal{C} = \left\{000000, 000111, 111000, 1111111\right\}\]

        Then, $d_{min}\left(\mathcal{C}\right) = 3$.
    \end{subparag}
\end{parag}

\begin{parag}{Example}
    Let's consider again the following code:
    \begin{center}
    \begin{tabular}{cccc}
        000 & $\mapsto$ & $0000000 = c_1$ \\
        001 & $\mapsto$ & $0011100 = c_2$ \\
        010 & $\mapsto$ & $0111011 = c_3$ \\
        011 & $\mapsto$ & $1110100 = c_4$ \\
        100 & $\mapsto$ & $0100111 = c_5$ \\
        101 & $\mapsto$ & $1101000 = c_6$ \\
        110 & $\mapsto$ & $1001111 = c_7$ \\
        111 & $\mapsto$ & $1010011 = c_8$ 
    \end{tabular}
    \end{center}

    To compute its minimum distance, there is no other way (for now) than drawing a table, and computing all $64$ distances (36 if we remove symmetry and 28 if we remove the cases were $c_i = c_j$):

    \begin{center}
    \begin{tabular}{c|cccccccc}
        & $c_1$ & $c_2$ & $c_3$ & $c_4$ & $c_5$ & $c_6$ & $c_7$ & $c_8$ \\
        \hline
            $c_1$ & 0 & 3 & 5 & 4 & 4 & 3 & 5 & 4 \\ 
            $c_2$ &   & 0 & 4 & 3 & 5 & 4 & 4 & 5 \\ 
            $c_3$ &   &   & 0 & 5 & 3 & 4 & 4 & 3 \\ 
            $c_4$ &   &   &   & 0 & 4 & 3 & 5 & 4 \\ 
            $c_5$ &   &   &   &   & 0 & 5 & 3 & 4 \\ 
            $c_6$ &   &   &   &   &   & 0 & 4 & 5 \\ 
            $c_7$ &   &   &   &   &   &   & 0 & 3 \\ 
            $c_8$ &   &   &   &   &   &   &   & 0 \\ 
    \end{tabular}
    \end{center}

    We thus get $d_{min}\left(\mathcal{C}\right) = 3$.
\end{parag}

\begin{parag}{Theorem: Erasure channel}
    Let's consider an erasure channel.

    If the erasure weight is such that $p < d_{min}\left(\mathcal{C}\right)$, then the MD-decoder is guaranteed to find the correct codeword.

    \begin{subparag}{Proof}
        We know by hypothesis that $p < d_{min}\left(\mathcal{C}\right)$.

        Let's suppose for contradiction that there are two codewords $c \neq \widetilde{c}$ such that they are both consistent with all non-erased symbols. This would mean that $c$ and $\widetilde{c}$ are equal in all non erasaed symbol, meaning:
        \[d\left(c, \widetilde{c}\right) \leq p\]

        But since $c \neq \widetilde{c}$, this is a contradiction to the property $d_{min}\left(\mathcal{C}\right) > p$.

        \qed
    \end{subparag}

    \begin{subparag}{Examples}
        Let's consider the same code as in the last example. We know that $d_{min}\left(\mathcal{C}\right) = 3$.

        If $y = ?01110?$, then we get $\hat{c} = 0011100$, which is the only possibility (as expected, since $p < d_{min}\left(\mathcal{C}\right)$).

        If $y = 11???00$, then we get $\hat{c} \in \left\{1110100, 0100111\right\}$.

        If $y = ???0011$, then we get $\hat{c} = 1010011$. We had luck, we found the correct codeword even though $p \geq d_{min}\left(\mathcal{C}\right)$.
    \end{subparag}

    \begin{subparag}{Remark}
        The greatest $p$, $p_{max}$, such that for any erasure weight $p \leq p_{max}$, we can be certain that the correct codeword can be found, is given by: 
        \[p_{max} = d_{min} - 1\]
    \end{subparag}
    
\end{parag}

\begin{parag}{Theorem: Error channel}
    Let's consider an error channel.

    If the error weight is such that $p < \frac{d_{min}\left(\mathcal{C}\right)}{2}$, then the MD-decoder is guaranteed to find the correct codeword.

    \begin{subparag}{Proof}
        By hypothesis, we know that the error weight is such that $p < \frac{d_{min}\left(\mathcal{C}\right)}{2}$. Let's say that the true codeword is $c$, the channel output is $y$, and $\hat{c}$ is the output of the MD-decoder.

        Then, we know that $d\left(c, y\right) = p$. Also, we get that:
        \[d\left(\hat{c}, y\right) \leq d\left(c, y\right) = p\]

        Thus, using the triangle inequality:
        \[d\left(c, \hat{c}\right) \leq \underbrace{d\left(c, y\right)}_{= p} + \underbrace{d\left(y, \hat{c}\right)}_{\leq p} \leq 2p \over{<}{hyp} 2\cdot \frac{d_{min}\left(\mathcal{C}\right)}{2} < d_{min}\left(\mathcal{C}\right)\]

        And, the only possibility for $d\left(c, \hat{c}\right) < d_{min}\left(\mathcal{C}\right)$, is that $c = \hat{c}$, meaning that the MD-decoder gave the right result.
    \end{subparag}

    \begin{subparag}{Remark 1}
        The greatest $p$, $p_{max}$, such that for any error weight $p \leq p_{max}$, we can be certain that the correct codeword can be found, is given by: 
        \[p_{max} = \left\lfloor \frac{d_{min} - 1}{2} \right\rfloor\]
    \end{subparag}

    \begin{subparag}{Remark 2}
        The greatest $p$, $p_{max}$, such that for any error weight $p \leq p_{max}$, we can be certain that an error \textit{can be detected} is given by:
        \[p_{max} = d_{min} - 1\]

        Indeed, if $p \leq d_{min} - 1$, then we cannot mistake what we get with another codeword.
    \end{subparag}
    
\end{parag}

\begin{parag}{Observation}
    We want $d_{min}\left(\mathcal{C}\right)$ to be large in a code, since, in both channel, the larger it is, the surer we are to get the good result. Then, we also want $n$ to be small, since it would mean less to transmit; and we want to have $\left|\mathcal{C}\right|$ to also be large. Those two last conditions are equivalent to wanting $R = \frac{k}{n}$ to be large.

    We want both $d_{min}\left(\mathcal{C}\right)$ and $R$ to be large, but there is some tension between them: increasing them decreases the other. This leads us to the following bound.
\end{parag}

\begin{parag}{Theorem: Singleton's bound}
    For any code $\mathcal{C}$, regardless of the alphabet $\mathcal{A}$, we have:
    \[d_{min}\left(\mathcal{C}\right) \leq n -k + 1\]
    where $n$ is the block length and $k = \log_{\left|\mathcal{A}\right|}\left(\left|\mathcal{C}\right|\right)$.

    \begin{subparag}{Proof}
        We have $\left|\mathcal{C}\right| = \left|\mathcal{A}\right|^k$ codewords. Let's say we have the following code (each column represents one of the digit of the code, and each row the $i$\Th codeword):
        \begin{center}
        \begin{tabular}{c|ccccccccc}
            & 1 & 2 & 3 & $\cdots$ & $d_{min} -1$ & $d_{min}$ & $d_{min} + 1$ & $\cdots$ & $n$ \\
            \hline
            1 & 0 & 0 & 0 & $\cdots$ & 0 & 0 & 0 & $\cdots$ & 0 \\
            2 & 0 & 1 & 0 & $\cdots$ & 1 & 1 & 0 & $\cdots$ & 1 \\
            $\vdots$ \\
            $\left|\mathcal{A}\right|^k$ & 0 & 0 & 0 & $\cdots$ & 1 & 1 & 1 & $\cdots$ & 1
        \end{tabular}
        \end{center}

        Now, let's erase the $d_{min} - 1$ first columns:
        \begin{center}
        \begin{tabular}{c|cccc}
            & $d_{min}$ & $d_{min} + 1$ & $\cdots$ & $n$ \\
            \hline
            1 & 0 & 0 & $\cdots$ & 0 \\
            2 & 1 & 0 & $\cdots$ & 1 \\
            $\vdots$ \\
            $\left|\mathcal{A}\right|^k$ & 1 & 1 & $\cdots$ & 1
        \end{tabular}
        \end{center}

        In this new table, there cannot be two rows that are equal since, then, $d_{min}$ could not be the one it is (since we erased $d_{min} - 1$ columns). However, since we only have $n - d_{min} + 1$ columns, we cannot make more that $\left|\mathcal{A}\right|^{n - d_{min} + 1}$ different rows. Since we have $\left|\mathcal{C}\right| = \left|\mathcal{A}\right|^k$ codewords, we get:
        \[\left|\mathcal{A}\right|^k \leq \left|\mathcal{A}\right|^{n - d_{min} + 1} \iff k \leq n - d_{min} + 1 \iff d_{min} \leq n - k + 1\]

        \qed
    \end{subparag}

\end{parag}

\begin{parag}{Definition: MDS code}
    A code $\mathcal{C}$ that attains Singleton's Bound with equality is called \important{Maximum Distance Separable (MDS) code}.
\end{parag}

\begin{parag}{Example}
    Let's consider again the following code:
    \begin{center}
    \begin{tabular}{cccc}
        000 & $\mapsto$ & $0000000 = c_1$ \\
        001 & $\mapsto$ & $0011100 = c_2$ \\
        010 & $\mapsto$ & $0111011 = c_3$ \\
        011 & $\mapsto$ & $1110100 = c_4$ \\
        100 & $\mapsto$ & $0100111 = c_5$ \\
        101 & $\mapsto$ & $1101000 = c_6$ \\
        110 & $\mapsto$ & $1001111 = c_7$ \\
        111 & $\mapsto$ & $1010011 = c_8$ 
    \end{tabular}
    \end{center}

    We know that $n = 7$, $k = 3$ and we have already computed $d_{min}\left(\mathcal{C}\right) = 3$. We observe that $n - k + 1 = 5 \neq 3$, meaning that it is not an MDS code.
\end{parag}

\begin{parag}{Geometrical interpretation}
    We notice that we can interpret codes geometrically.

    If $n = 2$ and $\mathcal{A} = \left\{0, 1\right\}$, then $\mathcal{A}^n = \mathcal{A}^2 = \left\{00, 01, 10, 11\right\}$. If our code is $\mathcal{C} =\left\{00, 11\right\}$, then we can draw the following diagram:
    \svghere[0.25]{DiagramGeometricalInterpretationCode2D.svg}

    We notice that the Hamming distance between two vertices is the shortest number of edges we have to take to go from the first point to the second.

    We can do a similar diagram if $n = 3$, if $\mathcal{C} = \left\{100\right\}, \left\{010\right\}, \left\{001\right\}$:
    \svghere[0.25]{DiagramGeometricalInterpretationCode3D.svg}
\end{parag}

\begin{parag}{Example: Procedure mod 97--10}
    We defined the procedure mod 97--10 as: 
    \[u \mapsto v = \left(100u\right) + \left(98 - \left[100u\right]_{97}\right)\]

    This is a code, just like what we have seen so far. Indeed, we have $k$ to be the number of digits of $u$, and: 
    \[\mathcal{A} = \left\{0, 1, \ldots, 9\right\}, \mathspace n = k+2, \mathspace R = \frac{k}{n} = \frac{k}{k + 2}\]
    
    We proved that any codeword must satisfy $\left[v\right]_{97} = 1$, and we will now prove that any single error will be detected.

    \begin{subparag}{Proof}
        Let's say that our channel maps: 
        \[v \mapsto v' = v + a\cdot 10^\ell, \mathspace \text{ for some } \ell \in \mathbb{Z}, a \in \left\{-9, -8, \ldots, 9\right\}\]
        
        Our claim is that $\left[v'\right]_{97} = 1$ if and only if $a = 0$. We see that: 
        \[\left[v'\right]_{97} = \left[v + a 10^\ell\right]_{97} = \left[\left[v\right]_{97} + \left[a 10^{\ell }\right]_{97}\right]_{97} = \left[1 + \left[a 10^\ell\right]_{97}\right]_{97}\]
        
        So, we must have $\left[a 10^\ell \right]_{97} = 0$, which is equivalent to $\left[a\right]_{97} = 0$ since $10^\ell$ and 97 are coprime and thus $10^\ell $ has an inverse mod 97. Also, this is equivalent to $a = 0$, since $a \in \left\{-9, -8, \ldots, 9\right\}$.

        To sum up, we have indeed found that: 
        \[\left[v'\right]_{97} = 1 \iff \left[a 10^\ell \right]_{97} = 0 \iff \left[a\right]_{97} = 0 \iff a = 0\]
        
        \qed
    \end{subparag}
    
\end{parag}


\end{document}
