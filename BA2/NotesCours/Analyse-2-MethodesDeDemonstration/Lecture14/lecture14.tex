% !TeX program = lualatex

\documentclass[a4paper]{article}

% Expanded on 2022-04-04 at 19:21:21.

\usepackage{../../style}

\title{Analyse 2 --- Méthode de démonstration}
\author{Joachim Favre}
\date{Mercredi 06 avril 2022}

\begin{document}
\maketitle

\lecture{14}{2022-04-06}{Récurrence forte}{
\begin{itemize}[left=0pt]
    \item Explication de la méthode dite de récurrence forte.
    \item Résumé des trois méthodes de récurrence que nous avons vues.
\end{itemize}
}

\begin{parag}{Récurrence forte}
    Soit $n_0 \in \mathbb{N}$ et $P\left(n\right)$ une proposition qui dépend de $n \in \mathbb{N}$, où $n \geq n_0$.

    Supposons que:
    \begin{enumerate}
        \item $P\left(n_0\right)$ est vraie.
        \item $\left\{P\left(n_0\right), P\left(n_0 + 1\right), \ldots, P\left(n\right)\right\}$ impliquent $P\left(n+1\right)$, pour tout $n \geq n_0$, $n \in \mathbb{N}$.
    \end{enumerate}
    
    Alors, $P\left(n\right)$ est vraie pour tout $n \geq n_0$, où $n \in \mathbb{N}$.
\end{parag}

\begin{parag}{Exemple}
    Pour tout $n \geq 2$, où $n \in \mathbb{N}$, $n$ est un produit de facteurs premiers.

    \begin{subparag}{Preuve}
        Nous faisons notre démonstration par récurrence forte. 
        \begin{enumerate}[left=0pt]
            \item Nous voulons montrer $P\left(2\right)$. Nous savons que $2$ est premier, donc nous pouvons écrire: 
            \[2 = 2\]
            
            Ce qui montre qu'elle est vraie.
        \item Nous considérons que $\left\{P\left(2\right), \ldots, P\left(n\right)\right\}$ sont vraies, et nous voulons montrer que $P\left(n+1\right)$ est vraie. 

            Considérons $\left(n + 1\right) \in \mathbb{N}$. S'il est premier, alors, par le même argument que pour la base, $P\left(n+1\right)$ est vraie.

            Sinon, nous savons que $\left(n + 1\right) = m\cdot k$ où $m, k \in \mathbb{N}$ sont tels que $2 \leq m, k < n + 1$. Par notre hypothèse de récurrence, nous savons que $P\left(m\right)$ et $P\left(k\right)$ sont vraies, donc $m$ et $k$ sont des produits de facteurs premiers. Ceci implique que $\left(n + 1\right) = m\cdot k$ est aussi un produit de facteurs premiers. Ainsi, $P\left(n+1\right)$ est vraie.
        \item Puisque la base et l'hérédité tiennent, nous avons démontré par récurrence forte $P\left(n\right)$ pour tout $n \geq 2$, où $n \in \mathbb{N}$.
        \end{enumerate}
        
        \qed
    \end{subparag}
\end{parag}

\begin{parag}{Résumé: Récurrence}
    Nos trois méthodes nous permettent de démontrer que $P\left(n\right)$ est vraie pour tout $n \geq n_0$.

    \begin{subparag}{Récurrence simple}
        \begin{enumerate}[left=0pt]
            \item $P\left(n_0\right)$ est vraie.
            \item $P\left(n\right) \implies P\left(n+1\right)$ pour tout $n \geq n_0$.
        \end{enumerate}
    \end{subparag}

    \begin{subparag}{Récurrence généralisée}
        Soit $k \geq 1$.
        \begin{enumerate}[left=0pt]
            \item $P\left(n_0\right), \ldots, P\left(n_0 + k\right)$ sont vraies.
            \item $\left\{P\left(n\right), \ldots, P\left(n+k\right)\right\} \implies P\left(n+k+1\right)$ pour tout $n \geq n_0$.
        \end{enumerate}
    \end{subparag}

    \begin{subparag}{Récurrence forte}
        \begin{enumerate}[left=0pt]
            \item $P\left(n_0\right)$ est vraie.
            \item $\left\{P\left(n_0\right), \ldots, P\left(n\right)\right\} \implies P\left(n+1\right)$ pour tout $n \geq n_0$.
        \end{enumerate}
    \end{subparag}
\end{parag}



\end{document}
