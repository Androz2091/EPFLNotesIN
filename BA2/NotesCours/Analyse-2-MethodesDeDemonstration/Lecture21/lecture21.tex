% !TeX program = lualatex

\documentclass[a4paper]{article}

% Expanded on 2022-05-11 at 13:12:09.

\usepackage{../../style}

\title{Méthodes de démonstration}
\author{Joachim Favre}
\date{Mercredi 11 mai 2022}

\begin{document}
\maketitle

\lecture{21}{2022-05-11}{Résumé}{
\begin{itemize}[left=0pt]
    \item Résumé des différentes méthodes que nous avons vues.
\end{itemize}

}

\subsection{Résumé}
\begin{parag}{Résumé}
    \begin{subparag}{Démonstration directe}
        Nous partons des conditions données ou de faits connus, $Q$, et nous faisons des implications logiques pour obtenir notre proposition désirée, $P$: 
        \[Q \implies \ldots \implies P\]
    \end{subparag}

    \begin{subparag}{Par contraposée}
        Pour montrer $P \implies Q$, nous montrons que $\lnot Q \implies \lnot P$.
    \end{subparag}

    \begin{subparag}{Disjonctions de cas}
        Nous séparons $P$ en plusieurs sous-cas, puis démontrons chaque cas séparément.
    \end{subparag}
    
    \begin{subparag}{''Si et seulement si''}
        Pour démontrer $P \iff Q$, nous pouvons soit démontrer que $P\implies Q$ et $Q \implies P$, soit faire une suite d'équivalence entre $P$ et $Q$: 
        \[P \iff \ldots \iff Q\]
    \end{subparag}
    
    \begin{subparag}{Par absurde}
        Pour démontrer $P$, nous démontrons que $\lnot P \implies Q$, où $Q$ est une proposition connue d'être fausse.
    \end{subparag}

    \begin{subparag}{Par le principe des tiroirs}
        Si nous avons $n$ objets distribués dans $k$ tirons, alors au moins un tiroir contient $\left\lceil \frac{n}{k} \right\rceil $ objets.
    \end{subparag}

    \begin{subparag}{Par récurrence simple}
        Pour démontrer $P\left(n\right)$ pour tout $n \geq n_0$, nous démontrons que $P\left(n_0\right)$ est vraie, et que $P\left(n\right) \implies P\left(n+1\right)$ pour tout $n \geq n_0$.
    \end{subparag}

    \begin{subparag}{Par récurrence généralisée}
        Pour démontrer $P\left(n\right)$ pour tout $n \geq n_0$, nous démontrons $P\left(n_0\right), \ldots, P\left(n_0 + k\right)$ et $\left\{P\left(n\right), \ldots, P\left(n+k\right)\right\} \implies P\left(n + k + 1\right)$ pour tout $n \geq n_0$, pour un $k \geq 1$ fixé.
    \end{subparag}
    
    \begin{subparag}{Par récurrence forte}
        Pour démontrer $P\left(n\right)$ pour tout $n \geq n_0$, nous démontrons $P\left(n_0\right)$, et $\left\{P\left(n_0\right), \ldots, P\left(n\right)\right\} \implies P\left(n+1\right)$ pour tout $n \geq n_0$.
    \end{subparag}

    \begin{subparag}{Par récurrence sur deux variables, méthode du carré}
        Pour démontrer $P\left(n, m\right)$ pour tout $m, n \geq 0$, nous démontrons $P\left(0, 0\right)$, $P\left(m, 0\right) \implies P\left(m + 1, 0\right)$ pour tout $m$ et $P\left(m, n\right) \implies P\left(m, n + 1\right)$ pour tout $m, n$.
    \end{subparag}

    \begin{subparag}{Par récurrence sur deux variables, méthode de la diagonale}
        Pour démontrer $P\left(n, m\right)$ pour tout $m, n \geq 0$, nous démontrons $P\left(m, 0\right) \implies P\left(m + 1, 0\right)$ pour tout $m$ et $P\left(m+1, n\right) \implies P\left(m, n+1\right)$ pour tout $m, n$.
    \end{subparag}
\end{parag}

\begin{parag}{Exemple}
    Il est parfois dur de trouver la méthode de démonstration à utiliser. Prenons les propositions suivantes en tant qu'exemples:
    \begin{enumerate}
        \item Dans une classe de 210 étudiants, il existe au moins 9 personnes avec la même initiale de prénoms.
        \item Il n'existe pas de nombres entiers $a, b \in \mathbb{Z}$ tels que $63a = 81 - 21b$.
        \item Pour tout $n \geq 2$ naturel, nous avons: 
            \[\prod_{k=2}^{n} \left(1 - \frac{1}{k^2}\right) = \frac{n+1}{2n}\]
        \item Si $f: E \mapsto \mathbb{R}$, où $E \subset \mathbb{R}^n$, est dérivable en $\bvec{x}= \bvec{a} \in E$, alors elle est continue en $\bvec{x} = \bvec{a}$.
        \item Pour tout $n \geq 1$, nous avons: 
            \[\sum_{k=1}^{n} \frac{1}{k\left(k+1\right)} = \frac{n}{n+1}\]
        \item Pour tout couple $a, b \in \mathbb{Z}$, le nombre $\left(7a + 4b\right)\left(a - 3b\right)\left(2a + b\right)$ est pair.
    \end{enumerate}

    \vspace{1em}

    La meilleure méthode pour les démontrer est probablement:
    \begin{enumerate}
        \item Par le principe des tiroirs et des chaussettes.
        \item Par l'absurde.
        \item Par récurrence simple.
        \item Par démonstration directe.
        \item Par démonstration directe (en voyant que c'est une série télescopique) ou par récurrence.
        \item Par disjonctions de cas: si $a$ est pair, si $b$ est pair, et si $a$ et $b$ sont impairs.
    \end{enumerate}
    
    Faire ces démonstrations est un bon exercice.
\end{parag}


\end{document}
