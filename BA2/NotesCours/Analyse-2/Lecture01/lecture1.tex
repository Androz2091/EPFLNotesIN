\documentclass[a4paper]{article}

% Expanded on 2022-02-21 at 10:07:54.

\usepackage{../../style}

\title{Analyse 2}
\author{Joachim Favre}
\date{Lundi 21 février 2022}

\begin{document}
\maketitle

\lecture{1}{2022-02-21}{Le meilleur sujet}{
\begin{itemize}[left=0pt]
    \item Définition des équations différentielles ordinaires.
    \item Explication d'un petit peu de terminologie autour de ces équations.
    \item Introduction au théorème de l'existence et de l'unicité d'une solution des EDVS.
\end{itemize}
}

\section{Équations différentielles ordinaires}
\subsection{Définitions et exemples}
\parag{Exemple 1}{
    Supposons qu'on a une fonction $y\left(x\right) = y$, où $x \in \mathbb{R}$. De plus, disons que nous savons que: 
    \[y' = 0, \mathspace \forall x \in \mathbb{R}\]
    
    On remarque alors que $y\left(x\right) = 2$ est une solution sur $\mathbb{R}$. De plus, on peut même obtenir une solution plus générale:
    \[y\left(x\right) = C, \mathspace \text{où } C \in \mathbb{R}, \forall x \in \mathbb{R}\]
    
    On verra plus tard que c'est la solution générale à cette équation.
}

\parag{Définition}{
    Une \important{équation différentielle ordinaire} est une expression: 
    \[E\left(x, y, y', \ldots, y^{\left(n\right)}\right) = 0\]
    où $E$ est une expression fonctionnelle, $n \in \mathbb{N}_0$, et $y = y\left(x\right)$ est une fonction inconnue de $x$.

    On cherche un intervalle ouvert $I \subset \mathbb{R}$ et une fonction $y : I \mapsto \mathbb{R}$ de classe $C^n$ telle que l'équation donnée est satisfaite $\forall x \in I$.

    \subparag{Remarque personnelle}{
        Cette définition peut paraître très formelle, mais l'idée c'est qu'une équation différentielle ordinaire est une équation où on a une fonction, ses dérivées, et la variable $x$. Cela permet de faire une opposition avec les équations intégrales comme la \textit{Rendering Equation} (où $L_i$ est liée à $L_o$, ce qui fait que c'est bien une équation et non une formule): 
        \[L_o(x, \bvec{\omega}) = L_e(x, \bvec{\omega}) + \int_\Omega{L_i(x, \bvec{\omega}')}f_r(\bvec{\omega}, x, \bvec{\omega}')\cos(\theta)d\bvec{\omega}'\]
        
        Cela permet aussi de faire une différence avec les équations à dérivées partielles, comme l'équation de la chaleur: 
        \[\frac{\partial T\left(x, t\right)}{\partial t} - k \frac{\partial^2 T\left(x, t\right)}{\partial x^2} = 0\]
    }
    
}

\parag{Exemple 2}{
    Considérons l'équation différentielle ordinaire suivante: 
    \[y'' = 0\]
    
    On remarque déjà que, nécessairement, $y' = C_1 \in \mathbb{R}$ pour tout $x \in \mathbb{R}$. Ainsi, pour finir, on obtient: 
    \[y = C_1 x + C_2, \mathspace \forall C_1, C_2 \in \mathbb{R}, \forall x \in \mathbb{R}\]
}

\parag{Exemple 3}{
    Prenons maintenant l'équation suivante: 
    \[y + y' = 0 \iff y' = -y\]
    
    On sait que $f\left(x\right) = e^x$ est telle que $f'\left(x\right) = f\left(x\right)$. Il nous suffit donc de la modifier légèrement pour obtenir une solution à notre équation: 
    \[y = e^{-x}, \mathspace \forall x \in \mathbb{R}\]
    
    On peut finalement en déduire la solution générale: 
    \[y\left(x\right) = Ce^{-x}, \mathspace \forall C \in \mathbb{R}, x \in \mathbb{R}\]
}

\parag{Équation la plus simple}{
    On remarque que l'exemple 1 et l'exemple 2 sont de la forme: 
    \[y^{\left(n\right)} = f\left(x\right)\]
    où $f\left(x\right)$ est une fonction connue et continue sur $I$.
    
    Dans ce cas là, nous pouvons résoudre cette équation par intégration.
}

\parag{Équation à variables séparées}{
    On appelle le troisième exemple une \important{équation à variables séparées}. En effet, on peut écrire: 
    \[\frac{dy}{dx} = y' \implies \frac{dy}{dx} = -y \over{\implies}{$y \neq 0$} \frac{dy}{y} = -dx\]

    Ainsi, nos variables sont séparées: nous avons $y$ d'un côté et $x$ de l'autre. Ceci nous permet d'obtenir une dépendance entre le changement infinitésimal en $y$ et le changement infinitésimal en $x$. Puisque c'est vrai pour tout $x$ sur notre intervalle, on peut intégrer en préservant l'égalité: 
    \[\int \frac{dy}{y} = -\int dx \implies \log\left|y\right| = -x + C_1\]
    pour un $C_1 \in \mathbb{R}$ arbitraire.

    On peut continuer notre équation:
    \[\left|y\right| = e^{-x} \underbrace{e^{C_1}}_{C_2 > 0} \implies y = \pm C_2 e^{-x}\]
    
    Finalement, on peut aussi remarque que $y = 0$ est une solution. Tout ceci nous permet d'obtenir la solution générale à notre équation: 
    \[y\left(x\right) = Ce^{-x}, \mathspace C \in \mathbb{R}, x \in \mathbb{R}\]
}

\parag{Terminologie}{
    On appelle $E\left(X, y, \ldots, y^{\left(n\right)}\right) = 0$ une équation différentielle (ED).

    \subparag{Ordre}{
        Un nombre naturel $n \in \mathbb{N}^*$ est \important{l'ordre} d'une équation différentielle si $n$ est l'ordre maximal de dérivée de $y\left(x\right)$ dans l'équation.
    }

    \subparag{Équation linéaire}{
        Si notre équation différentielle est un polynôme linéaire en $y, y', \ldots, y^{\left(n\right)}$, alors l'équation est dite \important{linéaire}.
    }

    \subparag{Équation autonome}{
        Si l'expression ne contient pas de $x$, l'équation est dite \important{autonome}.
    }

    \subparag{Solution générale}{
        La \important{solution générale} d'une équation différentielle est l'ensemble de toutes les solutions de l'équation.
    }
}

\parag{Exemple 4}{
    Considérons les équations suivantes:
    \begin{enumerate}
        \item $\displaystyle y + y' = 5x + 1$ 
        \item $\displaystyle 2x^2 y + y''' = 0$
        \item $\displaystyle y' + 3y'' = 0$
    \end{enumerate}
    
    On peut voir les propriétés suivantes:
    \begin{enumerate}
        \item Équation différentielle linéaire d'ordre 1 qui n'est pas autonome.
        \item ED linéaire d'ordre 3 qui n'est pas autonome.
        \item ED linéaire d'ordre 2 autonome.
    \end{enumerate}
}

\parag{Définition du problème de Cauchy}{
    Résoudre le \important{problème de Cauchy} (équation différentielle avec des conditions initiales) pour l'équation $E\left(x, y, y', \ldots, y^{\left(n\right)}\right) = 0$ consiste à trouver l'intervalle ouvert $I \subset \mathbb{R}$ et une fonction $y : I \mapsto \mathbb{R}$ de classe $C^n\left(I\right)$ telle que $E\left(x, y, \ldots, y^{\left(n\right)}\right) = 0$ sur $I$ et pour laquelle des conditions initiales sont satisfaites:
    \[y\left(x_0\right) = b_0, y\left(x_1\right) = b_1, y'\left(x_2\right) = b_2, \ldots\]
    
    Le nombre des conditions initiales dépend du type de l'équation différentielle.
}

\parag{Retour à l'exemple 2}{
    Nous avions trouvé: 
    \[y'' = 0 \implies y\left(x\right) = C_1 x + C_2, \mathspace \forall C_1, C_2 \in \mathbb{R}, x \in \mathbb{R}\]
    
    Résolvons maintenant le problème de Cauchy pour $y'' = 0$ avec les conditions initiales: 
    \[y\left(0\right) = 1, \mathspace y\left(2\right) = 4\]
    
    Nous pouvons mettre ces conditions initiales dans notre solution générale: 
    \[1 = y\left(0\right) = C_1\cdot 0 + C_2 = C_2 \implies C_2 = 1\]
    \[4 = y\left(2\right) = C_1 \cdot 2 + C_2 = 2C_1 + 1 \implies C_1 = \frac{3}{2}\]
    
    On obtient ainsi la \important{solution particulière} satisfaisant les conditions initiales: 
    \[y\left(x\right) = \frac{3}{2}x + 1\]
}

\subsection{Équations différentielles à variables séparées}
\parag{Définition}{
    On appelle une \important{équation différentielle à variables séparées} (EDVS) une équation sous la forme: 
    \[f\left(y\right) \cdot y' = g\left(x\right)\]
    où $f: I \mapsto \mathbb{R}$ est une fonction continue sur $I \subset \mathbb{R}$ et $g : J \mapsto \mathbb{R}$ est une fonction continue sur $J \in\mathbb{R}$.

    Une fonction $y: J' \subset J \mapsto I$ de classe $C^1$ satisfaisant l'équation $f\left(y\right) y' = g\left(x\right)$ est une solution.

    \subparag{Explication}{
        Nous pouvons manipuler notre équation de la manière suivante: 
        \[f\left(y\right) \frac{dy}{dx} = g\left(x\right) \over{\implies}{en gros}  \int f\left(y\right) dy = \int g\left(x\right) dx\]

        Nous verrons pourquoi ceci marche et pourquoi cette méthode est formelle à l'aide d'un théorème ci-après.
    }
}

\parag{Exemple}{
    Par exemple, $y' = -y$ est une EDVS: 
    \[y' = -y \iff \frac{1}{y} y' = -1 \implies f\left(y\right) = \frac{1}{y}, g\left(x\right) = -1\]
    
}

\parag{Théorème (existence et unicité d'une solution des EDVS)}{
    Soit $f: I \mapsto \mathbb{R}$ une fonction continue telle que $f\left(y\right) \neq 0$ pour tout $y \in I$, et soit $g : J \mapsto \mathbb{R}$ une fonction continue.

    \important{Existence:} Alors, pour tout couple $\left(x_0, b_0\right)$ où $x_0 \in J$ et $b_0 \in I$, l'équation 
    \[f\left(y\right) y' = g\left(x\right)\]
    admet une solution $y : J' \subset J \mapsto I$ vérifiant la condition initiale.

    \important{Unicité:} Si $y_1 : J_1 \mapsto I$ et $y_2 : J_2 \mapsto I$ sont deux solutions telles que $y_1\left(x_0\right) = y_2\left(x_0\right) = b_0$, alors: 
    \[y_1\left(x\right) = y_2\left(x\right), \mathspace \forall x \in J_1 \cap J_2\]

    \subparag{Note personnelle}{
        Ce théorème implique que si nous pouvons écrire la solution générale à une EDVS de manière complètement générale avec des constantes, alors il y a une et exactement une constante. Par exemple, les ensembles de fonctions suivants ne pourraient pas être les solutions générales d'une EDVS: 
        \[f\left(x\right) = C_1 \sin\left(x\right) + C_2 \cos\left(x\right), \mathspace g\left(x\right) = x^2\]
        où $C_1, C_2, \in \mathbb{R}$ et $x \in \mathbb{R}$.

        Cependant, comme on l'a vu plus tôt, l'ensemble de fonctions suivant est la solution générale à l'EDVS $y' = -y$: 
        \[y\left(x\right) = Ce^{-x}, \mathspace \text{où } C \in \mathbb{R}, x \in \mathbb{R}\]
        
    }
    
    \subparag{Preuve}{
        Nous allons seulement montrer l'existence de la solution. 

        Soit la fonction suivante: 
        \[F\left(y\right) = \int_{b_0}^{y} f\left(t\right)dt\]

        On sait que $F\left(y\right)$ est dérivable par le théorème fondamental du calcul intégral. De plus, on sait que $F'\left(y\right) = f\left(y\right) \neq 0$ sur $I$, donc $f\left(y\right)$ ne change pas pas de signe et donc $F\left(y\right)$ est monotone. Puisque $F\left(y\right)$ est continue et monotone, on sait qu'elle est inversible sur $I$. 
        
        Soit aussi la fonction suivante: 
        \[G\left(x\right) = \int_{x_0}^{x} g\left(t\right)dt\]
        
        Par le théorème fondamental du calcul intégral, on sait aussi que $G\left(x_0\right) = 0$ et que $G$ est dérivable sur $J$.

        Définissons aussi la fonction suivante dans un voisinage de $x_0$ (on sait que $F$ est inversible sur $I$, et $F^{-1}\left(G\left(x_0\right)\right) = b_0 \in I$): 
        \[y\left(x\right) = F^{-1}\left(G\left(x\right)\right)\]
        
        Nous allons démontrer que $y\left(x\right)$ est une solution de l'équation $f\left(y\right) y'\left(x\right) = g\left(x\right)$ dans un voisinage de $x_0 \in J$, et qu'elle satisfait $y\left(x_0\right) = b_0$.

        En manipulant notre définition, on obtient que, dans un voisinage de $x_0 \in J$: 
        \[F\left(y\left(x\right)\right) = G\left(x\right) \over{\implies}{$\frac{d}{dx}$}  F'\left(y\left(x\right)\right) y'\left(x\right) = G'\left(x\right) \implies f\left(y\right)y'\left(x\right) =  g\left(x\right)\]

        De plus, nous savons par la définition de $G$ et $F$ que $G\left(x_0\right) = 0$ et $F\left(b_0\right) = 0$, donc:
        \[y\left(x_0\right) = F^{-1}\left(G\left(x_0\right)\right) = F^{-1}\left(0\right) = b_0\]

        \qed
    }

    \subparag{Idée de la preuve}{
        Nous partons de notre équation: 
        \[g\left(y\right) \frac{dy}{dx} = f\left(x\right)\]
        
        Et, notre théorème nous dit que c'est plus ou moins équivalent à: 
        \[\int f\left(y\right)dy = \int g\left(x\right) dx \iff F\left(y\right) = G\left(x\right)\]
    }

}

\end{document}
