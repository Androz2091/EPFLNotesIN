\documentclass[a4paper]{article}

% Expanded on 2022-03-08 at 13:18:32.

\usepackage{../../style}

\title{Analyse 2}
\author{Joachim Favre}
\date{Mercredi 09 mars 2022}

\begin{document}
\maketitle

\lecture{6}{2022-03-09}{Fin des équations différentielles}{
\begin{itemize}[left=0pt]
    \item Explication de la méthode des coefficients indéterminés.
    \item Résumé des méthodes pour résoudre les équations différentielles que nous avons vues.
\end{itemize}
}

\parag{Méthode des coefficients indéterminés}{
    Cette méthode permet de trouver une solution particulière à l'équation $y''\left(x\right) + p y'\left(x\right) + qy\left(x\right) = f\left(x\right), \mathspace p, q \in \mathbb{R}$

    Pour que cette méthode fonctionne, il faut que $f\left(x\right)$ soit une combinaison linéaire de:
    \[e^{cx} R_n\left(x\right) \mathspace \text{ et } \mathspace e^{ax}\left(\cos\left(bx\right) P_n\left(x\right) + \sin\left(bx\right)Q_m\left(x\right)\right)\]
    où $R_n\left(x\right), P_n\left(x\right), Q_m\left(x\right)$ sont des polynômes de degré $n$ et $m$, et $c, a, b \in \mathbb{R}$.

    Ainsi, si $f\left(x\right)$ est une combinaison linéaire de ces deux fonctions, disons $f\left(x\right) = c_1 f_1\left(x\right) + c_2 f_2\left(x\right)$, nous avons seulement besoin de trouver une méthode pour trouver une solution pour $f_1\left(x\right)$ et pour $f_2\left(x\right)$, puis utiliser le principe de la superposition des solutions.

    Si $f\left(x\right) = e^{cx} R_n\left(x\right)$:
    \begin{center}
    \begin{tabularx}{\textwidth}{|>{\hsize=.25\textwidth}C|C|}
        \hline
        $c$ est une racine de $\lambda^2 + p\lambda + q = 0$ & Ansatz \\
        \hhline{|=|=|}
         Non & $y_{part} = e^{cx} T_n\left(x\right)$  \\
         \hline
         Oui & $y_{part} = x^r e^{cx} T_n\left(x\right)$ \\
        \hline
    \end{tabularx}
    \end{center}
    où $r$ est la multiplicité de la racine $\lambda = c$, et $T_n\left(x\right)$ est polynôme de degré $n$ à coefficients indéterminé.

    Si $f\left(x\right) = e^{ax}\left(\cos\left(bx\right) P_n\left(x\right) + \sin\left(bx\right) Q_n\left(x\right)\right)$:
    \begin{center}
    \begin{tabularx}{\textwidth}{|>{\hsize=.25\textwidth}C|C|}
        \hline
        $a + ib$ est une racine de $\lambda^2 + p\lambda + q = 0$ & Ansatz \\
        \hhline{|=|=|}
        Non & $y_{part} = e^{ax} \left(T_N\left(x\right) \cos\left(bx\right) + S_N\left(x\right) \sin\left(bx\right)\right)$  \\
         \hline
         Oui & $y_{part} = xe^{ax} \left(T_N\left(x\right) \cos\left(bx\right) + S_N\left(x\right) \sin\left(bx\right)\right)$ \\
        \hline
    \end{tabularx}
    \end{center}
    où $N = \max\left(n, m\right)$, et $T_N\left(x\right)$ et $S_N\left(x\right)$ sont des polynômes de degré $N$ à coefficients indéterminés.

    \subparag{Note personnelle: Mnémotechnie}{
        Dans les quatre cas, nous prenons exactement l'équation en tant qu'\textit{Ansatz}, dans laquelle nous remplaçons les polynômes par des polynômes inconnus, et nous multiplions par la multiplicité d'une solution potentielle.

        En effet, on remarque que, si $c$ n'est pas une racine de $\lambda^2 + p\lambda + q = 0$, alors c'est tout comme si c'était une racine de multiplicité 0 (bien évidemment, ne jamais dire cette phrase à un oral d'analyse). Ainsi, $x^r = 1$. Nous pouvons faire le même raisonnement pour la deuxième fonction, mais nous pouvons aussi voir qu'il est impossible que $a + ib$ ait une multiplicité 2 (sinon cela voudrait dire que $b = 0$, puisque $z$ et $\bar{z}$ sont des solutions pour les nombres complexes, et donc nous serions dans le premier scénario). En d'autres mots, on obtient donc qu'on doit multiplier par $x$ et non pas $x^r$.

        Par rapport à la valeur pour laquelle on doit considérer la multiplicité de la racine, nous pouvons faire deux observations. Premièrement, dans la deuxième fonction, si $b = 0$, alors nous sommes de retour dans la première fonction. Deuxièmement, il est probable qu'il y ait un lien avec l'exponentielle complexe, qui a une forme similaire, et que nous avons déjà utilisée pour résoudre ce genre de problèmes:
        \[e^{a + ib} = e^{a} \left(\cos\left(b\right) + i\sin\left(b\right)\right)\]

        Pour l'intuition derrière cette méthode, il est probable que tout cela découle du fait qu'il est impossible de ``tuer'' l'exponentielle et les fonctions trigonométriques basiques en les dérivant (elles tournent en rond), alors que les polynômes descendent en degré.
    }

}

\parag{Exemple}{
    Nous voulons trouver la solution générale de l'équation suivante, avec $x \in \mathbb{R}$:
    \[2y'' - y' - y = 100x e^{2x}\]

    \subparag{Équation homogène associée}{
        Commençons par résoudre l'équation homogène associée:
        \[y'' - \frac{1}{2} y' - \frac{1}{2}y = 0\]

        L'équation caractéristique est $\lambda^2 - \frac{1}{2}\lambda - \frac{1}{2} = 0$, et ses solutions sont $\lambda = 1, -\frac{1}{2}$. On obtient donc:
        \[y_{hom} = C_1 e^{x} + C_2 e^{\frac{-x}{2}}, \mathspace C_1, C_2 \in \mathbb{R}, x \in \mathbb{R}\]

    }

    \subparag{Solution particulière équation complète}{
        Cherchons une solution particulière de l'équation complète par la méthode de coefficients indéterminés.

        On remarque que $f\left(x\right) = 50xe^{2x}$ est de la forme $R_n\left(x\right) e^{cx}$ où $R_n\left(x\right) = 50x$ et $n = 1, c = 2$. Ainsi, $f\left(x\right)$ est de forme acceptée par la méthode des coefficients indéterminés.

        On remarque que $c = 2$ n'est pas une racine de l'équation caractéristique $\lambda^2 - \frac{1}{2} \lambda - \frac{1}{2} = 0$. Ainsi, nous pouvons prendre l'\textit{Ansatz}:
        \[y_{part} = \left(Ax + B\right)e^{2x}\]
        \[\implies y'_{part} = Ae^{2x} + 2\left(Ax + B\right) e^{2x}\]
        \[\implies y''_{part} = 2Ae^{2x} + 2Ae^{2x} + 4\left(Ax + B\right) e^{2x}\]

        Remplaçons la dans l'équation pour obtenir des contraintes sur $A$ et $B$:
        \[\underbrace{4Ae^{2x} + 4\left(Ax + B\right) e^{2x}}_{y''_{part}} \underbrace{- \frac{1}{2} Ae^{2x} - \left(Ax + B\right) e^{2x}}_{-\frac{1}{2} y_{part}'} \underbrace{- \frac{1}{2}\left(ax + B\right)e^{2x}}_{\frac{-1}{2} y_{part}} = 50xe^{2x}\]

        Ainsi, nous obtenons que:
        \[xe^{2x} \left(4 - 1 - \frac{1}{2}\right)A + e^{2x} \left(\left(4 - \frac{1}{2}\right)A + \left(4 - 1 - \frac{1}{2}\right)B\right) = 50x e^{2x}\]

        Ce qui implique que:
        \[xe^{2x} \left(\frac{5}{2} A\right) + e^{2x} \left(\frac{7}{2}A + \frac{5}{2} B\right) = 50x e^{2x}\]


        Ceci nous donne le système d'équations suivant:
        \[\begin{systemofequations} \frac{5}{2} A = 50 \\ \frac{7}{2} A + \frac{5}{2}B = 0\end{systemofequations}\]

        Ce qui nous permet de trouver que:
        \[A = 20, \mathspace B = -28\]

        C'est ensuite une bonne idée de vérifier le résultat.
    }

    \subparag{Solution générale}{
        Nous trouvons finalement que la solution générale de $2x'' - y'- y = 100xe^{2x}$ est:
        \[y\left(x\right) = C_1e^{x} + C_2e^{\frac{-x}{2}} + \left(20x - 28\right)e^{2x}, \mathspace C_1, C_2 \in \mathbb{R}, x \in \mathbb{R}\]
    }

    \subparag{Méthode de la variation de la constante}{
        Nous pouvons trouver le même résultat en utilisant la méthode de la variation de la constante.

        Nous savons que $v_1\left(x\right) = e^{x}$ et $v_2\left(x\right) = e^{-\frac{x}{2}}$ sont deux solutions linéairement indépendantes de l'équation homogène. Calculons leur Wronskien:
        \[W\left[v_1, v_2\right]\left(x\right) = \det\begin{pmatrix} e^{x} & e^{-\frac{x}{2}} \\ e^{x} & -\frac{1}{2}e^{-\frac{x}{2}} \end{pmatrix} = -\frac{1}{2}e^{\frac{x}{2}} - e^{\frac{x}{2}} = -\frac{3}{2} e^{\frac{x}{2}} \]

        Nous devons maintenant calculer $c_1\left(x\right)$ et $c_2\left(x\right)$:
        \[c_1\left(x\right) = - \int \frac{f\left(x\right) v_2\left(x\right)}{W\left[v_1, v_2\right]} dx = - \int \frac{50x e^{2x} e^{-\frac{x}{2}}}{- \frac{3}{2} e^{\frac{x}{2}}} dx = \ldots = \frac{100}{3} \left(x - 1\right)e^{x}\]
        \[c_2\left(x\right) = \int \frac{f\left(x\right) v_1\left(x\right)}{W\left[v_1, v_2\right]\left(x\right)}dx = \int \frac{50x e^{2x} e^{x}}{-\frac{3}{2} e^{\frac{x}{2}}}dx = \ldots = -\frac{40}{3} xe^{\frac{5}{2}x} + \frac{16}{3} e^{\frac{5}{2}}x\]

        Ainsi, on obtient:
        \[v_0\left(x\right) = c_1\left(x\right)v_1\left(x\right) + c_2\left(x\right) v_2\left(x\right) = \ldots = 20x e^{2x} - 28e^{2x}\]
        comme attendu.
    }
}

\parag{Note personnelle: Résumé}{
    \subparag{EDVS}{
        Une équation de la forme $f\left(y\right)y' = g\left(x\right)$ se résout en séparant les variables:
        \[f\left(y\right) \frac{dy}{dx} = g\left(x\right) \implies f\left(y\right) dy = g\left(x\right) dx \implies \int f\left(y\right) dy = \int g\left(x\right)dx\]
    }

    \subparag{EDL1 homogène}{
        La solution générale d'une EDL1 homogène $y'\left(x\right) + p\left(x\right) y\left(x\right) = 0$ est donnée par:
        \[y\left(x\right) = Ce^{-P\left(x\right)}, \mathspace C \in\mathbb{R}\]
        où $P\left(x\right)$ est une primitive (sans la constante) de $p\left(x\right)$.
    }

    \subparag{EDL1 complète}{
        Pour résoudre une EDL1 complète, $y'\left(x\right) + p\left(x\right)y\left(x\right) = f\left(x\right)$, nous utilisons plusieurs principes. La méthode de la variation de la constante nous donne une solution particulière:
        \[v\left(x\right) = c\left(x\right)e^{-P\left(x\right)}, \mathspace \text{avec } c\left(x\right) = \int f\left(x\right) e^{P\left(x\right)}\]
        où $P\left(x\right)$ est une primitive (sans la constante) de $p\left(x\right)$.

        Ensuite, en trouvant $v_0\left(x\right)$, la solution générale à l'équation homogène associée, nous utilisons le principe de superposition des solutions pour trouver la solution générale à notre équation:
        \[y\left(x\right) = v\left(x\right) + v_0\left(x\right)\]
    }

    \subparag{EDL2 homogène à coefficients constants}{
        Pour résoudre une EDL2 homogène à coefficients constants, $y''\left(x\right) + py\left(x\right) + qy\left(x\right) = 0$ où $p, q \in \mathbb{R}$, nous cherchons les racines $a, b \in \mathbb{C}$ du polynôme caractéristique:
        \[\lambda^2 + p\lambda + q = 0\]

        \begin{functionbypart}{y\left(x\right)}
        C_1 e^{ax} + C_2 e^{bx}, \mathspace \text{ si } a, b \in \mathbb{R}, a \neq b \\
        C_1 e^{ax} + C_2 xe^{ax}, \mathspace \text{ si } a = b \\
        C_1 e^{\alpha x} \cos\left(\beta x\right) + C_2 e^{\alpha x} \sin\left(\beta x\right), \mathspace \text{ si } a = \alpha + i\beta = \bar{b} \not \in \mathbb{R}
        \end{functionbypart}
    }

    \subparag{EDL2 homogène}{
        Pour résoudre une EDL2 homogène, $y''\left(x\right) + p\left(x\right)y'\left(x\right) + q\left(x\right)y\left(x\right) = 0$, nous avons besoin de deviner une solution particulière $v_1\left(x\right)$. De là, nous pouvons calculer une deuxième solution linéairement indépendante:
        \[v_2\left(x\right) = c\left(x\right) v_1\left(x\right), \mathspace \text{avec } c\left(x\right) = \int \frac{e^{-P\left(x\right)}}{v_1^2\left(x\right)}dx\]
        où $P\left(x\right)$ est une primitive (sans la constante) de $p\left(x\right)$.

        Finalement, la solution générale est donnée par:
        \[y\left(x\right) = C_1 v_1\left(x\right) + C_2 v_2\left(x\right), \mathspace \forall C_1, C_2 \in \mathbb{R}\]
    }

    \subparag{EDL2 à coefficients constants}{
        Pour résoudre une EDL2 à coefficients constants, $y''\left(x\right) + py'\left(x\right) + qy\left(x\right) = f\left(x\right)$ où $p, q \in \mathbb{R}$, nous pouvons parfois aller plus vite que la méthode de la variation de la constante en utilisant la méthode des coefficients indéterminés, si $f\left(x\right)$ a une forme particulière.
    }

    \subparag{EDL2 complète}{
        Pour résoudre une EDL2 complète, $y''\left(x\right) + p\left(x\right)y'\left(x\right) + q\left(x\right)y\left(x\right) = f\left(x\right)$, nous cherchons d'abord deux solutions particulières linéairement indépendantes à l'EDL2 homogène associée $v_1\left(x\right)$ et $v_2\left(x\right)$ (qui nous donnent donc aussi la solution générale à cette EDL2 homogène associée). De là, nous obtenons une solution particulière à notre EDL2 complète à l'aide de la méthode de la variation de la constante:
        \[v\left(x\right) = c_1\left(x\right) v_1\left(x\right) + c_2\left(x\right) v_2\left(x\right)\]

        Avec:
        \[c_1\left(x\right) = - \int \frac{f\left(x\right)v_2\left(x\right)}{W\left[v_1, v_2\right]\left(x\right)}dx\]
        \[c_2\left(x\right) = \int \frac{f\left(x\right) v_1\left(x\right)}{W\left[v_1, v_2\right]\left(x\right)}dx\]
        \[W\left[v_1, v_2\right]\left(x\right) = \det\begin{pmatrix} v_1\left(x\right) & v_2\left(x\right) \\ v_1'\left(x\right) & v_2'\left(x\right) \end{pmatrix} \]

        De là, nous pouvons trouver notre solution générale:
        \[y\left(x\right) = C_1 v_1\left(x\right) + C_2 v_2\left(x\right) + v\left(x\right), \mathspace \forall C_1, C_2 \in\mathbb{R}\]
    }
}


\end{document}
