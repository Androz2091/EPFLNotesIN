\documentclass[a4paper]{article}

% Expanded on 2022-03-15 at 13:45:36.

\usepackage{../../style}

\title{Analyse 2}
\author{Joachim Favre}
\date{Mercredi 16 mars 2022}

\begin{document}
\maketitle

\lecture{8}{2022-03-16}{J'ai failli ajouter un d à ce nom de théorème}{
\begin{itemize}[left=0pt]
    \item Définition d'adhérence et de frontière.
    \item Définition des suites dans $\mathbb{R}^n$, de leur convergence, et du concept de suite bornée. 
    \item Explication du théorème de Bolzano-Weierstrass.
    \item Explication et preuve du théorème qui fait un lien entre les suites dans $\mathbb{R}^n$ et la topologie.
    \item Définition d'ensemble borné, d'ensemble compact, et de recouvrement.
    \item Explication du théorème de Heine-Borel-Lebesgue.
\end{itemize}
}

\parag{Définition: Adhérence}{
    Soit $E \subset \mathbb{R}^n$ un sous-ensemble non-vide.

    L'intersection de tous les sous-ensembles fermés contenant $E$ est appelée \important{l'adhérence} de $E$. $\bar{E}$ est la notation de l'adhérence de $E$ dans $\mathbb{R}^n$.

    \subparag{Remarque}{
        On voit que si $E \subset \mathbb{R}^n$ est fermé, alors on a $E = \bar{E}$ par définition.
    }

    \subparag{Note personnelle: Intuition}{
        Nous pouvons considérer $\mathbb{R}^1 = \mathbb{R}$.

        Prenons par exemple $E = \left]1, 2\right] \cup \left\{6\right\}$. Alors, son adhérence est donnée par: 
        \[\bar{E} = \left[1, 2\right] \cup \left\{6\right\}\]
        
        L'adhérence est donnée par l'ensemble en union avec sa frontière (concept qu'on va définir juste après).
    }
}

\parag{Définition: Frontière}{
    Soit un ensemble $E \subset \mathbb{R}^n$ non vide, où $E \neq \mathbb{R}^n$. 

    Un point $\bvec{x} \in \mathbb{R}^n$ est un \important{point de frontière} de $E$ si toute boule ouverte de centre $\bvec{x}$ contient au moins un point de $E$ et au moins un point de $CE$.

    L'ensemble des points de frontière de $E$ s'appelle la \important{frontière} de $E$, notée $\partial E$.

    Nous définissons aussi les deux frontières suivantes: 
    \[\partial \o = \o, \mathspace \partial \mathbb{R}^n = \o\]
    
}

\parag{Exemple}{
    Prenons le sous-ensemble suivant: 
    \[E = \left\{\bvec{x} \in \mathbb{R}^n \telque x_i > 0, i = 1, \ldots, n\right\}\]
    
    C'est un sous-ensemble ouvert, comme nous l'avons déjà montré.

    Pour trouver l'adhérence, nous pouvons prendre le plus petit sous-ensemble fermé de $E$: 
    \[\bar{E} = \left\{\bvec{x} \in \mathbb{R}^n \telque x_i \geq 0, i = 1, \ldots n\right\}\]
    
    Nous pouvons aussi trouver la frontière: 
    \[\partial E = \left\{\bvec{x} \in \mathbb{R}^n \telque \exists i\ x_i \neq 0, \forall j \neq i\ x_j \geq 0\right\}\]
}

\parag{Propriétés}{
    Soit $E \subset \mathbb{R}^n$ non-vide. Alors, nous avons les propriétés suivantes:
    \begin{enumerate}
        \item $\mathring{E} \cap \partial E = \o$
        \item $\mathring{E} \cup \partial E = \bar{E}$
        \item $\partial E = \bar{E} \setminus \mathring{E}$
    \end{enumerate}
}

\subsection{Suites dans $\mathbb{R}^n$}
\parag{Définition: Suite dans $\mathbb{R}^n$}{
    Une suite d'éléments de $\mathbb{R}^n$ est une application $f: \mathbb{N} \mapsto \mathbb{R}^n$: 
    \[f: k \mapsto \bvec{x_k} = \begin{pmatrix} x_{1k} & \cdots & x_{nk} \end{pmatrix} \in \mathbb{R}^n\]
    
    $\left\{\bvec{x_k}\right\}_{k=0}^{\infty}$ est une suite d'éléments de $\mathbb{R}^n$.
}

\parag{Définition: Convergence}{
    Une suite $\left\{x_k\right\}_{k=0}^{\infty}$ est \important{convergente} et admet pour \important{limite} $\bvec{x} \in \mathbb{R}^n$ si pour tout $\epsilon > 0$, il existe $k_0 \in \mathbb{N}$ tel que $\forall k \geq k_0$, nous avons: 
    \[\left\|\bvec{x_k} - \bvec{x}\right\| \leq \epsilon\]
    
    Nous notons: 
    \[\lim_{k \to \infty} \bvec{x_k} = \bvec{x}\]

    \subparag{Remarque}{
        Notez que $\left\|\bvec{x_k} - \bvec{x}\right\| \leq \epsilon$ est équivalent à $\bvec{x_k} \in \bar{B\left(\bvec{x}, \epsilon\right)}$.
    }
}

\parag{Proposition}{
    Soit $\bvec{x} = \begin{pmatrix} x_1 & \cdots & x_n \end{pmatrix} \in \mathbb{R}^n$. Alors:
    \[\lim_{k \to \infty} \bvec{x_k} = \bvec{x} \iff \lim_{k \to \infty} x_{j, k} = x_j, \mathspace \forall j = 1, \ldots, n\]


    \subparag{Preuve}{
        En effet, nous savons que: 
        \[\epsilon \geq \left\|\bvec{x_k} - \bvec{x}\right\| = \left(\sum_{j=1}^{n} \left(x_{j, k} - x_{j}\right)^2\right)^{\frac{1}{2}}\]

        Puisque c'est une somme, de termes positifs, nous avons $\left(x_{j, k} - x_{j}\right)^2 \leq \epsilon_j \leq \epsilon$. Ainsi, nous pouvons rendre $x_{j,k}$ et $x_j$ arbitrairement proches en augmentant $k$, ce qui est la définition de la limite.

        \qed
    }
}


\parag{Définition: Suite bornée}{
    Une suite $\left\{\bvec{x_k}\right\}$ est \important{bornée} s'il existe un $M > 0$ tel que $\left\{\bvec{x_k}\right\}$ est contenue dans la boule fermée $\bar{B\left(\bvec{0}, M\right)}$.
}


\parag{Propriétés}{
    \begin{enumerate}[left=0pt]
        \item La liste d'une suite $\left\{\bvec{x_k}\right\}$, si elle existe, est unique.
        \item Toute suite convergente $\left\{\bvec{x_k}\right\}$ est bornée.
    \end{enumerate}
    
    \subparag{Justification de la propriété 2}{
        Par définition, si une suite est convergente, alors $\lim_{k \to \infty} \bvec{x_k} = \bvec{x}$. Cela implique donc que, pour $\epsilon = 1$ par exemple, il existe $k_0 \in \mathbb{N}$ tel que pour tout $k \geq k_0$, nous avons $\left\|\bvec{x_k} - \bvec{x}\right\| \leq\epsilon = 1$, donc que pour $k \geq k_0$, $x_k \in \bar{B\left(\bvec{x}, \epsilon\right)}$. Puisqu'il y a un un nombre fini de points $\bvec{x_0}, \ldots, \bvec{x_{k_{0} - 1}}$. En prenant en compte la distance maximale à l'origine de cette boule, et la distance maximale de ces points (qui existe puisqu'il y en a un nombre finis), nous avons trouvé un $M > 0$ tel que $\bvec{x_k} \in \bar{B\left(\bvec{0}, M\right)}$.

        \imagehere[0.7]{SuiteConvergenteEstBorneeDansRn.png}
    }
}

\parag{Théorème de Bolzano-Weierstrass}{
    Nous pouvons extraire une sous-suite convergente de toute suite bornée $\left\{\bvec{x_k}\right\} \subset \mathbb{R}^n$.
}

\parag{Théorème: Lien entre les suites dans $\mathbb{R}^n$ et la topologie}{
    Un sous-ensemble non-vide $E \subset \mathbb{R}^n$ est fermé si et seulement si toute suite $\left\{\bvec{x_k}\right\} \subset E$ d'éléments de $E$ qui converge a pour limite un élément de $E$.

    \demonstrationaconnaitre

    \subparag{Preuve $\implies$}{
        Nous supposons que $E \subset \mathbb{R}^n$ est fermé.

        Supposons par l'absurde qu'il existe une suite $\left\{\bvec{x_k}\right\} \subset E$ d'éléments de $E$ qui converge et qui a pour limite $\bvec{x} \not\in E$. Ainsi, on sait que $\bvec{x} \in CE$, qui est un ensemble ouvert dans $\mathbb{R}^n$ (puisque $E$ est fermé par hypothèse). Puisque cet ensemble ouvert nous savons, par définition, que $\exists \delta > 0$ tel que: 
        \[B\left(\bvec{x}, \delta\right) \subset CE\]
      
        Or, cela implique que: 
        \[\underbrace{\left\{\bvec{x_k}\ \forall k \in \mathbb{N}\right\}}_{\subset E} \cap \underbrace{B\left(\bvec{x}, \delta\right)}_{\subset CE} = \o\]

        En d'autres mots, aucun élément de la suite ne fait partie de cette boule ouverte.

        De l'autre côté, puisque $\lim_{k \to \infty} \bvec{x_k}$, nous avons qu'il existe un $k_0 \in \mathbb{N}$ tel que pour tout $k \geq k_0$: 
        \[\bvec{x_k} \in \bar{B\left(\bvec{x}, \frac{\delta}{2}\right)} \subset B\left(\bvec{x}, \delta\right)\]

        En d'autres mots, pour $k \geq k_0$, $\bvec{x_k}$ fait partie de notre boule ouverte. Ceci entre en contradiction avec ce que nous avions vu ci-dessus.

        \qed
    }
    
    \subparag{Preuve $\impliedby$}{
        Nous allons démontrer notre proposition par la contraposée: nous voulons montrer que si $E \subset \mathbb{R}^n$ n'est pas fermé, alors il existe une suite $\left\{\bvec{x_k} \subset E\right\}$ d'éléments de $E$ qui converge et qui a pour un limite un élément qui n'est pas dans $E$.

        Puisque nous savons que $E$ n'est pas fermé, nous savons que $CE$ n'est pas ouvert. Ainsi, $\exists \bvec{y} \in CE$ tel que, pour tout $\epsilon > 0$:  
        \[B\left(\bvec{y}, \epsilon\right) \cap E \neq \o\]
        
        Plus précisément, on peut prendre $\epsilon = \frac{1}{k}$, ce qui nous donne:
        \[\forall k \in \mathbb{N}_+, \ B\left(\bvec{y}, \frac{1}{k}\right)\]
        
        Ceci implique que, pour tout $k$, on sait qu'il existe un $\bvec{y_k}$ tel que $\bvec{y_k} \in B\left(\bvec{y}, \frac{1}{k}\right)$ et $\bvec{y_k} \in E$. Ceci nous donne une suite $\left\{\bvec{y_k}\right\}_{k \in \mathbb{N}_+} \subset E$ telle que $\lim_{k \to \infty} \bvec{y_k} = \bvec{y} \in CE$, et donc $\bvec{y} \not\in E$.

        \qed
    }
}

\parag{Remarque}{
    Pour construire l'adhérence $\bar{E}$ d'un sous-ensemble non-vide $E \subset \mathbb{R}^n$, il faut et il suffit d'ajouter les limites de toutes suites convergentes d'éléments de $E$.

    \subparag{Exemple}{
        Dans $\mathbb{R}^1 = \mathbb{R}$, si on prend toutes les suites d'éléments de $\left[0, 1\right[ $, il y a des suites qui convergent vers toutes les valeurs entre 0 et 1, compris.
    }
}

\parag{Définition: Ensemble borné}{
    Un ensemble $E \subset \mathbb{R}^n$ est \important{borné} s'il existe un $M > 0$ tel que: 
    \[E \subset \bar{B\left(\bvec{0}, M\right)}\]
    
}


\parag{Définition: Ensemble compact}{
    Un sous-ensemble non-vide $E \subset \mathbb{R}^n$ est \important{compact} s'il est fermé et borné.
}

\parag{Exemple 1}{
    Prenons une boule fermée quelconque: 
    \[\bar{B\left(\bvec{x}, \delta\right)} = \left\{\bvec{y} \in \mathbb{R}^n \telque \left\|\bvec{x} - \bvec{y}\right\| \leq \delta\right\}\]
    
    Nous savons qu'elle est bornée car: 
    \[\bar{B\left(\bvec{x} , \delta\right)} \subset \bar{B\left(\bvec{0}, \left\|\bvec{x}\right\| + \delta\right)}\]
    
    On peut donc en déduire qu'une boule fermée est compacte.
}

\parag{Exemple 2}{
    Prenons l'ensemble suivant, où $n \geq 2$: 
    \[E = \left\{\bvec{x} \in \mathbb{R}^n \telque x_1 = 0\right\}\]

    Il est clairement fermé, on peut prendre n'importe quelle boule centrée sur la droite définie par $CE$ ($x_1 = 0$), il est impossible de trouver un $\delta$ qui fait qu'elle est entièrement inclue dans $CE$.

    Cependant, nous pouvons aussi voir que $E$ n'est pas borné. Prenons la suite suivante: 
    \[\left\{\bvec{a_k} = \left(0, k, 0, 0, \cdots\right)\right\}_{k \in \mathbb{N}} \subset E\]
    
    Les normes forment la suite $\left\|\bvec{a_k}\right\| = k \in \mathbb{N}$. La distance entre l'origine et les éléments de $\left(\bvec{a_k}\right)$ n'est donc pas bornée, ce qui nous permet d'en déduire que $E$ n'est pas borné. Ainsi, $E$ n'est pas compact.
}

\parag{Définition: Recouvrement}{
    Un \important{recouvrement} d'un ensemble $E$ est défini par: 
    \[E \subset \bigcup_{i \in I} A_i, \mathspace A_i \subset \mathbb{R}^n \text{ ouverts}\]

    Notez que $I$ peut être innombrable.

    \svghere[0.5]{DefinitionRecouvrement.svg}
}


\parag{Théorème de Heine-Borel-Lebesgue}{
    Un sous-ensemble non-vide $E \subset \mathbb{R}^n$ est compact si et seulement si de \textit{tout} recouvrement de $E$ par des sous-ensembles ouverts dans $\mathbb{R}^n$ on peut extraire une famille finie d'ensembles qui forment un recouvrement de $E$.
}

\parag{Exemple 1.1}{
    Une droite dans $\mathbb{R}^n$, où $n \geq 2$, est fermée mais pas bornée, et donc pas compacte.

    Prenons une origine quelconque sur la droite, et numérotons là comme la droite des réelles. Alors, nous pouvons prendre le recouvrement suivant: 
    \[E \subset \bigcup_{n \in \mathbb{Z}} B\left(n, \frac{2}{3}\right)\]

    \imagehere[1]{HeineBorelLebesgueNonExample1.png}
    
    On ne peut clairement pas choisir de sous-recouvrement fini qui recouvre $E$. Si on jette une-seule boule, ce qui reste ne couvrira pas notre ensemble.
}

\parag{Exemple 1.2}{
    Un intervalle ouvert dans $\mathbb{R}^1 = \mathbb{R}$, $E = \left]0, 1\right[ \subset \mathbb{R}$ n'est pas compact. Il est bien borné, mais pas fermé.

    Prenons le recouvrement suivant: 
    \[E \subset \bigcup_{i \in \mathbb{N}_+} \left]0, \frac{i}{i+1}\right[ \]
    
    \imagehere{HeineBorelLebesgueNonExample2.png}

    Il est aussi impossible de choisir un sous-recouvrement fini. En effet, avec un nombre fini d'intervalles, alors il existe un $i$ qui est le plus grand, et donc nous n'atteignons pas les valeurs entre $\frac{i}{i+1}$ et $1$.
}

\parag{Exemple 2.1}{
    Prenons l'ensemble suivant: 
    \[A = \left\{\left(x, y\right) \in \mathbb{R}^2 \telque 1 > \sin\left(x + y\right) \geq -2\right\}\]

    Nous remarquons que $\sin\left(x + y\right) \geq -2$ tient toujours, $\forall \left(x, y\right) \in \mathbb{R}^2$. Pour l'autre condition, nous avons besoin de: 
    \[\sin\left(x + y\right) \neq 1 \iff x + y \neq \frac{\pi}{2} + 2k\pi \iff y \neq -x + \frac{\pi}{2} + 2k\pi\]
    
    On obtient donc que: 
    \[CA = \left\{\left(x, y\right) \in \mathbb{R}^2 \telque y = -x + \frac{\pi}{2} + 2k\pi\right\}\]
    
    Or, c'est une union de droites, donc il est clairement fermé. On obtient ainsi que $A$ est ouvert.

}

\parag{Exemple 2.2}{
    Prenons l'ensemble suivant: 
    \[B = \left\{\left(x, y\right)\in \mathbb{R}^2 \telque \sqrt{xy} < 2\right\}\]
    
    Pour que $\sqrt{xy}$ existe, nous avons besoin de $xy \geq 0$. De plus, nous avons aussi la condition $\sqrt{xy} < 2 \implies xy < 4$.

    Nous avons donc deux cas. Si $x > 0$, alors $y \geq 0$ et: 
    \[xy < 4 \iff y < \frac{4}{x}\]
    
    Si $x < 0$, alors $y \leq 0$ et: 
    \[xy < 4 \iff y > \frac{4}{x}\]
    
    Si $x = 0$, alors $y$ est arbitraire.

    \imagehere[0.5]{TopologieExempleOuvertsFermes.png}

    Les axes sont inclus dans notre sous-ensembles, mais pas les courbes. Ainsi, $B$ n'est ni ouvert ni fermé.
}




\end{document}
