\documentclass[a4paper]{article}

% Expanded on 2022-04-04 at 10:17:44.

\usepackage{../../style}

\title{Analyse 2}
\author{Joachim Favre}
\date{Lundi 04 avril 2022}

\begin{document}
\maketitle

\lecture{13}{2022-04-04}{On monte en ordres}{
\begin{itemize}[left=0pt]
    \item Explication du théorème 2 sur la dérivabilité.
    \item Définition des dérivées partielles d'ordres supérieurs, et du concept de classe pour une fonction.
    \item Explication du théorème de Schwarz.
    \item Définition de la matrice Hessienne.
    \item Grand résumé, suivit d'un grand nombre d'exemples.
\end{itemize}
}

\parag{Rappel}{
    Nous avons vu que $f$ est dérivable en $\bvec{a}$ implique que les dérivées directionnelles existent. De plus, si les dérivées directionnelles existent dans toutes les direction, alors les dérivées partielles existent. Notez que les deux réciproques sont fausses en général, l'existence des dérivées directionnelles n'implique pas la dérivabilité de la fonction, et l'existence des dérivées partielles n'implique pas l'existence des dérivées directionnelles.

    Cependant, nous avons le théorème qui suit.
}

\parag{Théorème 2 sur la dérivabilité}{
    Soit $E \subset \mathbb{R}^n$ un ensemble ouvert, $f : E \mapsto \mathbb{R}$ et un point $\bvec{a} \in E$.

    Supposons qu'il existe $\delta > 0$ tel que toutes les dérivées partielles $\frac{\partial f}{\partial x_k}\left(\bvec{a}\right)$ existent sur $B\left(\bvec{a}, \delta\right)$ et sont continues en $\bvec{a}$. Alors, $f$ est dérivable en $\bvec{a}\in E$.
}

\subsection{Dérivées partielles d'ordre supérieur}
\parag{Définition: Fonction dérivée partielle}{
    Soit $f: E \mapsto \mathbb{R}$, où $E \subset \mathbb{R}^n$ est ouvert, une fonction telle que $\frac{\partial f}{\partial x_k}$ existe pour un $k$, avec $1 \leq k \leq n$, en tout point de $E$. Alors $\frac{\partial f}{\partial x_k}\left(\bvec{x}\right)$ où $\bvec{x} \in E$, est la \important{fonction $k$-ème dérivée partielle}.
}

\parag{Définition: Dérivée partielle d'ordre supérieur}{
    Soit $f: E \mapsto \mathbb{R}$ une fonction telle que $\frac{\partial f}{\partial x_k}$ existe en tout $\bvec{x} \in E$. Si la fonction $\frac{\partial f}{\partial x_k}$ admet à son tour une dérivée partielle par rapport à $x_i$ (potentiellement une autre variable), on pose: 
    \[\frac{\partial }{\partial x_i} \left(\frac{\partial f}{\partial x_k}\right) \over{=}{déf} \frac{\partial^2 f}{\partial x_i \partial x_k}\]

    Nous appelons ceci la \important{dérivée partielle seconde}. Nous pouvons définir ainsi, lorsqu'elles existent, les dérivées partielles d'ordre $p$. Par exemple: 
    \[\frac{\partial}{\partial x_j} \left(\frac{\partial}{\partial x_i} \left(\frac{\partial f}{\partial x_k}\right)\right) = \frac{\partial^3 f}{\partial x_j \partial x_i \partial x_k}\]

    \subparag{Remarque}{
        Notez que la dérivée partielle qui se calcule en premier est celle de droite. Ceci est très cohérent avec la plupart des opérateurs, notamment les matrices et les application linéaires. Cependant, dans le livre de référence Douchet-Swahlen, l'ordre des dérivées partielles est échangé (et c'est probable que ce soit le seul livre qui utilise cette convention). 
    }
}

\parag{Définition: Classe}{
    Soit $E \subset \mathbb{R}^n$ un ensemble ouvert.

    Une fonction $f: E \mapsto \mathbb{R}$ est dite de \important{classe} $C^p$ sur $E$, si toutes les dérivées partielles d'ordre $\leq p$ existent et sont continues sur $E$.
    
    \subparag{Remarque 1}{
        $C^{\infty}$ veut dire que la fonction est de classe $p$ pour tout $p \in \mathbb{N}$.
    }

    \subparag{Remarque 2}{
       Le théorème 2 sur la dérivabilité nous dit que, si $f$ est de classe $C^1$ sur $E$, alors $f$ est dérivable sur $E$.
    }
}


\parag{Théorème de Schwarz}{
    Soit $f : E \mapsto \mathbb{R}$ et $\bvec{a} \in E$ tel que les dérivées partielles secondes $\frac{\partial^2 f}{\partial x_i \partial x_j}$ et $\frac{\partial^2 f}{\partial x_j \partial x_i}$ existent dans un voisinage de $\bvec{a}$ et sont continues en $\bvec{a}$ (en d'autres mots, $f$ est de classe $C^2$ sur un ensemble ouvert contenant $\bvec{a}$).

    Alors, nous avons: 
    \[\frac{\partial^2 f}{\partial x_i \partial x_j}\left(\bvec{a}\right) = \frac{\partial^2 f}{\partial x_j \partial x_i}\left(\bvec{a}\right)\]

    \subparag{Remarque}{
        De manière générale, on peut démontrer que si $f$ est de classe $C^{p}$ sur $E$, alors nous pouvons échanger l'ordre des dérivées partielles jusqu'à l'ordre $p$.
    }
}

\parag{Définition: Matrice Hessienne}{
    La \important{matrice Hessienne} est la matrice des dérivées partielles d'ordre 2 pour une fonction $E \mapsto \mathbb{R}^n$, où $E \subset \mathbb{R}^n$ est un sous-ensemble ouvert, notée: 
    \[\Hess\left(f\right)\left(\bvec{a}\right) = \begin{pmatrix} \dfrac{\partial^2 f}{\partial x_1^2}\left(\bvec{a}\right) & \dfrac{\partial^2 f}{\partial x_2 x_1}\left(\bvec{a}\right) & \cdots & \dfrac{\partial^2 f}{\partial x_n \partial x_1}\left(\bvec{a}\right) \\ \dfrac{\partial^2 f}{\partial x_1 \partial x_2}\left(\bvec{a}\right) & \dfrac{\partial^2 f}{\partial x_2^2}\left(\bvec{a}\right) & \cdots & \dfrac{\partial^2 f}{\partial x_n \partial x_2}\left(\bvec{a}\right) \\ \vdots & \vdots & \ddots  & \vdots \\ \dfrac{\partial^2 f}{\partial x_1 \partial x_n}\left(\bvec{a}\right) & \dfrac{\partial^2 f}{\partial x_2 \partial x_n}\left(\bvec{a}\right) & \cdots & \dfrac{\partial^2 f}{\partial x_n^2}\left(\bvec{a}\right) \end{pmatrix} \]
    
    \subparag{Remarque}{
        Si $f$ est de classe $C^2$ sur $E$, alors la matrice Hessienne est symétrique, c'est-à-dire: 
        \[\Hess\left(f\right)\left(\bvec{a}\right) = \Hess\left(f\right)\left(\bvec{a}\right)^T\]
    }
}

\parag{Exemple}{
    Soit la fonction définie sur $f : \mathbb{R}^2 \mapsto \mathbb{R}$ suivante: 
    \[f\left(x, y\right) = \sin\left(xy\right)\]

    Nous allons utiliser cet exemple afin d'illustrer le théorème de Schwarz, et les matrices Hessiennes.
    
    \subparag{Dérivées partielles premières}{
        Nous pouvons calculer les dérivées premières:
        \[\frac{\partial f}{\partial x} = y\cos\left(xy\right), \mathspace \frac{\partial f}{\partial y} = x\cos\left(xy\right)\]
    }
    
    \subparag{Dérivées partielles secondes}{
        Calculons maintenant les dérivées partielles secondes mixtes: 
        \[\frac{\partial^2 f}{\partial y \partial x} = \frac{\partial}{\partial y}\left(y \cos\left(xy\right)\right) = \cos\left(xy\right) - xy\sin\left(xy\right)\] 
        \[\frac{\partial^2 f}{\partial x \partial y} = \frac{\partial}{\partial x} \left(x \cos\left(xy\right)\right) = \cos\left(xy\right) - xy \sin\left(xy\right)\]
        
        Nous remarquons qu'ici elles sont égales. Nous pouvons aussi calculer les autres dérivées partielles: 
        \[\frac{\partial^2 f}{\partial x^2} = \frac{\partial}{\partial x} \left(\frac{\partial f}{\partial x}\right) = \frac{\partial}{\partial x}\left(y \cos\left(xy\right)\right) = -y^2 \sin\left(xy\right)\]
        \[\frac{\partial^2 f}{\partial y^2} = \frac{\partial}{\partial y}\left(\frac{\partial f}{\partial y}\right) = \frac{\partial}{\partial y}\left(x \cos\left(xy\right)\right) = -x^2 \sin\left(xy\right)\]
    }

    \subparag{Dérivées partielles d'ordre 3}{
        Nous pouvons maintenant aussi calculer les dérivées d'ordre 3 pour cette fonction. Par exemple: 
        \[\frac{\partial^3 f}{\partial x \partial y \partial x} = \frac{\partial}{x} \left(\frac{\partial^2 f}{\partial y \partial x}\right) = -2y\sin\left(xy\right) - xy^2 \cos\left(xy\right)\]
        \[\frac{\partial^3 f}{\partial x \partial x \partial y} = \frac{\partial}{\partial x} \left(\frac{\partial^2 f}{\partial x \partial y}\right) = \frac{\partial}{\partial x} \left(\frac{\partial^2 f}{\partial y \partial x}\right) = \frac{\partial^3 f}{\partial x \partial y \partial x}\]
        \[\frac{\partial^3 f}{\partial y \partial x^2} = \frac{\partial}{\partial y}\left(\frac{\partial^2 f}{\partial x^2}\right) = -2y\sin\left(xy\right) - xy^2 \cos\left(xy\right)\]

        On remarque que nous avons: 
        \[\frac{\partial^3 f}{\partial x^2 \partial y} = \frac{\partial^3 f}{\partial x \partial y \partial x} = \frac{\partial^3 f}{\partial y \partial x^2}, \mathspace \forall \left(x, y\right) \in \mathbb{R}^2\]
        
        Qui sont différentes de: 
        \[\frac{\partial^3 f}{\partial y^2 \partial x} = \frac{\partial^3 f}{\partial y \partial x \partial y} = \frac{\partial^3 f}{\partial x \partial y^2}, \mathspace \forall \left(x, y\right) \in \mathbb{R}^2\]
        
        Nous pouvons calculer les dérivées partielles d'ordre $p$, où $p \in \mathbb{N}^*$, pour $f\left(x, y\right) = \sin\left(xy\right)$. Elle existent et sont continues sur $\mathbb{R}^2$, ainsi l'ordre de dérivation ne fait pas de différence pour cette fonction.
    }
    
    \subparag{Matrice Hessienne}{
        Finalement, nous pouvons utiliser ce que nous avons calculé pour construire notre matrice Hessienne:
        \begin{multiequality}
        \Hess\left(f\right)\left(x, y\right) =\ & \begin{pmatrix} \frac{\partial^2 f}{\partial x^2} & \frac{\partial^2 f}{\partial y \partial x} \\ \frac{\partial^2 f}{\partial x \partial y} & \frac{\partial^2 f}{\partial y^2} \end{pmatrix} \\
        =\ & \begin{pmatrix} -y^2 \sin\left(xy\right) & \cos\left(xy\right) - xy\sin\left(xy\right) \\ \cos\left(xy\right) - xy\sin\left(xy\right) & -x^2 \sin\left(xy\right) \end{pmatrix} 
        \end{multiequality}
    }
}


\parag{Résumé}{
    Nous avons vu beaucoup de théorie, de laquelle nous pouvons faire le schéma suivant. Nous allons voir un certain nombre de contre-exemples sur les réciproques de nos propositions, ainsi elles sont déjà écrites sur le schéma pour plus de clarté.

    Soit $f: E \mapsto \mathbb{R}$, où $E$ est ouvert, alors:
    \svghere[0.5]{DiagrammeDerivabilite.svg}
}

\parag{Exemple 1}{
    Soit la fonction suivante:
    \begin{functionbypart}{f\left(x, y\right)}
        \frac{xy}{x^2 + y^2}, \mathspace \left(x, y\right) \neq \left(0, 0\right) \\
        0, \mathspace \left(x, y\right) = \left(0, 0\right)
    \end{functionbypart}

    Nous allons montrer que toutes les dérivées partielles de cette fonction existent, mais qu'elle n'est pas continue et que les dérivées directionnelles n'existent pas en $\left(0, 0\right)$.

    \subparag{Continuité}{
        On remarque que $f\left(x, y\right)$ n'est pas continue en $\left(0, 0\right)$: 
        \[\lim_{k \to \infty} f\left(\frac{1}{k}, \frac{1}{k}\right) = \lim_{k \to \infty} \frac{\frac{1}{k^2}}{\frac{1}{k^2} + \frac{1}{k^2}} = \frac{1}{2}\]
        \[\lim_{k \to \infty} f\left(0, \frac{1}{k}\right) = \lim_{k \to \infty} \frac{0 \cdot \frac{1}{k}}{\frac{1}{k}} = 0\]

        Ceci implique que la fonction n'est pas dérivable, par la contraposée de notre premier théorème. 
    }
    
    \subparag{Dérivées partielles}{
        Calculons maintenant les dérivées partielles. Si $\left(x, y\right) \neq \left(0, 0\right)$, alors nous avons: 
        \[\frac{\partial f}{\partial x} = \frac{y\left(x^2 + y^2\right) - 2x xy}{\left(x^2 + y^2\right)^2} = \frac{y^3 - x^2 y}{\left(x^2 + y^2\right)^2}\]
        
        Or, on remarque qu'elle ne peut pas être continue en $\bvec{0}$, puisque la limite suivante n'existe pas:
        \[\lim_{k \to \infty} \frac{\partial f}{\partial x}\left(0, \frac{1}{k}\right) = \lim_{k \to \infty} \frac{\frac{1}{k^3}}{\frac{1}{k^4}}\ = \lim_{k \to \infty} k\]

        Si nous voulons trouver la dérivée partielle selon $y$, alors nous pouvons utiliser la symétrie de notre fonction et échanger les $x$ et les $y$ dans $\frac{\partial f}{\partial x}$: 
        \[\frac{\partial f}{\partial y} = \frac{x^3 - y^2 x}{\left(x^2 + y^2\right)^2}\]

        De manière similaire, on trouve que cette dérivée partielle ne peut pas être continue en $\bvec{0}$.
    }
    
    \subparag{Dérivées directionnelles}{
        Calculons les dérivées directionnelles en $\left(0, 0\right)$. Ainsi, soit $\bvec{v} = \left(v_1, v_2\right) \neq \left(0, 0\right)$, cela nous donne: 
        \begin{multiequality}
        D f\left(\bvec{0}, \bvec{v}\right) =\ & \lim_{t \to 0} \frac{f\left(\bvec{0} + t \bvec{v}\right) - f\left(\bvec{0}\right)}{t}  \\
        =\ & \lim_{t \to 0} \frac{1}{t} \left(\frac{t^2 v_1 v_2}{t^2 \left(v_1^2 + v_2^2\right)} - 0\right)  \\
        =\ & \lim_{t \to 0} \frac{v_1 v_2}{t \left(v_1^2 + v_2^2\right)} 
        \end{multiequality}
        
        Cette limite n'existe pas, sauf si $v_1 = 0$ ou $v_2 = 0$, auquel cas elle est égale à 0. Ainsi, seules nos dérivées partielles existent, et elles sont égales à 0.
    }
}

\parag{Exemple 2}{
    Soit la fonction suivante:
    \begin{functionbypart}{f\left(x, y\right)}
        \frac{x^2 y^3}{x^4 + y^6}, \mathspace \left(x, y\right) \neq \left(0, 0\right) \\
        0, \mathspace \left(x, y\right) = \left(0, 0\right)
    \end{functionbypart}

    Nous allons montrer que toutes les dérivées directionnelles de cette fonction existent en $\left(0, 0\right)$, mais qu'elle n'est ni continue ni dérivable en ce point.

    \subparag{Continuité}{
        Nous remarquons qu'elle n'est pas continue en $\left(0, 0\right)$: 
        \[\lim_{k \to \infty} f\left(\frac{1}{k}, \frac{1}{k}\right) = \lim_{k \to \infty} \frac{\frac{1}{k^5}}{\frac{1}{k^4}\left(1 + \frac{1}{k^2}\right)} = 0\] 
        \[\lim_{k \to \infty} f\left(\frac{1}{k^3}, \frac{1}{k^2}\right) = \lim_{k \to \infty} \frac{\frac{1}{k^6} \cdot \frac{1}{k^6}}{\frac{1}{k^{12}} + \frac{1}{k^{12}}} = \frac{1}{2}\]

        Nous en déduisons que cette fonction n'est pas dérivable en $\left(0, 0\right)$.
    }
    
    \subparag{Dérivées directionnelles}{
         Calculons maintenant les dérivées directionnelles en $\left(0, 0\right)$. Ainsi, soit $\bvec{v} = \left(v_1, v_2\right) \neq 0$. Nous avons: 
         \begin{multiequality}
         D f\left(\bvec{0}, \bvec{v}\right) =\ & \lim_{t \to 0} \frac{f\left(\bvec{0} + t \bvec{v}\right) - f\left(\bvec{0}\right)}{t} \\
         =\ & \lim_{t \to 0} \frac{t^5 v_1^2 v_2^3}{t\left(t^4 v_1^4 + t^6 v_2^6\right)} \\
         =\ & \lim_{t \to 0} \frac{v_1^2 v_2^3}{v_1^4 + t^2 v_2^6} 
         \end{multiequality}
        
        Ce qui nous donne que:
        \begin{functionbypart}{Df\left(\bvec{0}, \bvec{v}\right)}
        \frac{v_1^2 v_2^3}{v_1^4} = \frac{v_2^3}{v_1^2}, \mathspace \text{si } v_1 \neq 0 \\
        \lim_{t \to 0} \frac{0}{0 + t^2 v_2^2} = 0, \mathspace \text{si } v_1 = 0
        \end{functionbypart}
        
        Nous en déduisons que les dérivées directionnelles existent en $\left(0, 0\right)$ pour tout $\bvec{v} \in \mathbb{R}^2$, $\bvec{v} \neq \left(0, 0\right)$.
    }

    \subparag{Dérivées partielles}{
        En particulier, nous avons: 
        \[\frac{\partial f}{\partial x}\left(\bvec{0}\right) = Df\left(\bvec{0}, \left(1, 0\right)\right) = 0\] 
        \[\frac{\partial f}{\partial y}\left(\bvec{0}\right) = Df\left(\bvec{0}, \left(0, 1\right)\right) = 0\]
        
        Ainsi, $\nabla f\left(\bvec{0}\right) = \bvec{0}$. Nous pouvons montrer que les dérivées partielles ne sont pas continues (cette propriété est nécessaire par notre deuxième théorème).
    }

    \subparag{Plan tangent}{
        Nous savons que $f$ n'est pas dérivable en $\left(0, 0\right)$, mais nous avons que $\nabla f\left(0, 0\right) = \left(0, 0\right)$. Ainsi, nous pouvons essayer d'écrire tout de même l'équation d'un plan: 
        \[z = f\left(0, 0\right) + \left<\nabla f\left(0, 0\right), \left(x, y\right)\right> = 0\]
        
        Cependant, comme nous pouvons le voir sur l'image suivante, ce plan ne fait aucun sens, ce n'est pas un plan tangent au graphique de la fonction. Cela montre que, si la fonction n'est pas dérivable, alors il n'existe pas de plan tangent.
        \imagehere[0.8]{PlanTangentFonctionNonDerivable.png}
        
        Ceci nous amène à la remarque suivante.
    }
}

\parag{Remarque}{
    Nous avions trouvé que si $f\left(x, y\right)$ est dérivable en $\left(x_0, y_0\right)$, alors le plan tangent à la surface $z = f\left(x, y\right)$ au point $\left(x_0, y_0, f\left(x_0, y_0\right)\right)$ est défini par l'équation: 
    \[z = f\left(x_0, y_0\right) + \left<\nabla f\left(x_0, y_0\right), \left(x - x_0, y - y_0\right)\right>\]
    
    Si $f$ n'est pas dérivable en ce point, alors ce plan n'est pas un plan tangent, même si le gradient $\nabla f\left(x_0, y_0\right)$ existe. Le plan tangent n'est simplement pas défini dans ce cas.
}


\end{document}
