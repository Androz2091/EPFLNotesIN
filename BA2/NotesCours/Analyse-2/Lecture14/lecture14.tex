\documentclass[a4paper]{article}

% Expanded on 2022-04-04 at 17:19:53.

\usepackage{../../style}

\title{Analyse 2}
\author{Joachim Favre}
\date{Mercredi 06 avril 2022}

\begin{document}
\maketitle

\lecture{14}{2022-04-06}{Toujours plus d'exemples}{
\begin{itemize}[left=0pt]
    \item Explication de comment démontrer qu'une fonction est dérivable, ou qu'elle ne l'est pas.
    \item Beaucoup d'exemples.
\end{itemize}

}

\parag{Démonstration de la dérivabilité}{
    Voici deux méthodes pour démontrer qu'une fonction est dérivable.

    \subparag{Méthode 1}{
        Nous savons que si toutes les dérivées partielles d'ordre 1 sont continues au point donné, alors nous savons que cela implique que $f$ est dérivable.

        Il est important de voir que le fait qu'une ou plusieurs dérivées partielles $\frac{\partial f}{\partial x_i}$ ne soient pas continues en $\bvec{a}$ n'implique pas nécessairement que $f$ n'est pas dérivable en $\bvec{a}$, comme nous le verrons dans l'exemple 5, et comme nous pouvons voir sur le schéma de résumé.
    }

    \subparag{Méthode 2}{
        Si le gradient $\nabla f\left(\bvec{a}\right)$ n'existe pas, alors nous savons que $f$ n'est pas dérivable en $\bvec{a}$. S'il existe, nous pouvons poser:
        \[r\left(\bvec{x}\right) = f\left(\bvec{x}\right) - f\left(\bvec{a}\right) - \left<\nabla f\left(\bvec{a}\right), \bvec{x} - \bvec{a}\right>\]

        Alors, si $\lim_{\bvec{x} \to \bvec{a}} \frac{r\left(\bvec{x}\right)}{\left\|\bvec{x} - \bvec{a}\right\|} = 0$, nous savons que $f$ est dérivable en $\bvec{a}$ par définition.

        De manière similaire, si $\lim_{\bvec{x} \to \bvec{a}} \frac{r\left(\bvec{x}\right)}{\left\|\bvec{x} - \bvec{a}\right\|} \neq 0$, alors $f$ n'est pas dérivable par notre premier théorème. En effet, si une fonction est dérivable, alors $L_{\bvec{a}}\cdot \bvec{v} = \left<\nabla f\left(\bvec{a}\right), \bvec{v}\right>$, et donc $r\left(\bvec{x}\right)$ et telle que donnée ci-dessus. Ainsi, si notre limite ne donne pas 0, c'est une contradiction avec le fait que la fonction soit dérivable.
    }
}

\parag{Exemple 3}{
    Soit la fonction suivante:
    \begin{functionbypart}{f\left(x, y\right)}
        \frac{xy^3}{x^2+ y^2}, \mathspace \left(x, y\right) \neq \left(0, 0\right) \\
        0, \mathspace \left(x, y\right) = \left(0, 0\right)
    \end{functionbypart}

    Nous allons montrer que cette fonction est de classe $C^1$, mais pas de classe $C^2$. De plus, les dérivées partielles secondes existent, mais nous ne pouvons pas changer l'ordre de dérivation.

    \subparag{Dérivées partielles premières}{
        Nous pouvons calculer ses dérivées partielles en $\left(0, 0\right)$:
        \[\frac{\partial f}{\partial x}\left(0, 0\right) = \lim_{t \to 0} \frac{f\left(t, 0\right) - f\left(0, 0\right)}{t} = \lim_{t \to 0} \frac{t\cdot 0}{t\left(t^2 + 0\right)} = 0\]
        \[\frac{\partial f}{\partial y}\left(0, 0\right) = \lim_{t \to 0} \frac{f\left(0, t\right) - f\left(0, 0\right)}{t} = \lim_{t \to 0} \frac{0\cdot t^3}{t\left(0 + t^2\right)} = 0\]

        Calculons aussi les dérivées partielles pour $\left(x, y\right) \neq \left(0, 0\right)$:
        \[\frac{\partial f}{\partial x} = \frac{y^3 \left(x^2 + y^2\right) - 2x\left(xy^3\right)}{\left(x^2+ y^2\right)^2} = \frac{y^5 -x^2 y^3}{\left(x^2 + y^2\right)^2}\]
        \[\frac{\partial f}{\partial y} = \frac{3 y^2 x\left(x^2 y^2\right) - 2y\left(x y^3\right)}{\left(x^2 + y^2\right)^2} = \frac{y^4 x + 3x^3 y^2}{\left(x^2 + y^2\right)^2}\]


        Nous pouvons aussi calculer leur limite:
        \[\lim_{\left(x, y\right) \to \left(0, 0\right)}  \frac{\partial f}{\partial x} = \lim_{r \to 0} \frac{r^{5} \overbrace{\left(\sin^5\left(\phi\right) - \cos^2\left(\phi\right) \sin^3\left(\phi\right)\right)}^{\text{borné}}}{r^4} = 0\]
        \[\lim_{\left(x, y\right) \to \left(0, 0\right)} \frac{\partial f}{\partial y} = \lim_{r \to 0} \frac{r^5 \overbrace{\left(\sin^4\left(\phi\right) \cos\left(\phi\right) + 3\cos^3\left(\phi\right) \sin^2\left(\phi\right)\right)}^{\text{borné}}}{r^4} = 0\]

        Ainsi, puisque les dérivées partielles existent et sont continues sur $\mathbb{R}^2$, nous savons que $f$ est dérivable sur $\mathbb{R}^2$ par notre deuxième théorème.
    }

    \subparag{Dérivées partielles secondes}{
        Calculons maintenant les dérivées partielles secondes:
        \[\frac{\partial }{\partial y} \left(\frac{\partial f}{\partial x}\right)\left(0, 0\right) = \lim_{t \to 0} \frac{t^5 - 0 \cdot t^3}{t\left(t^2 + 0\right)^2} = \lim_{t \to 0} \frac{t^5}{t^5} = 1\]
        \[\frac{\partial}{\partial x}\left(\frac{\partial f}{\partial y}\right)\left(0, 0\right) = \lim_{t \to 0} \frac{t\cdot 0 - 0\cdot t^3}{t\left(0 + t^2\right)^2} = \lim_{t \to 0} \frac{0}{t^5} = 0\]

        Nous avons donc trouvé une fonction telle que:
        \[\frac{\partial^2 f}{\partial x \partial y}\left(0, 0\right) \neq \frac{\partial^2 f}{\partial y \partial x}\left(0, 0\right)\]
        Par la contraposée du théorème de Schwarz, nous savons que $\frac{\partial^2 f}{\partial x \partial y}$ n'est pas continue en $\left(0, 0\right)$. En effet, pour $\left(x, y\right) \neq \left(0, 0\right)$:
        \[\frac{\partial^2 f}{\partial x \partial y} = \frac{6x^2 y^4 + y^6 - 3x^4 y^2}{\left(x^2 + y^2\right)^3} = \frac{\partial^2 f}{\partial y \partial x}\]

        Or, la limite n'existe pas en $\left(0, 0\right)$, donc elles ne peuvent pas être continues à ce point.
    }

    \subparag{Dérivabilité}{
        Nous savons déjà que cette fonction est dérivable, puisqu'elle est de classe $C^1$, mais utilisons la deuxième méthode pour l'illustrer.

        Nous avions trouvé $\nabla f\left(0, 0\right) = \left(0, 0\right)$. Ainsi, posons:
        \[r\left(x, y\right) = f\left(x, y\right) - \underbrace{f\left(0, 0\right)}_{= 0} - \underbrace{\left<\nabla f\left(0,0\right), \left(x, y\right)\right>}_{= 0} = f\left(x, y\right) = \frac{x y^3}{x^2 + y^2}\]

        Nous pouvons maintenant calculer la limite:
        \begin{multiequality}
        \lim_{\left(x, y\right) \to \left(0, 0\right)} \frac{r\left(x, y\right)}{\left\|\left(x, y\right) - \left(0, 0\right)\right\|} =\ & \lim_{\left(x, y\right) \to \left(0, 0\right)}\frac{x y^3}{\left(x^2 + y^2\right)\sqrt{x^2 + y^2}} \\
        =\ & \lim_{r \to 0} \frac{r^4 \cos\left(\phi\right) \sin^3\left(\phi\right)}{r^3}  \\
        =\ & \lim_{r \to 0} r \underbrace{\cos\left(\phi\right)\sin^3\left(\phi\right)}_{\text{borné}}  \\
        =\ & 0
        \end{multiequality}

        Ainsi, cette fonction est bien dérivable.
    }

}

\parag{Exemple 4}{
    Prenons la fonction suivante:
    \begin{functionbypart}{f\left(x, y\right)}
        x^2 \sin\left(\frac{1}{x}\right), \mathspace x \neq 0 \\
        0, \mathspace x = 0
    \end{functionbypart}

    Nous allons montrer que est dérivable en $\left(0, y_0\right)$, mais qu'elle n'est pas de classe $C^1$ en $\left(0, y_0\right)$, pour $y_0 \in \mathbb{R}$.

    \subparag{Dérivée partielles}{
        Clairement, $f\left(x, y\right)$ est continue sur $\mathbb{R}^2$. Calculons la dérivée partielle selon $x$. Si $x\neq 0$:
        \[\frac{\partial f}{\partial x} = 2x\sin\left(\frac{1}{x}\right) + x^2 \cos\left(\frac{1}{x}\right)\left(-\frac{1}{x^2}\right) = 2x\sin\left(\frac{1}{x}\right) - \cos\left(\frac{1}{x}\right)\]

        Et, si $x = 0$:
        \[\frac{\partial f}{\partial x}\left(0, y\right) = \lim_{t \to 0} \frac{t^2 \sin\left(\frac{1}{t}\right) - 0}{t} = \lim_{t \to 0} t \underbrace{\sin\left(\frac{1}{t}\right)}_{\text{borné}} = 0\]

        Regardons maintenant si cette dérivée partielle est continue en $\left(0, y_0\right)$:
        \[\lim_{\left(x, y\right) \to \left(0, y_0\right)} \frac{\partial f}{\partial x}\left(x, y\right) = \lim_{x \to 0} \left(\underbrace{2x\sin\left(\frac{1}{x}\right)}_{\to  0} - \underbrace{\cos\left(\frac{1}{x}\right)}_{\text{n'existe pas}}\right)\]
        qui n'existe pas. Ainsi, $\frac{\partial f}{\partial x}$ n'est pas continue si $x = 0$.
    }

    \subparag{Dérivabilité}{
        Si nous voulons savoir si cette fonction est dérivable, nous devons faire la deuxième méthode (la première méthode ne peut pas fonctionner puisque les dérivées partielles ne sont pas continues, et donc la fonction n'est pas de classe $C^1$). Posons:
    \[r\left(x, y\right) = f\left(x, y\right) - \underbrace{f\left(0, y_0\right)}_{= 0} - \underbrace{\left<\nabla f\left(0, y_0\right), \left(x, y - y_0\right)\right>}_{= 0} = x^2 \sin\left(\frac{1}{x}\right)\]

        Nous avons maintenant:
        \begin{multiequality}
         \lim_{\left(x, y\right) \to \left(0, y_0\right)}\frac{\left|r\left(x, y\right)\right|}{\left\|\left(x, y\right) - \left(0, y_0\right)\right\|} =\ & \lim_{\left(x, y\right) \to \left(0, y_0\right)} \frac{x^2 \left|\sin\left(\frac{1}{x}\right)\right|}{\sqrt{x^2 + \underbrace{\left(y - y_0\right)^2}_{\geq 0}}} \\
        \leq\ & \lim_{x \to 0} \frac{x^2 \left|\sin\left(\frac{1}{x}\right)\right|}{\left|x\right|}  \\
        =\ & \lim_{x \to 0} \left|x\right| \underbrace{\left|\sin\left(\frac{1}{x}\right)\right|}_{\text{borné}} \\
        =\ & 0
        \end{multiequality}

        Ceci implique que notre fonction est dérivable en $\left(0, y_0\right)$.
    }

    \subparag{Plan tangent}{
        $f$ est dérivable en $\left(0, y_0\right)$, et nous savons que $\nabla f\left(0, y_0\right) = \bvec{0}$. Ainsi, nous savons que le plan tangent en $\left(0, y_0, f\left(0, y_0\right)\right)$ est:
        \[z = 0 + \left<\nabla f\left(0, y_0\right), \left(x, y - y_0\right)\right> = 0\]

        Ceci est cohérent, comme nous pouvons le voir sur l'image suivante:
        \imagehere[0.8]{PlanTangentExempleDerivabilite4.png}
    }
}

\parag{Exemple 5}{
    Prenons la fonction suivante:
    \begin{functionbypart}{f\left(x, y\right)}
        \frac{x^4 y}{\left(x^2 + y^2\right)^2}, \mathspace \left(x, y\right) \neq \left(0, 0\right) \\
        0, \mathspace \left(x, y\right) = \left(0, 0\right)
    \end{functionbypart}

    Nous allons montrer que cette fonction est continue, mais pas dérivable en $\bvec{0}$.

    \subparag{Continuité}{
        Nous pouvons voir que $f$ est continue sur $\mathbb{R}$:
        \[\lim_{\left(x, y\right) \to \left(0, 0\right)} \left|f\left(x, y\right)\right| = \lim_{r \to 0} \frac{r^5 \overbrace{\left|\cos^4\left(\phi\right)\sin\left(\phi\right)\right|}^{\text{borné}}}{r^4} = 0\]
    }

    \subparag{Dérivées partielles}{
        Regardons les dérivées partielles en $\left(0, 0\right)$:
        \[\frac{\partial f}{\partial x}\left(0, 0\right) = \lim_{t \to 0} \frac{t^4 -  0}{t\left(t^2 + 0\right)^2} = 0\]
        \[\frac{\partial f}{\partial y}\left(0, 0\right) = \lim_{t \to 0} \frac{0\cdot t}{t\left(0 + t^2\right)^2} = 0\]

        Ceci nous donne que le gradient est donné par $\nabla f\left(0, 0\right) = \left(0, 0\right)$.
    }

    \subparag{Dérivabilité}{
        Nous voulons montrer que la fonction n'est pas dérivable en $\left(0, 0\right)$. Pour ce faire, nous allons utiliser la deuxième méthode, puisque le gradient et la fonction sont nuls à ce point. Ainsi, posons:
        \[r\left(x, y\right) = f\left(x, y\right) - f\left(0, 0\right) - \left<\nabla f\left(0, 0\right), \left(x, y\right)\right> = f\left(x, y\right) = \frac{x^4 y}{\left(x^2 + y^2\right)^2}\]

        Calculons maintenant la limite suivante:
        \begin{multiequality}
        \lim_{\left(x,y\right) \to \left(0, 0\right)} \frac{r\left(x, y\right)}{\left\|\left(x, y\right)\right\|} =\ & \lim_{\left(x,y\right) \to \left(0, 0\right)} \frac{x^4 y}{\left(x^2 + y^2\right)^2 \sqrt{x^2 + y^2}}  \\
        =\ & \lim_{\left(x, y\right) \to \left(0, 0\right)} \frac{x^4 y}{\left(x^2 + y^2\right)^{\frac{5}{2}}}
        \end{multiequality}


        Le fait que le degré du dénominateur est égal au degré du numérateur nous donne envie de montrer que cette limite n'existe pas:
        \[a_k = \left(\frac{1}{k}, \frac{1}{k}\right) \implies \lim_{k \to \infty} \frac{\frac{1}{k^4} \cdot\frac{1}{k}}{\left(\frac{2}{k^2}\right)^{\frac{5}{2}}} = \frac{1}{2^{\frac{5}{2}}}\]
        \[b_k = \left(0, \frac{1}{k}\right) \implies \lim_{k \to \infty} \frac{0 \cdot \frac{1}{k}}{\frac{1}{k^{5}}} = 0\]

        Ainsi, nous avons vu que la limite n'existe pas, et donc que $f\left(x, y\right)$ n'est pas dérivable en $\left(0, 0\right)$.
    }

    \subparag{Plan tangent}{
         En particulier, le plan $z = 0$ n'est pas le plan tangent à la surface $z = f\left(x, y\right)$ en $\left(0, 0\right)$:
         \imagehere[0.8]{NonPlanTangentExempleDerivabilite5.png}
    }
}

\parag{Résumé}{
    Nous pouvons à nouveau voir notre résumé. Soit $f: E \mapsto \mathbb{R}$, où $E$ est ouvert, alors:
    \svghere[0.5]{DiagrammeDerivabilite.svg}

    Ce schéma est très important, et nous devons le connaître.


    \subparag{Remarque}{
        Nous pouvons le comparer avec ce que nous avions en Analyse 1:
        \begin{multiequation}
        & \text{Classe $C^2$ sur $E$} \\
        \implies & \text{Classe $C^1$ sur $E$} \\
        \implies & \text{Dérivable en $a \in E$} \\
        \iff & \text{$f'\left(a\right)$ existe} \\
        \implies & \text{Continue en $a \in E$}
        \end{multiequation}
    }

}




\end{document}
