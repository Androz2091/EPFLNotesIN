% !TeX program = lualatex

\documentclass[a4paper]{article}

% Expanded on 2022-05-16 at 10:14:40.

\usepackage{../../style}

\title{Analyse 2}
\author{Joachim Favre}
\date{Lundi 16 mai 2022}

\begin{document}
\maketitle

\lecture{22}{2022-05-16}{Mon intégrale elle est douce}{
\begin{itemize}[left=0pt]
    \item Parce que l'épilation intégrale (je sais pas si c'est la meilleure référence à avoir, mais elle est là (Cyprien --- L'école 2)!).
    \item Définition de pavé, de subdivision et de sommes de Darboux.
    \item Définition des fonctions intégrables, et preuve que les fonctions continues sont intégrables.
    \item Explication des propriétés de l'intégrale, et du théorème de Fubini.
\end{itemize}
}
\section[Calcul intégral]{Calcul intégral des fonctions de plusieurs variables}
\subsection{Intégrale sur un pavé fermé}

\begin{parag}{Définition: Pavé}
    Un \important{pavé fermé} est un sous-ensemble de $\mathbb{R}^n$ qui est le produit Cartésien de $n$ intervalles fermés bornés: 
    \[P = \left[a_i, b_i\right] \times \left[a_2, b_2\right] \times \ldots \times \left[a_n, b_n\right], \mathspace a_i < b_i \ \forall i = 1, \ldots, n\]
    
    Nous notons le \important{pavé ouvert} par: 
    \[\mathring{P} = \left]a_1, b_1\right[ \times \ldots \times \left]a_n b_n\right[ \]

    \begin{subparag}{Exemple}
        Un pavé de dimension 1 est donné par: 
        \[P_1 = \left[a, b\right]\]
        c'est un intervalle fermé borné.

        Un pavé de dimension 2 est donné par: 
        \[P_2 = \left[a_1, b_1\right] \times \left[a_2, b_2\right]\]
        
        Nous pouvons les représenter géométriquement:
        \imagehere[0.7]{PaveFerme1D2D.png}
    \end{subparag}
\end{parag}

\begin{parag}{Définition: Volume}
    Le \important{volume} d'un pavé fermé est défini par: 
    \[\left|P\right| = \left(b_1 - a_1\right)\left(b_2 - a_2\right)\cdots\left(b_n -a_n\right)\]

    \begin{subparag}{Exemple}
        En reprenant nos deux pavés ci-dessus, nous avons: 
        \[\left|P_1\right| = b - a, \mathspace \left|P_2\right| = \left(b_1 - a_1\right)\left(b_2 - a_2\right)\]
        
    \end{subparag}
\end{parag}

\begin{parag}{Définition: Subdivision}
    Soit $\sigma_j$ une subdivision de $\left[a_j, b_j\right]$ (comme nous l'avions définie en Analyse 1), où $a_j < b_j$. Notez que chaque subdivision n'a pas besoin d'être régulière. Nous avons donc: 
    \[\sigma_j = \left\{a_j = x_{j,0} < x_{j,1} < \ldots < x_{j, n_j} < b_j\right\}\]

    Alors, $\sigma = \left(\sigma_1, \ldots, \sigma_n\right)$ est appelée une \important{subdivision} de $P$.

    Nous notons $D\left(\sigma\right)$ la collection des pavés engendrés par la subdivision. 

    \begin{subparag}{Exemple}
        Considérons $P_2 = \left[a_1, b_1\right] \times \left[a_2, b_2\right]$. Alors, nous pourrions prendre: 
        \[\sigma = \left\{\left\{a, x_1, x_2, b\right\}, \left\{a_2, y_1, y_2, y_3, b_2\right\}\right\}\]
        
        Nous pouvons représenter cela graphiquement
        \imagehere[0.7]{ExempleSubdivision2D.png}

        La subdivision nous donne: 
        \[P_2 = \bigcup_{Q \in D\left(\sigma\right)} Q = \bigcup_{\substack{i = 1, \ldots, 3 \\j = 1, \ldots, 4}} Q_{ij}\]
        
        Nous pouvons aussi calculer le volume du pavé: 
        \[\left|P_2\right| = \sum_{Q \in D\left(\sigma\right)}^{} \left|Q\right| = \sum_{\substack{i = 1, \ldots, 3 \\j = 1, \ldots, 4}} \left|Q_{ij}\right|\]
    \end{subparag}
\end{parag}

\begin{parag}{Définition: Sommes de Darboux}
    Soit $P$ un pavé fermé et soit $f: P \mapsto \mathbb{R}$ une fonction bornée sur $P$. Alors, on définit les sommes de Darboux de $f$ sur $P$.

    Soit $D\left(\sigma\right)$ une collection de pavés fermés engendrée par la subdivision $\sigma$. Alors: 
    \[\underline{S}_{\sigma}\left(f\right) \over{=}{déf} \sum_{Q \in D\left(\sigma\right)}^{} m\left(Q\right) \left|Q\right|, \mathspace \text{où } m\left(Q\right) = \inf_{\bvec{x} \in Q}\left(f\left(\bvec{x}\right)\right)\]
    \[{\color{red}\overline{S}_{\sigma}\left(f\right)} \over{=}{déf} \sum_{Q \in D\left(\sigma\right)}^{} {\color{red}M\left(Q\right)} \left|Q\right|, \mathspace \text{où } {\color{red}M\left(Q\right)} = {\color{red}\sup_{\bvec{x} \in Q}}\left(f\left(\bvec{x}\right)\right)\]
    
    Notez que nous ne savons pas s'il existe un minimum et un maximum, puisque la fonction n'est pas supposée continue. Cependant, puisqu'elle est bornée, nous savons qu'il existe un infimum et un suprémum. Aussi, nous remarquons que plus nous rajoutons de points, plus $\underline{S}_{\sigma}\left(f\right)$ augmente et plus $\overline{S}_{\sigma}\left(f\right)$ diminue, ce qui nous amène aux définitions suivantes.

    La \important{somme de Darboux inférieure} est définie par: 
    \[\underline{S}\left(f\right) \over{=}{déf} \sup\left\{\underline{S}_{\sigma}\left(f\right) : \sigma \text{ est une subdivision de } P\right\}\]
    
    La \important{somme de Darboux supérieure} est définie par: 
    \[{\color{red}\overline{S}\left(f\right)} \over{=}{déf} {\color{red}\inf}\left\{{\color{red}\overline{S}_{\sigma}\left(f\right)} : \sigma \text{ est une subdivision de } P\right\}\]

    \begin{subparag}{Exemple}
        La somme de Darboux inférieure $\underline{S}_{\sigma}$ associé à une subdivision donnée est la somme des parallélépipèdes rouges sur l'image suivante:
        \imagehere[0.7]{ExempleSommeDarbouxInferieure.png}
    \end{subparag}
    

    \begin{subparag}{Observations}
        Nous remarquons que, toujours $\underline{S}\left(f\right) \leq \overline{S}\left(f\right)$.
    \end{subparag}
\end{parag}

\begin{parag}{Définition: Fonction intégrable}
    Soit $P \subset \mathbb{R}^n$ un pavé fermé et $f: P \mapsto \mathbb{R}$ une fonction bornée.

    $f$ est \important{intégrable} sur $P$ si et seulement si: 
    \[\underline{S}\left(f\right) = \overline{S}\left(f\right)\]
    
    Dans ce cas, \important{l'intégrale de $f$ sur $P$} est définie par: 
    \[\int_P f\left(\bvec{x}\right)d \bvec{x} = \idotsint_P f\left(x_1, \ldots, x_n\right) dx_1 \ldots dx_n \over{=}{déf} \underline{S}\left(f\right) = \overline{S}\left(f\right)\]
    
    Notez que le deuxième terme n'est qu'une notation, donc nous ne pouvons par échanger les intégrales par exemples (pour l'instant). Nous verrons avec le théorème de Fubini que cette notation est, en fait, très cohérente.

    \begin{subparag}{Remarque mathématique}
        Quand nous définissons un objet en mathématiques, il faut toujours donner un exemple immédiatement après, car nous pourrions avoir défini un objet qui n'existe pas. Il est mieux que cet exemple soit trivial.
    \end{subparag}
\end{parag}

\begin{parag}{Exemple}
    Soit $P \subset \mathbb{R}^n$ un pavé fermé, et soit $f: P\mapsto \mathbb{R}$ une fonction constante, i.e. $f\left(\bvec{x}\right) = C \in \mathbb{R}$. Clairement, $f$ est bornée sur $P$. Nous allons montrer qu'elle est aussi intégrable sur $P$.

    Soit $\sigma$ une division de $P$. Alors, nous avons: 
    \[\underline{S}_{\sigma}\left(f\right) = \sum_{Q \in D\left(\sigma\right)}^{} \underbrace{\inf_{Q}\left(f\right)}_{= C} \left|Q\right| = C \sum_{Q \in D\left(\sigma\right)}^{} \left|Q\right| = C\left|P\right|\]
    \[\overline{S}_{\sigma}\left(f\right) = \sum_{Q \in D\left(\sigma\right)}^{} \underbrace{\sup_{Q}\left(f\right)}_{= C} \left|Q\right| = C \sum_{Q \in D\left(\sigma\right)}^{} \left|Q\right| = C\left|P\right|\]
    pour n'importe quelle subdivision $\sigma$.
    
    Donc, nous avons: 
    \[\underline{S}\left(f\right) = \overline{S}\left(f\right) = C \left|P\right| = \int_{P} C d \bvec{x}\]
    
    Nous avons bien démontré que $f$ était intégrable.

    \begin{subparag}{Observation}
        En particulier, si $C = 1$: 
        \[\int_{P} d \bvec{x} = \idotsint_{P} dx_1 \ldots dx_n = \left|P\right|\]
        
        Ceci nous donne donc le volume de $P$: 
        \[P = \left[a, b\right] \implies \left|P\right| = \int_{a}^{b} 1 dx = b- a\]
        \[P = \left[a_1, b_1\right] \times\left[a_2, b_2\right] \implies \iint_{P} 1 dx dy = \left(b_1 - a_1\right)\left(b_2 - a_2\right)\]
    \end{subparag}
\end{parag}

\begin{parag}{Théorème}
    Toute fonction continue est intégrable sur un pavé fermé.

    \begin{subparag}{Idée de preuve (assez complète)}
        Soit $P \subset \mathbb{R}^n$ un pavé fermé, et soit $f : P \mapsto \mathbb{R}$ une fonction continue.

        Pour commencer, puisque $P$ est un sous-ensemble compact et $f$ est une fonction continue, elle atteint son minimum et son maximum, et donc, en particulier, $f$ est bornée sur $P$.

        Considérons maintenant l'intégrabilité, nous voulons montrer $\underline{S}\left(f\right) = \overline{S}\left(f\right)$.

        Soit $\epsilon > 0$ \textit{(``l'Analyse sérieuse commence si on commence à parler d'epsilons''} Prof. Lachowska). Puisque $f$ est continue en chaque point de $P$, nous savons que pour tout $\bvec{x}_0 \in P$, il existe $\delta_{\bvec{x}_0}$ tel que: 
        \[\left\|\bvec{x}_0 - \bvec{x}\right\| \leq \delta_{\bvec{x}_0} \implies \left|f\left(\bvec{x}\right) - f\left(\bvec{x}_0\right)\right| \leq \frac{\epsilon}{2}\]
        
        Notez que nous pouvons prendre $\frac{\epsilon}{2}$ car cette proposition est vraie pour toute constante positive.

        Nous considérons maintenant le recouvrement de $P$ par les boules ouvertes $B\left(\bvec{x}_0, \delta_{\bvec{x}_0}\right)$. C'est bien un recouvrement car chaque point de $P$ est contenu dans une boule. Il y a beaucoup d'intersections, mais nous avons bien: 
        \[P \subset \bigcup_{\bvec{x}_0 \in P} B\left(\bvec{x}_0, \delta_{\bvec{x}_0}\right)\]
        
        Par le théorème de Heine-Borel-Lebesgue, il existe un sous-recouvrement fini:
        \[P \subset \bigcup_{\bvec{x}_j \in P} B\left(\bvec{x}_j, \delta_{\bvec{x}_j}\right)\]

        Or, si $\bvec{x}_1, \bvec{x}_2 \in B\left(\bvec{x}_j, \delta_j\right)$, nous avons: 
        \begin{multiequality}
            \left|f\left(\bvec{x}_1\right) - f\left(\bvec{x}_2\right)\right| \over{\leq}{$\dagger$}\ & \left|f\left(\bvec{x}_1\right) - f\left(\bvec{x}_j\right)\right| + \left|f\left(\bvec{x}_j\right) - f\left(\bvec{x}_2\right)\right|  \\
            \leq\ & \frac{\epsilon}{2} + \frac{\epsilon}{2} \\
            =\ & \epsilon 
        \end{multiequality}
        où l'inégalité $\dagger$ est l'inégalité triangulaire.
        
        Il existe une subdivision de $P$ qui correspond à ce recouvrement fini (cette affirmation est la partie de la preuve où nous ne donnons pas l'argument complet). Ainsi, soit $\sigma$ une telle subdivision. Puisque, sur chaque sous-pavé, le minimum et le maximum sont dans la boule ouverte, nous pouvons utiliser notre inégalité ci-dessus, ce qui nous donne: 
        \[\max_{Q} f - \min_{Q} f \leq \epsilon, \mathspace \forall Q \in D\left(\sigma\right)\]
        
        Nous trouvons donc que: 
        \[\overline{S}_{\sigma}\left(f\right) - \underline{S}_{\sigma}\left(f\right) = \sum_{Q \in D\left(\sigma\right)}^{} \left(\max_{Q} f - \min_{Q} f\right) \left|Q\right| \leq \epsilon \sum_{Q \in D\left(\sigma\right)}^{} \left|Q\right| = \epsilon \left|P\right|\]

        Mais, nous savons que $\overline{S}_{\sigma}\left(f\right) \geq \overline{S}\left(f\right)$ et $\underline{S}_{\sigma}\left(f\right) \leq \underline{S}\left(f\right)$, ce qui nous dit que: 
        \[0 \leq \overline{S}\left(f\right) - \underline{S}\left(f\right) \leq \overline{S}_{\sigma}\left(f\right) - \underline{S}_{\sigma}\left(f\right) \leq \epsilon \left|P\right|\]
        
        Cependant, c'est vrai pour tout $\epsilon > 0$, et le seul nombre qui est tel que $0 \leq x < \alpha$ pour tout $\alpha > 0$ est $x = 0$, donc nous avons bien trouvé que: 
        \[\overline{S}\left(f\right) - \underline{S}\left(f\right) = 0 \iff \overline{S}\left(f\right) = \underline{S}\left(f\right)\]
    \end{subparag}
\end{parag}

\begin{parag}{Propriétés de l'intégrale}
    \begin{subparag}{Propriété 1: Additivité}
        Soit $P$ un pavé fermé, et $\left\{P_i\right\}_{i \in I}$ une famille dénombrable de pavés fermés disjoints (l'intersection entre l'intérieur de n'importe quel deux pavé est vide, $\mathring{P}_i \cap \mathring{P}_j = \o$ pour $i \neq j$) telle que $P = \bigcup_{i \in I} P_i$. Par exemple, cela pourrait ressembler à: 
        \imagehere[0.7]{ExempleFamillePavesFermes.png}

        Alors, pour toute fonction continue $f: P \mapsto \mathbb{R}$, nous avons: 
        \[\int_{P} f\left(\bvec{x}\right) d \bvec{x} = \sum_{i \in I}^{} \int_{P_i} f\left(\bvec{x}\right) d \bvec{x}\]
    \end{subparag}

    \begin{subparag}{Propriété 2: Linéarité}
        Soit $P$ un pavé fermé, et soient $f, g : P \mapsto \mathbb{R}$ deux fonctions continues. Alors, pour tout $\alpha, \beta \in \mathbb{R}$:
        \[\int_{P} \left(\alpha f\left(\bvec{x}\right) + \beta g\left(\bvec{x}\right)\right)d \bvec{x} = \alpha \int_{P} f\left(\bvec{x}\right)d \bvec{x} + \beta \int_{P} g\left(\bvec{x}\right) d \bvec{x}\]
    \end{subparag}
    
    \begin{subparag}{Propriété 3}
        Soit $P$ un pavé fermé, et soit $f : P \mapsto\mathbb{R}$ une fonction bornée, intégrable sur $P$, et telle que: 
        \[\left|f\left(\bvec{x}\right)\right| \leq K \in \mathbb{R}_{\geq 0} \iff -K \leq f\left(\bvec{x}\right) \leq K, \mathspace \forall \bvec{x} \in P\]

        Alors: 
        \[-K \left|P\right| \leq \int_P f\left(\bvec{x}\right)d \bvec{x} \leq K \left|P\right|\]
    \end{subparag}
    
    Les deux premières propriétés découlent directement de la définition des sommes de Darboux.
\end{parag}

\begin{parag}{Définition: Volume}
    Soit $P \subset\mathbb{R}^2$ un pavé fermé de dimension 2, et soit $f : P \mapsto \mathbb{R}_+$ une fonction intégrable. Alors, le \important{volume} de l'ensemble sous la surface $z = f\left(x, y\right) \geq0$ est défini par: 
    \[V \over{=}{déf} \iint_P f\left(x, y\right) dx dy\]
    
    En d'autres mots, $V$ est le volume du sous-ensemble entre $z = 0$ et $z = f\left(x, y\right) \geq 0$ au dessus du pavé fermé $P \subset \mathbb{R}^2$.
\end{parag}

\begin{parag}{Théorème de Fubini}
    Soit $P = \left[a_1, b_1\right] \times \ldots \times \left[a_n, b_n\right]\subset \mathbb{R}^{n}$ un pavé fermé, et soit $f: P \mapsto \mathbb{R}$ une fonction continue.

    Alors, $f$ est intégrable sur $P$, et on a: 
    \[\int_{P} f\left(\bvec{x}\right)d \bvec{x} = \int_{a_n}^{b_n} \left(\int_{a_{n-1}}^{b_{n-1}} \cdots \left(\int_{a_1}^{b_1} f\left(x_1, \ldots, x_n\right)dx_1\right) \cdots dx_{n-1}\right)dx_{n}\]

    Ceci marche pour n'importe quel choix de l'ordre d'intégration.

    \begin{subparag}{Reformulation}
        En d'autres mots, nous pouvons calculer: 
        \[g_1\left(x_2, \ldots, x_n\right) = \int_{a_1}^{b_1} f\left(x_1, \ldots, x_n\right)dx_1\]
        où $x_1$ est une variable d'intégration, et $x_2, \ldots, x_n$ sont des paramètres. Nous pouvons ensuite prendre: 
        \[g_2\left(x_3, \ldots, x_n\right) = \int_{a_2}^{b_2} g_1\left(x_2, \ldots, x_n\right)dx_2\]
        
        Nous pouvons continuer ainsi de suite, jusqu'à calculer: 
        \[\int_{a_n}^{b_n} g_{n-1}\left(x_n\right)dx_{n} = \alpha \in \mathbb{R}\]
    \end{subparag}
    
    \begin{subparag}{Remarque}
        Ce théorème n'est uniquement valide sur un pavé fermé, nous verrons ensuite une autre version de ce théorème pour d'autres ensembles.
    \end{subparag}
    
\end{parag}

\begin{parag}{Cas $n = 2$}
    Soit $P = \left[a, b\right] \times \left[c, d\right] \subset \mathbb{R}^2$ un pavé fermé de dimension 2, et soit $f: P \mapsto \mathbb{R}$ une fonction continue. Alors, nous avons: 
    \[\int_{c}^{d} \left(\int_{a}^{b} f\left(x, y\right)dx\right)dy = \int_{a}^{b} \left(\int_{c}^{d} f\left(x, y\right)dy\right)dx = \iint_{P} f\left(x, y\right)dx dy\]

    \begin{subparag}{Idée de preuve}
        Nous pouvons trouver une subdivision de $P$ assez fine telle que $P = \bigcup_{i, j} P_{i, j}$ et $f\left(x, y\right) \approx C_{i, j} \in \mathbb{R}$ sur $P_{ij}$ (en d'autres mots, $f$ est presque constante sur chaque subdivision).

        Dans ce cas, nous pouvons écrire: 
        \begin{multiequality}
        \int_{c}^{d} \left(\int_{a}^{b} f\left(x, y\right)dx\right)dy \over{=}{add.}\ & \sum_{j}^{} \int_{y_{j-1}}^{y_j} \left(\sum_{i}^{} \int_{x_{i-1}}^{x_i} f\left(x, y\right)dx\right)dy \\
        \over{=}{lin.} \ & \sum_{i,j}^{} \int_{y_{j-1}}^{y_j} \int_{x_{i-1}}^{x_i} f\left(x, y\right)dx dy  \\
        \approx\ & \sum_{i, j}^{} C_{ij} \left(x_i - x_{i-1}\right)\left(y_j - y_{j-1}\right)  \\
        =\ & \sum_{i, j}^{} C_{ij}\left(y_j - y_{j-1}\right)\left(x_i - x_{i-1}\right)  \\
        \approx\ & \sum_{i, j}^{} \int_{x_{i-1}}^{x_i} \left(\int_{y_{j-1}}^{y_j} f\left(x, y\right)dy\right)dx \\
        =\ & \int_{a}^{b} \left(\int_{c}^{d} f\left(x, y\right)dy\right)dx 
        \end{multiequality}
    \end{subparag}
\end{parag}

\end{document}
