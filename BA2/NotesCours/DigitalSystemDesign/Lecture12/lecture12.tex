\documentclass[a4paper]{article}

% Expanded on 2022-03-29 at 13:54:27.

\usepackage{../../style}

\title{DSD}
\author{Joachim Favre}
\date{Jeudi 31 mars 2022}

\begin{document}
\maketitle

\lecture{12}{2022-03-31}{FPGA}{
\begin{itemize}[left=0pt]
    \item Explanation of how FPGAs work, and how it evolved through time.
\end{itemize}

}

\parag{FPGA}{
    We have seen CPLD was not really scalable because of its routing arrays, so it led to the introduction of the FPGA (Field Programmable Gate Array). Instead of representing logic functions using the sum of products canonical form, we use a \important{Look Up Table} (LUT): we represent a truthtable in a multiplexer. We need to add memory in order to store each of the entries in the truth table. This is represented using D-flipflops in a shift-register configuration, and it is named the \important{configuration logic}. During configuration, the configuration pane is loaded with the values.
    \imagehere{LookUpTable.png}

     To overcome the scaling-problem of the CPLD, each LUT has its own routing array (this is named distributed routing). The structure can be copied horizontally and vertically, and to provide interconnects, the routing arrays are connected horizontally and vertically:
    \imagehere{DistributedRouting.png}

    The main difference with the routing arrays of the CPLD, is that every macrocell cannot be connected to every macrocell.

    For example, we can represent the function $X = \bar{A} \cdot \left(B \oplus C\right) + D \cdot \bar{C}$ on a FPGA-structure. Here, we only need one of the Look Up Table, so it gives:
    \imagehere{FPGAStructureExample.png}

    We had two problems: know on which LUT we map our function (placement), and know how to connect the inputs to the selected LUT and how to connect the LUT to the output (routing). They are problems since, as mentioned earlier, any macrocell cannot be connected to any macrocell. Both those problems are NP-hard (they are at least as hard as the hardest problems in the class NP), but we have Place And Route softwares (P\&R-softwares) that solve these problems. For example, there is Intel's Quartus which performs this task.

    For the routing array, apart from the number of connections, it is very similar to the CPLD, since we use a matrix structure (named cross-bar). However, we replace the E$^2$-switched by pass-gates (tri-states buffers). Again, we have a use-pane (the routing array), and a programming pane (the configuration logic, a bunch of D-flipflops in a shift-register structure). 
    \imagehere[0.7]{RoutingArrayCrossBar.png}

    We notice that there is a problem since we are limited to four inputs, whereas in the CPLD we could have more than 22 inputs. Let's consider another example to see how to implement functions with more than four inputs on an FPGA. Let's take the following table: 
    \begin{center}
        \begin{tabular}{ccccc|c}
            $E$ & $D$ & $C$ & $B$ & $A$ & $X$ \\
            \hline
            0 & 0 & 0 & 0 & 0 & 0 \\
            0 & 0 & 0 & 0 & 1 & 1 \\
            0 & 0 & 0 & 1 & 0 & 0 \\
            0 & 0 & 0 & 1 & 1 & 1 \\
            0 & 0 & 1 & 0 & 0 & 0 \\
            0 & 0 & 1 & 0 & 1 & 0 \\
            0 & 0 & 1 & 1 & 0 & 1 \\
            0 & 0 & 1 & 1 & 1 & 1 \\
            0 & 1 & 0 & 0 & 0 & 1 \\
            0 & 1 & 0 & 0 & 1 & 0 \\
            0 & 1 & 0 & 1 & 0 & 1 \\
            0 & 1 & 0 & 1 & 1 & 0 \\
            0 & 1 & 1 & 0 & 0 & 0 \\
            0 & 1 & 1 & 0 & 1 & 0 \\
            0 & 1 & 1 & 1 & 0 & 0 \\
            0 & 1 & 1 & 1 & 1 & 0 \\
        \end{tabular}
        \hspace{1em}
        \begin{tabular}{ccccc|c}
            $E$ & $D$ & $C$ & $B$ & $A$ & $X$ \\
            \hline
            1 & 0 & 0 & 0 & 0 & 1 \\
            1 & 0 & 0 & 0 & 1 & 0 \\
            1 & 0 & 0 & 1 & 0 & 1 \\
            1 & 0 & 0 & 1 & 1 & 0 \\
            1 & 0 & 1 & 0 & 0 & 1 \\
            1 & 0 & 1 & 0 & 1 & 1 \\
            1 & 0 & 1 & 1 & 0 & 1 \\
            1 & 0 & 1 & 1 & 1 & 0 \\
            1 & 1 & 0 & 0 & 0 & 0 \\
            1 & 1 & 0 & 0 & 1 & 0 \\
            1 & 1 & 0 & 1 & 0 & 1 \\
            1 & 1 & 0 & 1 & 1 & 1 \\
            1 & 1 & 1 & 0 & 0 & 0 \\
            1 & 1 & 1 & 0 & 1 & 0 \\
            1 & 1 & 1 & 1 & 0 & 0 \\
            1 & 1 & 1 & 1 & 1 & 1 \\
        \end{tabular}
    \end{center}

    The key observation we can make is to see that we can split our truth table into two truthtables with only four inputs. So, we can make a LUT for the first one, a LUT for the second one, and connect their result to a multiplexer (a third LUT) controlled by $E$. Thus, we need 3 LUTs for a logic functions with 3 inputs.

    Again, there are IOB (Input Output Bloc), similarly as for the CPLD (where the E$^2$ switches are replaced by mutliplexers and tri-state buffers, naturally), connected to the outer routing arrays, in order to connect inputs and outputs to the outside world.
    
    FPGA are very scalable: nowadays FPGA have more than $800\,000$ LUTs, whereas the CPLD are more or less limited to 512 macrocells. The problem of the FPGA when compared to the CPLD is that it is volatile. In other words, when the power is switched off, the FPGA looses its functionalities. Each time the FPGA is turned on, we have to configure it. This is done using a bit-file, which contains the content for each of the D-flipflop in the configuration-pane, and it contains the initial values of the D-flipflops in the LUTs. Such file is generated by the P\&R-software.

    An interesting fact is that most of the time delays inside an FPGA is due to the distributed routing-network. Indeed, in ``big'' designs, about 80\% of the delay is caused by routing, and only 20\% by the LUTs.
}

\parag{Evolution of the FPGA}{
    This was how the FPGA worked until the 1990s. However, FPGA started to be used for implementing calculations, and due to all the delays the computations were slow. Hence, hardware multipliers were added to the FPGA. Using flipflops in the LUTs was a waste of ressource (remember that most of the time is spent in the routing network), so RAM (Random Access Memory) was added to the FPGA. This architecture provided very fast MACs (Multiply Acumulate).

    Moreover, as technology evloved, there was a requirement for faster IO. This has led to the integration of hardware Intellectual Property (IP) blocks, and the integration of specialised IO-blocks (such as an Ethernet interface). 

    Then, we wondered why putting a processor outside of the FPGA, so they started integrate it in the FPGA. This is named a SOC (System On Chip). 

    \imagehere{EvolvedFPGA.png}

    Xilinx's Zynq FPGA and Intel's Cyclone V FPGA are examples of these SOCs.
}

\parag{Success}{
    FPGAs are still very used today. Indeed, Intel began in 2018 the HARP-program: they combined a XEON processor with a FPGA. Intel even bought Altera, one of the two leaders in the FPGA-market. 

    An FPGA runs at approximately $\SI{300\text{--}600}{\mega\hertz }$, which is slow since a Xeon CPU runs up to a factor 10 faster. Moreover, the parallelism of a FPGA is not that useful since GPUs are better. However, a FPGA can do massive parallelism and can be tuned to each task thread. In other words, they can be used as reconfigurable accelerators for both the CPU and the GPU, whereas CPUs and GPUs are fixed architectures. This is why they are still used today.
}

\parag{Usage}{
    To use FPGA, graphic-entry programs like logism are not very productive. We need the means to design systems at a higher level (we no longer program in Assembly). For this reason, Hardware Description Languages (HDL) were invented. There exist two types: VHDL (mostly used in Europe) and Verilog (mostly used in the rest of the world).
}




\end{document}
