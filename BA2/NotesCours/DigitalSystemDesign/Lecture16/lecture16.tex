% !TeX program = lualatex

\documentclass[a4paper]{article}

% Expanded on 2022-04-28 at 17:52:46.

\usepackage{../../style}

\title{DSD}
\author{Joachim Favre}
\date{Jeudi 28 avril 2022}

\begin{document}
\maketitle

\lecture{16}{2022-04-28}{More complex circuits}{
\begin{itemize}[left=0pt]
    \item Explanation on how to describe conditional and unconditional multiplexers, in explicit and implicit processes.
    \item Explanation on how to describe a D-Flipflop.
\end{itemize}

}

\subsection{More advanced VHDL}

\begin{filecontents*}[overwrite]{muxImplicit.code}
<signal> <= <true_value> WHEN <condition> ELSE <false_value>;
\end{filecontents*}

\begin{filecontents*}[overwrite]{muxExplicit.code}
IF <condition> 
    THEN <true_body>
    ELSE <false_body>
END IF;
\end{filecontents*}

\begin{filecontents*}[overwrite]{prioritisedMuxImplicit.code}
<signal> <= <value1> WHEN <condition1>
            ELSE <value2> WHEN <condition2>
            ELSE <value3>;
\end{filecontents*}


\begin{parag}{Multiplexer}
    We have two ways to describe a multiplexer. The first one is through implicit processes:
    \importcode{muxImplicit.code}{VHDL}
    
    We can also use the following syntax, to use multiple multiplexers: 
    \importcode{prioritisedMuxImplicit.code}{VHDL}
    
    This does what is called ``prioritised multiplexers'', leading to a much longer sequence of multiplexers, and may lead to long gate delays:
    \imagehere[0.5]{prioritisedMuxImplicit.png}

    The second way is through the explicit processes:
    \importcode{muxExplicit.code}{VHDL}
    
    The advantage of the explicit version is that we can put as many clauses as we want in the body, including another multiplexer. We can nest the clauses in the bodies, and thus making some kind of ``binary tree'' of multiplexers.
\end{parag}

\begin{filecontents*}[overwrite]{uncodnditionalMuxImplicit.code}
WITH <cnd_signal> SELECT <result_signal> <=
    <value_1> WHEN <c1>,
    <value_2> WHEN <c2>,
    <value_3> WHEN <c3>,
    <value_4> WHEN OTHERS;
\end{filecontents*}

\begin{filecontents*}[overwrite]{uncodnditionalMuxExplicit.code}
CASE <cnd_signal> IS
    WHEN <c1> => <body_c1>;
    WHEN <c2> => <body_c2>;
    WHEN <c3> => <body_c3>;
    WHEN OTHERS => <body_default>;
END_CASE;
\end{filecontents*}

\begin{parag}{Unconditional multiplexer}
    We can also write unconditional multiplexers, similar to switch cases (over the \texttt{<cnd\_signal>} value) in programming languages. In the implicit processes, it is spelled as:
    \importcode{uncodnditionalMuxImplicit.code}{VHDL}
    
    And, for the explicit processes:
    \importcode{uncodnditionalMuxExplicit.code}{VHDL}
\end{parag}

\begin{filecontents*}[overwrite]{flipflopAsynchronous.code}
dff:PROCESS(<rst_signal>, <clk_signal>)
BEGIN
    IF (<rst_signal> = '1') THEN
        <state_signal> <= <rst_value>;
    ELSIF (rising_edge(<clk_signal>)) THEN
        <state_signal> <= <d_value>;
    END IF;
END PROCESS dff;
\end{filecontents*}

\begin{filecontents*}[overwrite]{flipflopSynchronous.code}
dff:PROCESS(<clk_signal>)
BEGIN
    IF (rising_edge(<clk_signal>)) THEN
        IF (<rst_signal> = '1') THEN
            <state_signal> <= <rst_value>;
        ELSE
            <state_signal> <= <d_value>;
    END IF;
END PROCESS dff;
\end{filecontents*}


\begin{parag}{D-FlipFlop}
    To make a D-FlipFlop, we need to write it as an explicit process. An asynchrously-reset D-Flipflop could be written as:
    \importcode{flipflopAsynchronous.code}{VHDL}
    
    And a synchronously-reset D-FlipFlop could be:
    \importcode{flipflopSynchronous.code}{VHDL}

\end{parag}


\end{document}
