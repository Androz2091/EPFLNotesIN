% !TeX program = lualatex

\documentclass[a4paper]{article}

% Expanded on 2022-05-02 at 13:15:23.

\usepackage{../../style}

\title{DSD}
\author{Joachim Favre}
\date{Lundi 02 mai 2022}

\begin{document}
\maketitle

\lecture{17}{2022-05-02}{Port mapping and types}{
\begin{itemize}[left=0pt]
    \item Explanation of port mapping.
    \item Explanation on how to use types.
    \item Example of a serial adder.
\end{itemize}
}


\begin{filecontents*}[overwrite]{ArchitectureComponent.code}
COMPONENT <component_name> IS
    PORTS(
        <port_list>
    );
END COMPONENT;
\end{filecontents*}

\begin{filecontents*}[overwrite]{ArchitectureComponentPortMap.code}
<name> : <component_name>
PORT MAP(
    <component_port_1> => <signal_1>,
    <component_port_2> => <signal_2>,
    ...,
    <component_port_n> => <signal_n>
);
\end{filecontents*}

\begin{parag}{Port mapping}
    To use a previously defined component, we need to tell that we are going to use it in the declaration section of the architecture (by copy-pasting the component's entity section): 
    \importcode{ArchitectureComponent.code}{VHDL}

    Then, in the architecture, we do a port map, using the \texttt{=>} symbol, telling which input/output of the blackbox gets connected to which signal. Also, note that we use comas in the \texttt{PORT MAP}, just like when we call a function in a regular programming language.
    \importcode{ArchitectureComponentPortMap.code}{VHDL}
\end{parag}

\begin{parag}{Type}
    We can define types in VHDL in the declaration section of an architecture. This is some kind of enumeration in C. This follows the following syntax: 
    \begin{center}
        \texttt{TYPE <type\_name> IS (<name\_1>, ..., <name\_n>);}
    \end{center}

    Note that names need to be unique, as usual. Thus, we have to be careful not to have \texttt{reset} as an input and \texttt{reset} in our type.
\end{parag}

\begin{filecontents*}[overwrite]{FullAdder.code}
LIBRARY ieee;
USE ieee.std_logic_1164.all;

ENTITY FullAdder IS
    PORT(
        A, B, Cin : IN std_logic;
        Sum, Cout: OUT std_logic
    );
END FullAdder;
\end{filecontents*}

\begin{filecontents*}[overwrite]{SerialAdd.code}
LIBRARY ieee;
USE ieee.std_logic_1164.all;
    PORTS (
        A, B, start, clock : IN std_logic;
        Sums, Overflow : OUT std_logic
    );
ENTITY SerialAdd IS
END SerialAdd;

ARCHITECTURE datapath OF SerialAdd IS
    COMPONENT FullAdder IS
        PORTS (
            A, B, Cin : IN std_logic;
            Sum, Cout : OUT std_logic;
        );
    END COMPONENT;

    SIGNAL s_Cin, s_Cout, s_Cout_del : std_logic;
BEGIN
    s_Cin <= s_Cout_del AND NOT (start);
    
    dff : PROCESS(clock) IS
    BEGIN
        IF(falling_edge(clock)) THEN
            s_Cout_del <= s_Cout;
        END IF;
    END PROCESS;  -- is already reset by the start signal

    add : FullAdder
    PORT MAP (
        A => A,
        B => B,
        Cin => s_Cin,
        Sum => Sums,
        Cout => s_Cout
    );
END datapath;
\end{filecontents*}

\begin{filecontents*}[overwrite]{controlmachine.code}
ARCHITECTURE UseType OF ControlMachine IS
    TYPE StateType IS (IDLE, BIT0, BIT1, BIT2, BIT3);
    SIGNAL s_cur_state, s_next_state : StateType;
BEGIN
    overflow <= '1' WHEN s_cure_state = BIT3 ELSE '0';
    valid <= '1' WHEN s_cur_state = BIT0
                      OR s_cur_state = BIT1
                      OR s_cur_state = BIT2
                      OR s_cur_state = BIT3
                 ELSE '0';
    dff : PROCESS(clock)
    BEGIN
        IF (rising_edge(clock)) THEN
            IF(reset = '1') THEN s_cur_state <= IDLE
                            ELSE s_cur_state <= s_next_state;
            END IF;
        END IF;
    END PROCESS dff;

    transition : PROCESS(start, s_cur_state) IS
    BEGIN
        CASE(s_cur_state) IS
            WHEN IDLE => IF(start = '1') THEN s_next_state
                             s_next_state <= BIT0;
                         ELSE
                             s_next_state <= IDLE;
                         END IF;
            WHEN BIT0 => s_next_state <= BIT1;
            WHEN BIT1 => s_next_state <= BIT2;
            WHEN BIT2 => s_next_state <= BIT3;
            WHEN OTHERS => s_next_state <= IDLE;
        END CASE;
    END PROCESS transition;
END UseType;
\end{filecontents*}


\begin{parag}{Example}
    Let's say we want to make a 4-bit serial adder. The idea is that we want to add up numbers, where we have one bit of each number at a time (LSB first, MSB last):
    \imagehere{SerialAdderDiagram.png}

    Let's first say we have the following full adder:
    \importcode{FullAdder.code}{VHDL}

    We note that, looking at the timing diagram, the bits arrive at the rising edges of the clock. Also, when the LSB arrives, we want \texttt{CIn} to be 0, but then for the others, we want to be the carry out of the last bit (we need a memory action, thus we will need a flip-flop). This means that we can draw the following circuit and timing diagram:
    \imagehere{SerialAdderExtendedDiagram.png}

    Which implies the following code:
    \importcode{SerialAdd.code}{VHDL}
    
    Now, we want to make the sequence. This leads to the following timing diagram: 
    \imagehere{SerialAdderSequence.png}

    Note that, this time, everything is happening on the positive edge of the clock.

    \importcode{controlmachine.code}{VHDL}
\end{parag}




\end{document}
