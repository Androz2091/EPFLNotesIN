% !TeX program = lualatex
% Using VimTeX, you need to reload the plugin (\lx) after having saved the document in order to use LuaLaTeX (thanks to the line above)

\documentclass[a4paper]{article}

% Expanded on 2022-10-03 at 08:18:55.

\usepackage{../../style}

\title{Analyse-2}
\author{Joachim Favre}
\date{Lundi 03 octobre 2022}

\begin{document}
\maketitle

\lecture{2}{2022-10-03}{Des courbes}{
\begin{itemize}[left=0pt]
    \item Définition de courbe régulière, simple, fermée et par morceaux.
    \item Définition des intégrales curvilignes, sur des champs scalaires et vectoriels.
\end{itemize}

}

\section{Intégrales curvilignes}
\subsection{Courbes}
\parag{Définition: Courbe régulière}{
    Un sous-ensemble $\Gamma \subseteq \mathbb{R}^n$ est une \important{courbe régulière} s'il existe une fonction:
    \[\begin{split}
    \gamma: \left[a, b\right] &\longmapsto \mathbb{R}^n \\
    t &\longmapsto \gamma\left(t\right) = \left(\gamma_1\left(t\right), \ldots, \gamma_n\left(t\right)\right)
    \end{split}\]
    
    Telle que:
    \begin{enumerate}
        \item $\gamma\left(\left[a, b\right]\right) = \Gamma$ où $\gamma\left(\left[a, b\right]\right) = \left\{x \in \mathbb{R}^n |  \exists t \in \left[a, b\right] \text{ avec } \gamma\left(t\right) = x\right\}$.
        \item $\gamma \in C^0\left(\left[a, b\right]; \mathbb{R}^n\right) \cap C^1\left(\left]a, b\right[ ; \mathbb{R}^n\right)$; donc elle est continue et différentiable.
        \item $\left\|\gamma'\left(t\right)\right\| \neq 0$ pour tout $t \in \left[a, b\right]$.
    \end{enumerate}
    
    Un tel $\gamma$ s'appelle une paramétrisation de $\Gamma$.
    
    \subparag{Remarques}{
        Le point 3 est équivalent à $\gamma'\left(t\right) \neq \left(0, \ldots, 0\right)$ pour tout $t \in \left[a, b\right]$.

        Moralement, $\gamma$ représente un point qui se déplace sur la courbe $\Gamma$. On trouve ainsi que $t$ c'est le temps, $\gamma\left(t\right)$ c'est la position au temps $t$, et $\gamma'\left(t\right)$ c'est la vitesse. Donc, $\gamma'\left(t\right) \neq 0$ veut dire qu'on ne s'arrête jamais.
    }
}

\parag{Exemple}{
    Nous voulons savoir si le demi-cercle $\Gamma$ de rayon $1$, sous-ensemble de $\mathbb{R}^2$, est une courbe régulière. Pour cela, nous devons trouver une paramétrisation.

    Nous pouvons utiliser les coordonnées polaires:
    
    \[\begin{split}
    \gamma: \left[0, \pi\right] &\longmapsto \mathbb{R}^2 \\
    t &\longmapsto \left(\cos\left(t\right), \sin\left(t\right)\right)
    \end{split}\]
    
    Calculons la norme de la dérivée: 
    \[\gamma'\left(t\right) = \left(-\sin\left(t\right), \cos\left(t\right)\right) \implies \left\|\gamma'\left(t\right)\right\|^2 = 1 \neq 0\]
    
    Nous obtenons donc bien que $\Gamma$ est une courbe régulière.

    \subparag{Remarque}{
        Nous remarquons qu'une courbe $\Gamma$ peut avoir plusieurs paramétrisations. Nous aurions pu considérer, pour la même courbe:
        
        \[\begin{split}
        \bar{\gamma}: \left[-1, 1\right] &\longmapsto \mathbb{R}^2 \\
        t &\longmapsto \left(t, \sqrt{1 - t^2}\right)
        \end{split}\]
        
        Nous avons bien $\bar{\gamma}\left(\left[a, b\right]\right) = \Gamma$ car $\Gamma$ est le graphe de $\sqrt{1 - t^2}$. Nous obtenons aussi: 
        \[\bar{\gamma}'\left(t\right) = \left(1, \frac{-t}{\sqrt{1 - t^2}}\right) \implies \bar{\gamma}'\left(t\right) \neq 0\]
        
        C'est donc aussi une paramétrisation du cercle, mais sa ``vitesse'' n'est cependant pas constante. 
    }
}

\parag{Définition: Courbe régulière simple}{
    Une courbe régulière $\Gamma \subset \mathbb{R}^n$ est \important{simple} s'il existe une paramétrisation $\gamma: \left[a, b\right] \mapsto \Gamma$ telle que, pour tout $s, t \in \left[a, b\right]$, nous avons: 
    \[\gamma\left(s\right) = \gamma\left(t\right) \implies \begin{systemofequations} s = a, t = b\\ s = b, t = a \\ s = t \end{systemofequations}\]
    
    \subparag{Remarque}{
        Cela veut simplement dire que la courbe ne se recoupe pas.
    }
}

\parag{Définition: Courbe régulière fermée}{
    Une courbe régulière $\Gamma \in \mathbb{R}^n$ est \important{fermée} s'il existe une paramétrisation $\gamma : \left[a, b\right] \mapsto \Gamma$ avec $\gamma\left(a\right) = \gamma\left(b\right)$.
}

\parag{Exemples}{
    Nous pouvons considérer les courbes suivantes:
    \svghere{ExempleCourbes.svg}
}

\parag{Définition: Courbe régulière par morceaux}{
    $\Gamma$ est une courbe régulière par morceaux si $\Gamma = \Gamma_1 \cup \ldots \cup \Gamma_k$, où les $\Gamma_i$ sont des courbes régulières telles que $\Gamma_{i+1}$ commence là où $\Gamma_i$ s'arrête (afin que la courbe finale soit continue).

    \subparag{Exemple}{
        Nous pouvons par exemple prendre $\Gamma = \Gamma_1 \cup \Gamma_2 \cup \Gamma_3$: 
        \svghere[0.5]{ReguliereParMorceaux.svg}
    }
    
}

\parag{Exemple 1}{
    Considérons le cercle de centre $\left(x_0, y_0\right)$ et de rayon $r$: 
    \[\Gamma = \left\{\left(x, y\right) \in \mathbb{R}^2 | \left(x - x_0\right)^2 + \left(y - y_0\right)^2 = r^2\right\}\]

    Nous pouvons poser la paramétrisation suivante:
    
    \[\begin{split}
    \gamma: \left[0, 2\pi\right] &\longmapsto \mathbb{R}^2 \\
    t &\longmapsto \left(x_0 + r\cos\left(t\right), y_0 + r\sin\left(t\right)\right)
    \end{split}\]

    Nous avons alors: 
    \[\gamma'\left(t\right) = \left(-r\sin\left(t\right), r\cos\left(t\right)\right) \implies \left\|\gamma'\left(t\right)\right\| = r \neq 0\]
    
    C'est donc bien une paramétrisation régulière. De plus, nous avons $\gamma\left(0\right) = \gamma\left(2\pi\right)$, ce qui implique que $\Gamma$ est fermée. Nous trouvons finalement qu'il n'y a pas d'intersection, donc que $\Gamma$ est simple.
}

\parag{Exemple 2}{
    Nous voulons paramétriser le triangle de sommets $\left(0, 0\right), \left(2, 0\right), \left(0, 1\right)$. Ce n'est pas une courbe régulière puisqu'il y a des angles, mais nous allons la paramétriser par morceaux.
    \svghere[0.4]{ParametrisationTriangle.svg}

    Nous pouvons poser $\Gamma_1 = \left\{\left(t, 0\right) | t\in\left[0, 2\right]\right\}$, ce qui nous donne:
    \[\begin{split}
    \gamma: \left[0, 2\right] &\longmapsto \mathbb{R}^2 \\
    t &\longmapsto \left(t, 0\right)
    \end{split}\]
    
    Pour trouver $\Gamma_2$, nous pouvons faire une interpolation entre $\left(2, 0\right)$ et $\left(0, 1\right)$, ce qui nous donne: 
    \[\Gamma_2 = \left\{\left(2, 0\right) + t\left[\left(0, 1\right) - \left(2, 0\right)\right] |t \in \left[0, 1\right]\right\} = \left\{\left(2, 0\right) + t\left(-2, 1\right) | t \in \left[0, 1\right]\right\} \]
    
    Cela nous permet de trouver $\gamma_2$: 
    \[\begin{split}
    \gamma_2: \left[0, 1\right] &\longmapsto \mathbb{R}^2 \\
    t &\longmapsto \left(2 - 2t, t\right)
    \end{split}\]
    
    Pour finir, nous pouvons trouver que $\Gamma_3 = \gamma_3\left(\left[0, 1\right]\right)$ où:
    \[\begin{split}
    \gamma_3: \left[0, 1\right] &\longmapsto \mathbb{R}^2 \\
    t &\longmapsto \left(0, 1 - t\right)
    \end{split}\]
    
    Nous trouvons finalement que $\Gamma = \Gamma_1 \cup \Gamma_2 \cup \Gamma_3$ est une courbe régulière par morceaux.
}

\parag{Exemple 3}{
    Nous voulons paramétriser une hélice dans $\mathbb{R}^3$.
    \imagehere[0.5]{ParametrisationHelice.png}

    Nous trouvons que $\Gamma = \gamma\left(\left[0, 4\pi\right]\right)$, où:
    \[\begin{split}
    \gamma : \left[0, 4\pi\right] &\longmapsto \mathbb{R}^3 \\
    t &\longmapsto \left(\cos\left(t\right), \sin\left(t\right), t\right)
    \end{split}\]
}

\parag{Exemple 4}{
    Considérons la courbe composée d'une droite allant de $\left(-1, 0\right)$ à $\left(0, 1\right)$, et d'une droite allant de $\left(0, 1\right)$ à $\left(1, 0\right)$. 

    Nous savons que cette courbe n'est pas régulière, mais nous pourrions trouver la paramétrisation suivante: 
    \[\begin{split}
        \hat{\gamma}: \left[-1, 1\right] &\longmapsto \mathbb{R}^2 \\
    t &\longmapsto \left(t^3, 1 - \left|t^3\right|\right)
    \end{split}\]
    
    Elle paramétrise bien $f\left(x\right) = 1 - \left|x\right|$, et elle est de classe $C^1\left(\left]-1, 1\right[ , \mathbb{R}^2\right)$: 
    \[\hat{\gamma}'\left(t\right) = \left(3t^2, -3t\left|t\right|\right)\]
    
    Cependant, nous avons $\hat{\gamma}'\left(0\right) = \left(0, 0\right)$, donc ce n'est pas une paramétrisation régulière.
}

\parag{Définition: Opposé d'une courbe}{
    Soit $\Gamma$ une courbe paramétrisée par $\gamma : \left[a, b\right] \mapsto \Gamma$. Alors, nous pouvons définir $-\Gamma$ par $\Gamma$ parcourue dans l'autre sens, c'est-à-dire paramétrisée par:
    \[\begin{split}
        \widetilde{\gamma}: \left[a, b\right] &\longmapsto \Gamma \\
    t &\longmapsto \gamma\left(-t + a + b\right)
    \end{split}\]

    \subparag{Remarque}{
        Nous avons alors:
        \[\widetilde{\gamma}'\left(t\right) = -\gamma'\left(t\right)\]
    }
}

\subsection{Intégrales curvilignes}
\parag{Idée}{
    Le but de cette partie est d'intégrer une fonction le long d'une courbe.
}

\parag{Exemple}{
    Disons que nous voulons connaître l'altitude moyenne qu'on avait lors d'une randonnée. Pour cela, nous pouvons considérer la courbe $\Gamma \subset \mathbb{R}^2$ qui donne notre trajet (en coordonnées GPS par exemple) avec $f\left(x, y\right)$ qui donne l'altitude. Alors, pour obtenir la valeur moyenne, nous cherchons à intégrer $f$ le long de $\Gamma$.
}

\parag{Définition: Intégrale curviligne}{
    Soient $\Omega \subset \mathbb{R}^n$ un ensemble ouvert, $\Gamma \subset \Omega$ une courbe régulière, et $\gamma : \left[a, b\right] \mapsto \Gamma$ une paramétrisation de cette courbe.

    Pour $f \in C^0\left(\Omega\right)$, nous définissons l'intégrale curviligne de $f$ le long de $\Gamma$:
    \[\int_\Gamma f d \ell = \int_{a}^{b} f\left(\gamma\left(t\right)\right) \left\|\gamma'\left(t\right)\right\| dt\]
    
    Pour un champ vectoriel, $F \in C^0\left(\Omega, \mathbb{R}^n\right)$, nous définissons: 
    \[\int_{\Gamma} F d \ell = \int_\Gamma F \dotprod d \ell = \int_{a}^{b} F\left(\gamma\left(t\right)\right) \dotprod \gamma'\left(t\right) dt\]
    
    Si $\Gamma = \Gamma_1 \cup \ldots \cup \Gamma_k$ est régulière par morceaux, nous définissons: 
    \[\int_{\Gamma} f d \ell = \sum_{i=1}^{k} \int_{\Gamma_{i}} f d \ell \]

    Et de la même manière pour un champ vectoriel:
    \[\int_{\Gamma} F \dotprod d \ell = \sum_{i=1}^{k} \int_{\Gamma} F \dotprod d \ell \]

    \subparag{Remarque 1}{
        Si nous avons une paramétrisation telle que $\left\|\gamma'\left(t\right)\right\| = 1$ (donc qu'elle va à vitesse constante), alors nous avons simplement: 
        \[\int_{\Gamma} f d \ell = \int_{a}^{b} f\left(\gamma\left(t\right)\right) dt\]
    }

    \subparag{Remarque 2}{
        L'intégrale $\int_{\Gamma} f d \ell $ ne dépend pas de la paramétrisation choisie, au signe près. Nous lèverons cette ambigüité plus tard en parlant d'orientation.

        En effet, nous avons par exemple:
        \[\int_{-\Gamma} F \dotprod d \ell = -\int_{\Gamma} F \dotprod d \ell \]
        car $\widetilde{\gamma}'\left(t\right) = -\gamma'\left(t\right)$.
    }
    
    \subparag{Remarque 3}{
        La longueur de la courbe $\Gamma$ est donnée par: 
        \[\int_\Gamma 1 d \ell\]
    }
    
    \subparag{Remarque 4}{
        L'intégrale suivante s'interprète physiquement comme le travail pour déplacer une particule le long de $\Gamma$ dans le champ de force $F$: 
        \[\int_{\Gamma} F \dotprod d \ell \]
        
    }
}

\parag{Exemple 1}{
    Calculons la longueur du cercle de rayon $r$, centré en 0. Commençons par prendre une paramétrisation:
    
    \[\begin{split}
    \gamma: \left[0, 2\pi\right] &\longmapsto \mathbb{R}^2 \\
    t &\longmapsto \left(r \cos\left(t\right), r\sin\left(t\right)\right)
    \end{split}\]
    
    Nous trouvons donc que la longueur de $\Gamma$ est donnée par: 
    \[\int_{\Gamma} 1 d \ell = \int_{0}^{2\pi} 1 \underbrace{\left\|\gamma'\left(t\right)\right\|}_{r} dt = 2\pi r\]
}

\parag{Exemple 2}{
    Calculons la hauteur moyenne de l'hélice $\Gamma \subset\mathbb{R}^3$, donnée par: 
    \[\Gamma = \left\{\left(\cos\left(t\right), \sin\left(t\right), t\right) | t \in \left[0, 4\pi\right]\right\}\]
    
    La hauteur moyenne est donnée par: 
    \[\frac{\int_{\Gamma} \text{hauteur} d \ell }{\text{longueur de $\Gamma$}} = \frac{\int_{\Gamma} \text{hauteur} d \ell}{\int_{\Gamma} 1 d \ell }\]
    
    Nous pouvons simplement prendre la paramétrisation suivante:
    \[\begin{split}
    \gamma: \left[0, 4\pi\right] &\longmapsto \mathbb{R}^3 \\
    t &\longmapsto \left(\cos\left(t\right), \sin\left(t\right), t\right)
    \end{split}\]
    
    Et nous pouvons calculer sa dérivée: 
    \[\gamma'\left(t\right) = \left(-\sin\left(t\right), \cos\left(t\right), 1\right) \implies \left\|\gamma'\left(t\right)\right\| = \sqrt{1 + 1} = \sqrt{2}\]
    
    Nous pouvons maintenant calculer la longueur de notre hélice: 
    \[\int_{\Gamma} 1 d \ell = \int_{0}^{4\pi} 1 \cdot  \sqrt{2} dt = 4\pi\sqrt{2}\]
    
    La hauteur est donnée par la troisième composante de $\gamma$, donc $t$: 
    \[\int_{\Gamma} \text{hauteur} d \ell = \int_{\Gamma} z d \ell = \int_{0}^{4\pi} t \sqrt{2} dt = \sqrt{2} \frac{t^2}{2} \eval_{4\pi}^{0} = \sqrt{2} 8 \pi^2\]
    
    On trouve finalement que la hauteur moyenne est donnée par: 
    \[h_{moy} = \frac{\sqrt{2} 8 \pi^2}{\sqrt{2} 4 \pi} = 2\pi\]
    
    Ce à quoi on pouvait s'attendre.
}

\parag{Exemple 3}{
    Considérons le champ vectoriel suivant:
    \[\begin{split}
    F: \mathbb{R}^2 &\longmapsto \mathbb{R}^2 \\
    \left(x, y\right) &\longmapsto \left(-y, x\right)
    \end{split}\]
    
    Nous voulons intégrer cette fonction sur le cercle unitaire. Nous pouvons ainsi prendre la paramétrisation suivante:
    \[\begin{split}
    \gamma: \left[0, 2\pi\right] &\longmapsto \mathbb{R}^2 \\
    t &\longmapsto \left(\cos\left(t\right), \sin\left(t\right)\right)
    \end{split}\]
    
    L'intégrale de notre fonction est le travail d'une particule qui voyage dans le champ $F$ le long de $\Gamma$: 
    \autoeq{\unexpanded{\int_{\Gamma} F \dotprod d \ell = \int_{0}^{2\pi} F\left(\gamma\left(t\right)\right) \dotprod \gamma'\left(t\right) dt = \int_{0}^{2\pi} \left(-\sin\left(t\right), \cos\left(t\right)\right) \dotprod \left(-\sin\left(t\right), \cos\left(t\right)\right) dt = \int_{0}^{2\pi} \left(\sin^2\left(t\right) + \cos^2\left(t\right)\right)dt = \int_{0}^{2\pi} dt = 2\pi}}

    Il est intéressant de remarquer que nous avons pris une intégrale sur un cercle fermé, mais que le résultat n'est pas 0.
}

\end{document}
