% !TeX program = lualatex
% Using VimTeX, you need to reload the plugin (\lx) after having saved the document in order to use LuaLaTeX (thanks to the line above)

\documentclass[a4paper]{article}

% Expanded on 2022-10-10 at 08:15:44.

\usepackage{../../style}

\title{Analyse 3}
\author{Joachim Favre}
\date{Lundi 10 octobre 2022}

\begin{document}
\maketitle

\lecture{3}{2022-10-10}{Des notions à haut potentiel}{
\begin{enumerate}[left=0pt]
    \item Définition de fonctions qui dérivent de potentiel.
    \item Preuve que l'intégrale d'un champ vectoriel qui dépend d'un potentiel ne dépend que de son point de départ et de son point d'arrivée.
    \item Explication de comment montrer qu'un champ dérive d'un potentiel, et preuve de 2 théorèmes sur ce sujet.
\end{enumerate}
}

\subsection{Champs qui dérivent d'un potentiel}
\parag{Définition: Dériver d'un potentiel}{
    Soient $\Omega \subset \mathbb{R}^n$ un ensemble ouvert, et $F \in C^{0}\left(\Omega;\mathbb{R}^n\right)$.

    Alors, le champ vectoriel $F$ \important{dérive du potentiel} $f \in C^{1}\left(\Omega\right)$ si $F = \nabla f$.

    \subparag{Remarque 1}{
        Ceci est une généralisation du concept de primitive pour les champs vectoriels.
    }

    \subparag{Remarque 2}{
        Tous les champs vectoriels ne dérivent pas d'un potentiel.
    }
    
}

\parag{Proposition}{
    Si $F \in C^{1}\left(\Omega; \mathbb{R}^n\right)$ dérive du potentiel $f \in C^{1}\left(\Omega\right)$ et $\Gamma$ est une courbe de régulière de paramétrisation $\gamma : \left[a, b\right] \mapsto \Omega$, alors: 
    \[\int_{\Gamma} F \dotprod d \ell = f\left(\gamma\left(b\right)\right) - f\left(\gamma\left(a\right)\right)\]
    
    \subparag{Preuve}{
        Calculons la dérivée suivante: 
        \autoeq{\unexpanded{\frac{\partial }{\partial t} \left(f \circ \gamma\left(t\right)\right) = \frac{\partial }{\partial t} \left(f\left(\gamma_1\left(t\right)\right), \ldots, f\left(\gamma_n\left(t\right)\right)\right) = \frac{\partial f}{\partial x_1} \left(\gamma\left(t\right)\right) \frac{\partial \gamma_1}{\partial t} \left(t\right) + \ldots + \frac{\partial f}{\partial x_n} \left(\gamma\left(t\right)\right) \frac{\partial \gamma_n}{\partial t} \left(t\right) = \nabla f\left(\gamma\left(t\right)\right) \dotprod \gamma't\left(t\right) = F\left(\gamma\left(t\right)\right) \dotprod  \gamma'\left(t\right)}}
        
        Ce résultat est très intéressant, il nous permet de trouver que: 
        \autoeq{\unexpanded{\int_{\Gamma} F \dotprod d \ell = \int_{a}^{b} F\left(\gamma\left(t\right)\right) \dotprod \gamma'\left(t\right) dt = \int_{a}^{b} \frac{\partial }{\partial t} \left(f \circ \gamma\left(t\right)\right) dt = \left[f \circ \gamma\left(t\right)\right]_{a}^b = f\left(\gamma\left(b\right)\right) - f\left(\gamma\left(a\right)\right)}}

        \qed
        
    }
}

\parag{Savoir si une fonction dérive d'un potentiel}{
    Nous savons que, pour tout champ scalaire $f$, nous avons $\rot \nabla f = 0$. Ainsi, si nous voulons connaître si une fonction dérive d'un potentiel, il est très intéressant de commencer par calculer le rotationnel: s'il est non-nul, nous savons que notre fonction en dérive pas d'un potentiel (si $F = \nabla f$, alors $\rot F = \rot \nabla f = 0$).

    S'il est nul, alors nous devons plus travailler. Nous allons voir des théorèmes intéressants, mais nous pouvons essayer aussi de trouver le potentiel dont dérive notre fonction (si elle en dérive de un). Afin d'y parvenir, nous pouvons soit essayer de tâtonner, soit considérer $F = \nabla f$ comme un système de plusieurs équations différentielles (qui ne donnent pas toujours une solution).

    \subparag{Observation}{
        Un autre critère ressemblant à la vérification que $\rot F = 0$, est que pour toute courbe fermée $\Gamma \subset \mathbb{R}^n$ (on a une paramétrisation $\gamma : \left[a, b\right] \mapsto \Gamma$ telle que $\gamma\left(a\right) = \gamma\left(b\right)$), alors: 
        \[\int_{\Gamma} F \dotprod d \ell = 0\]

        En effet: 
        \[\int_{\Gamma} F \dotprod d \ell = f\left(\gamma\left(b\right)\right) - f\left(\gamma\left(a\right)\right) = f\left(\gamma\left(b\right)\right) - f\left(\gamma\left(b\right)\right) = 0\]

        En pratique, ce critère est non-trivial à appliquer, mais peut être très intéressant dans certains scénarios. 
    }
    
}


\parag{Exemple}{
    Considérons le champ vectoriel suivant, défini sur $\Omega = \mathbb{R}^2$: 
    \[F\left(x, y\right) = \left(y^2 + x^2, 2xy + e^y\right)\]
    
    Nous voulons savoir si $F$ dérive d'un potentiel. La première étape est de vérifier que le rotationnel est nul partout:
    \[\rot F = \frac{\partial }{\partial y} f_1 - \frac{\partial }{\partial x} f_2 = 2y  - 2y = 0\]
    
    La deuxième étape est de calculer. En tâtonnant, il est complètement possible de trouver:
    \[f\left(x, y\right) = y^2x + \frac{x^3}{3} + e^y\]
    
    On remarque que, toujours, il y a plusieurs réponses possibles: nous aurions pu ajouter n'importe quelle constante.

    Si nous n'étions par arrivé à trouver notre résultat en tâtonnant, nous pouvons résoudre le système donné par $\frac{\partial f}{\partial x} = y^2 + x^2$ et $\frac{\partial f}{\partial y} = 2xy + e^y$. On utilise la première équation pour trouver la fonction avec une constante qui dépend de $y$:
    \[f\left(x, y\right) = \int y^2 + x^2 dx = y^2 x + \frac{x^3}{3} + c\left(y\right)\]
    
    Remplaçons maintenant ce résultat dans notre deuxième équation, afin d'obtenir un résultat (ou une contradiction): 
    \[\frac{\partial f}{\partial y} = 2xy + e^y \iff 2xy + c'\left(y\right) = 2xy + e^y \iff c'\left(y\right) = e^y \iff c\left(y\right) = e^y + C\]
    
    Nous remarquons que $c'\left(y\right)$ et $c\left(y\right)$ ne dépendent que de $y$, donc notre résultat fonctionne (sinon nous aurions une contradiction et notre fonction ne dériverait pas d'un potentiel). Nous avons terminé, et nous trouvons finalement que: 
    \[f\left(x, y\right) = y^2 x + \frac{x^3}{3} + c\left(y\right) = y^2 x + \frac{x^3}{3} + e^y + C\]

}

\parag{Définition: Domaine étoilé}{
    Un ensemble ouvert $\Omega$ est un \important{domaine étoilé} s'il existe un centre $x_0 \in \Omega$ tel que pour tout $x \in \Omega$, le segment $\left[x_0, x\right] \subseteq \Omega$.
    
    \svghere[0.25]{DomaineEtoile.svg}
}

\parag{Définition: Simplement connexe}{
    Un ensemble ouvert qui n'a pas de trou est appelé \important{simplement connexe}.

    \subparag{Remarque 1}{
        Cette notion est uniquement importante pour le théorème qui suit. Ce n'est pas grave si nous n'en avons qu'une intuition.
    }

    \subparag{Remarque 2}{
        Tout domaine étoilé est un ensemble connexe.
    }

    \subparag{Exemples}{
        \svghere{DomainsEtoilesSimplementConvexe.svg}
    }
}


\parag{Théorème 1: Caractérisation par le rotationnel}{
    Soient $n = 2, 3$, $\Omega \subset \mathbb{R}^n$ un ensemble ouvert, et $F \in C^1\left(\Omega; \mathbb{R}^n\right)$.

    \begin{enumerate}[left=0pt]
        \item Si $F$ dérive d'un potentiel, alors $\rot F = 0$.
        \item Si $\Omega$ est simplement connexe, alors cette propriété est une condition nécessaire et suffisante. En d'autres mots, dans ce cas, $F$ dérive d'un potentiel si et seulement si $\rot F = 0$.
    \end{enumerate}

    \subparag{Preuve 1}{
        Si $F$ dérive d'un potentiel, alors il existe une fonction $f$ telle que $F = \nabla f$, et donc: 
        \[\rot F = \rot \nabla f = 0\]
    }

    \subparag{Preuve 2}{
        Nous allons seulement prouver le cas plus faible où $\Omega$ est un domaine étoilé, pour simplifier notre preuve.

        On définit $f$ telle que $f\left(x_0\right) = 0$ et 
        \[f\left(x\right) \int_{\left[x_0, x\right]} F \dotprod d \ell \]
        où $\left[x_0, x\right]$ est le segment partant de $x_0$ vers $x$. Nous pouvons poser la paramétrisation suivante pour cette courbe: 
        
        \[\begin{split}
        \gamma: \left[0, 1\right] &\longmapsto \Gamma = \left[x_0, x\right] \\
        t &\longmapsto x_0 + t\left(x - x_0\right)
        \end{split}\]

        Alors, nous avons: 
        \autoeq{\unexpanded{\frac{\partial f}{\partial x_i} \left(x\right) = \frac{\partial }{\partial x_i} \int_{0}^{1} F\left(x_0 + t\left(x - x_0\right)\right) \dotprod \left(x - x_0\right) dt = \int_{0}^1 \frac{\partial}{\partial x_i} F\left(x_0 + t\left(x - x_0\right)\right) \dotprod \left(x - x_0\right)dt }}

        Nous pouvons démontrer que cela donne:
        \autoeq{\unexpanded{\int_{0}^{1} \frac{\partial }{\partial x_i} F\left(x_0 + t\left(x - x_0\right)\right) \dotprod \left(x - x_0\right) dt = \left[t F_i \left(x_0 + t\left(x - x_0\right)\right)\right]_0^1 = F_i\left(x\right)}}
        
        Il suit donc bien que $\nabla f = F$.

        \qed
    }
}

\parag{Théorème 2: Caractérisation par les courbes}{
    Soient $\Omega \subset \mathbb{R}^n$ un ensemble ouvert (peu importe le $n$), et $F \in C^{0}\left(\Omega; \mathbb{R}^n\right)$.

    Alors, les conditions suivantes sont équivalentes:
    \begin{enumerate}[left=0pt]
        \item $F$ dérive d'un potentiel.
        \item $\displaystyle \int_{\Gamma} F \dotprod d \ell = 0$ pour toute courbe régulière par morceau simple fermée $\Gamma \subset \Omega$.
        \item $\displaystyle \int_{\Gamma_1} F \dotprod d \ell = \int_{\Gamma_2} F \dotprod d \ell$ pour toutes courbes régulières par moreaux simple avec les mêmes extrémités.
    \end{enumerate}

    \subparag{Preuve $2 \iff 3$}{
        Considérons le cas où l'intégrale selon $\Gamma_1$ est égale à l'intégrale selon $\Gamma_2$ (deux courbes avec les mêmes extrémités). Alors, en posant $\Gamma = -\Gamma_1 \cup \Gamma_2$, une courbe fermée:
        \autoeq{\unexpanded{\int_{\Gamma_1} = \int_{\Gamma_2} = \int_{\Gamma_2} + \int_{\Gamma_1} - \int_{\Gamma_1} = \int_{\Gamma_2} + \int_{-\Gamma_1} + \int_{\Gamma_1} = \int_{\Gamma} + \int_{\Gamma_1} \iff \int_{\Gamma} = 0}}

        Cela nous dit donc bien que n'importe quelle intégrale sur une courbe fermée est nulle
        
        Dans l'autre sens, nous pouvons séparer n'importe quel courbe régulière par morceaux simple fermée en deux sous-courbes telles que l'intégrale sur ces deux chemins seront toujours égales, par une intuition similaire.
        
        \svghere[0.5]{CaracterisationCourbeFermees.svg}
    }

    \subparag{Preuve $1 \implies 2$}{
        Nous savons que c'est vrai par l'observation que nous avons faite quand nous cherchions à comprendre comment savoir si une fonction dérive d'un potentiel.
    }

    \subparag{Preuve $3 \implies 1$}{
        Posons la fonction suivante: 
        \[f\left(x\right) = \int_{x_0 \to x} F \dotprod d \ell\]
        où $x_0$ est fixé, et $x_0 \to x$ représente n'importe quelle courbe allant de $x_0$ à $x$.

        Par hypothèse, l'intégrale selon n'importe quelle courbe donne le même résultat, ainsi:
        \autoeq{\unexpanded{\frac{\partial f}{\partial x_i} \left(x\right) = \lim_{h \to 0} \frac{1}{h} \left(\int_{x_0 \to x + h e_i} F \dotprod d \ell - \int_{x_0 \to x} F \dotprod d \ell \right) = \lim_{h \to 0} \frac{1}{h} \left(\int_{x_0 \to x + h e_i} F \dotprod d \ell + \int_{x \to x_0} F \dotprod d \ell \right) = \lim_{h \to 0} \frac{1}{h} \left(\int_{x \to x + h e_i} F \dotprod d \ell\right)}}
        
        Puisque cette intégrale ne dépend pas du chemin, alors nous pouvons choisir comme chemin le segment $\left[x, x + he_i\right]$, avec $\gamma\left(t\right) = x + \left(x + he_i - x\right)t$: 
        \autoeq{\unexpanded{\lim_{h \to 0} \frac{1}{h}\int_{x \to x + e_i h} F \dotprod d \ell = \lim_{h \to 0} \int_{0}^{1} F\left(x + e_i h\right) \dotprod e_i h dt = \int_{0}^{1} \lim_{h \to 0} \frac{1}{h} h F_i\left(x + h e_i\right) dt = F_1\left(x\right)}}

        Ainsi, on obtient que $\frac{\partial }{\partial x_i} f = F_i\left(x\right)$, et donc que $\nabla f = F$.
        
        \qed
    }
}

\end{document}
