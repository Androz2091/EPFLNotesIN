% !TeX program = lualatex
% Using VimTeX, you need to reload the plugin (\lx) after having saved the document in order to use LuaLaTeX (thanks to the line above)

\documentclass[a4paper]{article}

% Expanded on 2022-10-24 at 08:15:45.

\usepackage{../../style}

\title{Analyse}
\author{Joachim Favre}
\date{Lundi 24 octobre 2022}

\begin{document}
\maketitle

\lecture{5}{2022-10-24}{Théorème de la divergence en deux dimensions}{
\begin{itemize}[left=0pt]
    \item Preuve d'un corollaire du théorème de Green permettant de calculer l'aire d'un domaine.
    \item Définition de la normale extérieure, et preuve d'une proposition permettant de la calculer.
    \item Explication et preuve du théorème de la divergence dans $\mathbb{R}^2$.
    \item Utilisation du théorème de la divergence dans $\mathbb{R}^2$ pour calculer une intégrale.
\end{itemize}

}

\begin{parag}{Corolaire: Formule de l'Aire}
    Soit $\Omega$ un domaine régulier avec le bord $\partial \Omega$ orienté positivement. Nous avons alors:
    \[\text{Aire}\left(\Omega\right) = \iint_{\Omega}1 dxdy = \int_{\partial \Omega} \left(0, x\right) \dotprod d \ell = \int_{\partial \Omega} \left(-y, 0\right) \dotprod d \ell \]
    
    \begin{subparag}{Preuve}
        Nous avons bien que: 
        \[1 = \rot\left(0, x\right) = \rot\left(-y, 0\right)\]
        
        Puis, notre résultat découle directement par le théorème de Green.

        \qed
    \end{subparag}
\end{parag}

\begin{parag}{Exemple}
    Nous voulons vérifier le théorème de Green pour le domaine suivant: 
    \[\Omega = \left\{\left(x, y\right) \in \mathbb{R}^2 | x, y > 0 \text{ et } 4 < x^2 + y^2 < 9\right\}\]
    
    Et avec le champ vectoriel suivant: 
    \[F\left(x, y\right) = \left(\frac{x}{x^2 + y^2}, \frac{1}{x^2 + y^2}\right)\]
    
    Nous remarquons que notre domaine a l'allure d'un cercle, ainsi utilisons des coordonnées polaires: 
    \[\Omega = \left\{\left(r\cos\left(\theta\right), r\sin\left(\theta\right)\right) | \theta \in \left]0, \frac{\pi}{2}\right[ , r \in \left]2, 3\right[ \right\}\]

    De plus, nous trouvons que le rotationnel de notre fonction est donné par: 
    \[\rot F = \frac{-2x + 2xy}{\left(x^2 + y^2\right)^2}\]
    
    Calculons maintenant notre première intégrale:
    \autoeq{\unexpanded{\iint_{\Omega} \rot F dx dy = \int_{2}^{3} \int_{0}^{\frac{\pi}{2}} \rot F \left(r \cos\left(\theta\right), r\sin\left(\theta\right)\right)\underbrace{r}_{\text{Jacobien}} d \theta dr = \int_{2}^{3} \int_{0}^{\frac{\pi}{2}} \frac{-2r \cos\left(\theta\right) + 2r^2 \cos\left(\theta\right)\sin\left(\theta\right)}{r^4} r d \theta dr = \int_{2}^{3} \left(\frac{-2}{r^2} \left[\sin\left(\theta\right)\right]_{0}^{\frac{\pi}{2}} + \frac{2}{r} \left[\frac{\sin^2\left(\theta\right)}{2}\right]_{0}^{\frac{\pi}{2}}\right) dr = \int_{2}^{3} \left(\frac{-2}{r^2} + \frac{1}{r}\right) dr = \left[\frac{2}{r} + \log\left(r\right)\right]_2^3 = \frac{2}{3} + \log\left(3\right) - 1 - \log\left(2\right) = \frac{-1}{3} + \log\left(\frac{3}{2}\right)}}
    
    Nous pouvons ensuite calculer notre deuxième intégrale. Pour y arriver, nous devons définir une paramétrisation de $\partial \Omega$, de manière positive:
    \begin{center}
    \begin{minipage}{0.45\textwidth}
    \[\begin{split}
    \gamma_1: \left[0, \frac{\pi}{2}\right] &\longmapsto \Gamma_1 \\
    t &\longmapsto 3\left(\cos\left(t\right), \sin\left(t\right)\right)
    \end{split}\]
    \end{minipage}
    \hfill
    \begin{minipage}{0.45\textwidth}
    \[\begin{split}
    \gamma_2: \left[2, 3\right] &\longmapsto -\Gamma_2 \\
    t &\longmapsto \left(0, t\right)
    \end{split}\]
    \end{minipage}
    \end{center}
    
    \begin{center}
    \begin{minipage}{0.45\textwidth}
    \[\begin{split}
    \gamma_3: \left[0, \frac{\pi}{2}\right] &\longmapsto -\Gamma_3 \\
    t &\longmapsto 2\left(\cos\left(t\right), \sin\left(t\right)\right)
    \end{split}\]
    \end{minipage}
    \hfill
    \begin{minipage}{0.45\textwidth}
    \[\begin{split}
    \gamma_4: \left[2, 3\right] &\longmapsto \Gamma_4 \\
    t &\longmapsto \left(t, 0\right)
    \end{split}\]
    \end{minipage}
    \end{center}

    \svghere[0.5]{TheoremGreenParametrisationExemple.svg}

    Or, nous savons que: 
    \[\partial \Omega = \Gamma_1 \cup \Gamma_2 \cup \Gamma_3 \cup \Gamma_4 = \Gamma_1 \cup -\left(-\Gamma_2\right) \cup -\left(-\Gamma_3\right) \cup \Gamma_4\]
    
    Et ainsi: 
    \autoeq{\unexpanded{\int_{\partial \Omega} F \dotprod d \ell = \int_{\Gamma_1} F \dotprod d \ell - \int_{-\Gamma_2} F \dotprod d \ell - \int_{-\Gamma_3} F \dotprod d \ell + \int_{\Gamma_4} F \dotprod d \ell = \int_{0}^{\frac{\pi}{2}} \left(\frac{3\cos\left(t\right)}{9}, \frac{1}{9}\right)\dotprod 3\left(-\sin\left(t\right), \cos\left(t\right)\right)dt - \int_{2}^{3} \left(0, \frac{1}{t^2}\right) \dotprod \left(0, 1\right)dt \fakeequal - \int_{0}^{\frac{\pi}{2}} \left(\frac{2\cos\left(t\right)}{4}, \frac{1}{4}\right) \dotprod 2\left(-\sin\left(t\right), \cos\left(t\right)\right)dt + \int_{2}^{3} \left(\frac{t}{t^2}, \frac{1}{t^2}\right) \dotprod \left(1, 0\right)dt = \ldots = -\frac{1}{3} + \log\left(\frac{3}{2}\right)}}
    comme attendu.
\end{parag}

\subsection{Théorème de la divergence sur le plan}
\begin{parag}{Définition: Normale extérieure}
    Soit $\Omega \subset \mathbb{R}^2$ un domaine régulier et $P \in \partial \Omega$ un point de la frontière de $\Omega$. Le vecteur $\nu_P \in \mathbb{R}^2$ est la \important{normale extérieure} à $\Omega$ en $P$ si:
    \begin{enumerate}
        \item $\left\|\nu_P\right\| = 1$
        \item Si $\gamma: \left[a, b\right]\mapsto \partial \Omega$ est une paramétrisation telle que $\gamma\left(t_0\right)$ existe (où $t_0$ est tel que $\gamma\left(t_0\right) = P$), alors $\gamma'\left(t_0\right) \dotprod \nu_P = 0$.
        \item Pour tout $\epsilon > 0$ suffisamment petit, alors $P + \epsilon \nu_P \not\in \Omega$
    \end{enumerate}

    \svghere[0.5]{NormeExterieure.svg}
    
    \begin{subparag}{Intuition}
        Le point 2 dit que notre vecteur est orthogonal au bord, et le point 3 nous donne la direction de ce vecteur (parmi les deux possibles).
    \end{subparag}
    
    \begin{subparag}{Remarque}
        À un point où la dérivée n'existe pas (dans une courbe simple par morceaux), nous ne considérons pas le deuxième point, ce qui nous donne plein de vecteurs normaux extérieurs.
    \end{subparag}
    
    
\end{parag}

\begin{parag}{Proposition}
    Si $\Omega$ est un domaine régulier de bord régulier orienté positivement, et $\gamma: \left[a, b\right] \mapsto \partial \Omega$ est une paramétrisation régulière, alors pour tout $t \in\left[a, b\right]$, nous avons: 
    \[\nu_{\gamma\left(t\right)} = \frac{\left(\gamma_2'\left(t\right), - \gamma_1'\left(t\right)\right)}{\left\|\gamma'\left(t\right)\right\|}\]
        
    \begin{subparag}{Preuve}
        Pour le point 1, nous avons bien: 
        \[\left\|\nu_{\gamma\left(t\right)}\right\| = \frac{\left\|\left(\gamma_2', -\gamma_1'\right)\right\|}{\left\|\left(\gamma_1', \gamma_2'\right)\right\|} = 1\]

        Pour le point 2, nous trouvons: 
        \[\nu_{\gamma\left(t\right)} \dotprod \gamma'\left(t\right) = \frac{1}{\left\|\gamma'\right\|} \left(\gamma_2', -\gamma_1'\right) \dotprod \left(\gamma_1', \gamma_2'\right) = \frac{1}{\left\|\gamma'\right\|} \left(\gamma_2' \gamma_1' - \gamma_1' \gamma_2'\right) = 0\]

        Pour le point 3, nous pouvons appliquer une translation et une rotation à notre domaine, afin d'avoir $\gamma'\left(t\right) = \left(0, y\right)$, où $y > 0$. Ainsi, cela nous donne le vecteur: 
        \[\nu_{\gamma\left(t\right)} = \frac{\left(y, 0\right)}{y} = \left(1, 0\right)\]
        qui pointe bien vers l'extérieur (comme on peut le voir sur l'image ci-dessous).

        \svghere[0.65]{PreuvePropositionNormaleExterieure.svg}
    \end{subparag}
\end{parag}

\begin{parag}{Théorème de la divergence dans $\mathbb{R}^2$}
    Soit $\Omega \subset \mathbb{R}^2$ régulier, $F \in C^1\left(\Omega;\mathbb{R}^2\right)$ un champ vectoriel, et $\nu_P$ la normale extérieure à $\Omega$ en $P \in \partial \Omega$. Alors, nous avons: 
    \[\iint_{\Omega} \Div F dxdy = \int_{\partial \Omega} \left(F \dotprod \nu\right) d \ell \]
    
    \begin{subparag}{Remarque 1}
        La quantité $\int_{\partial \Omega} F \dotprod \nu d \ell $ s'appelle le \important{flux} de $F$ à travers $\partial \Omega$.
    \end{subparag}

    \begin{subparag}{Remarque 2}
        Si $\gamma: \left[a, b\right] \mapsto \partial\Omega$ est une paramétrisation positive de $\partial\Omega$, alors: 

        \autoeq{\unexpanded{\int_{\partial \Omega} F \dotprod \nu d \ell = \int_{a}^{b} F\left(\gamma\left(t\right)\right) \dotprod \nu_{\gamma\left(t\right)} \left\|\gamma'\left(t\right)\right\|dt = \int_{a}^{b} F\left(\gamma\left(t\right)\right) \dotprod \left(\gamma'_2\left(t\right), -\gamma_1'\left(t\right)\right) \frac{\left\|\gamma'\left(t\right)\right\|}{\left\|\gamma'\left(t\right)\right\|} dt}}
        
        Nous avons donc trouvé que: 
        \[\int_{\partial \Omega} F \dotprod \nu d \ell = \int_{a}^{b} F\left(\gamma\left(t\right)\right) \dotprod \left(\gamma_2'\left(t\right), -\gamma_1'\left(t\right)\right)dt\]
    \end{subparag}

    \begin{subparag}{Intuition}
        Dans l'idée $\iint_{\Omega} \Div F dxdy$ est ``ce qui sort'' de $\Omega$, et $\int_{\partial\Omega} F \dotprod \nu d \ell $ est ``ce qui passe à travers'' $\partial \Omega$. Tout ce qui sort doit passer à travers la frontière, donc ces deux valeurs doivent être égales.
    \end{subparag}
    
    \begin{subparag}{Preuve}
        Si $F = \left(F_1, F_2\right)$, alors:
        \[\Div F = \rot\left(-F_2, F_1\right)\]

        Nous obtenons ainsi, par le théorème de Green: 
        \[\iint_{\Omega} \Div F dxdy = \iint_{\Omega} \rot\left(-F_2, F_1\right)dxdy = \int_{\partial \Omega} \left(-F_2, F_1\right) \dotprod d \ell \]
        
        Prenons maintenant une paramétrisation $\gamma: \left[a, b\right] \mapsto \partial \Omega$. Cela nous donne que notre intégrale est égale à: 
        \autoeq[s]{\unexpanded{\int_{a}^{b} \left(-F_2\left(\gamma\left(t\right)\right), F_1\left(\gamma\left(t\right)\right)\right) \dotprod \left(\gamma_1'\left(t\right), \gamma_2'\left(t\right)\right)dt = \int_{a}^{b} \left(F_1, F_2\right) \dotprod \left(\gamma_2'\left(t\right), -\gamma_1'\left(t\right)\right) = \int_{\partial\Omega} \left(F \dotprod \nu\right)d \ell}}
        par notre proposition ci-dessus.

        \qed
    \end{subparag}
\end{parag}

\begin{parag}{Exemple 1}
    Nous voulons calculer $\int_{\Gamma} g\left(x, y\right) d \ell $ où $\Gamma$ est le cercle de rayon 3 centré en 0, et: 
    \[g\left(x, y\right) = x\sin^4\left(y\right) + ye^{x^2 + 2x}\]

    Avec la définition des intégrales curvilignes, ce calcul est compliqué. Cependant, nous pouvons plutôt faire les observations suivantes. Comme $\Gamma$ est le cercle de rayon 3 centre en 0, nous trouvons que $\Gamma = \partial\Omega$, où: 
    \[\Omega = \left\{\left(x, y\right) \in \mathbb{R}^2 | x^2 + y^2 < 9\right\}\]
    
    \svghere[0.5]{Exemple1TheoremDivergence.svg}

    Nous pouvons voir sur l'image ci-dessus que la normale extérieure au point $P\left(x, y\right) \in \partial \Omega$ est parallèle au vecteur allant de l'origine vers ce point. Ainsi: 
    \[\nu_{\left(x, y\right)} = \frac{1}{\sqrt{x^2 + y^2}} \left(x, y\right) = \frac{1}{r} \left(x,y\right) = \frac{1}{3} \left(x, y\right)\]
    
    Ceci est extrêmement intéressant car nous pouvons écrire notre intégrale comme:
    \[\int_{\Gamma} g d \ell = \int_{\partial\Omega} \underbrace{\left(\sin^4\left(y\right), e^{x^2+ 2x}\right)}_{F} \dotprod \frac{1}{3} \left(x, y\right) 3 d \ell = 3\int_{\partial \Omega} F \dotprod \nu d \ell \]
     
    Or, nous pouvons voir que: 
    \[\Div F = \partial_x \left(\sin^4 \left(y\right)\right) + \partial_y\left(e^{x^2 + 2x}\right) = 0\]
    
    Ce qui nous permet de simplifier notre intégrale, par le théorème de la divergence: 
    \[\int_{\Gamma} g d \ell = 3\int_{\partial \Omega}F \dotprod \nu d \ell = 3\iint_{\Omega} \Div F dxdy = 0\]

    \begin{subparag}{Remarque}
        Cette méthode est à retenir pour le calcul d'intégrales.
    \end{subparag}
    
\end{parag}




\end{document}
