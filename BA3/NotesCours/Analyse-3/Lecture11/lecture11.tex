% !TeX program = lualatex
% Using VimTeX, you need to reload the plugin (\lx) after having saved the document in order to use LuaLaTeX (thanks to the line above)

\documentclass[a4paper]{article}

% Expanded on 2022-12-05 at 08:18:29.

\usepackage{../../style}

\title{Analyse}
\author{Joachim Favre}
\date{Lundi 05 décembre 2022}

\begin{document}
\maketitle

\lecture{11}{2022-12-05}{Transformées de Fourier}{
\begin{itemize}[left=0pt]
    \item Définition des transformées de Fourier.
    \item Définition des transformées inverses de Fourier, et explication du théorème de la formule d'inversion.
    \item Explication des propriétés des transformées de Fourier.
\end{itemize}

}

\section{Transformée de Fourier}
\parag{But}{
    Le plus gros problème avec les séries de Fourier c'est que nous sommes obligés de travailler avec des fonctions périodiques. De plus, nous avons toujours cet objectif de travailler avec des fréquences.

    L'idée est donc de construire une fonction $\hat{f}$ à partir de $f$ telle que $\hat{f}\left(\alpha\right)$ représente ``la quantité de fréquence $\alpha$ dans $f$''.

    Pour les séries de Fourier, nous avions défini un produit scalaire pour des fonctions $T$-périodiques, puis nous avions trouvé que nous pouvions faire une décomposition sur les vecteurs de bases, comme pour $\mathbb{R}^n$: 
    \[f\left(x\right) = F f\left(x\right) = \sum_{n=-\infty}^{\infty} \left\langle f, e^{i\frac{2\pi}{T}nx} \right\rangle e^{i \frac{2\pi}{T}nx} \]
    où $\left\langle f, e^{i\frac{2\pi}{T}nx} \right\rangle$  représente la ``quantité de $e^{i\frac{2\pi}{T}nx}$ dans $f$''. Cependant, puisque $e^{i\alpha x}$ est une fonction typique de fréquence $\alpha$, $\left\langle f, e^{i \frac{2\pi}{T}nx} \right\rangle$ est ``la quantité de fréquence $\frac{2\pi}{T}$ dans $f$''.

    Nous voulons généraliser cette idée à des fonctions non périodiques (avec $T \to \infty$), et à des fréquences arbitraires.
}

\parag{Définition: Transformée de Fourier}{
    Soit $f: \mathbb{R} \mapsto \mathbb{R}$  (ou $f: \mathbb{R} \mapsto \mathbb{C}$) telle que: 
    \[\int_{-\infty}^{\infty} \left|f\left(x\right)\right|dx < +\infty\]
    
    La \important{transformée de Fourier} de $f$ est la fonction complexe $\mathcal{F}f : \mathbb{R} \mapsto \mathbb{C}$ donnée par: 
    \[\mathcal{F} f\left(\alpha\right) = \frac{1}{\sqrt{2\pi}} \int_{-\infty}^{\infty} f\left(x\right)e^{-i\alpha x}dx\]
    
    \subparag{Remarque 1}{
        On note souvent $\hat{f}\left(\alpha\right) = \mathcal{F}f\left(\alpha\right)$.
    }

    \subparag{Remarque 2}{
        La condition $\int_{-\infty}^{\infty} \left|f\left(x\right)\right|dx < \infty$ est un peu ennuyeuse. Cependant, nous pouvons nous en débarrasser en utilisant la théorie des distributions de Schwartz (que nous ne verrons pas dans ce cours).
    }
    
    \subparag{Intuition}{
        Nous pouvons voir $\mathcal{F} f\left(\alpha\right)$ comme ``la quantité de $\alpha$ dans $f$.''
    }
}

\parag{Exemple 1}{
    Considérons la fonction suivante:
    \begin{functionbypart}{f\left(x\right)}
        e^{i \beta x}, \mathspace \text{si } x \in \left[-L, L\right]  \\
        0, \mathspace \text{sinon}
    \end{functionbypart}

    En appliquant la définition des transformées de Fourier, nous avons: 
    \[\mathcal{F} f\left(\alpha\right) = \frac{1}{\sqrt{2\pi}} \int_{-\infty}^{\infty} f\left(x\right) e^{-i\alpha x} dx = \frac{1}{\sqrt{2\pi}} \int_{-L}^{L} e^{i \left(\beta - \alpha\right)x}dx\]
    
    Si $\alpha = \beta$ alors on trouve: 
    \[\mathcal{F} f\left(\beta\right) = \frac{1}{\sqrt{2\pi}} \int_{-L}^{L} dx = \frac{2L}{\sqrt{2\pi}}\]
    
    Si $\alpha \neq \beta$, alors: 
    \autoeq{\unexpanded{\mathcal{F} f\left(\alpha\right) = \frac{1}{\sqrt{2\pi}} \int_{-L}^{L}  e^{i\left(\beta - \alpha\right)}dx = \frac{1}{\sqrt{2\pi} \left(\beta - \alpha\right)i} \left[e^{i\left(\beta - \alpha\right) L} - e^{-i\left(\beta - \alpha\right)L} \right] = \frac{2}{\sqrt{2\pi} \left(\beta - \alpha\right)} \sin\left(\left(\beta - \alpha\right)L\right)}}

    Regardons ce qui se passe quand $\alpha$ est proche de $\beta$: 
    \[\mathcal{F} f \left(\alpha\right) = \frac{2}{\sqrt{2\pi} \left(\beta -\alpha\right)} \sin\left(\left(\beta -\alpha\right)L\right) = \frac{2L}{\sqrt{2\pi}} \underbrace{\frac{\sin\left(\left(\beta - \alpha\right)L\right)}{\left(\beta - \alpha\right) L}}_{\to 1} \to \frac{2L}{2\pi}\]
    
    Et maintenant quand $\alpha$ est loin de $\beta$:
    \[\mathcal{F} f\left(\alpha\right) = \underbrace{\frac{1}{\beta - \alpha}}_{\to 0} \cdot  \underbrace{\frac{2 \sin\left(\left(\beta - \alpha\right)L\right)}{\sqrt{2\pi}}}_{\text{borné}} \to 0\]

    Nous avons donc bien trouvé que notre transformée de Fourier met en évidence la fréquence $\beta$.
}

\parag{Exemple 2}{
    Considérons la fonction suivante:
    \begin{functionbypart}{f\left(x\right)}
    \gamma, \mathspace \text{si } x \in \left[a, b\right]  \\
    0, \mathspace \text{sinon}
    \end{functionbypart}

    Nous pouvons trouver sa transformée de Fourier: 
    \[\hat{f}\left(\alpha\right) = \mathcal{F}f\left(\alpha\right) = \frac{1}{\sqrt{2\pi}} \int_{-\infty}^{\infty} f\left(x\right)e^{-i\alpha x} dx = \frac{1}{\sqrt{2\pi}} \int_{a}^{b} \gamma e^{-i\alpha x} dx = \]

    Cela nous donne ainsi:
    \begin{functionbypart}{\hat{f}\left(\alpha\right)}
         \frac{-\gamma}{\sqrt{2\pi}} \frac{e^{-i\alpha b} - e^{-i \alpha a}}{i \alpha}, \mathspace \text{si } \alpha \neq 0    \\
         \frac{\gamma}{\sqrt{2\pi}}\left(b - a\right), \mathspace \text{si } \alpha = 0
    \end{functionbypart}
}

\parag{Exemple 3}{
    Prenons la fonction suivante: 
    \[f\left(x\right) = e^{-\left|x\right|}\]

    Nous trouvons alors que: 
    \autoeq{\unexpanded{\hat{f}\left(\alpha\right) = \frac{1}{\sqrt{2\pi}} \int_{-\infty}^{\infty} f\left(x\right) e^{-i \alpha x}dx = \frac{1}{\sqrt{2\pi}} \int_{-\infty}^{\infty} e^{-\left|x\right| - i \alpha x} dx = \frac{1}{\sqrt{2\pi}} \int_{-\infty}^{0} e^{\left(1 - i\alpha\right)x}dx + \frac{1}{\sqrt{2\pi}} \int_{0}^{\infty} e^{\left(-1 - i\alpha\right)x}dx}}

    Ce qui nous donne que:
    \autoeq{\unexpanded{\hat{f}\left(\alpha\right) = \frac{1}{\sqrt{2\pi}} \left[\frac{e^{\left(1 - i\alpha\right)x}}{1 - i \alpha}\right]_{-\infty}^0 + \frac{1}{\sqrt{2\pi}} \left[\frac{e^{-\left(1 + i \alpha\right)x}}{-\left(1 + i \alpha\right)}\right]_0^{\infty} = \frac{1}{\sqrt{2\pi}} \left(\frac{1}{1 - i\alpha} - 0 + 0 - \frac{1}{-\left(1 + i\alpha\right)}\right) = \frac{1}{\sqrt{2\pi}} \frac{1 + i\alpha + 1 - i\alpha}{1 + \alpha ^2} = \frac{2}{\sqrt{2\pi}\left(1 + \alpha ^2\right)}}}
    
}

\parag{Définition: Transformée de Fourier inverse}{
    Soit $g: \mathbb{R} \mapsto \mathbb{C}$, une fonction telle que: 
    \[\int_{-\infty}^{\infty} \left|g\left(\alpha\right)\right| d\alpha < +\infty\]
    
    La \important{transformée de Fourier inverse} de $g$, $\mathcal{F}^{-1} g : \mathbb{R} \mapsto \mathbb{C}$, est définie par: 
    \[\mathcal{F}^{-1} g\left(x\right) = \frac{1}{\sqrt{2\pi}} \int_{-\infty}^{\infty} g\left(x\right) e^{i \alpha x} d\alpha\]

    \subparag{Observation}{
        C'est la même formule que la transformée de Fourier classique, mais sans le moins dans l'exponentielle.
    }
    
}

\parag{Théorème: Formule d'inversion}{
    Soit $f: \mathbb{R} \mapsto \mathbb{C}$ une fonction telle que: 
    \[\int_{-\infty}^{\infty} \left|f\left(x\right)\right| dx < \infty, \mathspace \int_{-\infty}^{\infty} \left|\hat{f}\left(\alpha\right)\right|d\alpha < \infty\]
    où $\hat{f}\left(\alpha\right) = \mathcal{F} f\left(\alpha\right)$.

    Alors: 
    \[\mathcal{F}^{-1} \left(\mathcal{F} \left(f\right)\right) \left(x\right) = \left(\mathcal{F}^{-1} \hat{f}\right)\left(x\right) = f\left(x\right)\]
    
    \subparag{Remarque}{
        Il suit aussi que: 
        \[\mathcal{F}\left(\mathcal{F}^{-1}\left(\hat{f}\right)\right)\left(\alpha\right) = \mathcal{F}\left(f\right)\left(\alpha\right) = \hat{f}\left(\alpha\right)\]
    }
}

\parag{Exemple}{
    Soit $f\left(x\right) = e^{-\left|x\right|}$. Nous avions trouvé que: 
    \[\hat{f}\left(\alpha\right) = \frac{2}{\sqrt{2\pi} \left(1 + \alpha ^2\right)}\]
    
    Commençons par vérifier que la transformée de Fourier est intégrable en valeur absolue sur $\left]-\infty, \infty\right[ $: 
    \autoeq{\unexpanded{\int_{-\infty}^{\infty} \left|\hat{f}\left(\alpha\right)\right|d\alpha = \frac{2}{\sqrt{2\pi}} \int_{-\infty}^{\infty} \frac{1}{1 + \alpha ^2} d\alpha = \frac{2}{\sqrt{2\pi}} \left[\arctan\left(\alpha\right)\right]_{-\infty}^{\infty} = \sqrt{2\pi} < \infty}}
    
    Nous pouvons donc appliquer le théorème d'inversion. 

    Notre intégrale ci-dessus nous donne déjà que 
    \[\mathcal{F}^{-1} \hat{f}\left(0\right) = \frac{1}{\sqrt{2\pi}} \int_{-\infty}^{\infty} \hat{f}\left(\alpha\right) d \alpha = \frac{\sqrt{2\pi}}{\sqrt{2\pi}} = 1 = f\left(0\right)\]

    Maintenant, pour $x \neq 0$: 
    \[\mathcal{F}^{-1} \left(\frac{2}{\sqrt{2\pi} \left(1 + \alpha ^2\right)}\right)\left(x\right) = \frac{1}{\pi} \int_{-\infty}^{\infty} \frac{1}{1 + \alpha ^2}  e^{i \alpha x} dx\]
    
    Cependant, puisque $\frac{1}{1 + \alpha ^2}$ est pair et $\sin\left(\alpha x\right)$ est impair, leur produit est impair et donc la composante complexe de cette intégrale donne 0. Nous trouvons pour la composante réelle que: 
    \[\mathcal{F}^{-1} \left(\hat{f}\right)\left(x\right) = \frac{1}{\pi} \int_{-\infty}^{\infty} \frac{1}{1 + \alpha ^2} \cos\left(\alpha x\right) dx\]
    
    Maintenant que nous avons ce résultat, nous pouvons l'utiliser pour calculer une intégrale. Puisque $\mathcal{F}^{-1}\left(\hat{f}\right)\left(1\right) = f\left(1\right)$, nous trouvons que: 
    \[e^{-\left|1\right|} = \frac{1}{\pi} \int_{-\infty}^{\infty} \frac{\cos\left(\alpha\right)}{1 + \alpha ^2} d \alpha \implies \int_{-\infty}^{\infty} \frac{\cos\left(x\right)}{1 + x^2} dx = \frac{\pi}{e}\]

    Ce résultat est très difficile à obtenir autrement.

    \subparag{Remarque personnelle}{
        Le résultat de cette intégrale est absolument magnifique, et je pense qu'il faut prendre quelques secondes pour l'apprécier à sa juste valeur!
    }
    
}

\parag{Propriétés}{
    Soient $f, g : \mathbb{R} \mapsto \mathbb{C}$ telles que: 
    \[\int_{-\infty}^{\infty} \left|f\left(x\right)\right|dx < \infty, \mathspace \int_{-\infty}^{\infty} \left|g\left(x\right)\right|dx < \infty\]
    
    Alors, nous avons les propriétés suivantes.

    \subparag{Continuité}{
        $\mathcal{F}f\left(\alpha\right)$ est continue en $\alpha$.
    }

    \subparag{Linéarité}{
        La transformée de Fourier est linéaire:
        
        \[\mathcal{F}\left(a f + b g\right) = a \mathcal{F} f + b \mathcal{F}g, \mathspace \forall a, b \in \mathbb{C}\]
    }

    \subparag{Décalage}{
        Faire un décalage est équivalent à multiplier par une exponentielle:
        \[ \mathcal{F} \left(f\left(x + b\right)\right)\left(\alpha\right) = e^{ib \alpha} \mathcal{F}\left(f\right)\left(\alpha\right)\]
        \[\mathcal{F}\left(f\right)\left(\alpha + \beta\right) = \mathcal{F}\left(e^{-i \beta x} f\left(x\right)\right)\]
    }
    
    \subparag{Dilatation}{
        Faire une dilatation et équivalent à faire une autre dilatation et multiplier par une constante:
        \[ \mathcal{F}\left(f\left(cx\right)\right) = \frac{1}{\left|c\right|} \mathcal{F}\left(f\right)\left(\frac{\alpha}{c}\right), \mathspace \forall c \neq 0\]   
        \[\mathcal{F}\left(f\right)\left(\gamma \alpha\right) = \mathcal{F}\left(\frac{1}{\left|\gamma\right|} f\left(\frac{x}{\gamma}\right)\right), \mathspace \forall \gamma \neq 0\]
    }

    \subparag{Dérivée}{
        Si de plus $f \in C^1\left(\mathbb{R}\right)$, et $\int_{-\infty}^{\infty} \left|f'\left(x\right)\right|dx < \infty$, alors:
        \[\mathcal{F}\left(\frac{d}{dx} f\right)\left(\alpha\right) = \mathcal{F}\left(f'\right)\left(\alpha\right) = i \alpha \mathcal{F}\left(f\right)\left(\alpha\right)\]
    }

    \subparag{Dérivée de la transformée}{
        Si de plus $f \in C^1\left(\mathbb{R}\right)$, et est telle que $\int_{-\infty}^{\infty} \left|x f\left(x\right)\right| < \infty$, alors:
        \[\frac{d}{d \alpha}  \mathcal{F}\left(f\right)\left(\alpha\right) = \left(\mathcal{F} f\right)'\left(\alpha\right) = \mathcal{F}\left(-ix f\left(x\right)\right)\left(\alpha\right)\]
    }
    

    \subparag{Remarque}{
        Nous pouvons aussi calculer la $n$-ème dérivée, si toutes les hypothèses sont correctement vérifiées: 
        \[\mathcal{F} \left(f^{\left(n\right)}\right)\left(\alpha\right) = \left(i\alpha\right)^n \mathcal{F}\left(f\right)\left(\alpha\right), \mathspace \left(\mathcal{F} f\right)^{\left(n\right)} \left(\alpha\right) = \mathcal{F}\left(\left(-ix\right)^n f\left(x\right)\right)\left(\alpha\right)\]
    }
    
    \subparag{Preuve}{
        Démontrons uniquement les propriétés de dérivation.
        
        Pour la première dérivée, nous pouvons faire une intégrale par parties:
        \autoeq{\unexpanded{\mathcal{F}\left(f'\right)\left(\alpha\right) = \frac{1}{\sqrt{2\pi}}\int_{-\infty}^{\infty} f'\left(x\right) e^{-i \alpha x} dx = \frac{1}{\sqrt{2\pi}} \left[f\left(x\right) e^{-i \alpha x}\right]_{-\infty}^{\infty} - \frac{1}{\sqrt{2\pi}} \int_{-\infty}^{\infty} f\left(x\right) e^{-i \alpha x} \left(-i \alpha\right) dx}}
        
        Puisque $f\left(x\right)$ est intégrable en valeur absolue sur $\left]-\infty, \infty\right[ $, nous obtenons qu'elle tend vers 0 à ces bornes. Or, puisque $\left|e^{-i \alpha x}\right| = 1$ pour tout $x$, on obtient que: 
        \[\mathcal{F}\left(f'\right)\left(\alpha\right) = 0 - 0 + \frac{i\alpha}{\sqrt{2\pi}} \int_{-\infty}^{\infty} f\left(x\right) e^{-i \alpha x}dx =  i \alpha \mathcal{F}\left(f\right)\left(\alpha\right)\]
        
        Passons maintenant à la dérivée de la transformée. Nous aurions besoin de justifier l'échange de la dérivée et de l'intégrale, mais en supposant qu'on ait le droit: 
        \autoeq{\unexpanded{\frac{d}{d\alpha} \mathcal{F} f\left(\alpha\right) = \frac{1}{\sqrt{2\pi}} \int_{-\infty}^{\infty} f\left(x\right) \frac{d}{d \alpha} e^{-i \alpha x}dx = \frac{1}{\sqrt{2\pi}} \int_{-\infty}^{\infty} -i x f\left(x\right) e^{-i\alpha x}dx = \mathcal{F}\left(-i x f\left(x\right)\right)\left(\alpha\right)}}
        
        \qed
    }
}

\parag{Théorème: Identité de Parseval}{
    Soit $f: \mathbb{R} \mapsto \mathbb{C}$ une fonction telle que: 
    \[\int_{-\infty}^{\infty} \left|f\left(x\right)\right|dx < \infty, \mathspace \int_{-\infty}^{\infty} \left|f\left(x\right)\right|^2 dx < \infty\]
    
    Alors: 
    \[\int_{-\infty}^{\infty} \left|f\left(x\right)\right|^2 dx = \int_{-\infty}^{\infty} \left|\hat{f}\left(\alpha\right)\right|^2 d \alpha\]
    où $\hat{f}\left(\alpha\right) = \mathcal{F}\left(f\right)\left(\alpha\right)$.
    
    \subparag{Preuve}{
        Si $f$ est réelle, alors: 
        \autoeq{\unexpanded{\int_{-\infty}^{\infty} f\left(x\right)^2 dx = \int_{-\infty}^{\infty} f\left(x\right) \frac{1}{\sqrt{2\pi}} \underbrace{\int_{-\infty}^{\infty} \hat{f}\left(\alpha\right) e^{i \alpha x} d \alpha}_{= \mathcal{F}^{-1}\left(\hat{f}\right) = f\left(x\right)} dx = \int_{-\infty}^{\infty} \hat{f}\left(\alpha\right) \frac{1}{\sqrt{2\pi}} \int_{-\infty}^{\infty} f\left(x\right)e^{i \alpha x} dx d\alpha = \int_{-\infty}^{\infty} \hat{f}\left(\alpha\right) \bar{\hat{f}\left(\alpha\right)} d\alpha = \int_{-\infty}^{\infty} \left|\hat{f}\left(\alpha\right)\right|^2 d \alpha}}

        \qed
    }
}

\parag{Proposition: Point fixe des transformées}{
    Soit $f\left(x\right) = e^{-\frac{x^2}{2}}$. Alors $f$ est un point fixe de la transformée: 
    \[\mathcal{F} f\left(\alpha\right) = e^{- \frac{\alpha ^2}{2}}\]
    
    \subparag{Preuve}{
        Pour commencer, vérifions que notre fonction est bien intégrable en valeur absolue sur $\left]-\infty, +\infty\right[ $:
        \autoeq{\unexpanded{\int_{-\infty}^{\infty} \left|f\left(x\right)\right|dx = \sqrt{\left(\int_{-\infty}^{\infty} e^{-\frac{x^2}{2}} dx\right)^2} = \sqrt{\int_{-\infty}^{\infty} e^{-\frac{x^2}{2}} dx \int_{-\infty}^{\infty} e^{-\frac{y^2}{2}}dy} = \sqrt{\int_{-\infty}^{\infty} \int_{-\infty}^{\infty} e^{-\frac{x^2 + y^2}{2}}dxdy}}}

        En faisant un changement de variable polaire, on trouve que c'est égal à:
        \autoeq{\unexpanded{\int_{-\infty}^{\infty} \left|f\left(x\right)\right|dx = \sqrt{\int_{0}^{2\pi} \int_{0}^{\infty} e^{\frac{-r^2}{2}} r dr d \theta} = \sqrt{\int_{0}^{2\pi} d \theta \left[-e^{\frac{-r^2}{2}}\right]_{0}^{+\infty}} = \sqrt{2\pi \cdot  1} = \sqrt{2\pi}}}
        ce qui est bien inférieur à $\infty$. Cela montre au passage que: 
        \[\mathcal{F} f\left(0\right) = \frac{1}{\sqrt{2\pi}} \int_{-\infty}^{\infty} f\left(x\right)dx = \frac{\sqrt{2\pi}}{\sqrt{2\pi}} = 1\]
        
        Maintenant, nous voulons calculer la transformée de Fourier. Appliquer la formule semble compliquée, donc utilisons une autre méthode. Nous remarquons que, pour notre fonction $f\left(x\right) = e^{-\frac{x^2}{2}}$, nous avons $f'\left(x\right) = -xf\left(x\right)$. Nous appliquons maintenant les propriétés de dérivations (que nous avons le droit d'utiliser puisque $f, f'$ et $xf\left(x\right)$ sont intégrables en valeur absolue sur $\left]-\infty, \infty\right[ $): 
        \autoeq{\unexpanded{\frac{d}{d\alpha} \mathcal{F} f\left(\alpha\right) = \mathcal{F}\left(-ix f\left(x\right)\right)\left(\alpha\right) = i \mathcal{F} \left(\underbrace{-x f\left(x\right)}_{= f'\left(x\right)}\right)\left(\alpha\right) = i \mathcal{F}\left(f'\right)\left(\alpha\right) = i \cdot  i\alpha \mathcal{F}\left(f\right)\left(\alpha\right) = -\alpha \mathcal{F} f\left(\alpha\right)}}
        
        Nous avons donc trouvé que la transformée de Fourier de notre fonction respecte l'équation:
        \[\hat{f}'\left(\alpha\right) = -\alpha \hat{f}\left(\alpha\right)\]
        
        Cette équation est une EDO homogène qui peut être résolue facilement: 
        \[\hat{f}\left(\alpha\right) = C e^{-\frac{\alpha ^2}{2}}\]
        
        Or, nous avions trouvé ci-dessus que $\hat{f}\left(0\right) = 1$, ce qui nous donne que $C = 1$ et donc: 
        \[\hat{f}\left(\alpha\right) = e^{- \frac{\alpha ^2 }{2}}\]
        comme voulu.

        \qed
    }
}


\end{document}
