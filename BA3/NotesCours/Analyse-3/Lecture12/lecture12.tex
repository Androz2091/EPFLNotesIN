% !TeX program = lualatex
% Using VimTeX, you need to reload the plugin (\lx) after having saved the document in order to use LuaLaTeX (thanks to the line above)

\documentclass[a4paper]{article}

% Expanded on 2022-12-12 at 08:25:40.

\usepackage{../../style}

\title{Analyse 2}
\author{Joachim Favre}
\date{Lundi 12 décembre 2022}

\begin{document}
\maketitle

\lecture{12}{2022-12-12}{On fait comme s'il n'y avait pas de problème}{
\begin{itemize}[left=0pt]
    \item Définition du produit de convolution.
    \item Explication des propriétés du produit de convolution.
    \item Preuve du lien entre produits de convolutions et produits avec les transformées de Fourier.
    \item Application des transformées de Fourier à la résolution d'équations différentielles et fonctionnelles.
\end{itemize}

}

\parag{Proposition: Parité des transformées}{
    Soit $f: \mathbb{R} \mapsto \mathbb{R}$ une fonction intégrable en valeur absolue sur $\left]-\infty, \infty\right[ $.

    \begin{enumerate}
        \item Si $f$ est paire, alors: 
        \[\hat{f}\left(\alpha\right) = \sqrt{\frac{2}{\pi}} \int_{0}^{\infty} f\left(x\right) \cos\left(\alpha x\right)dx\]

        En particulier, $\hat{f}$ est paire.
        \item Si $f$ est impaire, alors: 
        \[\hat{f}\left(\alpha\right) = -i\sqrt{\frac{2}{\pi}} \int_{0}^{\infty} f\left(x\right)\sin\left(\alpha x\right)dx\]
        
        En particulier, $\hat{f}$ est impaire.
    \end{enumerate}

    \subparag{Preuve}{
        La preuve est considérée triviale et laissée en exercice au lecteur.
    }
}

\parag{Remarque}{
    Comme pour les séries de Fourier, si $f: \left[0, \infty\right[  \mapsto \mathbb{R}$, on peut considérer la transformée de Fourier en sinus de $f$ en l'étendant sur $\mathbb{R}$ de manière à la rendre impaire.

    De manière similaire, en étendant $f$ de manière la rendre paire, nous pouvons considérer la transformée de Fourier en cosinus de $f$.
}

\subsection{Produit de convolution}
\parag{Définition: Produit de convolution}{
    Soient $f, g : \mathbb{R} \mapsto \mathbb{R}$ des fonctions intégrables en valeur absolue sur $\left]-\infty, \infty\right[ $. Alors, le \important{produit de convolution} de $f$ et $g$, noté $f * g$, est la fonction:
    \[\begin{split}
    f * g: \mathbb{R} &\longmapsto \mathbb{R} \\
    x &\longmapsto \int_{-\infty}^{\infty} f\left(x - t\right)g\left(t\right)dt
    \end{split}\]

    L'idée est que nous gardons la fonction $g$ fixée, que nous retournons (prenons la symétrie de) la fonction $f$ et la déplaçons pour que son centre soit en $x$, puis calculons l'intégrale de leur produit.

    \subparag{Remarque personnelle}{
        Il y a une vidéo de 3Blue1Brown sur ce sujet:
        \begin{center}
            \url{https://www.youtube.com/watch?v=KuXjwB4LzSA}
        \end{center}
    }
}

\parag{Exemple 1}{
    Considérons les fonctions suivantes:
    \begin{functionbypart}{f\left(x\right) = g\left(x\right)}
        1, & x \in \left[1, 1\right] \\
        0, & \text{sinon}
    \end{functionbypart}

    Intuitivement, en regardant le dessin ci-dessous, nous pouvons voir que si $x > 2$, alors $f * g\left(x\right) = 0$, mais $f * g \left(\frac{3}{2}\right) = \frac{1}{2}$.
    \svghere{ExempleProduitDeConvolution.svg}  

    Calculons maintenant notre produit de convolution, en prenant le changement de variable $y = x-t$:
    \[f * g \left(x\right) = \int_{-\infty}^{\infty} f\left(x - t\right)g\left(x\right) dt = \int_{-1}^{1} f\left(x - t\right)dt = -\int_{x+1}^{x-1} f\left(y\right)dy = \int_{x-1}^{x+1} f\left(y\right)dy\]
    
    Ainsi, nous obtenons que $f * g\left(x\right) = 0$ si $x \in \left[2, \infty\right[ $ ou $x \in \left]-\infty, -2\right] $. De plus, pour $x \in \left[0, 2\right[ $: 
    \[f * g\left(x\right) = \int_{x-1}^{1} 1dx = 1 - x + 1 = 2 - x\]
    
    Et de manière similaire, si $x \in \left[-2, 0\right]$: 
    \[f * g\left(x\right) = \int_{-1}^{x+ 1} 1dx = 2 + x\]
    
    Nous obtenons ainsi que:
    \begin{functionbypart}{f * g\left(x\right)}
        2 - \left|x\right|, & \text{si } x \in \left[-2, 2\right] \\
        0, & \text{sinon}
    \end{functionbypart}

    \svghere[0.5]{ResultatExempleProduitDeConvolution.svg}
}

\parag{Exemple 2}{
    Prenons $f\left(x\right) = g\left(x\right) = e^{-\left|x\right|}$. Alors, nous obtenons, si $x > 0$: 
    \autoeq{\unexpanded{f * g\left(x\right) = \int_{-\infty}^{\infty} e^{-\left|x-t\right|} e^{-\left|t\right|} dt = \int_{-\infty}^{0} e^{-\left(x - t\right)} e^t dt + \int_{0}^{x} e^{-\left(x - t\right)} e^{-t} dt + \int_{x}^{\infty} e^{x-t} e^{-t} dt = \left[\frac{e^{2t - x}}{2}\right]_{-\infty}^{0} + xe^{-x} + \left[\frac{e^{x - 2t}}{-2}\right]_{x}^{\infty} = \frac{e^{-x}}{2} + xe^{-x} + \frac{e^{-x}}{2} = e^{-x}\left(1 + x\right)}}
    
    Similairement, si $x \leq 0$, on peut trouver que: 
    \[f * g\left(x\right) = e^{x}\left( 1 - x\right)\]
    
    En mettant tout cela ensemble, nous obtenons que, pour tout $x \in \mathbb{R}$: 
    \[f * g\left(x\right) = e^{-\left|x\right|}\left(1 + \left|x\right|\right)\]
}

\parag{Application physique}{
    En physique, la loi de désintégration radioactive nous dit que la masse d'un produit radioactif au fil du temps est décrit par la loi suivante: 
    \[M\left(t\right) = M_0 e^{-c t}\]
    où $c > 0$ est une constante dépendante du matériau et $M_0 > 0$ la masse de départ.

    Nous nous demandons maintenant quelle est la masse radioactive au temps $T$ si une centrale produit $m\left(t\right)$ déchets radioactifs par seconde pendant $\left[0, T\right]$ (qui se désintègrent naturellement en partie).

    Commençons par discrétiser notre fonction. Nous pouvons approximer la masse produite entre $t_i$ et $t_{i+1} = t_i + \Delta t_i$ par $m\left(t_i\right) \Delta t$. Cette masse se désintègre entre $t_i$ et $T$, et donc elle contribue environ $m\left(t_i\right) \Delta t e^{-c \left(T - t_i\right)}$ à la radioactivité en $T$. La masse au temps $T$ est donc approximée par: 
    \[M\left(t\right) \approx \sum_{i=0}^{n} m\left(t_i\right) \Delta t e^{-c \left(T - t_i\right)}\]
    
    Nous pouvons faire tendre $\Delta t$ vers 0 pour que ce ne soit plus une approximation, ce qui nous donne: 
    \[M\left(t\right) = \int_{0}^{T} m\left(t\right)e^{-c\left(T - t\right)} dt = \left(m\left(t\right) * e^{-ct}\right)\left(T\right)\]
    où $m\left(t\right)$ est prise telle que $m\left(t\right) = 0$ pour tout $x \in \mathbb{R} \setminus \left[0, T\right]$.
}

\parag{Propriétés}{
    Soient $f, g, h : \mathbb{R} \mapsto \mathbb{R}$ des fonctions intégrables en valeur absolue sur $\left]-\infty, \infty\right[ $. Alors:
    \begin{enumerate}
        \item $\left(\alpha f + \beta g\right) * h = \alpha f * h + \beta g * h$
        \item $f * g = g * f$
        \item $\left(f * g\right) * h = f * \left(g * h\right)$
    \end{enumerate}
}

\parag{Proposition: Convolutions et transformées}{
    Soient $f, g: \mathbb{R} \mapsto \mathbb{R}$ des fonctions intégrables en valeur absolue sur $\left]-\infty, \infty\right[ $. Alors:
    \begin{enumerate}
        \item $\displaystyle \int_{-\infty}^{\infty} \left|f * g\left(x\right)\right|dx \in \mathbb{R}$ 
        \item $\displaystyle \mathcal{F}\left(f * g\right) = \sqrt{2\pi} \mathcal{F}\left(f\right) \mathcal{F}\left(g\right)$
    \end{enumerate}

    \subparag{Preuve 1}{
        Vérifions que notre intégrale est bien réelle (qu'elle n'est pas infinie): 
       \autoeq{\unexpanded{\int_{-\infty}^{\infty} \left|f * g\left(x\right)\right|dx = \int_{-\infty}^{\infty} \left|\int_{-\infty}^{\infty} f\left(x - t\right) g\left(t\right) dt\right|dx \leq \int_{-\infty}^{\infty} \int_{-\infty}^{\infty} \left|f\left(x-t\right)\right|\left|g\left(t\right)\right|dt dx = \int_{-\infty}^{\infty} \left|g\left(t\right)\right| \int_{-\infty}^{\infty} \left|f\left(x-t\right)\right|dxdt =  \underbrace{\int_{-\infty}^{\infty} \left|g\left(t\right)\right|dt}_{\in \mathbb{R}} \cdot \underbrace{\int_{-\infty}^{\infty} \left|f\left(y\right)\right|dy}_{\in \mathbb{R}} \in \mathbb{R}}}
       en prenant à nouveau le changement de variable $y = x - t$.
    }

    \subparag{Preuve 2}{
        Considérons maintenant la transformée de Fourier de la convolution (en prenant encore le changement de variable $y = x - t$): 
        \autoeq{\unexpanded{\mathcal{F}\left(f * g\right)\left(\alpha\right) = \frac{1}{\sqrt{2\pi}} \int_{-\infty}^{\infty} \int_{-\infty}^{\infty} f\left(x-t\right)g\left(t\right)dt e^{-i \alpha x} dx = \int_{-\infty}^{\infty} g\left(t\right) e^{-i\alpha t} \cdot \frac{1}{\sqrt{2\pi}} \int_{-\infty}^{\infty} f\left(x-t\right) e^{-i \alpha\left(x - t\right)}dx dt = \int_{-\infty}^{\infty} g\left(t\right) e^{-i\alpha t} \cdot \underbrace{\frac{1}{\sqrt{2\pi}} \int_{-\infty}^{\infty} f\left(y\right) e^{-i \alpha\left(y\right)}dy}_{= \mathcal{F}\left(f\right)\left(\alpha\right)} dt = \mathcal{F}\left(f\right)\left(\alpha\right) \sqrt{2\pi} \frac{1}{\sqrt{2\pi}} \int_{-\infty}^{\infty} g\left(t\right)e^{-i \alpha t} dt = \sqrt{2\pi} \mathcal{F}\left(f\right)\left(\alpha\right) \mathcal{F}\left(g\right)\left(\alpha\right)}}

        \qed
    }
}

\parag{Proposition: Convolution et transformées inverses}{
    Soient $f, g: \mathbb{R} \mapsto \mathbb{R}$ des fonctions telles que $\hat{f} = \mathcal{F}\left(f\right)$ et $\hat{g} = \mathcal{F}\left(g\right)$ sont intégrables en valeur absolue sur $\left]-\infty, \infty\right[ $. Alors: 
    \[\mathcal{F}^{-1}\left(\hat{f} * \hat{g}\right) = \sqrt{2\pi} \mathcal{F}^{-1}\left(\hat{f}\right) \mathcal{F}^{-1}\left(\hat{g}\right) = \sqrt{2\pi} fg\]

    À partir de là, nous pouvons trouver que: 
    \[\hat{f} * \hat{g} = \sqrt{2\pi} \mathcal{F} \left(fg\right) \implies \mathcal{F}\left(fg\right) = \frac{1}{\sqrt{2\pi}} \hat{f} * \hat{g} = \frac{1}{\sqrt{2\pi}} \mathcal{F}\left(f\right) * \mathcal{F}\left(g\right)\]

    Ainsi, à un facteur de $\sqrt{2\pi}$ près, $\mathcal{F}$ transforme les convolutions en produits et inversement.
}

\subsection{Application à des équations fonctionnelles}

\parag{Exemple 1: EDO}{
    Soit $f: \mathbb{R} \mapsto \mathbb{R}$ une fonction fixée. Nous cherchons une solution $u: \mathbb{R} \mapsto \mathbb{R}$ à l'équation: 
    \[-u''\left(t\right) + u\left(t\right) = f\left(t\right)\]

    Nous obtenons ce genre d'équation si nous avons un ressort dans le vide, sur lequel nous appliquons une force $f\left(t\right)$.

    Appliquons la transformée de Fourier sur notre équation: 
    \autoeq{\unexpanded{-\mathcal{F}\left(u''\right)\left(\alpha\right) + \mathcal{F}\left(u\right)\left(\alpha\right) = \mathcal{F}\left(f\right)\left(\alpha\right) \iff -\left(-\alpha ^2\right) \hat{u}\left(\alpha\right) + \hat{u}\left(\alpha\right) = \hat{f}\left(\alpha\right) \iff \hat{u}\left(\alpha\right) = \frac{1}{1 + \alpha ^2} \hat{f}\left(\alpha\right)}}
    
    Nous cherchons $u$, donc prenons la transformée de Fourier inverse: 
    \[u\left(t\right) = \mathcal{F}^{-1} \left(\frac{1}{1 + \alpha ^2} \hat{f}\left(\alpha\right)\right) = \frac{1}{\sqrt{2\pi}} \mathcal{F}^{-1}\left(\frac{1}{1 + \alpha ^2}\right) * \mathcal{F}^{-1}\left(\hat{f}\right) = \frac{1}{\sqrt{2\pi}} \left(\sqrt{\frac{\pi}{2}} e^{-\left|t\right|} * f\right)\left(t\right)\]
    puisque $\mathcal{F}\left(\frac{\sqrt{\pi}}{2} e^{-\left|t\right|}\right) = \frac{1}{1 + \alpha ^2}$.
    
    Nous obtenons ainsi finalement que: 
    \[u\left(t\right) = \frac{1}{2} \int_{-\infty}^{\infty} e^{-\left|t-x\right|}f\left(x\right)dx\]
    
    Par exemple, pour $f\left(t\right) = e^{-\left|t\right|}$, nous pouvons trouver que: 
    \[u\left(t\right) = \frac{1}{2} e^{-\left|t\right|} \left(1 + \left|t\right|\right)\]

    Nous avions vu cette convolution plus tôt dans ce cours.
}

\parag{Exemple 2: EDO}{
    Considérons l'équation d'Airy (qui a des applications en physique quantique et en optique): 
    \[u''\left(x\right) - xu\left(x\right) = 0\]
    
    À nouveau, en appliquant la transformée de Fourier: 
    \[\mathcal{F}\left(u''\right)\left(\alpha\right) - \mathcal{F}\left(x u\left(x\right)\right)\left(\alpha\right) = 0 \iff -\alpha ^2 \hat{u}\left(\alpha\right) - i \hat{u}'\left(\alpha\right) = 0 \iff \alpha ^2 \hat{u}\left(\alpha\right) = -i \hat{u}'\left(\alpha\right)\]
    
    C'est une EDL1 homogène, ce qui est facile à résoudre: 
    \[\hat{u}\left(\alpha\right) = Ce^{i \frac{\alpha^3}{3}}\]
    
    Cette fonction n'est malheureusement pas intégrable en valeur absolue sur $\left]-\infty, \infty\right[ $, puisque $\int_{-\infty}^{\infty} \left|\hat{u}\left(\alpha\right)\right| d \alpha = +\infty$. Ainsi, $\mathcal{F}^{-1}$ n'est pas applicable. Mais, en l'appliquant quand même en faisant comme s'il n'y avait pas de problème, cela nous permet tout de même de deviner la solution: 
    \[u\left(x\right) = \text{``}\mathcal{F}^{-1} \left(C e^{i \frac{\alpha^3}{3}}\right)\text{''} =  \text{``}\frac{C}{\sqrt{2\pi}} \int_{-\infty}^{\infty} e^{i \left(\frac{\alpha^3}{3} + \alpha x\right)} d\alpha \text{''} = \frac{2C}{\sqrt{2\pi}} \int_{0}^{\infty} \cos\left(\frac{\alpha^3}{3} + \alpha x\right)d \alpha\]

    Cette fonction (à une constante multiplicative près) s'appelle la fonction d'Airy, et elle n'est exprimable à partir de fonctions élémentaires. On peut vérifier qu'elle est bien la solution à notre équation et que cette intégrale converge bien.

    \subparag{Remarque}{
        Comme mentionné précédemment, à l'aide de la théorie des distributions nous pouvons enlever la nécessité pour $\hat{f}$ d'être intégrable en valeur absolue sur $\left]-\infty, \infty\right[ $ afin de calculer sa transformée inverse. C'est la raison pour laquelle ça a quand même marché ici.
    }
}

\parag{Exemple 3: Équation fonctionnelle}{
    Soit $f: \mathbb{R} \mapsto \mathbb{R}$ une fonction fixée. Nous cherchons une solution $u: \mathbb{R} \mapsto \mathbb{R}$ à l'équation fonctionnelle suivante: 
    \[9 u\left(x\right) + 8\left(u* f\right)\left(x\right) = f\left(x\right)\]
    
    En prenant une transformée de Fourier, nous obtenons: 
    \autoeq{\unexpanded{9 \hat{u}\left(\alpha\right) + 8\sqrt{2\pi} \hat{u}\left(\alpha\right) \hat{f}\left(\alpha\right) = \hat{f}\left(\alpha\right) \iff \hat{u}\left(\alpha\right) \left(9 + 8\sqrt{2\pi} \hat{f}\left(\alpha\right)\right) = \hat{f}\left(\alpha\right) \iff \hat{u}\left(\alpha\right) = \frac{\hat{f}\left(\alpha\right)}{9 + 8\sqrt{2\pi} \hat{f}\left(\alpha\right)}}}
    
    En posant par exemple $f\left(x\right) = e^{-\left|x\right|} \implies \hat{f}\left(\alpha\right) = \sqrt{\frac{2}{\pi}} \frac{1}{1 + \alpha ^2}$, on obtient: 
    \[\hat{u}\left(\alpha\right) = \sqrt{\frac{2}{\pi}} \frac{1}{1 + \alpha ^2} \left(9 + 8\sqrt{2\pi} \sqrt{\frac{2}{\pi}} \frac{1}{1 + \alpha ^2}\right)^{-1} = \ldots = \sqrt{\frac{2}{\pi}} \frac{1}{25 + 9 \alpha ^2}\]
    
    Cela ressemble à $\frac{1}{1 + \alpha ^2}$, donc essayons de le faire ressortir: 
    \[\hat{u}\left(\alpha\right) = \frac{1}{25} \sqrt{\frac{2}{\pi}} \frac{1}{1 + \left(\frac{3}{5} \alpha\right)^2} = \frac{1}{25} \hat{f}\left(\frac{3}{5}\alpha\right) = \frac{1}{25} \mathcal{F}\left(f\right)\left(\frac{3}{5} \alpha\right) = \frac{1}{25} \frac{5}{3} \mathcal{F}\left(f\left(\frac{5}{3}x\right)\right)\left(\alpha\right)\]

    Ce qui nous donne finalement que: 
    \[u\left(x\right) = \frac{1}{15} f\left(\frac{5}{3}x\right) = \frac{1}{15} e^{-\frac{5}{3}\left|x\right|}\]
}

\end{document}
