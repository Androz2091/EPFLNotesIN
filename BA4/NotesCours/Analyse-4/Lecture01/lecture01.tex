% !TeX program = lualatex
% Using VimTeX, you need to reload the plugin (\lx) after having saved the document in order to use LuaLaTeX (thanks to the line above)

\documentclass[a4paper]{article}

% Expanded on 2023-02-23 at 15:05:15.

\usepackage{../../style}

\title{Analyse 4}
\author{Joachim Favre}
\date{Jeudi 23 février 2023}

\begin{document}
\maketitle

\lecture{1}{2023-02-23}{Fonctions holomorolophes}{
\begin{itemize}[left=0pt]
    \item Explication des notations et quelques rappels.
    \item Définition des limites et de la continuité pour les fonctions complexes.
    \item Explication des équations de Cauchy-Riemann.
\end{itemize}

}

\section{Analyse complexe}
\subsection{Rappels et notations}
\parag{Définition: Nombres complexes}{
    L'ensemble des nombres complexes, noté $\mathbb{C}$, est défini par: 
    \[\mathbb{C} = \left\{z = x + iy \ |\ x, y \in \mathbb{R}\right\}\]
    où $i$ est tel que $i^2 = -1$.
}

\parag{Définition: Forme cartésienne}{
    Soit $z = x + iy \in \mathbb{C}$. Alors, $x = \cre\left(z\right)$ est la \important{partie réelle} de $z$, et $y = \cim\left(z\right)$ est sa \important{partie imaginaires}.

    \subparag{Remarque}{
        Nous pouvons ainsi représenter n'importe quel nombre complexe comme un point sur le plan complexe, où l'axe horizontal est la partie réelle et l'axe vertical est la partie complexe.
    }
}

\parag{Définition: Conjugué complexe}{
    Soit $z = x + iy \in \mathbb{C}$. Nous définissons le \important{conjugué complexe} de $z$ comme: 
    \[\bar{z} = x - iy\]

    \subparag{Remarque}{
        Le conjugué de complexe a beaucoup de propriétés, ce qui permet, de manière générale, de simplement remplacer tous les $i$ par $-i$ dans $z$ pour trouver $\bar{z}$.
    }
}

\parag{Définition: Forme polaire}{
    Soit $z = x + iy \in\mathbb{C}$. Nous pouvons représenter n'importe quel point avec des coordonnées polaires, et puisqu'un nombre complexe est représentable comme un point sur le plan complexe, nous pouvons représenter un nombre complexe avec des coordonnées polaires. Nous obtenons $x = r\cos\left(\theta\right)$ et $y = r\sin\left(\theta\right)$, et donc: 
    \[z = x + yi = r\left[\cos\left(\theta\right) + i\sin\left(\theta\right)\right]\]
    
    Cette forme est appelée la \important{forme polaire}.

    Nous pouvons obtenir $r$ et $\theta$ grâce à leur formule: 
    \[r = \sqrt{x^2 + y^2}\] 
    \[\theta = \arg\left(z\right)\]
    
    L'argument, $\theta$, est une fonction multivoque: elle est définie à un multiple de $2\pi$ près. La \important{détermination principale de l'argument} est le seul angle dans $\left]-\pi, \pi\right] $. Une fois que cet intervalle est fixé, cela nous donne: 
    \begin{functionbypart}{\arg\left(z\right)}
        \arctan\left(\frac{y}{x}\right), & x > 0, y \in \mathbb{R} \\
        \frac{\pi}{2}, & x = 0, y > 0 \\
        \arctan\left(\frac{y}{x}\right) + \pi, & x < 0, y \geq 0 \\
        \arctan\left(\frac{y}{x}\right) - \pi, & x < 0, y < 0 \\
        -\frac{\pi}{2}, & x = 0, y < 0
    \end{functionbypart}

    De plus, $\arg\left(0\right)$ est indéfini.
}

\parag{Identité d'Euler}{
    L'identité d'Euler nous dit que:
    \[e^{i\theta} = \cos\left(\theta\right) + i\sin\left(\theta\right)\]

    Ainsi, $\left\|e^{i\theta}\right\| = 1$ et $\arg\left(e^{i\theta}\right) = \theta$. Ces égalités sont probablement celles qui seront le plus utilisées dans ce cours.

    \subparag{Preuve}{
        Cette égalité découle directement de la définition de l'exponentielle complexe, présentée dans le cours suivant.
    }

    \subparag{Note personnelle}{
        Pour comprendre intuitivement cette identité, je recommande toujours cette incroyable vidéo de 3Blue1Brown:
        \begin{center}
            \url{https://youtu.be/v0YEaeIClKY}
        \end{center}
    }
}

\parag{Définition: Forme exponentielle}{
    Soit $z = x + iy \in \mathbb{C}$. Nous pouvons utiliser les deux derniers paragraphes, ce qui nous donne:
    \[z = r e^{i \theta}\]
}

\parag{Exemple}{
    Considérons $z = 1 + i$. Nous pouvons trouver simplement que $r = \sqrt{1^2 + 1^2} = \sqrt{2}$. De plus, ce point fait un angle de $\frac{\pi}{4}$, donc nous trouvons $\arg\left(z\right) = \frac{\pi}{4}\ \left(\Mod 2\pi\right)$. Ainsi: 
    \[r = \sqrt{2}\left(\cos\left(\frac{\pi}{4}\right) + i\sin\left(\frac{\pi}{4}\right)\right) = \sqrt{2} e^{i \frac{\pi}{4}}\]
}

\subsection{Fonctions holomorphes}
\subsubsection{Définitions issues de l'Analyse 1}
\parag{Notation: Fonction complexe}{
    Une fonction complexe est notée:
    \[\begin{split}
    f: \mathbb{C} &\longmapsto \mathbb{C} \\
    z &\longmapsto f\left(z\right) = f\left(x + iy\right)
    \end{split}\]
    
    On notera aussi:
    \[f\left(x + iy\right) = u\left(x, y\right) + iv\left(x, y\right) = \cre\left(f\left(x + iy\right)\right) + i \cim\left(f\left(x + iy\right)\right)\]
    avec $u, v : \mathbb{R}^2 \mapsto \mathbb{R}$.
}

\parag{Isomorphisme}{
    Par abus de langage, on assimilera $\mathbb{C} = \mathbb{R}^2$. Ceci peut être justifié puisqu'il peut être associé un unique point de $\mathbb{R}^2$ dans $\mathbb{C}$ et inversement.
}

\parag{Exemple 1}{
    Considérons la fonction suivante: 
    \[f\left(z\right) = z^2\]
    
    Nous trouvons ainsi que: 
    \[f\left(z\right) = \left(x + iy\right)^2 = x^2 + 2x \cdot iy + \left(iy\right)^2 = \left(x^2 - y^2\right) + i\left(2xy\right)\]
    
    Nous trouvons donc que: 
    \[u\left(x, y\right) = x^2 - y^2, \mathspace v\left(x, y\right) = 2xy\]

    Il est très important de remarquer que $v\left(x, y\right) \in \mathbb{R}$, il n'y a pas de $i$.
}

\parag{Exemple 2}{
    Considérons la fonction suivante: 
    \[f\left(z\right) = x\]
    
    Alors, nous obtenons trivialement que: 
    \[u\left(x, y\right) = x, \mathspace v\left(x, y\right) = 0\]
}

\parag{Définition: Limite}{
    Soient $f: \mathbb{C} \mapsto \mathbb{C}$ et $z_0 \in \mathbb{C}$. La \important{limite} de $f\left(z\right)$ quand $z$ tend vers $z_0$ est l'unique nombre $L \in \mathbb{C}$ tel que, pour tout $\epsilon > 0$, il existe $\delta > 0$ tel que, si $\left|z - z_0\right| < \delta$, alors: 
    \[\left|f\left(z\right) - L\right| < \epsilon\]

    On note: 
    \[L = \lim_{z \to z_0} f\left(z\right)\]
}

\parag{Définition: Continuité}{
    Soient $f: \mathbb{C} \mapsto \mathbb{C}$ et $z_0 \in \mathbb{C}$. $f$ est \important{continue} en $z_0$ si: 
    \[\lim_{z \to z_0} f\left(z\right) = f\left(z_0\right)\]
}

\parag{Exemple 1}{
    Nous voulons montrer que $f\left(z\right) = z^2$ est continue en $z_0 = 0$, par la définition.

    Soit un $\epsilon > 0$ arbitraire. Nous voulons trouver un $\delta$ tel que, si $\left|z - 0\right| < \delta$, alors $\left|z^2 - 0^2\right| < \epsilon$. Nous pouvons faire ceci en supposant que $\left|z - 0\right| < \delta$, puis en développant l'inégalité sur le $\epsilon$, et en trouvant une contrainte sur $\delta$: 
    \[\left|z^2 - 0^2\right| = \left|z^2\right| = \left|z\right|^2 < \delta ^2\]
    
    Nous voulons que ce soit plus petit ou égal à $\epsilon$, donc nous pouvons ainsi définir $\delta = \sqrt{\epsilon}$, ce qui nous donne bien que:
    \[\left|z\right| < \delta \implies\left|z^2\right| < \delta ^2 = \epsilon\]
}

\parag{Exemple 2}{
    Nous voulons montrer que $f\left(z\right) = \frac{\bar{z}}{z}$ n'est pas continue en $z_0 = 0$. Pour se faire, nous pouvons simplement chercher deux sous-suites convergentes vers des valeurs différentes. 

    Commençons par considérer $z = x + i0$, et donc $f\left(z\right) = \frac{x}{x} = 1$. Ainsi, dans ce cas:
    \[\lim_{\substack{z \to 0 \\ \cim\left(z\right) = 0}} f\left(z\right) = 1\]

    Considérons maintenant $z = 0 + iy$, ce qui nous donne $f\left(z\right) = \frac{y}{-y} = -1$. Dans ce cas, nous trouvons:
    \[\lim_{\substack{z \to 0 \\ \cre\left(z\right) = 0}} f\left(z\right) = -1\]

    Nous pouvons en déduire que $f$ n'est pas continue en $z_0 = 0$.
}

\subsection{Fonctions holomorphes et équations de Cauchy-Riemann}

\parag{Définition: Fonction holomorphe}{
    Soient $\Omega \subset \mathbb{C}$ un ensemble ouvert (dont la définition est exactement la même que dans les réels), $f: \Omega \mapsto \mathbb{C}$ et $z_0 \in \mathbb{C}$.

    $f$ est \important{holomorphe} en $z_0$ si la limite suivante existe: 
    \[\lim_{h \to 0} \frac{f\left(z_0 + h\right) - f\left(z_0\right)}{h}\]

    Si cette limite existe, alors nous la notons $f'\left(z_0\right)$.

    De plus, nous disons que $f$ est holomorphe sur $\Omega$ si $f$ est holomorphe pour tout $z_0 \in \Omega$.
    
    \subparag{Remarque 1}{
        Une fonction est donc holomorphe en un point si elle est dérivable à ce point. Cependant, si cette limite existe, alors cela a beaucoup plus d'influence sur $f$ que dans les réels, donc nous lui donnons un nom spécial.
    }
    
    \subparag{Remarque 2}{
        La limite de $f'\left(z_0\right)$ s'écrit aussi avec: 
        \[\lim_{h \to 0} \frac{f\left(z_0 + h\right) - f\left(z_0\right)}{h} = \lim_{z \to z_0} \frac{f\left(z\right) - f\left(z_0\right)}{z - z_0}\]

        De plus, comme mentionnée dans l'autre remarque, cette limite est beaucoup plus ``forte'' que son équivalente dans $\mathbb{R}$. En effet, dans $\mathbb{R}$, pour qu'elle existe, il faut calculer deux limites: la limite à gauche et la limite à droite. Cependant, dans $\mathbb{C}$, il y a une infinité de chemins pour que $z$ s'approche de $z_0$. En d'autres mots, la notion de dérivées dans $\mathbb{C}$ est plus forte, puisque plus restrictive.
    }
}

\parag{Exemple 1}{
   Il est possible de montrer que $f\left(z\right) = z^2$ est holomorphe dans $\mathbb{C}$, et que $f'\left(z\right) = 2z$. La démonstration est complètement analogue à ce que nous faisons dans les réels.
}

\parag{Exemple 2}{
    Considérons la fonction suivante: 
    \[f\left(z\right) = \bar{z} = x - iy\]
    
    Soit $z_0 \in \mathbb{C}$ arbitraire, nous voulons voir si notre fonction est holomorphe à ce point: 
    \[\lim_{h \to 0} \frac{\bar{z_0 + h} - \bar{z_0}}{h} = \lim_{h \to 0} \frac{\bar{z_0} + \bar{h} - \bar{z_0}}{h} = \lim_{h \to 0} \frac{\bar{h}}{h}\]

    Cependant, nous avions vu que cette limite n'existe pas. Nous en déduisons que $f\left(z\right) = \bar{z}$ est holomorphe nulle part sur $\mathbb{C}$.
}

\parag{Théorème: Équations de Cauchy-Riemann}{
    Soit $\Omega \subset \mathbb{C}$ ouvert, $f = u + iv : \Omega \mapsto \mathbb{C}$.

    $f$ est holomorphe sur $\Omega$ si et seulement si les équations suivantes, appelées les équations de Cauchy-Riemann, sont satisfaites: 
    \[\begin{systemofequations} u, v = C^1\left(\Omega\right), \text{ i.e. leurs dérivées partielles doivent être continues} \\ \frac{\partial u}{\partial x} = \frac{\partial v}{\partial y} \iff u_x = v_y \\ \frac{\partial v}{\partial x} = - \frac{\partial u}{\partial y} \iff v_x = -u_y \end{systemofequations}\]
    où nous utilisons la notation abrégée $u_x = \frac{\partial u}{\partial x} $ et $v_y = \frac{\partial v}{\partial y} $.

    De plus, nous avons: 
    \[f'\left(z\right) = u_x + i v_x = u_y - i v_y\]

    \subparag{Esquisse de preuve $\implies$}{
        Soit $f\left(x + iy\right) = u\left(x, y\right) + iv\left(x, y\right)$ holomorphe. Nous voulons montrer que: 
        \[f'\left(x + yi\right) = u_x\left(x, y\right) + iv_x\left(x, y\right) = v_y\left(x, y\right) - i u_y\left(x, y\right)\]

        Nous utilisons la définition par la limite, en posant $h = s + it$: 
        \autoeq[s]{f'\left(x + yi\right) = \lim_{s + it \to 0} \frac{f\left(x + s + i\left(y + t\right)\right) - f\left(x + yi\right)}{s + it} = \lim_{s + it \to 0} \frac{u\left(x + s, y + t\right) + iv\left(x + s, y + t\right) - u\left(x, y\right) - iv\left(x, y\right)}{s + it}}
        
        Cependant, puisque la fonction est holomorphe, cette limite existe pour tous les chemins $s + it \to 0$. Nous pouvons ainsi choisir des chemins particuliers. Commençons par considérer $s \to 0$ et $t = 0$: 
        \autoeq{f'\left(x + yi\right) = \lim_{s \to 0} \frac{u\left(x + s, y\right) - u\left(x, y\right) + i \left(v\left(x + s, y\right) - v\left(x, y\right)\right)}{s} = \lim_{s \to 0} \frac{u\left(x + s, y\right) - u\left(x, y\right)}{s} + \lim_{s \to 0} i\frac{v\left(x + s, y\right) - v\left(x, y\right)}{s} = u_x\left(x, y\right) + iv_x\left(x, y\right)}
        par définition des dérivées partielles.

        Maintenant, si nous considérons $s = 0$ et $t \to 0$, alors nous trouvons similairement que: 
        \autoeq{f'\left(x + yi\right) = \lim_{t \to 0} \frac{u\left(x, y + t\right) - u\left(x, y\right) + i\left(v\left(x, y + t\right) - v\left(x, y\right)\right)}{t i} = v_y - i u_y}

        En combinant tout cela, nous trouvons que $u_x + i v_x = v_y - iu_y$, et donc que $u_x = v_y$ et $v_x = -u_y$.
    }

    \subparag{Preuve $\impliedby$}{
        La preuve dans l'autre direction est très difficile.
    }

    \subparag{Remarque}{
        Les équations de Cauchy Riemann disent que $u\left(x, y\right)$ est un potentiel de $\left(v_y, -v_x\right)$. De manière similaire, elles disent que $v\left(x, y\right)$ est un potentiel de $\left(-u_y, u_x\right)$.
    }
}

\end{document}
