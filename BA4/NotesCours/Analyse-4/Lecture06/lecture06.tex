% !TeX program = lualatex
% Using VimTeX, you need to reload the plugin (\lx) after having saved the document in order to use LuaLaTeX (thanks to the line above)

\documentclass[a4paper]{article}

% Expanded on 2023-03-30 at 15:18:40.

\usepackage{../../style}

\title{Analyse 4}
\author{Joachim Favre}
\date{Jeudi 30 mars 2023}

\begin{document}
\maketitle

\lecture{6}{2023-03-30}{Des exemples triviaux}{
\begin{itemize}[left=0pt]
    \item Beaucoup d'exemple de calcul de séries de Laurent, rayon de convergence, nature de points et résidus.
    \item Esquisse de preuve d'une méthode pour trouver si un point est pôle d'ordre $m$.
    \item Introduction à une méthode pour trouver le résidu à un point.
\end{itemize}

}

\begin{parag}{Exemple 3}
    Nous voulons trouver la série de Laurent, son rayon de convergence, la nature de $z_0$ et le résidu en $z_0$ pour $z_0 = 0$ et la fonction: 
    \[f\left(z\right) = \frac{e^{z}}{z^2}\]
    
    Nous savons que: 
    \[e^z = \sum_{n=0}^{\infty} \frac{z^n}{n!}, \mathspace \forall z \in\mathbb{C}\]
    
    Ainsi, en divisant par $z^2$, nous obtenons que, pour tout $z \in \mathbb{C}^*$: 
    \[f\left(z\right) = \frac{e^z}{z^2} = \sum_{n=0}^{\infty} \frac{z^n}{z^2 n!} = \sum_{n=0}^{\infty} \frac{z^{n-2}}{n!} = \sum_{n=-2}^{\infty} \frac{z^n}{\left(n+2\right)!}\]

    Le rayon de convergence est donc $R = \infty$. De plus, $0$ est un pôle d'ordre 2, et $\Res_0\left(f\right) = \frac{1}{\left(-1+2\right)!} = 1$.
\end{parag}

\begin{parag}{Exemple 4}
    Considérons la fonction suiante: 
    \[f\left(z\right) = \frac{\sin\left(z\right)}{z}\]
    
    Nous voulons calculer sa série de Laurent en $z_0 = 0$. Nous pouvons utiliser la série de Taylor du sinus: 
    \[L_0 f\left(z\right) = \frac{1}{z} \left(z - \frac{z^3}{3!} + \frac{z^5}{5!} - \ldots\right) = 1 - \frac{z^2}{3!} + \frac{z^4}{5!} - \ldots, \mathspace \forall z \in \mathbb{C}^*\]
    
    $z_0 = 0$ est un point régulier, nous appelons donc $z_0$ une singularité éliminable. Ainsi, $\Res_0\left(f\right) = 0$ et $R = \infty$.
\end{parag}

\begin{parag}{Exemple 5}
    Considérons la fonction et $z_0$ suivants: 
    \[f\left(z\right) = \frac{1}{z^2 + z} = \frac{1}{z\left(z + 1\right)}, \mathspace z_0 = 0\]
    
    Clairement, nous avons $R = 1$ à cause de la singularité en $z = -1$. En utilisant une décomposition par élément neutre (appelée la feinte du loup, en l'occurrence): 
    \[f\left(z\right) = \frac{1 + z - z}{z\left(z+1\right)} = \frac{1}{z} - \frac{1}{z+1}\]
    
    On connait la série géométrique pour le deuxième terme, donc: 
    \[f\left(z\right) = \frac{1}{z} - \sum_{n=0}^{\infty} \left(-z\right)^n = \frac{1}{z} - \sum_{n=0}^{\infty} \left(-1\right)^n z^n\]
    pour $\left|-z\right| < 1$ et $z \neq 0$. 

    Nous trouvons finalement que $0$ est un pôle d'ordre 1 et que $\Res_0\left(f\right) = 1$. De plus, puisque nous voulons $\left|-z\right| = \left|z\right| < 1$, nous avons $R = 1$ (comme attendu).
\end{parag}

\begin{parag}{Exemple 6}
    Considérons la fonction et $z_0$ suivants: 
    \[f\left(z\right) = e^{\frac{1}{z}}, \mathspace z_0 = 0\]
    
    À notre habitude, utilisons notre série de Taylor. Pour $z_0 \in \mathbb{C}^*$, nous avons: 
    \[e^{\frac{1}{z}} = \sum_{n=0}^{\infty} \frac{\left(\frac{1}{z}\right)^{n}}{n!} = \sum_{n=0}^{\infty} \frac{z^{-n}}{n!} = \sum_{n=-\infty}^{0} \frac{z^{n}}{\left(-n\right)!}, \mathspace \forall z \in \mathbb{C}^*\]
    
    0 est donc une SEI. De plus, nous avons $\Res_0\left(f\right) = 1$ et $R = \infty$.
\end{parag}

\begin{parag}{Exemple 7}
    Considérons la fonction suivante: 
    \[f\left(z\right) = \log\left(z\right)\]
    
    Nous voulons trouver sa série de Laurent en $z_0 = 0$. Cependant, la fonction n'est pas holomorphe pour $z \in \mathbb{R}_{\leq 0}$. Ainsi, le théorème de Laurent ne s'applique pas.
\end{parag}

\begin{parag}{Exemple 8}
    Considérons la fonction et le $z_0$ suivants: 
    \[f\left(z\right) = \frac{1}{z-3}, \mathspace z_0 = 1\]
    
    Nous remarquons que la fonction est même holomorphe en $1$. Puisque $z_{0}$ est à une distance de 2 de la seule singularité, en $z = 3$, le rayon de convergence est $R = 2$. Nous pouvons aussi en déduire que $1$ est un point régulier et que $\Res_1\left(f\right) = 0$.

    Pour notre série de Laurent, nous faisons une mise en évidence forcée (nous voulons quelque chose sous la forme $\frac{1}{a\left(z-z_0\right) + 1}$, afin d'utiliser la série géométrique): 
    \[L_1 f\left(z\right) = \frac{1}{z-3} = \frac{1}{z - 1 -2} = \frac{1}{-2} \frac{1}{\frac{z-1}{-2} + 1}\]

    Nous reconnaissons à nouveau la série géométrique: 
    \[L_1 f\left(z\right) = \frac{1}{-2} \sum_{n=0}^{\infty} \left(\frac{z-1}{-2}\right)^n \left(-1\right)^n = \frac{-1}{2} \sum_{n=0}^{\infty} \frac{\left(z-1\right)^n}{2^n}\]
    
    Cette astuce est importante et doit être retenue.
\end{parag}

\begin{parag}{Proposition: Caractérisation des pôles}
    Soit $\Omega \subset \mathbb{C}$ un ouvert, $z_0 \in \Omega$, $f: \Omega \setminus \left\{z_0\right\} \mapsto \mathbb{C}$ une fonction holomorphe, et $m \in \mathbb{N}^*$. Soit de plus: 
    \[F\left(z\right) = \left(z-z_0\right)^m f\left(z\right)\]
    
    $z_0$ est un pôle d'ordre $m$ si et seulement si $F\left(z\right)$ est holomorphe dans $\Omega \setminus \left\{z_0\right\}$ et $\lim_{z \to z_0} F\left(z\right)$ existe et est non-nulle.

    \begin{subparag}{Observation}
        L'hypothèse que la limite soit non-nulle empêche que $m$ soit trop grand, et l'hypothèse qu'elle existe empêche que $m$ soit trop petit.
    \end{subparag}
    
    \begin{subparag}{Remarque personnelle}
        Cette méthode est très similaire à celle que nous utilisions pour étudier la convergence de séries et d'intégrales en Analyse 1.
    \end{subparag}

    \begin{subparag}{Esquisse de preuve}
        Nous supposons par hypothèse que $z_0$ est un pôle d'ordre $m$. Nous allons uniquement utiliser des équivalences, ce qui va permettre de démontrer les deux directions de la preuve d'un coup.

        $z_0$ est un pôle d'ordre $m$ si et seulement si:
        \[f\left(z\right) = \sum_{n=-m}^{\infty} c_n \left(z-z_0\right)^n = \sum_{n=0}^{\infty} c_{n-m} \left(z-z_0\right)^{n-m}\]
        avec $c_{-m} \neq 0$.

        Puisque $z \neq z_0$, ceci est équivalent à:
        \autoeq{F\left(z\right) = f\left(z\right)\left(z-z_0\right)^m = \sum_{n=0}^{\infty} c_{n-m}\left(z-z_0\right)^{n-m}\left(z-z_0\right)^m = \sum_{n=0}^{\infty} c_{n-m} \left(z-z_0\right)^n = c_{-m} + c_{-m+1} \left(z-z_0\right) + \ldots}
        
        Ceci est donc équivalent au fait que $F\left(z\right)$ admet un développement en série de Taylor en $z_0$. Nous avions vu plus tôt qu'une fonction est holomorphe si est seulement si elle admet une série de Taylor, donc $F$ est holomorphe. Pour être rigoureux, il faudrait montrer que $F$ est bien convergente.

        On trouve aussi que le premier terme de la série de Taylor de $F\left(z\right)$ est $c_{-m}$ si et seulement si: 
        \[\lim_{z \to z_0} F\left(z\right) = c_{-m}\]
        
        Ainsi, l'existence de $c_{-m} \neq 0$ est équivalent au fait que cette limite existe et est non-nulle.

        \qed
    \end{subparag}
\end{parag}

\begin{parag}{Exemple}
    Considérons la fonction suivante: 
    \[f\left(z\right) = \frac{1}{z \sin\left(z\right)}\]
    
    Nous voulons connaitre la nature de $z_0 = 0$. En appliquant notre série de Taylor, nous voyons que: 
    \[f\left(z\right) = \frac{1}{z^2 + z^3\epsilon\left(z\right)}\]
    
    Nous en déduisons donc que $z_0$ est un pôle d'ordre 2. Démontrons le formellement:
    \[F\left(z\right) = z^2 f\left(z\right) = \frac{z}{\sin\left(z\right)} \implies \lim_{z \to 0} \frac{z}{\sin\left(z\right)} = 1\]

    Puisque notre limite existe est non-nulle, et puisque $\frac{z}{\sin\left(z\right)}$ est holomorphe autour de $0$, nous obtenons que $z = 0$ est un pôle d'ordre 2.
\end{parag}

\subsection{Théorème des résidus}
\begin{parag}{Exemple}
    Considérons la fonction et le $z_0$ suivants: 
    \[f\left(z\right) = \frac{1}{z \sin\left(z\right)}, \mathspace z_0 = 0\]
    
    Nous voulons calculer le résidu en $z_0$. Nous avons déjà trouvé que $z_0$ était un pôle d'ordre 2, donc nous pouvons écrire: 
    \[f\left(z\right) = L_0 f\left(z\right) = \frac{c_{-2}}{z^2} + \frac{c_{-1}}{z} + c_0 + \ldots\]
    
    Nous faisons plusieurs transformations, afin d'isoler $c_{-1}$: 
    \autoeq{z^2 f\left(z\right) = c_{-2} + c_{-1} z + c_{0}z^2 + \ldots \implies \left(z^2 f\left(z\right)\right)' = c_{-1} + 2 c_0 z + \ldots \implies \lim_{z \to z_0} \left(z^2 f\left(z\right)\right)' = c_{-1}}
    
    Ainsi, nous avons: 
    \[\Res_0\left(f\right) = \lim_{z \to 0} \left(\frac{z}{\sin\left(z\right)}\right)' = \lim_{z \to 0} \frac{-z\cos\left(z\right) + \sin\left(z\right)}{\sin^2\left(z\right)} = 0\]

    Nous pouvons généraliser cette méthode.
\end{parag}



\end{document}
