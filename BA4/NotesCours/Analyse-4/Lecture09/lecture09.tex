% !TeX program = lualatex
% Using VimTeX, you need to reload the plugin (\lx) after having saved the document in order to use LuaLaTeX (thanks to the line above)

\documentclass[a4paper]{article}

% Expanded on 2023-04-27 at 15:18:31.

\usepackage{../../style}

\title{Analyse 4}
\author{Joachim Favre}
\date{Jeudi 27 avril 2023}

\begin{document}
\maketitle

\lecture{9}{2023-04-27}{L'efficacité des convolutions}{
\begin{itemize}[left=0pt]
    \item Explication des propriétés des transformées de Laplace.
    \item Définition des transformées de Laplace inverse.
\end{itemize}

}

\parag{Observation}{
    La transformée de Laplace généralise la transformée de Fourier pour les fonctions sur $\mathbb{R}_+$. En d'autres mots, si $f: \left[0, \infty\right[ \mapsto \mathbb{R}$ admet une transformée de Fourier, alors elle admet une transformée de Laplace.

    En effet, pour admettre une transformée de Fourier, $f$ doit satisfaire: 
    \[\int_{0}^{\infty} \left|f\left(t\right)\right| dt < \infty\]
        
    Nous avons donc $\int_{0}^{\infty} \left|f\left(t\right)\right|e^{-0 t} dt < \infty$, ce qui nous dit bien que $f$ admet une transformée de Laplace.
}

\parag{Exemple 1}{
    Considérons la fonction de Heaviside: 
    \begin{functionbypart}{f\left(t\right)} 
    0, & \text{si } t < 0 \\
    1, & \text{si } t \geq 0
    \end{functionbypart}
    
    Nous remarquons que: 
    \[\int_{0}^{\infty} \left|f\left(t\right)\right| e^{- \gamma_0 t} dt = \int_{0}^{\infty} e^{- \gamma_0 t} dt\]

    Cette intégrale converge si et seulement si $\gamma_0 > 0$, donc la transformée de Laplace de $f$ n'est définie que pour des abscisses de convergence strictement positifs.

    Soit maintenant $z \in \mathbb{C}$ tel que $\cre\left(z\right) > 0$. Nous obtenons: 
    \[\mathcal{L}\left(f\right)\left(z\right) = \int_{0}^{\infty} e^{-zt}dt = \left[\frac{e^{-zt}}{-z}\right]_{0}^{\infty} = -\frac{1}{z}\left(\lim_{t \to \infty} e^{-zt} - 1\right) = \frac{1}{z} - \frac{1}{z} \lim_{t \to \infty} e^{-zt}\]
    
    Étudions notre limite. Calculons la limite de sa norme: 
    \[\lim_{t \to \infty} \left|e^{-zt}\right| = \lim_{t \to \infty} e^{-\cre\left(z\right)t} = 0\]
    puisque $\cre\left(z\right) > 0$ par hypothèse.

    Nous avons donc obtenu que: 
    \[\mathcal{L}\left(f\right)\left(z\right) = \frac{1}{z}\]
}

\parag{Exemple 2}{
    Soit $a \in \mathbb{R}$. Considérons la fonction suivante: 
    \begin{functionbypart}{f\left(t\right)} 
    0, & \text{si } t < 0 \\
    e^{at}, & \text{si } t \geq 0
    \end{functionbypart}
    
    Regardons l'intégrale suivante: 
    \[\int_{0}^{\infty} \left|f\left(t\right)\right|e^{-\gamma_0 t} dt = \int_{0}^{\infty} e^{\left(a - \gamma_0\right)t} dt\]
    
    Cette intégrale converge si et seulement si $\gamma_0 > a$. Calculons maintenant notre transformée de Laplace, pour un $z$ tel que $\cre\left(z\right) > a$: 
    \[\mathcal{L}\left(f\right)\left(z\right) = \int_{0}^{\infty} e^{\left(a - z\right)t}dt = \left[\frac{e^{\left(a - z\right)t}}{a - z}\right]_{0}^{\infty} = \frac{1}{z-a} \]
    en utilisant exactement le même raisonnement que l'exemple précédent.
}

\parag{Rappel: Convolution}{
    Soient $f, g: \left]-\infty, \infty\right[  \mapsto \mathbb{R}$. Leur \important{convolution} est définie par: 
    \[\left(f*g\right)\left(t\right) = \int_{-\infty}^{\infty} f\left(s\right) g\left(t - s\right)ds\]

    \subparag{Remarque}{
        Nous travaillons plutôt avec des fonctions $f, g: \left[0, \infty\right[ $. Dans ce cas, nous avons simplement: 
        \[\left(f*g\right)\left(t\right) = \int_{0}^{t} f\left(s\right) g\left(t-s\right)ds\]

        En effet, nous avons $s \geq 0$ et $t - s \geq 0$ si et seulement si $0 \leq s \leq t$.
    }
}

\parag{Théorème: Propriétés de la transformée de Laplace}{
    Soient $f, g: \left[0, \infty\right[ \mapsto \mathbb{R}$ avec un même abscisse de convergence $\gamma_0 > 0$, et soient $a, b \in \mathbb{R}$.
    \begin{itemize}
        \item (Linéarité) Pour tout $a, b \in \mathbb{R}$:
            \[\mathcal{L}\left(af + bg\right) = a\mathcal{L}\left(f\right) + b\mathcal{L}\left(g\right)\]
        \item (Décalage) Si $a > 0$, alors:
            \[\mathcal{L}\left(e^{-bt}f\left(at\right)\right)\left(z\right) = \frac{1}{a} \mathcal{L}\left(f\right)\left(\frac{z+b}{a}\right), \mathspace \cre\left(z\right) > a \gamma_0  - b\]
        \item (Convolution) Les convolutions sont transformées en produit:
            \[\mathcal{L}\left(f*g\right) = \mathcal{L}\left(f\right) \mathcal{L}\left(g\right)\]
        \item (Transformée des dérivées) Si $f \in C^1\left(\left[0, \infty\right[ \right)$ et $\int_{0}^{\infty} \left|f'\left(t\right)\right| e^{- \gamma_0 t} dt < \infty$, alors:
            \[\mathcal{L}\left(f'\right)\left(z\right) = z \mathcal{L}\left(f\right)\left(z\right) - f\left(0\right)\]
        \item (Transformée des dérivées généralisée) Si $f$ est de classe $C^n\left(\left[0, \infty\right[ \right)$ et $\int_{0}^{\infty} \left|f^{\left(k\right)}\left(t\right) e^{- \gamma_0 t} dt\right| < \infty$ pour tout $1 \leq k \leq n$, alors:
            \[\mathcal{L}\left(f^{\left(n\right)}\right)\left(z\right) = z^n \mathcal{L}\left(f\right)\left(z\right) - z^{n-1} f\left(0\right) - z^{n-2} f'\left(0\right) - \ldots - z f^{\left(n-2\right)}\left(0\right) - f^{\left(n-1\right)}\left(0\right)\]
        \item (Transformée des primitives) Si $h\left(t\right) = \int_{0}^{t} f\left(s\right)ds$ et $\int_{0}^{\infty} \left|h\left(t\right)\right| e^{-\gamma_0 t} dt < \infty$, alors:
            \[\mathcal{L}\left(h\right)\left(z\right) = \frac{1}{z} \mathcal{L}\left(f\right)\left(z\right)\]
        \item (Fourier) Si $\gamma_0 = 0$, alors:
            \[\mathcal{F}\left(f\right)\left(\alpha\right) = \frac{1}{\sqrt{2\pi}} \mathcal{L}\left(f\right)\left(i\alpha\right), \mathspace \alpha \in \mathbb{R}\]
        \item (Holomorphie) $\mathcal{L}\left(f\right)$ est holomorphe dans $\left\{z \in \mathbb{C} \suchthat \cre\left(z\right) > \gamma_0\right\}$, et:  
        \[\mathcal{L}'\left(f\right)\left(z\right) = \int_{0}^{\infty} f\left(t\right)e^{-tz} dt = \int_{0}^{\infty} -t f\left(t\right) e^{-tz}dt = \mathcal{L}\left(-t f\left(t\right)\right)\left(z\right)\] 
 
        Plus généralement:  
        \[\mathcal{L}^{\left(n\right)}\left(f\right)\left(z\right) = \mathcal{L}\left(\left(-t\right)^n f\left(t\right)\right)\left(z\right)\] 
    \end{itemize}

    \subparag{Preuve}{
        Les preuves sont considérées triviales et laissées en exercice au lecteur.

        Il est intéressant de noter que la propriété de la transformée des dérivées généralisées peut être montrée par intégration par partie ou par récurrence sur $n$.
    }
}

\parag{Exemple}{
    Considérons la fonction suivante: 
    \[f\left(t\right) = t^2, \mathspace t \geq 0\]
    
    Nous voulons la calculer sans passer par sa définition, en utilisant uniquement des propriétés. Nous savons que $f''\left(t\right) = 2$, et $\mathcal{L}\left(f''\right) = \mathcal{L}\left(2\right) = 2 \mathcal{L}\left(1\right)$. De plus: 
    \[\mathcal{L}\left(f''\right)\left(z\right) = z^2 \mathcal{L}\left(f\right) - z f\left(0\right) - f'\left(0\right) = z^2 \mathcal{L}\left(f\right)\]

    Aussi, la première transformée de Laplace que nous avons calculée était:
    \[\mathcal{L}\left(1\right)\left(z\right) = \frac{1}{z}\]

    Nous trouvons donc finalement que: 
    \[2 \mathcal{L}\left(1\right) = z^2 \mathcal{L}\left(f\right) \iff z^2 \mathcal{L}\left(f\right) = 2 \frac{1}{z} \iff \mathcal{L}\left(f\right) = \frac{2}{z^3}\]
}

\parag{Définition: Transformée de Laplace inverse}{
    On dit que $f\left(t\right)$ est la \important{transformée de Laplace inverse} de $F\left(z\right)$, notée $f\left(t\right) = \mathcal{L}^{-1}\left(F\right)\left(t\right)$, si $F\left(z\right) = \mathcal{L}\left(f\right)\left(z\right)$.
}

\parag{Exemple 1}{
    Soient $a, b \in \mathbb{R}$ et la fonction: 
    \[F\left(z\right) = \frac{1}{\left(z-a\right)\left(z-b\right)}\]
    
    Nous voulons trouver la transformée de Laplace inverse de $F$, donc une fonction $f$ telle que $\mathcal{L}\left(f\right) = F$. Nous savons déjà que: 
    \[\mathcal{L}\left(e^{at}\right)\left(z\right) = \frac{1}{z-a}\]
    
    Ainsi, nous pouvons utiliser la décomposition en éléments simples: 
    \[F\left(z\right) = \frac{1}{\left(z-a\right)\left(z-b\right)} = \frac{A}{z-a} + \frac{B}{z-b} = \frac{A\left(z-b\right) + B\left(z-A\right)}{\left(z-a\right)\left(z-b\right)}\]

    Ceci nous donne l'équation suivante: 
    \[1 + 0z = z\left(A + B\right) - \left(Ab + Ba\right)\]
    
    Deux polynômes sont égaux si et seulement si leurs coefficients sont égaux:
    \[\begin{systemofequations} A + B = 0 \\ -\left(Ab + Ba\right) = 1 \end{systemofequations} \iff \begin{systemofequations} A = \frac{1}{a-b} \\ B = \frac{1}{b-a} \end{systemofequations}\]

    Nous obtenons donc que: 
    \[F\left(z\right) = \frac{1}{a-b}\left(\frac{1}{z-a} - \frac{1}{z-b}\right) = \frac{1}{a-b} \left(\mathcal{L}\left(e^{at}\right)\left(z\right) - \mathcal{L}\left(e^{bt}\right)\left(z\right)\right) = \mathcal{L}\left(\frac{e^{at} - e^{bt}}{a-b}\right)\left(z\right)\]
    
    Nous pouvons donc identifier: 
    \[f\left(t\right) = \frac{e^{at} - e^{bt}}{a - b}\]

    \subparag{Remarque}{
        Cette méthode est très générale pour calculer la transformée de Laplace inverse de fonctions rationnelles.
    }
    
    \subparag{Autre méthode}{
        Nous aurions aussi pu utiliser une autre méthode, les produits de convolutions. Nous savons que: 
        \[F\left(z\right) = \frac{1}{z-a} \cdot  \frac{1}{z-b} = \mathcal{L}\left(e^{at}\right)\left(z\right) \mathcal{L}\left(e^{bt}\right)\left(z\right) = \mathcal{L}\left(e^{at}* e^{bt}\right)\left(z\right)\]
        
        En calculant cette convolution, nous obtenons bien le même résultat: 
        \[\left(e^{at}*e^{bt}\right)\left(t\right) = \int_{0}^{t} e^{as} e^{b\left(t-s\right)}ds = \ldots = \frac{e^{at} - e^{bt}}{a - b}\]
        
        Cette méthode est moins générale que celle d'avant, car le calcul de convolutions peut être lourd.
    }
}

\end{document}
