% !TeX program = lualatex
% Using VimTeX, you need to reload the plugin (\lx) after having saved the document in order to use LuaLaTeX (thanks to the line above)

\documentclass[a4paper]{article}

% Expanded on 2023-05-25 at 15:15:06.

\usepackage{../../style}

\title{Analyse 4}
\author{Joachim Favre}
\date{Jeudi 25 mai 2023}

\begin{document}
\maketitle

\lecture{12}{2023-05-25}{Il faut bien des cours qui aillent plus lentement}{
\begin{itemize}[left=0pt]
    \item Explication de la méthode de séparation des variables pour la résolution d'équations aux dérivées partielles.
\end{itemize}

}

\section{Équations aux dérivées partielles}

\parag{Notations}{
    Soit $u\left(x, t\right)$ une fonction réelle de deux paramètres $x, t \in \mathbb{R}$.

    Nous introduisons la notation raccourcie suivante: 
    \[u_x = \frac{\partial u}{\partial x} , \mathspace u_t = \frac{\partial u}{\partial t} , \mathspace u_{xx} = \frac{\partial^{2} u}{\partial x^{2}}\]
}

\parag{Distinction}{
    Dans une équation différentielle ordinaire (EDO), la dérivée porte uniquement sur un seule variable. Par exemple: 
    \[u''\left(t\right) + 3u\left(t\right) = \sin\left(t\right)\]
    
    Dans une équation aux dérivées partielles (EDP), la dérivée porte sur plusieurs variables.  La résolution d'EDPs est très différente de la résolution d'EDOs.
}

\parag{Équation de la chaleur dans une barre finie}{
    Soit $c \neq 0$, $L > 0$ et $f: \left[0, L\right] \mapsto \mathbb{R}$.

    Nous voulons trouver une fonction $u = u\left(x, t\right)$ pour $x \in \left[0, L\right]$ et $t \geq 0$ telle que: 
    \[\begin{systemofequations} u_t\left(x, t\right) = c^2 u_{x x}\left(x, t\right), & x \in \left]0, L\right[, t > 0 \\ u\left(0, t\right) = u\left(L, t\right) = 0 & t \geq 0 \\ u\left(x, 0\right) = f\left(x\right), & x \in \left[0, L\right]\end{systemofequations}\]
    
    $u\left(x, t\right)$ donne la température de la barre à la position $x$ et au temps $t$, où $f$ est la température initiale de la barre et le $c$ caractérise la matériau de la barre. La première équation explique comment la chaleur se propage dans la barre. La deuxième équation est la condition de bord, elle implique que nous refroidissons parfaitement les bouts, de telle manière à ce qu'ils soient toujours à la température ambiante de la pièce. La troisième est la chaleur initiale dans la barre.

    \subparag{Séparation de variable}{
        Nous supposons que $u$ a la forme suivante (c'est une grosse hypothèse, mais si nous obtenons un résultat, alors nous aurons gagné):  
        \[u\left(x, t\right) = v\left(x\right) w\left(t\right)\]

        En mettant cette hypothèse dans notre équation, nous obtenons: 
        \[\begin{systemofequations} v\left(x\right)w'\left(t\right) = c^2 v''\left(x\right) w\left(t\right) \\ v\left(0\right) w\left(t\right) = v\left(L\right)w\left(t\right) = 0 \end{systemofequations}\]
            
        Nous supposons que $v\left(x\right)$ et $w\left(t\right)$ ne sont pas la fonction 0 constante, donc que nous pouvons diviser par elle, ce qui nous donne: 
        \[\begin{systemofequations} \frac{w'\left(t\right)}{w\left(t\right)}\frac{1}{c^2} = \frac{v''\left(x\right)}{v\left(x\right)} \\ v\left(0\right) = v\left(L\right) = 0 \end{systemofequations}\]
        
        Cependant, puisque nous avons une égalité entre deux fonctions pour tout $x, t$, cela implique nécessairement que: 
        \[\frac{w'\left(t\right)}{w\left(t\right)} \frac{1}{c^2} = \frac{v''\left(x\right)}{v\left(x\right)} = -\lambda \in \mathbb{R}\]
    }
    
    \subparag{Découplage}{
        Nous pouvons découpler notre égalité en deux équations, la première étant une équation de Sturm-Liouville 1 et la deuxième une équation différentielle linéaire homogène: 
        \[\begin{systemofequations} v''\left(x\right) = \lambda v\left(x\right) = 0  \\ v\left(0\right) = v\left(L\right) = 0 \end{systemofequations}, \mathspace w'\left(t\right) + \lambda c^2 w\left(t\right) = 0\]
        pour $x \in \left[0, L\right]$ et $t > 0$.

        Nous savons résoudre ces deux équations. Nous commençons par la première: les solutions non-triviales de cette équation existent uniquement quand $\lambda = \left(\frac{n \pi}{L}\right)^2$ pour $n \in \mathbb{N}^*$, et: 
        \[v_n\left(x\right) = \alpha_n \sin\left(\frac{n \pi}{L} x\right)\]

        Nous pouvons ensuite résoudre la deuxième équation, en utilisant ce $\lambda$. Cette solution est trivialement donnée par: 
        \[w_n\left(t\right) = A_n e^{-c^2 \lambda_n t} = A_n e^{-\frac{c^2 n^2 \pi^2}{L^2}t}\]
    }
    
    \subparag{Conditions initiales}{
        Nous obtenons finalement: 
        \autoeq{u_n\left(x\right) = v_n\left(x\right) w_n\left(t\right) = A \alpha_n \sin\left(\frac{n\pi}{L}x\right)e^{-\frac{c^2 n^2 \pi^2}{L^2}t} = \beta_n \sin\left(\frac{n\pi}{L}x\right)e^{-\frac{c^2 n^2 \pi^2}{L^2}t}}
        où nous avons simplement pris $A \alpha_n = \beta_n$, car c'est une simple constante.

        De plus, nous pouvons remarquer que notre équation est linéaire, donc qu'une somme de solution est aussi une solution:
        \[u\left(x, t\right) = \sum_{n=1}^{\infty} u_n\left(x, t\right) = \sum_{n=1}^{\infty} \beta_n \sin\left(\frac{ n\pi}{L}x\right) e^{-\left(\frac{c n \pi}{L}\right)^2 t}\]

        Il est possible de démontrer que c'est la solution générale à notre équation. Il ne nous reste plus qu'à utiliser la condition initiale $u\left(x, 0\right) = f\left(x\right)$, pour trouver les $\beta_n \in \mathbb{R}$: 
        \[u\left(x, 0\right) = f\left(x\right) = \sum_{n=1}^{\infty} \beta_n \sin\left(\frac{n\pi}{L}x\right) = F_s\left(f\right)\left(x\right)\]
        
        Nous reconnaissons la série de Fourier en sinus de $f$. Cela nous donne donc: 
        \[\beta_n = \frac{2}{L} \int_{0}^{L} f\left(x\right) \sin\left(\frac{n\pi}{L}x\right)dx\]
    }
    
    \subparag{Remarque}{
        Il est possible de démontrer que la série qui définit notre $u\left(x, t\right)$ converge uniformément si les $\beta$ sont ceux d'une fonction $f$ de classe $C^1$.
    }
}

\parag{Application}{
    Nous considérons à nouveau l'équation de la chaleur: 
    \[\begin{systemofequations} u_t\left(x, t\right) = c^2 u_{x x}\left(x, t\right), & x \in \left]0, L\right[, t > 0 \\ u\left(0, t\right) = u\left(L, t\right) = 0 & t \geq 0 \\ u\left(x, 0\right) = f\left(x\right), & x \in \left[0, L\right]\end{systemofequations}\]
    
    Nous voulons la résoudre pour $f\left(x\right) = 2\sin\left(\frac{\pi}{L}x\right) - \sin\left(\frac{3\pi}{L}x\right)$. Nous pouvons trivialement identifier les coefficients: 
    \[\beta_1 = 2, \mathspace \beta_3 = -1, \mathspace \beta_n = 0, n \not\in \left\{1, 3\right\}\]
    
    Nous obtenons ainsi que: 
    \[u\left(x, t\right) = 2 \sin\left(\frac{\pi}{L}x\right) e^{-\left(\frac{c \pi}{L}\right)^2 t} - \sin\left(\frac{3\pi}{L}x\right) e^{-\left(\frac{3 c \pi}{L}\right)^2 t}\]

    \subparag{Remarque}{
        Nous n'aurons pas besoin de calculer de série de Fourier en exercices, les fonctions seront toujours sous la forme d'une somme de sinus de bonne fréquence.
    }
}

\parag{Méthode de séparation des variables}{
    La méthode que nous avons vue est extrêmement généralisable. Il est souvent possible de poser $u\left(x, t\right) = v\left(x\right)w\left(t\right)$, puis de manipuler l'équation afin d'obtenir quelque chose sous la forme: 
    \[f\left(x\right) = g\left(t\right)\]
    
    Nous pouvons ensuite poser $f\left(x\right) = g\left(t\right) = \lambda \in \mathbb{R}$, ce qui nous ramène dans le cas des équations différentielles ordinaires.
}


\parag{Équation des ondes}{
    Soient $c \in \mathbb{R}^2$, $L > 0$ et $f, g : \left[0, L\right] \mapsto \mathbb{R}$. Nous posons l'équation suivante: 
    \[\begin{systemofequations} u_{t t}\left(x, t\right) = c^2 u_{x x}\left(x, t\right) & x \in \left]0, L\right[ , t > 0\\u_x\left(0, t\right) = u_x\left(L, t\right) = 0, & t > 0\\ u\left(x, 0\right) = f\left(x\right) \\ u_t\left(x, 0\right) = g\left(x\right) \end{systemofequations}\]
    
    $u\left(x, t\right)$ représente la hauteur d'une corde que l'on fait vibrer, en fonction de la position $x$ et du temps $t$.

    \subparag{Résolution}{
        En utilisant à nouveau la méthode de séparation des variables et Sturm-Liouville 2, il est possible d'obtenir (en faisant attention au cas $n = 0$ lord de la résolution de l'équation pour $w\left(t\right)$, car il doit être considéré différemment): 
        \autoeq{u\left(x, t\right) = \frac{\alpha_0}{2} + \frac{\beta_0}{2}t \fakeequal + \sum_{n=1}^{\infty} \cos\left(\frac{n \pi}{L}x\right) \left(\alpha_n \cos\left(\frac{n \pi c}{L}t\right) + \beta_n \sin\left(\frac{n \pi c}{L}t\right)\right)}
        où $\alpha_n$ et $\beta_n$ sont les coefficients de Fourier en cosinus de $f\left(x\right)$ et $g\left(x\right)$, respectivement.
    }
}

\end{document}
