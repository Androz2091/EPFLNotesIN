% !TeX program = lualatex
% Using VimTeX, you need to reload the plugin (\lx) after having saved the document in order to use LuaLaTeX (thanks to the line above)

\documentclass[a4paper]{article}

% Expanded on 2023-03-02 at 13:15:59.

\usepackage{../../style}

\title{Proba stats}
\author{Joachim Favre}
\date{Jeudi 02 mars 2023}

\begin{document}
\maketitle

\lecture{4}{2023-03-02}{Computing the probability people are terrorists}{
\begin{itemize}[left=0pt]
    \item Definition of conditional probabilities.
    \item Explanation of the law of total probability and Bayes' theorem.
\end{itemize}

}

\subsection{Conditional probabilities}
\parag{Definition: Conditional probability}{
    Let $A$ and $B$ be events of the probability space $\left(\Omega, \mathcal{F}, \prob\right)$, such that $\prob\left(B\right) > 0$. Then, the \important{conditional probability} of $A$ given $B$ (knowing that $B$ took place) is: 
    \[\prob\left(A | B\right) = \frac{\prob\left(A \cap B\right)}{\prob\left(B\right)}\]
    
    If $\prob\left(B\right) = 0$, we are still allowed to consider the following quantity (since both sides are equal to 0, ignoring that $\prob\left(A | B\right) $ is undefined):
    \[\prob\left(A \cap B\right) = \prob\left(A | B\right)\prob\left(B\right)\]

    \subparag{Observation}{
        We note that any set $A$ can be split into: 
        \[A = \left(A \cap B\right) \cup \left(A \cap B^C\right)\]
        where $A \cap B$ and $A \cap B^C$ are disjoint.

        This yields that: 
        \[\prob\left(A\right) = \prob\left(A \cap B\right) + \prob\left(A \cap B^C\right) = \prob\left(A | B\right)\prob\left(B\right) + \prob\left(A | B^C\right)\prob\left(B^C\right)\]

        Note that this is correct even if $\prob\left(B\right) = 0$ or $\prob\left(B^C\right) = 0$. Reading this formula orally should make it very intuitive.
    }

    \subparag{Intuition}{
        The idea is to draw the following diagram:
        \svghere[0.5]{ConditionalProbability.svg}

        Since we know that $B$ happened, this is like if our new sample space is $B$ and that we are looking for the event $A \cap B$.
    }
}

\parag{Example}{
    We roll two fair dice, one red and one green. Let $A$ and $B$ be the events that the total exceeds 8, and that we get 6 on the red die, respectively. We want to compute the probability of $A$ knowing that $B$ has occurred.

    This can easily be computed thanks to our definition: 
    \[\prob\left(A | B\right) = \frac{\prob\left(A \cap B\right)}{\prob\left(B\right)} = \frac{\frac{4}{6^2}}{\frac{6}{6^2}} = \frac{2}{3}\]

    This is very intuitive: if we know that we rolled a 6 on the red die, then the green die must roll any number which is not 1 or 2.

    Note that we can show $\prob\left(A\right) = \frac{5}{18}$. We indeed see that conditioning the problem may have a big result on the probabilities.
}

\parag{Theorem: Conditional probability space}{
    Let $\left(\Omega, \mathcal{F}, \prob\right)$ be a probability space, and let $B \in \mathcal{F}$ such that $\prob\left(B\right) > 0$. Also, let $\hat{\prob}\left(A\right) = \prob\left(A | B\right)$.

    Then $\left(\Omega, \mathcal{F}, \hat{\prob}\right)$ is a probability space. In particular:
    \begin{itemize}
        \item If $A \in \mathcal{F}$, then $0 \leq \hat{\prob}\left(A\right) \leq 1$
        \item $\hat{\prob}\left(\Omega\right) = 1$
        \item If $\left\{A_i\right\}_{i=1}^{\infty}$ are pairwise disjoint, then: 
        \[\hat{\prob}\left(\bigcup_{i=1}^{\infty} A_i\right) = \sum_{j=1}^{\infty} \hat{\prob}\left(A_i\right)\]
    \end{itemize}
}

\parag{Theorem: Law of total probability}{
    Let $\left\{B_i\right\}_{i=1}^{\infty}$ be pairwise disjoint events (meaning that $B_i \cap B_j = \o$ for $i \neq j$) of the probability space $\left(\Omega, \mathcal{F}, \prob\right)$. Also, let $A$ be an event satisfying $A \subset \bigcup_{i=1}^{\infty} B_i$. 

    Then: 
    \[\prob\left(A\right) = \sum_{i=1}^{\infty} \prob\left(A \cap B_i\right) = \sum_{i=1}^{\infty} \prob\left(A | B_i\right)\prob\left(B_i\right)\]

    \subparag{Intuition}{
        Basically, if the $B_i$ cover entirely $A$, then we can express $A$ as a union of $B_i \cap A$, which are pairwise disjoint (since the $B_i$ originally were pairwise disjoint). 
    }

    \subparag{Remark}{
        We have seen a special case of this theorem above, in the definition of conditional probabilities, with $B$ and $B^C$: 
        \[\prob\left(A\right) = \prob\left(A|B\right)\prob\left(B\right) + \prob\left(A|B^C\right)\prob\left(B^C\right)\]
    }
}

\parag{Bayes' theorem}{
    Let $\left\{B_i\right\}_{i=1}^{\infty}$ be pairwise disjoint events (meaning that $B_i \cap B_j = \o$ for $i \neq j$) of the probability space $\left(\Omega, \mathcal{F}, \prob\right)$. Also, let $A$ be an event satisfying $A \subset \bigcup_{i=1}^{\infty} B_i$ and $\prob\left(A\right) > 0$.

    Then, for any $j \in \mathbb{N}$:
    \[\prob\left(B_j | A\right) = \frac{\prob\left(A | B_j\right)\prob\left(B_j\right)}{\sum_{i=1}^{\infty} \prob\left(A | B_i\right)\prob\left(B_i\right)}\]

    \subparag{Remark 1}{
        Note that this result is also true if the number of $B_i$ is finite, and if the $B_i$ partition $\Omega$ (which is a typical application of this theorem).
    }
    
    \subparag{Remark 2}{
        This theorem is very important. There always are questions on this theorem at exams.
    }

    \subparag{Proof}{
        This proof is very straighforward, using our definitions and our law of total probability: 
        \[\prob\left(B_j | A\right) = \frac{\prob\left(A \cap B_j\right)}{\prob\left(A\right)} = \frac{\prob\left(A | B_j\right)\prob\left(B_j\right)}{\prob\left(A\right)} = \frac{\prob\left(A | B_j\right)\prob\left(B_j\right)}{\sum_{i=1}^{\infty} \prob\left(A | B_i\right)\prob\left(B_i\right)}\]
    }
    
}

\parag{Example}{
    We suspect that the man in front of us at the security check at the airport is a terrorist. We know that one person out of $10^6$ is a terrorist, and that a terrorist is detected by security check with a probability of 0.9999, but that the alarm may still be triggered when a normal person goes through with a probability $10^{-5}$. We wonder what is the probability that he is a terrorist, given that the alarm is triggered when he passes through security.

    Let $A$ be the event that the alarm is triggered and $T$ be the one that he is a terrorist. We can apply Bayes' rule: 
    \[\prob\left(T | A\right) = \frac{\prob\left(A |T\right)\prob\left(T\right)}{\prob\left(A|T\right)\prob\left(T\right) + \prob\left(A|T^C\right)\prob\left(T^C\right)} = \frac{0.9999 \cdot 10^{-6}}{0.9999\cdot 10^{-6} + 10^{-5}\cdot \left(1 - 10^{-6}\right)}\]

    Which is approximately: 
    \[\prob\left(T|A\right) \approx 0.0909 = \SI{9.09}{\percent}\]
    
    
    This shows that $\prob\left(T | A\right) \neq \prob\left(A | T\right)$, rather unintuitively if we are not careful (it is a typical bias in statistics to assume they are equal).
}


\end{document}
