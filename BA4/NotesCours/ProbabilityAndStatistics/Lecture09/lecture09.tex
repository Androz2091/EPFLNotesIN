% !TeX program = lualatex
% Using VimTeX, you need to reload the plugin (\lx) after having saved the document in order to use LuaLaTeX (thanks to the line above)

\documentclass[a4paper]{article}

% Expanded on 2023-03-21 at 13:21:41.

\usepackage{../../style}

\title{Probability and statistics}
\author{Joachim Favre}
\date{Mardi 21 mars 2023}

\begin{document}
\maketitle

\lecture{9}{2023-03-21}{Going continuous}{
\begin{itemize}[left=0pt]
    \item Definition of conditional probability mass distributions, and proof that they indeed follow the axioms of PMFs.
    \item Definition of conditional expected value.
    \item Definition of convergence in distribution.
    \item Proof of the law of small numbers.
    \item Definition of continuous random variables, and their PDF and CDF.
\end{itemize}

}

\parag{Example}{
    We want to compute the expected value and variance of $X \followsdistr \text{Geom}\left(p\right)$.

    We know that $\prob\left(X \geq x\right) = \left(1-p\right)^{x-1}$. By our theorem, we get that: 
    \[\exval\left(X\right) = \sum_{x=1}^{\infty} \left(1-p\right)^{x-1} = \frac{1}{1 - \left(1-p\right)} = \frac{1}{p}\]

    Now, let's compute the variance. Let's first compute the following value: 
    \autoeq{\exval\left[X\left(X-1\right)\right] = 2 \sum_{x=2}^{\infty} \left(x-1\right)\left(1-p\right)^{x-1} = 2\left(1-p\right) \frac{d}{dp} \left[-\sum_{x=1}^{\infty} \left(1-p\right)^{x-1}\right] = 2\left(1-p\right) \frac{d}{dp} \left(\frac{-1}{p}\right) = \frac{2\left(1-p\right)}{p^2}}
    
    This trick of seeing that a series is the $n$\Th derivative of a sum we know how to compute (typically for power series) is very common and must be remembered. 

    We finally get that the variance is: 
    \[\Var\left(X\right) = \exval\left[X\left(X-1\right)\right] + \exval\left(X\right) - \exval\left(X\right)^2 = \frac{2\left(1-p\right)}{p^2} + \frac{1}{p} - \frac{1}{p^2} = \frac{1-p}{p^2}\]

    As a quick sanity check, we can notice that it gets smaller as $p\to1$ (there is less variance when we have more probability to success) and larger as $p\to0$ (there is a lot of variance when we have a small probability of success). This is what intuition tells us, so this is a good thing.
}

\subsection{Conditional probability distributions}
\parag{Definition: Conditional probability distribution}{
    Let $\left(\Omega, \mathcal{F}, \prob\right)$ be a probability space, on which we define a random variable $X$, and let $B \in \mathcal{F}$ such that $\prob\left(B\right) > 0$.

    Then, the \important{conditional probability mass function} of $X$ given $B$ is: 
    \[f_X\left(x |B\right) = \prob\left(X = x | B\right) = \frac{\prob\left(A_x \cap B\right)}{\prob\left(B\right)}\]
}

\parag{Theorem}{
    The function $f_X\left(x | B\right)$ is a well defined mass function. In other words: 
    \[f_X\left(x | B\right) \geq 0, \mathspace \sum_{x}^{} f_X\left(x | B\right) = 1\]
}

\parag{Proposition}{
    Let $B$ be an event of the form $X \in \mathcal{B}$, for some $\mathcal{B} \subset \mathbb{R}$. Then:
    \[f_X\left(x | B\right) = \frac{I\left(x \in \mathcal{B}\right)}{\prob\left(X \in \mathcal{B}\right)} f_X\left(x\right)\]

    \subparag{Intuition}{
        This means that:
        \begin{functionbypart}{f_X\left(x | B\right)}
            0, & x \not \in \mathcal{B} \\
            c f_X\left(x\right), & x \in \mathcal{B}
        \end{functionbypart}
        where $c = \frac{1}{\prob\left(X \in \mathcal{B}\right)}$ is a normalisation factor.

        We thus zero-out every terms which cannot happen, and normalise the rest.
    }
    
    \subparag{Proof}{
        This is a direct proof: 
        \autoeq{f_X\left(x | B\right) = \frac{\prob\left(X = x \cap X \in \mathcal{B}\right)}{\prob\left(X \in \mathcal{B}\right)} = \frac{\prob\left(X \in \mathcal{B} | X = x\right)}{\prob\left(X \in \mathcal{B}\right)}\prob\left(X = x\right) = \frac{I\left(x \in \mathcal{B}\right)}{\prob\left(X \in \mathcal{B}\right)} f_X\left(x\right)}
        
        Indeed, $\prob\left(X \in \mathcal{B} | X = x\right)$ is one if $x \in \mathcal{B}$ and 0 otherwise. It is thus an indicator variable.

        \qed
    }
}

\parag{Example}{
    We want to compute the conditional PMF of $X \followsdistr \text{Geom}\left(p\right)$ given that $X > n$.

    We know that the event $B = \left\{X > n\right\}$ has probability $\left(1-p\right)^n$ (the first success is after the $n$\Th trial is equivalent to saying that the first $n$ experiments failures). Thus:
    \[f_X\left(x | B\right) = \frac{\prob\left(X = x \cap X > n\right)}{\prob\left(X > n\right)} = \frac{I\left(x > n\right)}{\prob\left(X > n\right)} f_X\left(x\right) = p\left(1 - p\right)^{x-n-1}\]
    for $x = n+1, n+2, \ldots$. Note that the indicator variable hides in the fact that we only consider $x \geq n+1$.
}


\parag{Definition: Conditional expected value}{
    Let $\left(\Omega, \mathcal{F}, \prob\right)$ be a probability space, where we consider some random variable $X$ and event $B \in \mathcal{F}$. We also suppose that $\sum_{x}^{} \left|g\left(x\right)\right| f_X\left(x | B\right) < \infty$.

    The \important{conditional expected value} of $g\left(X\right)$ given $B$ is defined as: 
    \[\exval\left[g\left(X\right) | B\right] = \sum_{x}^{} g\left(x\right) f_X\left(x | B\right)\]
}

\parag{Theorem}{
    Let $X$ be random variable with expected value $\exval\left(X\right)$, and $\left\{B_i\right\}_{i=1}^{\infty}$ be partition such that $\prob\left(B_i\right) > 0$ for all $i$ and such that the following sum is absolutely convergent.

    Then: 
    \[\exval\left(X\right) = \sum_{i=1}^{\infty} \exval\left(X |B_i\right)\prob\left(B_i\right)\]

    This is very similar to the theorem of total probability.
    
    \subparag{Implication}{
        Let $B$ be an event with $\prob\left(B\right) > 0$ and $\prob\left(B^C\right) > 0$. Then, our theorem implies that: 
        \[\exval\left(X\right) = \exval\left(X|B\right)\prob\left(B\right) + \exval\left(X | B^c\right)\prob\left(B^c\right)\]

        This follows immediately from the theorem, by setting $B_1 = B$, $B_2 = B^C$ and $B_3 = B_4 = \ldots = \o$.
    }

    \subparag{Proof}{
        Just as one can think, we will use the theorem of total probability: 
        \[f\left(x\right) = \prob\left(X = x\right) = \sum_{i=1}^{\infty} \prob\left(X = x | B_i\right) \prob\left(B_i\right) = \sum_{i = 1}^{\infty} f\left(x | B_i\right)\prob\left(B_i\right)\]
        
        So, this gives that: 
        \autoeq{\exval\left(X\right) = \sum_{x}^{} xf\left(x\right) = \sum_{x}^{} x \sum_{i=1}^{\infty} f\left(x | B_i\right)\prob\left(B_i\right) = \sum_{i=1}^{\infty} \left[\sum_{x}^{} xf\left(x | B_i\right)\right]\prob\left(B_i\right) = \sum_{i=1}^{\infty} \exval\left(X | B_i\right)\prob\left(B_i\right)}

        \qed
    }
}

\parag{Example 1}{
    We want to compute the expected value of $X \followsdistr \text{Geom}\left(p\right)$ given that $X > n$.

    We have seen in the previous example that: 
    \[f_X\left(x | B\right) = p\left(1-p\right)^{x-n-1}, \mathspace x = n+1, \ldots\]
    
    This gives us that: 
    \[\exval\left(X | B\right) = \sum_{x=n+1}^{\infty} x f_X\left(x|B\right) = \sum_{x=n+1}^{\infty} xp\left(1-p\right)^{x-n-1}\]
    
    Setting $y = x -n$, this gives: 
    \[\exval\left(X | B\right) = \sum_{y=1}^{\infty} \left(n+y\right)p\left(1-p\right)^{y-1} = n \sum_{y=1}^{\infty} p\left(1-p\right)^{y-1} + \sum_{y=1}^{\infty} yp\left(1-p\right)^{y-1} = n + \frac{1}{p}\]
    
    Indeed, the first sum is the probability distribution of the geometric distributions (and thus sums to 1), and the second one is their expected value. This result is rather intuitive.
}

\parag{Example 2}{
    Let's now say that we want to compute the conditional probability of $X \followsdistr \text{Geom}\left(p\right)$ given that $B^C = \left\{X \leq n\right\}$.

    However, we know $\exval\left(X\right)$ and $\exval\left(X | B\right)$: the second one was just computed in the previous example. Thus, we know that: 
    \autoeq{\exval\left(X\right) = \exval\left(X | B\right)\prob\left(B\right) + \exval\left(X | B^C\right)\prob\left(B^C\right) \iff \exval\left(X | B^C\right) = \frac{\exval\left(X\right) - \exval\left(X | B\right)\prob\left(B\right)}{\prob\left(B^C\right)} = \frac{\frac{1}{p} - \left(n + \frac{1}{p}\right)\left(1-p\right)^n}{1 - \left(1 - p\right)^n}}
}

\subsection{Law of small numbers}
\parag{Definition: Convergence of distributions}{
    Let $\left\{X_n\right\}$ and $X$ be random variables with cumulative distribution functions $\left\{F_n\right\}$ and $F$, respectively.

    We say that $\left\{X_n\right\}$ \important{converge in distribution} (or converge in laws) to $X$ if, for all $x \in \mathbb{R}$ where $F$ is continuous: 
    \[\lim_{n \to \infty} F_n\left(x\right) = F\left(x\right)\]
    
    We write $X_n \over{\to}{$D$} X$ (or $X_n \over{\to}{$\left(d\right)$} X$) as $n \to \infty$.

    \subparag{Remark}{
        In the case of discrete random variables, meaning $D_X \subset \mathbb{Z}$, this is equivalent to: 
        \[\lim_{n \to \infty} f_n\left(x\right) = f\left(x\right)\]
    }
}

\parag{Theorem: Law of small numbers}{
    Let $\left(p_n\right)$ be a sequence of probabilities such that $\lim_{n \to \infty} n p_n = \lambda > 0$, and let $X_n \followsdistr B\left(n, p_n\right)$ be a sequence of random variables following binomial distributions. Finally, let $X \followsdistr \text{Pois}\left(\lambda\right)$.

    Then:
    \[X_n \over{\to}{D} X\]

    \subparag{Implication}{
        We can use this theorem to approximate binomial probabilities for large $n$ and small $p$, by using Poisson probabilities.
    }
    
    
    \subparag{Proof}{
        Let $r$ be fixed. We want to show that the PMF of the binomial distribution tends towards the one of the Poisson distribution:
        \autoeq{\lim_{n \to \infty} \binom{n}{r} p_n^{r} \left(1-p_n\right)^{n-r} = \lim_{n \to \infty} \underbrace{n^{-r} \binom{n}{r}}_{\to \frac{1}{r!}} \underbrace{\left(n p_n\right)^{r}}_{\lambda^r} \underbrace{\left(1 - \frac{np_n}{n}\right)^{n-r}}_{\to e^{-\lambda}} = \frac{1}{r!} \lambda^r e^{-\lambda}}
        
        We use the fact that: 
        \[\lim_{n \to \infty} n^{-r} \binom{n}{r} = \frac{1}{r!}\]

        \qed
    }
}

\parag{Proposition}{
    Let $X_n$ be a hypergeometric variable: 
    \[\prob\left(X_n = x\right) = \frac{\binom{m}{x} \binom{N-m}{n-x}}{\binom{N}{x}}\]
    
    Also, let $N, m \to \infty$ such that $\frac{m}{N} \to p$ for some $0 < p < 1$. 

    The, the limiting distribution of $X_N$ is $B\left(n, p\right)$: 
    \[\prob\left(X_N = x\right) \to \binom{n}{x} p^x \left(1-p\right)^{n-x}, \mathspace i = 0, \ldots, n\]
     
    \subparag{Proof}{
        Multiplying by some terms which divide out to 1, it gives: 
        \[\frac{\binom{m}{x} \binom{N-m}{n-x}}{\binom{N}{n}} = \frac{m^x}{N^x}\cdot \frac{\left(N - m\right)^{n-x}}{N^{n-x}} \cdot  \frac{\overbrace{m^{-x} \binom{m}{x}}^{\to \frac{1}{x!}}\overbrace{\left(N - m\right)^{-\left(n-x\right)}\binom{N-m}{n-x}}^{\frac{1}{\left(n-x\right)!}}}{\underbrace{N^{-n} \binom{N}{n}}_{\to \frac{1}{n!}}}\]
        which, as $N \to +\infty$, tends towards:
        \[p^{x} \left(1-p\right)^{n-x} \frac{n!}{x! \left(n-x\right)!} = p^x \left(1-p\right)^{n-x} \binom{n}{x}\]
        as required.

        \qed
    }
}

\section{Continuous random variables}
\subsection{Fundamentals}

\parag{Definition: Continuous random variable}{
    Let $X$ be a random variable, and $D_x$ be its support: 
    \[D_X = \left\{x \in \mathbb{R} \suchthat \exists \omega \in \Omega \text{ such that } X\left(\omega\right) = x\right\}\]
    
    This is a \important{continuous random variable} if $D_X$ is infinite non-countable.
}


\parag{Definition: CDF}{
    Let $\left(\Omega, \mathcal{F}, \prob\right)$ be a probability space.

    The \important{cumulative distribution function} of a continuous random variable $X$ on this probability space is defined just as for discrete random variables: 
    \[F\left(x\right) = \prob\left(X \leq x\right) = \prob\left(\mathcal{B}_x\right), \mathspace x \in \mathbb{R}\]
    where $\mathcal{B}_x = \left\{\omega \suchthat X\left(\omega\right) \leq x\right\} \subset \Omega$.
}

\parag{Observation}{
    The probability that a continuous random variable is an exact value is 0. For instance, the probability that we get exactly 0.5 when picking a random number in $\left[0, 1\right]$ is 0.

    We need to switch our point of view: we will consider masses. We know that the mass of any particle of some continuous matter weigh 0, but that the whole object still weighs something. When we consider point-wise functions, we must not consider mass, but densities. This thus leads to the following definition.
}

\parag{Definition: PDF}{
    A random variable $X$ is continuous if there exists a function $f\left(x\right)$, called the \important{probability density function} (or density), named PDF, of $X$, such that: 
    \[\prob\left(X \leq x\right) = F\left(x\right) = \int_{-\infty}^{x} f\left(u\right)du, \mathspace x \in \mathbb{R}\]
    
    By using the properties of $F$, we know that this implies that $f\left(x\right) \geq 0$ and $\int_{-\infty}^{\infty} f\left(x\right)dx = 1$.

    \subparag{Remark}{
        By using the fundamental theorem of calculus, we get that: 
        \[f\left(x\right) = \frac{dF\left(x\right)}{dx}\]

        This means that $F\left(x\right)$ needs to be continuous. This function $F\left(x\right)$ can only have jumps for discrete random variables.
    }

    \subparag{Observation}{
        We notice that: 
        \[\prob\left(X = x\right) = \lim_{\delta \to 0} \int_{x}^{x + \delta} f\left(x\right)dx = 0\]

        As expected, a single point has zero probability.
    }
}

\end{document}
