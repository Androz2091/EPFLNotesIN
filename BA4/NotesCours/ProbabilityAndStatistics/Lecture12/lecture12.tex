% !TeX program = lualatex
% Using VimTeX, you need to reload the plugin (\lx) after having saved the document in order to use LuaLaTeX (thanks to the line above)

\documentclass[a4paper]{article}

% Expanded on 2023-03-30 at 13:19:38.

\usepackage{../../style}

\title{Probability and statistics}
\author{Joachim Favre}
\date{Jeudi 30 mars 2023}

\begin{document}
\maketitle

\lecture{12}{2023-03-30}{Generalisation of our definitions}{
\begin{itemize}[left=0pt]
    \item Definition of marginal PMF and PDF, and proof that they work like we want them to.
    \item Definition of conditional PMF and PDF for an arbitrary set of random variable.
    \item Definition of the multinomial distribution.
\end{itemize}
}

\parag{Theorem: Marginal PMF and PDF}{
    The \important{marginal probability mass function} of a discrete random variable is: 
    \[f_X\left(x\right) = \prob\left(X = x\right) = \sum_{y}^{} f_{X, Y}\left(x, y\right), \mathspace x \in \mathbb{R}\]
    
    The \important{marginal probability density function} of a continuous random variable is: 
    \[f_X\left(x\right) = \int_{-\infty}^{\infty} f_{X, Y}\left(x, y\right)dy\]
    
    \subparag{Proof}{
        We consider the discrete case, the continuous case is similar.

        We are looking for $f_X\left(x\right) = \prob\left(X = x\right)$: 
        \autoeq{\prob\left(X = x\right) = \sum_{y}^{} \prob\left(X = x | Y= y\right)\prob\left(Y = y\right) = \sum_{y}^{} \prob\left(X= x, Y = y\right) = \sum_{y}^{} f_{X, Y}\left(x, y\right)}

        \qed
    }
}

\parag{Definition: Conditional PMF and PDF}{
    The \important{conditional probability mass/density function} of $Y$ given $X$ is, supposing that $f_X\left(x\right) > 0$: 
    \[f_{Y|X}\left(y|x\right) = \frac{f_{X, Y}\left(x, y\right)}{f_X\left(x\right)}, \mathspace y \in \mathbb{R}\]
    
    If $\left(X, Y\right)$ is discrete, we also have that: 
    \[f_{Y|X}\left(y|x\right) = \prob\left(Y = y | X = x\right)\]
}

\parag{Example}{
    Let's say every day we receive $X \followsdistr P\left(100\right)$ emails (following a Poisson distribution with $\lambda = 100$). Each is a spam independently with probability $p = 0.9$. We want to find the distribution of the total number of emails we received, given that we received 15 good ones.

    Let $Y$ be the number of good emails we received. This follows: 
    \[Y | X = x \followsdistr B\left(x, 0.1\right) \implies \prob\left(Y = y | X = x\right) = \binom{x}{y} p^y \left(1-p\right)^{x-y}\]
    
    Now, we know that: 
    \[f_Y\left(y\right) = \sum_{x}^{} f_{X, Y}\left(x, y\right) = \sum_{x}^{} f_{Y|X}\left(y|x\right) f_X\left(x\right) = \sum_{x}^{} \binom{x}{y} p^y \left(1-p\right)^{x-y} \frac{\lambda^x}{x!} e^{-\lambda}\]

    We can show that this is a Poisson mass function with parameter $\lambda\left(1-p\right)$: 
    \[f_Y\left(y\right) = \frac{\left[\lambda\left(1-p\right)^y\right]}{y!} e^{-\lambda\left(1-p\right)}\]
}

\parag{Definition: Joint CDF}{
    Let $X_1, \ldots, X_n$ be random variables defined on the same probability space. Their \important{joint cumulative distribution function} is: 
    \[F_{X_1, \ldots, X_n} \left(x_1, \ldots, x_n\right) = \prob\left(X_1 \leq x_1, \ldots, X_n \leq x_n\right)\]
}

\parag{Definition: Joint PMF/PDF}{
    Let $X_1, \ldots, X_n$ be random variables defined on the same probability space. 

    If they are continuous, their \important{joint probability density function} is: 
    \[f_{X_1, \ldots, X_n}\left(x_1, \ldots, x_n\right) = \frac{\partial^n F_{X_1, \ldots, X_n}\left(x_1, \ldots, x_n\right)}{\partial x_1 \cdots \partial x_n}\]
    where the order of derivatives does not mater with the functions we will consider.

    If they are discrete, their \important{joint probability mass function} is: 
    \[f_{X_1, \ldots, X_n}\left(x_1, \ldots, x_n\right) = \prob\left(X_1 = x_1, \ldots, X_n = x_n\right)\]
}

\parag{Definition: Marginal}{
    Let $\mathcal{A} \subset \left\{1, 2, \ldots, n\right\}$. We consider a $n$-tuple of random variables $X_{\left\{1, \ldots, n\right\}} = \left(X_1, \ldots, X_n\right)$, writing for instance $X_{\left\{2, 3, 5\right\}} = \left(X_2, X_3, X_5\right)$.

    If our random variables are continuous, the marginal distribution of $X_{\mathcal{A}}$ is: 
    \[f_{X_{\mathcal{A}}}\left(x\right) = \int_{\bar{\mathcal{A}}}^{} f_{X_{\left\{1, \ldots, n\right\}}}\left(x_1, \ldots, x_n\right) dx_{\mathcal{A}}\]
    
    If our random variables are discrete, the marginal distribution of $X_{\mathcal{A}}$ is: 
    \[f_{X_{\mathcal{A}}}\left(x\right) = \sum_{x_{\mathcal{A}} \in \bar{\mathcal{A}}} f_{X_{\left\{1, \ldots, n\right\}}}\left(x_1, \ldots, x_n\right)\]
}

\parag{Definition: Conditional density}{
    Let $\mathcal{A}, \mathcal{B} \subset \left\{1, 2, \ldots, n\right\}$ be such that $\mathcal{A} \cap \mathcal{B} = \o$. The conditional probability mass/density function of some set of random variables $Y_{\mathcal{B}}$ given another set of random variables $X_{\mathcal{A}}$ is, supposing that $f_{X_{\mathcal{A}}}\left(x\right) > 0$: 
    \[f_{Y_{\mathcal{B}}|X_{\mathcal{A}}}\left(y|x\right) = \frac{f_{X_{\mathcal{A}}, Y_{\mathcal{B}}}\left(x, y\right)}{f_{X_{\mathcal{A}}}\left(x\right)}, \mathspace y \in \mathbb{R}\]
}


\parag{Example}{
    Let's say that we have $F_{X, Y}\left(x, y\right)$ and that we want to find $F_{X}\left(x\right)$. To do so, we notice that: 
    \[F_{X}\left(x\right) = \prob\left(X \leq x\right) = \prob\left(X \leq x, Y < +\infty\right) = \lim_{y \to \infty} F_{X, Y}\left(x, y\right)\]
}

\parag{Definition: Multinomial distribution}{
    Let $m \in \mathbb{N}$ and $p_1, \ldots, p_k \in \left[0, 1\right]$ such that $p_1 + \ldots + p_k = 1$.

    The random variable $\left(X_1, \ldots, X_k\right)$ has the \important{multinomial distribution} of denominator $m$ and probabilities $\left(p_1, \ldots, p_k\right)$ if its mass function is: 
    \[f\left(x_1, \ldots, x_k\right) = \frac{m!}{x_1! \cdots x_k!} p_1^{x_1} \cdots p_k^{x_k} \mathspace x_1, \ldots, x_k \in \left\{0, \ldots, m\right\} \text{ and } \sum_{j=1}^{k} x_j = m\]

    This is written $\left(X_1, \ldots, X_k\right) \followsdistr B\left(m, p_1, \ldots, p_k\right)$

    \subparag{Intuition}{
        This is the probability that, having $m$ balls, we have $x_1$ balls of the first colour, $x_2$ balls of the second, and so on; knowing that the first colour has probability $p_1$ to appear, and so on.
    }
}

\parag{Example}{
    Let us say that $n$ people vote for three candidates, independently with probabilities $p_1 = 0.45$, $p_2 = 0.4$ and $p_3 = 0.15$ (where $p_i$ represents the probability to vote for the $i$\Th candidate). 

    Let $X_1, X_2, X_3$ represent the number of votes for each candidate. This follows exactly a multinomial distribution:
    \[\prob\left(X_1 = x_1, X_2 = x_2, X_3 = x_3\right) = \frac{n!}{x_1!x_2!x_3!} 0.45^{x_1} 0.4^{x_2} 0.15^{x_3}\]
}

\end{document}
