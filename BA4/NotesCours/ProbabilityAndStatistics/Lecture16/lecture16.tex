% !TeX program = lualatex
% Using VimTeX, you need to reload the plugin (\lx) after having saved the document in order to use LuaLaTeX (thanks to the line above)

\documentclass[a4paper]{article}

% Expanded on 2023-04-25 at 13:19:41.

\usepackage{../../style}

\title{ProbaStats}
\author{Joachim Favre}
\date{Mardi 25 avril 2023}

\begin{document}
\maketitle

\lecture{16}{2023-04-25}{The storm}{
\begin{itemize}[left=0pt]
    \item Definition of mean vector and variance matrix.
    \item Definition of the moment-generating function of random vectors.
    \item Explanation of some of the properties of MGFs of random vectors.
    \item Definition of the multivariate normal distribution.
    \item Proof of some of the properties of the multivariate normal distribution.
\end{itemize}

}

\parag{Definition: Mean vector}{
    Let $X = \left(X_1, \ldots, Xp\right)^T \in \mathbb{R}^{p}$ be a vector of random variables. Its \important{expectation} (or \important{mean vector}) is: 
    \[\exval\left(X\right)_{p \times 1} = \begin{pmatrix} \exval\left(X_1\right) \\ \vdots \\ \exval\left(X_p\right) \end{pmatrix} \]
}

\parag{Definition: PSD matrix}{
    Let $\Omega$ be a matrix. It is \important{positive semi-definite} (PSD), denoted $\Omega \succeq 0$, if it is symmetric and, for all vector $x \in \mathbb{R}^n$: 
    \[x^T \Omega x \geq 0\]
    
    \subparag{Implications}{
        Since it is symmetric, it is orthonormally diagonalisable with real eigenvalues by the spectral theorem. In other wrods, there exists a diagonal matrix $D$ and matrix $U$ for which $UU^T = I_p$, such that: 
        \[\Omega = U D U ^T\]

        Moreover, since $x^T \Omega x \geq 0$ for any vector $x$, it implies that all its eigenvalues are positive.
    }
}

\parag{Definition: positive definite matrix}{
    Let $\Omega$ be a matrix. It is \important{positive definite}, denoted $\Omega \succ 0$, if it is symmetric and, for all vector $x \in \mathbb{R}^n \setminus \left\{0\right\}$: 
    \[x^T \Omega x > 0\]
    
    \subparag{Implications}{
        A positive definite matrix has all the properties of a PSD matrix, except that all its eigenvalues are strictly positive, and thus it is full rank (meaning that it is inversible).
    }
}

\parag{Definition: (co)-variance matrix}{
    Let $X = \left(X_1, \ldots, X_p\right)^T$ be a vector of random variables. Its \important{(co)-variance matrix} is: 
    \[\Var\left(X\right)_{p\times p} = \begin{pmatrix} \Var\left(X_1\right) & \Cov\left(X_1, X_2\right) & \cdots & \Cov\left(X_1, X_p\right) \\ \Cov\left(X_1, X_2\right) & \Var\left(X_2\right) & \cdots & \Cov\left(X_2, X_p\right) \\ \vdots & \vdots & \ddots & \vdots \\ \Cov\left(X_1, X_p\right) & \Cov\left(X_2, X_p\right) & \cdots & \Var\left(X_p\right) \end{pmatrix} \]

    \subparag{Remark}{
        We notice that, for any vector $a = \left(a_1, \ldots, a_p\right)^T \in \mathbb{R}^p$, we have:
        \autoeq{\Var\left(\sum_{j=1}^{p} a_j X_j\right) = \sum_{i = 1}^{p} \sum_{j=1}^{p} \Cov\left(a_i X_i, a_j X_j\right)= \sum_{i=1}^{p} \sum_{j=1}^{p} a_i a_j \Cov\left(X_i, X_j\right) = a^T \Var\left(X\right) a}

        However this matrix is symmetric (since $\Cov\left(X, Y\right) = \Cov\left(Y, X\right)$). We can combine this with the following fact to get this matrix is positive semi-definite: 
        \[a^T \Var\left(X\right) a = \Var\left(\sum_{j=1}^{p} a_j X_j\right) \geq 0\]
    }
}

\parag{Definition: MGF of random vector}{
    Let $X_{p\times 1} = \left(X_1, \ldots, X_p\right) \in \mathbb{R}^p$ be a random vector. Its \important{moment-generating function} (MGF) is: 
    \[M_X\left(t\right) = \exval\left(e^{t \dotprod X}\right) = \exval\left(e^{t^T X}\right) = \exval\left(e^{\sum_{r=1}^{p} t_r X_r}\right), \mathspace \forall t \in \mathcal{T}\]
    where $\mathcal{T}$ is the set of $t \in \mathbb{R}^p$ where this converges, meaning:
    \[\mathcal{T} = \left\{t \in \mathbb{R}^p \suchthat M_X\left(t\right) < \infty\right\}\]
}

\parag{Proposition: Properties of MGF}{
    Let $X_{p\times 1}$ be a random vector. Its MGF has the following properties:
    \begin{itemize}
        \item $0 \in \mathcal{T} \subset \mathbb{R}^p$ and $M_X\left(0\right) = 1$.
        \item The mean vector of $X_{p\times1}$ is: 
        \[\exval\left(X\right)_{p \times 1} = M'_X\left(0\right) = \frac{\partial M_X}{\partial t}\left(0\right) = \begin{pmatrix} \frac{\partial M_x}{\partial t_1} \left(0\right) \\ \vdots \\ \frac{\partial M_x}{\partial t_p} \left(0\right) \end{pmatrix} \in \mathbb{R}^p\]
        
        \item The variance matrix of $X_{p \times 1}$ is: 
        \[\Var\left(X\right)_{p \times p} = \frac{\partial^{2} M_X}{\partial t \partial t^T}\left(0\right) - M'_X\left(0\right) M'_X\left(0\right)^T\]
        where $\frac{\partial^{2} M_X}{\partial t \partial t^T}$ is the Hessian matrix of $M_X$:
        \[\frac{\partial^{2} M_X}{\partial t \partial t^T}\left(t\right) = \begin{pmatrix} \frac{\partial^2 M_X}{\partial t_1^2}\left(t\right)  & \cdots & \frac{\partial^2 M_X}{\partial t_1 \partial t_p}\left(t\right) \\ \vdots & \ddots & \vdots \\ \frac{\partial^2 M_X}{\partial t_p \partial t_1}\left(t\right) & \cdots & \frac{\partial^2 M_X}{\partial t_p^2}\left(t\right) \end{pmatrix} \]
        \item Let $\mathcal{A}, \mathcal{B} \subset \left\{1, \ldots, p\right\}$ be such that they are disjoint (meaning $\mathcal{A} \cap \mathcal{B} = \o$). As usual, we denote $X_{\mathcal{A}}$ for the subvector of $X$ containing $\left\{X_j \suchthat j \in \mathcal{A}\right\}$ (for instance $X_{\left\{1, 5\right\}} = \left(X_1, X_5\right)^T$). $X_{\mathcal{A}}$ and $X_\mathcal{B}$ are independent if and only if: 
        \[M_X\left(t\right) = \exval\left(e^{t_{\mathcal{A}} \dotprod X_{\mathcal{A}} + t_{\mathcal{B}} \dotprod X_{\mathcal{B}}}\right) = M_{X_{\mathcal{A}}}\left(t_{\mathcal{A}}\right) M_{X_{\mathcal{B}}}\left(t_{\mathcal{B}}\right), \mathspace t \in \mathcal{T}\]
        \item There is an injective mapping from MGFs to probability distributions. In other words, different probability distributions which MGFs exist lead to different MGFs.
        \end{itemize}
}

\parag{Definition: Characteristic function}{
    The \important{characteristic function} of some random variable $X$ is: 
    \[\phi_X\left(t\right) = \exval\left(e^{i t X}\right), \mathspace t \in \mathbb{R}\]

    This is also known as a \important{Fourier transform}.

    \subparag{Remark}{
        Many distributions do not have a MGF, their Laplace transform does not converge. Thus, using a Fourier transform can be very useful for proofs, even though they are a bit trickier since they require complex analysis.
    }
}

\subsection{Multivariate normal distribution}
\parag{Definition: Multivariate normal distribution}{
    Let some random vector $X = \left(X_1, \ldots, X_p\right)^T$. It follows a \important{multivariate normal distribution} (or it is jointly Gaussian) if there exists a $p \times 1$ vector $\mu = \left(\mu_1, \ldots, \mu_p\right)^T \in \mathbb{R}^p$ and a $p \times p$ symmetric matrix with elements $\omega_{jk}$ such that: 
    \[u^T X \followsdistr \mathcal{N}\left(u^T \mu, u^T \Omega u\right), \mathspace u \in \mathbb{R}^p\]
    
    We write $X \followsdistr \mathcal{N}_p\left(\mu, \Omega\right)$.

    \subparag{Remark}{
        If they exist, then we have:
        \[\mu = \exval\left(X\right), \mathspace \Omega = \Var\left(X\right)\]
    }

    \subparag{Observation}{
        Since $0 \leq \Var\left(u^T X\right) = u^T \Omega u$ for any $\mathbb{R}^p$, this means that $\Omega$ must be positive semi-definite.
    }
    
    \subparag{Terminology}{
        Saying that a random vector follows a normal distribution is ambiguous. It may mean that each component follows a normal distribution, or that the vector follows a multivariate normal distribution.
    }

    \subparag{Intuition}{
        The idea is that it is constructed such that, no matter the line on which we project our variables, it is a 1D Gaussian distribution.
    }
    
}

\parag{Example: Degenerate case}{
    Let us consider: 
    \[X_1 \followsdistr  \mathcal{N}\left(0, 1\right), \mathspace X_2 = - X_1\]
    
    We notice that: 
    \[\Cov\left(X_1, X_2\right) = \Cov\left(X_1, -X_1\right) = -\Var\left(X_1\right) = -1\]

    This implies that:
    \[\begin{pmatrix} X_1 \\ X_2 \end{pmatrix} \followsdistr \mathcal{N}\left(\begin{pmatrix} 0 \\ 0 \end{pmatrix}, \begin{pmatrix} 1 & -1 \\ -1 & 1 \end{pmatrix} \right)\]
    
    However, in this case, if we consider $u = \left(1, 1\right)^T \in \mathbb{R}^2$: 
    \[u^T X = X_1 + X_2 \followsdistr \mathcal{N}\left(0, 0\right)\]
    
    This is a degenerate case, which is slightly problematic since we divide by the standard deviation in the PDF of the normal distribution. However, since this only happens when we have a linear dependence between $X_1$ and $X_2$, this can be fixed by not considering $X_2$ when doing our distribution computations, and then using $X_2 = -X_1$. 

    In other words, if we have a degenerate case, we can decrease the dimensions of our matrix (to $\mathbb{R}^{1\times 1}$ in this case) in order for it to match its rank, and thus make it positive definite. When $\Omega$ is positive definite (meaning that all its eigenvalues are strictly positive and thus that it is full rank), everything works perfectly fine.
}

\parag{Proposition: Mean and covariance of joint normal distribution}{
    Let $X \followsdistr N_p\left(\mu, \Omega\right)$, with $\omega_{ij} = \Omega_{ij}$. 

    Then: 
    \[\exval\left(X_j\right) = \mu_j, \mathspace \Var\left(X_j\right) = \omega_{jj}, \mathspace \Cov\left(X_j, X_k\right) = \omega_{jk}\]
    
    \subparag{Proof}{
        Let $e_j \in \mathbb{R}^p$ be the vector with a 1 at the $j$\Th place and zeros everywhere else. We then have that: 
        \[X_j = e_j^T X \followsdistr \mathcal{N}\left(\mu_j, \omega_j\right)\]
        
        It follows that: 
        \[\exval\left(X_j\right) = \mu_j, \mathspace \Var\left(X_j\right) = \omega_{jj}\]

        For the final property, we can see that: 
        \[X_j + X_k = \left(e_j + e_k\right)^T X \followsdistr \mathcal{N}\left(\mu_j + \mu_k, \omega_{jj} + \omega_{kk} + 2 \omega_{jk}\right)\]
        
        This gives us: 
        \autoeq{\Var\left(X_j + X_k\right) = \omega_{jj} + \omega_{kk} + 2\omega_{jk} \iff \Var\left(X_j\right) + \Var\left(X_k\right) + 2\Cov\left(X_j, X_k\right) = \omega_{jj} + \omega_{kk} + 2\omega_{jk} \iff \Cov\left(X_j, X_k\right) = \omega_{jk}}

        \qed
    }
}

\parag{Theorem: Properties of the joint normal distribution}{
    Let $X \followsdistr N_p\left(\mu, \Omega\right)$. We have the following properties:
    \begin{enumerate}
        \item The moment-generating function of $X$ is: 
        \[M_X\left(t\right) = \exp\left(t^T \mu + \frac{1}{2} t^T \Omega t\right), \mathspace t \in \mathbb{R}^p\]
        \item Let $\mathcal{A}, \mathcal{B} \subset \left\{1, \ldots, p\right\}$ be such that $\mathcal{A} \cap \mathcal{B} = \o$. $X_{\mathcal{A}}$ and $X_{\mathcal{B}}$ are independent ($X_{\mathcal{A}} \independent X_{\mathcal{B}}$) if and only if: 
        \[\Omega_{\mathcal{A}\times\mathcal{B}} = 0\]

        In other words, $X_1, \ldots, X_n$ are mutually independent if and only if $\Omega$ is diagonal.
        \item If moreover $X_1, \ldots, X_n \iid \mathcal{N}\left(\mu, \sigma^2\right)$, then: 
        \[X \followsdistr \mathcal{N}_n\left(\mu 1_n, \sigma^2 I_n\right)\]
        where $1_n \in \mathbb{R}^n$ is a vector with only ones, and $I_n \in \mathbb{R}^{n \times n}$ is the identity matrix.
        \item Any linear combination of our variables is also normal. In other words, letting $a \in \mathbb{R}^{r}$ and $B \in \mathbb{R}^{r \times p}$, we get:
        \[a + B X \followsdistr \mathcal{N}_r\left(a + B \mu, B \Omega B^T\right)\]
    \end{enumerate}
    
    \subparag{Remark}{
        The second property is very important. For instance, if $\left(X_1, X_2\right)$ is a multivariate Gaussian, we simply have: 
        \[\Cov\left(X_1, X_2\right) = 0 \iff X_1 \independent X_2\]
    }

    \subparag{Proof 1}{
        We know that $u^T X \followsdistr \mathcal{N}\left(u^T \mu, u^T \Omega u\right)$ by hypothesis. We know the MGFs of 1D normal distributions, giving us:
        \[M_{u^T X}\left(t\right) = \exval\left(e^{t u^T X}\right) = \exp\left(t u^T \mu + \frac{1}{2} t^2 u^T \Omega u\right)\]
        
        However, we notice that: 
        \[M_X\left(u\right) = \exval \left(e^{u^T X}\right) = \exval\left(e^{1\cdot  u^T X}\right) = M_{u^T X}\left(1\right)\]

        This thus gives us: 
        \[M_X\left(u\right) = \exp\left(u^T \mu + \frac{1}{2} u^T \Omega u\right), \mathspace \forall u \in \mathbb{R}^p\]
        as required.
    }

    \subparag{Proof 2}{
        We only prove the 2D case, but the complete proof is very similar.

        Let us consider the moment generating functions. By the first property, we have:
        \autoeq{M_X\left(t\right) = \exp\left(t^T \mu + \frac{1}{2} t^T \Sigma t \right) = \exp\left(t_1 \mu_1 + t_2 \mu_2 + \frac{1}{2}t_1^2 \sigma_1^2 + \frac{1}{2}t_2^2 \sigma_2^2 + t_1 t_2 \Sigma_{1, 2}\right)}

        Then, because both our variables $X_1 = \begin{pmatrix} 1 & 0 \end{pmatrix}^T X$ and $X_2 = \begin{pmatrix} 0 & 1 \end{pmatrix}^T X $ are 1D Gaussian, we know that: 
        \[M_{X_1}\left(t\right) = \exp\left(t_1 \mu_1 + \frac{1}{2}t_1^2 \sigma_1^2\right), \mathspace M_{X_2}\left(t\right) = \exp\left(t_2 \mu_2 + \frac{1}{2}t_2^2 \sigma_2^2\right)\]
        
        Now, we know that two polynomials are equal for all $t \in \mathbb{R}$ if and only if their coefficients are equal. Thus, we get that: 
        \[M_{X}\left(t\right) = M_{X_1}\left(t\right) M_{X_2}\left(t\right) \iff \Sigma_{1,2} = 0\]
        
        However, we know that $M_{X}\left(t\right) = M_{X_1}\left(t\right) M_{X_2}\left(t\right)$ if and only if $X_1$ and $X_2$ are independent, finishing our proof.
    }
    
    \subparag{Proof 3}{
        Let $X_1, \ldots, X_n \iid \mathcal{N}\left(\mu, \sigma^2\right)$. We notice that $X \followsdistr \mathcal{N}_n\left(\mu 1_n, \sigma^2 I_n\right)$ indeed works, since they all have a mean $\mu$, $\Var\left(X_i\right) = \sigma^2$ and $\Cov\left(X_i, X_j\right) = 0$ for $i \neq j$.
    }

    \subparag{Proof 4}{
        Let us consider the MGF of $a + BX$: 
        \autoeq{\exval\left[\exp\left(t^T \left(a + BX\right)\right)\right] = \exval\left[\exp\left(t^T a + \left(B^T t\right)^T X\right)\right] = e^{t^T a} \exval\left[\exp\left(\left(B^T t\right)^T X\right)\right] = e^{t^T a} M_X\left(B^T t\right) = \exp\left(t^T a + \left(B^T t\right)^T \mu + \frac{1}{2} \left(B^T t\right)^T \Omega \left(B^T t\right)\right) = \exp\left(t^T \left(a + B \mu\right) + \frac{1}{2} t^T \left(B \Omega B^T\right)t\right)}

        However, we recognise that this is the MGF of the $\mathcal{N}_r\left(a + B \mu, B \Omega B^T\right)$ distribution. Because of the injectivity of MGFs, we indeed get that: 
        \[a + BX \followsdistr \mathcal{N}_r\left(a + B \mu, B \Omega B^T\right)\]

        \qed
    }
}

\parag{Remark}{
   We notice that any independent Gaussian variables are jointly Gaussian. 

   However, if our variables are all Gaussian but are dependent, it does not necessarily mean that they are jointly Gaussian.
}


\parag{Example 1}{
    Let $X \followsdistr \mathcal{N}_p\left(0, \Omega\right)$. We know that $\Omega$ is positive semi-definite, and thus that it is orthonornormally diagonalisable: 
    \[\Omega = U D U^T\]
    
    If $X$ is non-degenerate (meaning that $\Omega$ is also positive definite and thus that its eigenvalues are non-zero), then: 
    \[D^{-\frac{1}{2}} U^T X \followsdistr \mathcal{N}_p\left(0, I_p\right)\]
    
    \subparag{Remark}{
        Since $D$ is diagonal, we can compute its inverse square root by computing the inverse square root of all of its elements. We notice that it indeed has the property: 
        \[\left(D^{-\frac{1}{2}}\right)^2 = \begin{pmatrix} \frac{1}{\sqrt{\lambda_1}} & \cdots & 0 \\ \vdots & \ddots & \vdots \\ 0 & \cdots & \frac{1}{\sqrt{\lambda_n}} \end{pmatrix}^2 = \begin{pmatrix} \frac{1}{\lambda_1} & \cdots & 0 \\ \vdots & \ddots & \vdots \\ 0 & \cdots & \frac{1}{\lambda_n} \end{pmatrix} = D^{-1}\]
    }

    \subparag{Intuition}{
        Let's try to make a parallel with the 1D case, where if $X \followsdistr \mathcal{N}\left(0, \sigma^2\right)$, then $\frac{1}{\sigma} X \followsdistr \mathcal{N}\left(0, 1\right)$. 

        We thus want to make sense of $\left(\sqrt{\Omega}\right)^{-1}$, since it plays the role of $\frac{1}{\sigma}$. To do so, we notice that: 
        \[\Omega = U D U^T = U D^{\frac{1}{2}} D^{\frac{1}{2}} U^T = \left(U D^{\frac{1}{2}}\right) \left(U D^{\frac{1}{2}}\right)^T\]
        where we used the fact that $D^{\frac{1}{2}}$ is diagonal.

        Thus, it makes a lot of sense to define $\sqrt{\Omega} = U D^{\frac{1}{2}}$, since it gives: 
        \[\Omega = \sqrt{\Omega} \sqrt{\Omega}^T\]
        
        Now, we can finally see that: 
        \[\left(\sqrt{\Omega}\right)^{-1} = \left(U D^{\frac{1}{2}}\right)^{-1} = \left(D^{\frac{1}{2}}\right)^{-1} U^{-1} = D^{-\frac{1}{2}} U^T\]
        since $U$ is orthogonal.

        Finally, we indeed get that it would make a lot of sense that we have $\left(\sqrt{\Omega}\right)^{-1} X \followsdistr \mathcal{N}_p\left(0, I_p\right)$
    }
    

    \subparag{Proof}{
        Let's make a constructive proof. We want to find an $A$ such that $A X \followsdistr \mathcal{N}_p\left(0, I_p\right)$. By our theorem, we know that: 
        \[AX \followsdistr  \mathcal{N}_r\left(0, A \Omega A^T\right)\]
        
        We thus want to solve $A \Omega A^T = I_p$. Since $\Omega$ is positive semi-definite, we can orthonormally diagonalise it:
        \[\Omega = U D U^T = U D^{\frac{1}{2}} D^{\frac{1}{2}} U^T\]
        where $D$ is diagonal, $D_{ii} > 0$ (thanks to the positive definitness of $\Omega$) and $U^T U = I_n$ (thanks to the orthonormality of $U$).

        Now, we can pick $A = D^{-\frac{1}{2}} U^T$ which indeed gives:
        \[A \Omega A^T = D^{-\frac{1}{2}} \underbrace{U^T U}_{I} D^{\frac{1}{2}} D^{\frac{1}{2}} \underbrace{U^T U}_{I} D^{-\frac{1}{2}} = D^{-\frac{1}{2}} D^{\frac{1}{2}} D^{\frac{1}{2}} D^{-\frac{1}{2}} = I\]
        where we used $\left(D^{-\frac{1}{2}}\right)^{T} = D^{-\frac{1}{2}}$ since it is diagonal.

    }
}

\end{document}
