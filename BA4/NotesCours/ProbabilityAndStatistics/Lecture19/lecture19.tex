% !TeX program = lualatex
% Using VimTeX, you need to reload the plugin (\lx) after having saved the document in order to use LuaLaTeX (thanks to the line above)

\documentclass[a4paper]{article}

% Expanded on 2023-05-04 at 13:19:38.

\usepackage{../../style}

\title{Probability and statistics}
\author{Joachim Favre}
\date{Jeudi 04 mai 2023}

\begin{document}
\maketitle

\lecture{19}{2023-05-04}{Slowly starting to do statistics}{
\begin{itemize}[left=0pt]
    \item Definition of convolutions.
    \item Proof of the theorem making a link between the sum of random variables and the convolution of their PMFs or PDFs.
    \item Definition of order statistics, minimum and maximum.
    \item Proof of a theorem giving the CDF of the minimum and maximum of IID random variables.
\end{itemize}

}

\parag{Example 1}{
    Let $X_1, X_2 \iid \mathcal{N}\left(0, 1\right)$. We want to compute the joint distribution $Y = \left(X_1 + X_2, X_1 - X_2\right)$.

    We want to use transformation theorem. We notice that: 
    \[\left|J\left(x_1, x_2\right)\right| = \left|\det B\right| = \left|-2\right| = 2\]

    Now, we also need to be able to write our $x$'s as functions of $y$'s: 
    \[\begin{pmatrix} X_1 \\ X_2 \end{pmatrix} = B^{-1} \begin{pmatrix} Y_1 \\ Y_2 \end{pmatrix} = -\frac{1}{2} \begin{pmatrix} -1 & -1 \\ -1 & 1 \end{pmatrix} \begin{pmatrix} Y_1 \\ Y_2 \end{pmatrix} \]
    
    This tells us that: 
    \autoeq{f_{Y_1, Y_2}\left(y_1, y_2\right) = f_{X_1, X_2}\left(\frac{y_1 + y_2}{2}, \frac{y_1 - y_2}{2}\right) \frac{1}{\left|J\left(\frac{y_1 + y_2}{2}, \frac{y_1 - y_2}{2}\right)\right|} = \frac{1}{2} \cdot  \frac{1}{2\pi} \exp\left(-\frac{1}{4} \left(y_1^2 + y_2^2\right)\right)}
    
    We thus see that $Y_1, Y_2 \iid \mathcal{N}\left(0, 2\right)$. Naturally, we could  also have used the properties of jointly Gaussian distributions, since IID Gaussian variables are jointly Gaussian.
}

\parag{Example 2}{
    Let us consider $X_1, X_2 \iid \exp\left(\lambda\right)$. We want to compute the joint density of: 
    \[Y = \begin{pmatrix} Y_1 \\ Y_2 \end{pmatrix} = \begin{pmatrix} X_1 + X_2 \\ \frac{X_1}{X_1 + X_2} \end{pmatrix} = g\left(X\right)\]
    
    Since they are IID, we have that: 
    \[f\left(x_1, x_2\right) = \lambda^2 \exp\left(-\lambda\left(x_1 + x_2\right)\right)I\left(x_1 \geq 0\right)I\left(x_2 \geq 0\right)\]
    
    Now, we need to compute the inverse map of $g$, $h$. This is not affine, so there is no general formula. From the problem, we notice that: 
    \[y_1 y_2 = x_1\]
    
    Now, we can use this to get: 
    \[y_1 = x_1 + x_2 = y_1 y_2 + x_2 \iff x_2 = y_1\left(1 - y_2\right)\]
    
    This finally gives us that: 
    \[x = h\left(y_1, y_2\right) = \begin{pmatrix} y_1y_2 \\ y_1\left(y_1 - y_2\right) \end{pmatrix} \]

    After that, we need to compute the Jacobian of $g$:
    \[\left|J_g\left(x_1, x_2\right)\right| = \left|\det \begin{pmatrix} 1 & 1 \\ \frac{x_2}{\left(x_1 + x_2\right)^2} & -\frac{x_1}{\left(x_1 + x_2\right)^2} \end{pmatrix}\right| = \left|-\frac{x_1 + x_2}{\left(x_1 + x_2\right)^2}\right| = \frac{1}{x_1 + x_2} = \frac{1}{y_1}\] 
    
    Another way to see this is that: 
    \[\left|J_g\left(x_1, x_2\right)\right| = \left|J_h\left(y_1, y_2\right)\right|^{-1} = \left|\det\begin{pmatrix} y_2 & y_1 \\ 1 - y_2 & -y_1 \end{pmatrix} \right|^{-1} = \left|y_1\right|^{-1} = \frac{1}{y_1}\]
    
    By using our transformation theorem, this yields that: 
    \autoeq{f_{Y_1, Y_2}\left(y_1, y_2\right) = \left|\frac{1}{J_g\left(x_1, x_2\right)}\right| f_{X_1, X_2}\left(x_1, x_2\right) = y_1 \lambda^2 \exp\left(-\lambda \left(x_1 + x_2\right)\right) I\left(x_1 > 0\right)I\left(x_2 > 0\right) = y_1 \lambda^2 \exp\left(-\lambda y_1\right)I\left(y_1 y_2 > 0\right) I \left(y_1 \left(1 - y_2\right) > 0\right) = \underbrace{y_1 \lambda^2 \exp\left(-\lambda y_1\right)I\left(y_1 > 0\right)}_{f_{Y_1}\left(y_1\right)}\underbrace{I\left(0 < y_2 < 1\right)}_{f_{Y_2}\left(y_2\right)}}
    
    We notice that this is a product of two densities meaning that they are independent. This tells us that $Y_1 \followsdistr \text{Gamma}\left(2, \lambda\right)$ and $Y_2 \followsdistr U\left(0, 1\right)$ independently.
}

\parag{Definition: Discrete convolution}{
    Let $X, Y$ be discrete random variables with PMFs $f_X, f_Y$. The convolution of their PMFs is: 
    \[f_X * f_Y\left(s\right) = \sum_{x}^{} f_X\left(x\right)f_Y\left(s- x\right)\]
    
}

\parag{Definition: Continuous convolution}{
    Let $X, Y$ be continuous random variable with PDFs $f_X, f_Y$. The convolution of their PDFs is: 
    \[f_X * f_Y\left(s\right) = \int_{-\infty}^{\infty} f_X\left(x\right)f_Y\left(s- x\right)dx\]
}

\parag{Remark}{
    The convolution $f_{X_1} * f_{X_2}$ produces a new function, which we can evaluate: $\left(f_{X_1}* f_{X_2}\right)\left(x\right)$. However, in this course, we simplify the notation by writing: 
    \[\left(f_{X_1}* f_{X_2}\right)\left(x\right) = f_{X_1}* f_{X_2}\left(x\right)\]
}


\parag{Theorem}{
    Let $X, Y$ be independent random variables with PMF or PDF $f_X, f_Y$.

    The PMF of PDF of their sum $S = X + Y$ is: 
    \[f_S\left(s\right) = f_X * f_Y\left(s\right)\]
    
    \subparag{Proof}{
        The idea is to make the following change of variable: 
        \[\begin{pmatrix} W \\ S \end{pmatrix} = \begin{pmatrix} X \\ X + Y \end{pmatrix} = \begin{pmatrix} 1 & 0 \\ 1 & 1 \end{pmatrix} \begin{pmatrix} X \\ Y \end{pmatrix} \implies \begin{pmatrix} X \\ Y \end{pmatrix} = \begin{pmatrix} W \\ S - W \end{pmatrix} \]
        
        The Jacobian is: 
        \[\left|J\left(x, y\right)\right| = \left|\det \begin{pmatrix} 1 & 0 \\ 1 & 1 \end{pmatrix} \right| = 1\]
        
        Thus, by applying our transformation theorem: 
        \[f_{W, S}\left(w, s\right) = f_{X, Y}\left(x, y\right) \frac{1}{\left|J\left(x, y\right)\right|} = f_{X}\left(x\right)f_Y\left(y\right) = f_X\left(w\right)f_Y\left(s-w\right)\]
        since $X$ and $Y$ are independent.

        We can now marginalise this distribution to get the one of $S$. In the continuous case, we get: 
        \[f_S\left(s\right) = \int_{-\infty}^{\infty} f_X\left(w\right)f_Y\left(s-w\right)dw\]
        as required.

        \qed
    }

    \subparag{Other proof}{
        We want to make another proof, which will be more intuitive and natural.

        Let's start with the CDF, since it is easier to reason about them. We notice that, using the theorem of total probability: 
        \autoeq{F_{X + Y}\left(y\right) = \prob\left(X + Y \leq y\right) = \int_{-\infty}^{\infty} \prob\left(X + Y \leq y | X = z\right)f_{X}\left(z\right)dz = \int_{-\infty}^{\infty} \prob\left(Y \leq y - z | X = z\right) f_{X}\left(z\right) dz}
        
        Now, we now that $X$ and $Y$ are independent, so $\prob\left(Y \leq y-z | X = z\right) = \prob\left(Y \leq y- z\right)$, giving us that: 
        \autoeq{F_{X+ Y}\left(y\right) = \int_{-\infty}^{\infty} \prob\left(Y \leq y - z\right) f_X\left(z\right)dz = \int_{-\infty}^{\infty} F_{Y}\left(y-z\right) f_{X}\left(z\right)dz}
        
        After that, we can differentiate to get the PDF. In this course, we are always working with sufficiently well-behaved functions, so we can bring the derivative inside of the integral: 
        \autoeq{f_{X + Y}\left(y\right) = \int_{-\infty}^{\infty} \frac{d}{dy} F_{Y}\left(y - z\right) f_X\left(z\right)dz = \int_{-\infty}^{\infty} f_Y\left(y-z\right)f_X\left(z\right)dz}
        
        \qed
    }
    

    \subparag{Personal remark}{
        There is a great 3Blue1Brown video where they explain the intuition behind this result:
        \begin{center}
            \url{https://www.youtube.com/watch?v=KuXjwB4LzSA}
        \end{center}
    }
}

\parag{Theorem}{
    Let $X_1, \ldots, X_n$ be independent random variables with PDF $f_{X_1}, \ldots, f_{X_n}$.

    The PDF of their sum $S = X_1 + \ldots + X_n$ is: 
    \[f_S\left(s\right) = f_{X_1} * \ldots * f_{X_n}\left(s\right)\]
    
    \subparag{Remark}{
        This theorem is nice, but it is often not usable in practice: convolutions can be really heavy to compute. 

        Often, if $n > 2$, it is easier to use MGFs to turn sum of independent random variables into products, and handle such expressions without using any convolution.
    }
}


\subsection{Order statistics}

\parag{Definition: Order statistics}{
    Let $X_1, \ldots, X_n$ be random variables.

    The \important{order statistics} of those random variables are the ordered values: 
    \[X_{\left(1\right)} \leq \ldots \leq X_{\left(n\right)}\]
    
    If moreover they are continuous, then we cannot have any equality, so: 
    \[X_{\left(1\right)} < \ldots < X_{\left(n\right)}\]
}

\parag{Definition: Minimum}{
     Let $X_1, \ldots, X_n$ be random variables. The \important{minimum} is $X_{\left(1\right)}$.
}

\parag{Definition: Maximum}{
     Let $X_1, \ldots, X_n$ be random variables. The \important{maximum} is $X_{\left(n\right)}$.
}


\parag{Definition: Median}{
     Let $X_1, \ldots, X_n$ be random variables. The \important{median}, the central value, is 
     \[\begin{systemofequations} X_{\left(m+1\right)}, & n = 2m + 1 \\ \dfrac{X_{\left(m\right)} + X_{\left(m+1\right)}}{2}, & n = 2m \end{systemofequations}\]

     \subparag{Remark}{
         This definition is not important.
     }
}

\parag{Theorem}{
    Let $X_1, \ldots, X_n \iid F$ be continuous random variables with PDF $f$ and CDF $F$. Then:
    \begin{enumerate}
        \item $\prob\left(X_{\left(1\right)} \leq x\right) = 1 - \left(1 - F\left(x\right)\right)^n$
        \item $\prob\left(X_{\left(n\right)} \leq x\right) = F\left(x\right)^n$
        \item $f_{X_{\left(r\right)}}\left(x\right) = \frac{n!}{\left(r-1\right)!\left(n-r\right)!} F\left(x\right)^{r-1} f\left(x\right)\left(1 - F\left(x\right)\right)^{n-r}$
    \end{enumerate}
    
    \subparag{Remark}{
        The last property does not have to be known.
    }
    
    \subparag{Proof 1}{
        We notice that: 
        \autoeq{\prob\left(X_{\left(1\right)} \leq y\right) = \prob\left(\min\left(X_1, \ldots, X_n\right) \leq y\right) = 1 - \prob\left(\min\left(X_1, \ldots, X_n\right) \geq y\right) = 1 - \prob\left(X_1 \geq y \cap \ldots \cap X_n \geq y\right) = 1 - \prob\left(X_1 > y\right) \cdots \prob\left(X_n > y\right) = 1 - \prob\left(X > n\right)^n = 1 - \left(1 - \prob\left(X < n\right)\right)^n = 1 - \left(1 - F\left(x\right)\right)^n}
        
    }

    \subparag{Proof 2}{
        This proof is very similar to the one we just did: 
        \[\prob\left(X_{\left(n\right)} \leq x\right) = \prob\left(\max\left(X_1, \ldots, X_n\right) \leq x\right) = \prob\left(X \leq x\right)^n = F\left(x\right)^n\]

        \qed
    }
}

\parag{Example}{
    Two people arrive late uniformly at random from 0 to 1 hour after the appointed time. We want to compute the expected time at which they will have both arrived.

    We have $X_1, X_2 \iid U\left(0, 1\right)$, and we want to compute the PDF of $V = X_{\left(2\right)}$. By our theorem, we know that: 
    \[F_{V}\left(v\right) = v^2\]
    
    Thus, we get that: 
    \[f_V\left(v\right) = \frac{d\left(v^2\right)}{dv} = 2v\]
    
    We can now compute the expected value: 
    \[\exval\left(V\right) = \int_{0}^{1} 2v \cdot v dv = \frac{2}{3}\]

    This means that, on average, they will both have arrived 40 minutes after the appointed time.

    We could do a similar analysis to get that $\exval\left(X_{\left(1\right)}\right) = \exval\left(U\right) = \frac{1}{3}$. Now, if we wanted to know the average waiting time of the first arrived $W = V - U$, we can use the linearity of the expectations to get: 
    \[\exval\left(W\right) = \exval\left(V - U\right) = \exval\left(V\right) - \exval\left(U\right) = \frac{1}{3}\]
    
    Note that $U$ and $V$ are definitely not independent, but we do not need this property for the expectations to be linear.
}

\end{document}
