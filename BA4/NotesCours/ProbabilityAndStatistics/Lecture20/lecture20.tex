% !TeX program = lualatex
% Using VimTeX, you need to reload the plugin (\lx) after having saved the document in order to use LuaLaTeX (thanks to the line above)

\documentclass[a4paper]{article}

% Expanded on 2023-05-09 at 13:19:00.

\usepackage{../../style}

\title{ProbaStats}
\author{Joachim Favre}
\date{Mardi 09 mai 2023}

\begin{document}
\maketitle

\lecture{20}{2023-05-09}{Approximations}{
\begin{itemize}[left=0pt]
    \item Explanation and proof of Markov's inequality and some of its corollaries. 
    \item Explanation of Jensen's inequality.
    \item Explanation of Hoeffding's inequality.
    \item Definition of almost sure, mean square, probability and distribution convergence, and explanation of which implies which.
\end{itemize}

}

\section{Approximations}

\subsection{Inequalities}

\parag{Theorem: Markov inequality}{
    Let $X$ be a random variable such that $X \geq 0$, and let $a > 0$.

    Then: 
    \[\prob\left(X \geq a\right) \leq \frac{\exval\left(X\right)}{a}\]
    
    \subparag{Proof}{
        Since $X \geq 0$, we can see the following inequality, using an indicator variable: 
        \[X \geq X I\left(X \geq a\right) \geq a I\left(X \geq a\right)\]
        
        Taking an expectation on both sides, we get that: 
        \[\exval\left(X\right) \geq a \exval\left(I\left(X \geq a\right)\right) = a \prob\left(X \geq a\right)\]
        
        This indeed allows us to conclude that: 
        \[\prob\left(X \geq a\right) \leq \frac{\exval\left(X\right)}{a}\]

        \qed
    }
}

\parag{Corollaries}{
    Let $X$ be an arbitrary random variables. We have the following corollaries.

    \begin{enumerate}
        \item Let $g$ be a function non-negative everywhere. Then:
        \[\prob\left(g\left(X\right) \geq a\right) \leq \frac{\exval\left(g\left(X\right)\right)}{a}\]
        \item We have that: 
        \[\prob\left(\left|X\right| \geq a\right) \leq \frac{\exval\left(\left|X\right|\right)}{a}\]
        \item (Chebyshov's inequality) We have that:
        \[\prob\left(X \geq a\right) \leq \frac{\exval\left(X^2\right)}{a^2}\]
        \item Let $g$ be an increasing function non-negative everywhere. Then: 
        \[\prob\left(X \geq a\right) \leq \frac{\exval\left(g\left(x\right)\right)}{g\left(a\right)}\]
        \item We have that: 
        \[\prob\left(\left|X - \exval\left(X\right)\right| \geq a\right) \leq \frac{\Var\left(X\right)}{a^2}\]
        
            
    \end{enumerate}
    
    \subparag{Proof 3}{
        We know that: 
        \[\prob\left(X \geq a\right) \leq \prob\left(\left|X\right| \geq a\right) = \prob\left(\left|X\right|^2 \geq a^2\right)\]
        
        However, we can now apply Markov's inequality to get: 
        \[\prob\left(X \geq a\right) \leq \prob\left(X^2 \geq a^2\right) \leq \frac{\exval\left(X^2\right)}{a^2}\]
    }

    \subparag{Proof 4}{
        We know that: 
        \[\prob\left(X \geq a\right) = \prob\left(g\left(X\right) \geq g\left(a\right)\right) \leq \frac{\exval\left(g\left(X\right)\right)}{g\left(a\right)}\]
    }
    
    \subparag{Proof 5}{
        We can use the fact that $\prob\left(\left|Y\right| \geq a\right) \leq \frac{\exval\left(Y^2\right)}{a^2}$ by letting $Y = X - \exval\left(X\right)$: 
        \[\prob\left(\left|X - \exval\left(X\right)\right| \geq a\right) \leq \frac{\exval\left(\left(X - \exval\left(X\right)^2\right)\right)}{a^2} = \frac{\Var\left(X\right)}{a^2}\]

        \qed
    }
    
}

\parag{Theorem: Jensen's inequality}{
    Let $g$ be a convex function, and $X$ be a random variable. Then: 
    \[g\left(\exval\left(X\right)\right) \leq \exval\left(g\left(X\right)\right)\]
    
    \subparag{Intuition}{
        We consider the simpler case where $X$ is a binary random variable: it outputs $a$ with probability $p$ and $b$ with probability $1-p$.

        We know that $g\left(\exval\left(X\right)\right) = g\left(pa + \left(1-p\right)b\right)$ and $\exval\left(g\left(X\right)\right) = pg\left(a\right) + \left(1-p\right)g\left(b\right)$. This can be visualised as:
        \svghere[0.9]{JensensInequality.svg}

        However, the fact that $g$ is convex means that any straight line which both endpoints are on the function, is above the function. This indeed means that: 
        \[\exval\left(g\left(X\right)\right) \geq g\left(\exval\left(X\right)\right)\]
        
        It is possible to use this argument to make a proof for any discrete random variable. For continuous random variables, we must use other arguments; this one is really only for the intuition.
    }

    \subparag{Remark}{
        This result is important.
    }
}

\parag{Exemple}{
    Let $X_1, \ldots, X_n \iid \text{Bern}\left(p\right)$, and: 
    \[\bar{X} = \frac{1}{n} \sum_{i=1}^{n} X_i\]
    
    We want to bound $\prob\left(\left|\bar{X} - p\right| \geq \epsilon\right)$ for any $\epsilon > 0$.

    The absolute value is not easy to manipulate, so let us begin with squaring both sides: 
    \[\prob\left(\left|\bar{X} - p\right| \geq \epsilon\right) = \prob\left(\left(X - p\right)^2 \geq \epsilon^2\right) \leq \frac{\exval\left(\left(\bar{X} - p\right)^2\right)}{\epsilon^2}\]
    
    However, we notice that the numerator is the variance of $\bar{X}$. Indeed, we have that: 
    \[\exval\left(\bar{X}\right) = \exval\left(\frac{1}{n}\sum_{i=1}^{n} X_i\right) = \frac{1}{n}\exval\left(\sum_{i=1}^{n} X_i\right) = \frac{1}{n} \cdot  n \exval\left(X_i\right) = p\]

    Thus, the numerator is indeed of the form $\exval\left(\left(\bar{X} - \exval\left(X\right)\right)^2\right) = \Var\left(\bar{X}\right)$. Let us compute this: 
    \[\Var\left(\bar{X}\right) = \Var\left(\frac{1}{n} \sum_{i=1}^{n} X_i\right) = \frac{1}{n^2} \Var\left(\sum_{i=1}^{n} X_i\right) = \frac{1}{n^2} \cdot n \Var\left(X_i\right) = \frac{p\left(1-p\right)}{n}\]
    since $X_1, \ldots, X_n$ are independent.
    
    Putting everything together, we get that: 
    \[\prob\left(\left|\bar{X} - p\right| > \epsilon\right) \leq \frac{\Var\left(\bar{X}\right)}{\epsilon^2} = \frac{p\left(1-p\right)}{n \epsilon^2}\]
    
    This is very interesting since the right hand side tends towards 0 as $n \to \infty$. This means that, the average of Bernoulli random variables tends towards their mean $p = \exval\left(X_i\right)$.
}

\parag{Theorem: Hoeffding's inequality}{
    Let $Z_1, \ldots, Z_n$ be independent random variables such that $\exval\left(Z_i\right) = 0$ and such that there exists constants $a_i < b_i$ for which $a_i \leq Z_i \leq b_i$ for all $i$. Also, let $\epsilon > 0$.

    Then, for all $t > 0$: 
    \[\prob\left(\sum_{i=1}^{n} Z_i \geq \epsilon\right) \leq e^{-t \epsilon} \prod_{i=1}^{n} \exp\left(\frac{t^2 \left(b_i - a_i\right)^2}{8}\right)\]
    
    \subparag{Remark}{
        Since this is true for any $t$, we can look for the $t$ which minimises the expression on the right hand side (which only requires to minimise a polynomial since the exponential is an increasing function). This usually yields an inequality which is much better than the theorems we saw before.
    }

    \subparag{Intuition}{
        Let $X \geq 0$. We know that: 
        \[\prob\left(X \geq a\right) \leq \frac{\exval\left(X\right)}{a}\]
        
        However, since $e^x$ is an increasing function which is non-negative everywhere, we get that, for an arbitrary $t > 0$: 
        \[\prob\left(X \geq a\right) = \prob\left(e^{tX} > e^{ta}\right) \leq \frac{\exval\left(e^{tX}\right)}{e^{ta}}\]
        
        This inequality is known as Chernhoff's bound. Hoeffding's inequality's idea is similar, except that it uses the fact that the random variables are bounded instead of non-negative.
    }
}

\parag{Example}{
    Let $X_1, \ldots, X_n \iid \text{Bern}\left(p\right)$, and: 
    \[\bar{X} = \frac{1}{n} \sum_{i=1}^{n} X_i\]
    
    We again want to bound $\prob\left(\left|\bar{X} - p\right| \geq \epsilon\right)$ for any $\epsilon > 0$, but using a better inequality.

    Let $Z_i = \frac{X_i - p}{n}$. We notice that we indeed have the main hypothesis for the Hoeffding's inequality:
    \[\exval\left(Z_i\right) = \frac{\exval\left(X_i\right) - p}{n} = \frac{p-p}{n} = 0\]
    
    We now see that our expression is equivalent to: 
    \[\prob\left(\left|\bar{X} - p\right| > \epsilon\right) = \prob\left(\bar{X} - p > \epsilon\right) + \prob\left(-\left(\bar{X} - p\right) > \epsilon\right) = \prob\left(\sum_{i=1}^{n} Z_i > \epsilon\right) + \prob\left(- \sum_{i}^{n} Z_i > \epsilon\right)\]

    We know that $0 \leq X_i \leq 1$ and thus $\frac{-p}{n} \leq Z_i \leq \frac{1-p}{n}$. We can thus let $a_i = \frac{-p}{n}$ and $b_i = \frac{1-p}{n}$, giving us that: 
    \[\left(b_i - a_i\right)^2 = \left(\frac{1}{n}\right)^2 = \frac{1}{n^2}\]
    
    We can then plug everything in our formula, noticing it gives the same result for $\sum_{i=1}^{n} Z_i$ and $\sum_{i=1}^{n} \left(-Z_i\right)$: 
    \autoeq{\prob\left(\left|\bar{X} - p\right| > \epsilon\right) = \prob\left(\sum_{i=1}^{n} Z_i > \epsilon\right) + \prob\left(-\sum_{i=1}^{n} Z_i > \epsilon\right) \leq 2 e^{t \epsilon} \exp\left(t^2 \frac{1}{n^2} \frac{1}{8}\right)^n = \exp\left(\frac{t^2}{8n} - t \epsilon\right)}

    We can then finally optimise this by minimising the parabola $\frac{t^2}{8n} - t \epsilon$, which is minimal at $t = 4 \epsilon n$. This gives us:
    \[\prob\left(\left|\bar{X} - p\right| > \epsilon\right) \leq 2e^{-2n \epsilon^2}\]

    This inequality converges much faster to 0: this is an exponential decay whereas Markov's inequality gave something decaying in $\Theta\left(\frac{1}{n}\right)$.
}

\subsection{Convergence}

\parag{Example}{
    Let us consider the sequence of random variables defined as: 
    \[\prob\left(X_n = 0\right) = 1 - \frac{1}{n}, \mathspace \prob\left(X_n = n^2\right) = \frac{1}{n}\]
    
    We notice that: 
    \[\lim_{n \to \infty} \prob\left(X_n = 0\right) = 1\]
    
    However, we also get that: 
    \[\lim_{n \to \infty} \exval\left(X_n\right) = \lim_{n \to \infty} \left(0\left(1 - \frac{1}{n}\right) + n^2 \frac{1}{n}\right) = +\infty\]
    
    We thus notice that $X_n$ converges in distribution to $X = 0$ constant, even though the moments do not converge to the ones of $X = 0$.
}

\parag{Definition: Convergence}{
    Let $X, X_1, X_2, \ldots$ be random variables with cumulative distribution function function $F, F_1, \ldots$.

    Then, we define:
    \begin{enumerate}
        \item $X_n$ converges to $X$ \important{almost surely}, written $X_n \over{\to}{a.s.} X$, if: 
        \[\prob\left(\lim_{n \to \infty} X_n = X\right) = 1\]
        \item $X_n$ converges to $X$ \important{in mean square}, written $X_n \over{\to}{2} X$, if $\exval\left(X^2_n\right) < \infty$ for all $n$, $\exval\left(X^2\right) < \infty$ and: 
        \[\lim_{n \to \infty} \exval\left(\left(X_n - X\right)^2\right) = 0\]
        
        \item $X_n$ converges to $X$ \important{in probability}, written $X_n \over{\to}{P} X$, if, for all $\epsilon > 0$: 
        \[\lim_{n \to \infty} \prob\left(\left|X_n - X\right| > \epsilon\right) = 0\]
        
        \item $X_n$ converges to $X$ \important{in distribution}, writen $X_n \over{\to}{D} X$, if: 
        \[\lim_{n \to \infty} F_n\left(x\right) = F\left(x\right)\]
        everywhere $F\left(x\right)$ is continuous.
    \end{enumerate}
    
    \subparag{Remark}{
        The first definition is not important and must not be remembered, but the third one is really important.
    }
}

\parag{Theorem}{
    Let $X, X_1, X_2 \ldots$ be random variables. The modes of convergence imply one another as:
    \svghere[0.5]{ConvergenceModeImplications.svg}

    Note that any other implication is false in general.

    \subparag{Proof}{
        We want to show the following proposition: 
        \[X_n \over{\to}{2} X \implies X_n \over{\to}{P} X\]

        Let $\epsilon > 0$. We want to show that: 
        \[\prob\left(\left|X_n - X\right| > \epsilon\right) = 0\]
        
        Markov's inequality yields: 
        \[\prob\left(\left|X_n - X\right| > \epsilon\right) = \prob\left(\left(X_n - X\right)^2 > \epsilon^2\right) \leq \frac{\exval\left(\left(X_n - X\right)^2\right)}{\epsilon^2}\]
        
        However, by the definition of mean square convergence, $\exval\left(\left(X_n - X\right)^2\right)$ tends towards 0 as $n \to \infty$. Thus, this indeed means that: 
        \[\lim_{n \to \infty} \prob\left(\left|X_n - X\right| > \epsilon\right) = 0\]
        
        \qed
    }
}

\end{document}
