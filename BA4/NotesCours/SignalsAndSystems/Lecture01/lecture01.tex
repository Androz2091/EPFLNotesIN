% !TeX program = lualatex
% Using VimTeX, you need to reload the plugin (\lx) after having saved the document in order to use LuaLaTeX (thanks to the line above)

\documentclass[a4paper]{article}

% Expanded on 2023-02-24 at 13:17:03.

\usepackage{../../style}

\title{Signals and systems}
\author{Joachim Favre}
\date{Vendredi 24 février 2023}

\begin{document}
\maketitle

\lecture{1}{2023-02-24}{Notations (more or less good)}{
\begin{itemize}[left=0pt]
    \item Definition of discrete time and continuous time signal and systems.
    \item Definition of periodic, even, odd, energy and power signals.
    \item Definition of linear, time invariant, memoryless, causal, invertible and stable systems.
\end{itemize}

}

\section{Introduction}
\subsection{Signals}
\parag{Definition: Signals}{
    A \important{discrete time} (DT) signal, written $x\left[n\right]$, is a function $x: \mathbb{Z} \mapsto \mathbb{R}$. A \important{continuous time} (CT) signal, written $x\left(t\right)$, is a function $x: \mathbb{R} \mapsto \mathbb{R}$ or $x: \mathbb{R} \mapsto \mathbb{C}$.

    \subparag{Remark}{
        We will always use square brackets for DT signals and parenthesis for CT signals.
    }
}

\parag{Definition: Periodic signal}{
    A discrete time signal is \important{periodic} if there exists $N \in \mathbb{N}$ such that, for all $n \in \mathbb{N}$: 
    \[x\left[n\right] = x\left[n + N\right]\]
    
    Similarly, a continuous time signal is periodic if there exists $T \in\mathbb{N}$ such that, for all $t \in \mathbb{R}$: 
    \[x\left(t\right) = x\left(t + T\right)\]
    
    $N$ and $T$ are named \important{periods}. The \important{fundamental period}, if it exists, is the smallest such $N$ or $T$.
}

\parag{Definition: Parity of signals}{
    A signal is \important{even} if: 
    \[x\left(t\right) = x\left(-t\right), \mathspace x\left[n\right] = x\left[-n\right]\]
    
    Similarly, a signal is \important{odd} if: 
    \[-x\left(t\right) = x\left(-t\right), \mathspace -x\left[n\right] = x\left[-n\right]\]
}

\parag{Definition: Energy of a signal}{
    The \important{energy} of a signal is defined as: 
    \[\mathcal{E} = \int_{-\infty}^{\infty} \left|x\left(t\right)\right|^2 dt, \mathspace \mathcal{E} = \sum_{n=-\infty}^{\infty} \left|x\left[n\right]\right|^2\]

    Some signals have an infinite energy. If it has a final energy, then we call it an \important{energy signal}.
}

\parag{Definition: Power of a signal}{
    The \important{power} of a signal is defined as: 
    \[\mathcal{P} = \lim_{T \to \infty} \frac{1}{2T} \int_{-T}^{T} \left|x\left(t\right)^2\right|dt, \mathspace \mathcal{P} = \lim_{N \to \infty} \frac{1}{2N + 1} \sum_{n=-N}^{N} \left|x\left[n\right]\right|^2\]

    Some signals have zero or infinite power. If a signal has a non-zero finite power, then we call it a \important{power signal}.

    \subparag{Remark}{
        If a signal is an energy signal, then it has a zero power, and thus it is not a power signal. Inversely, if a signal is a power signal, then it has an infinite energy and is thus not an energy signal.
    }
}

\parag{Example: Continuous}{
    Let's consider the following signal, which will be important in this course: 
    \[x\left(t\right) = C e^{a t}, \mathspace C, a \in \mathbb{C}\]
    
    We have multiple cases. If we have both $C, a \in \mathbb{R}$, then this is just a real exponential, which will either blow up over time or exponentially decay.

    If we have $a = \omega_0 j$ for $\omega_0 \in \mathbb{R}^*$ (it only has an imaginary component), then we get a periodic signal, with fundamental period $T = \frac{2\pi}{\omega_0}$:  
    \[e^{j \omega_0 \left(T + t\right)} = e^{j \omega_0 t + 2\pi j} = e^{j \omega_0 t}\]
    
    If we have $a = r + j \omega_0$, then we get a damped sinusoids (its real part is a sinusoidal function which amplitude follows an exponential).
}

\parag{Example: Discrete}{
    We can also consider the discrete exponential signal: 
    \[x\left[n\right] = C \alpha^n = C e^{\beta n}, \mathspace \alpha = e^{\beta}\]

    This is very similar to the continuous case, however we need to be careful about periodicity. For instance, $x\left[n\right] = e^{j \frac{3}{4} n}$ is not be periodic (there exists no integer period) but $y\left[n\right] = e^{j \frac{3}{4} \pi n}$ is periodic. 
}

\parag{Operations}{
    We can make different operations on our signals. We can make a \important{time shift}, for some $n_0 \in \mathbb{N}$ or $t_0 \in \mathbb{R}$:
    \[x\left[n - n_0\right], \mathspace x\left(t - t_0\right)\]

    We can also make a \important{time reversal}:
    \[x\left[-n\right], \mathspace x\left(-t\right)\] 

    We can finally make all sorts of scaling of the time variable, making a \important{time scaling}, for some $\alpha \in \mathbb{R}^*$:
    \[x\left(\alpha t\right)\]
}


\subsection{Systems}
\parag{Definition: System}{
   A \important{system}, $\mathcal{H}$, takes one or more input signals, and change them into one or more output signals. Formally, this is an operator.

   A \important{discrete time} system takes a discrete time input signal $x\left[n\right]$ and outputs a DT signal: 
   \[y\left[n\right] = \mathcal{H}\left\{x\left[n\right]\right\}\]
   
   Similarly, a \important{continuous time} system takes a CT signal $x\left(t\right)$ and outputs a CT signal: 
   \[y\left(t\right) = \mathcal{H}\left\{x\left(t\right)\right\}\]
   
   \subparag{Example}{
       For instance, the suspensions of a car can be seen as a system: it transforms the bumpiness of the road into something smoother. We can also see an audio compressor as a digital system.
   }

   \subparag{Personal remark}{
       The notation presented here is, in my opinion, not great. The best notation would instead to write: 
       \[y = \mathcal{H}\left\{x\right\} \implies y\left(t\right) = \mathcal{H}\left\{x\right\}\left(t\right)\]

       In other words, $\mathcal{H}$ is an operator: it takes a function $x$ and outputs another function $y$. Then, we evaluate both sides at $t$.
       
       Then, if we have a function $z\left(t\right) = x\left(t+2\right)$, we can make the following shortcut: 
       \[y\left(t\right) = \mathcal{H}\left\{z\right\}\left(t\right) = \mathcal{H}\left\{x\left(t+2\right)\right\}\left(t\right)\]
       
       In this course, we forget about the final parenthesis, writing: 
       \[\mathcal{H}\left\{x\left(t+2\right)\right\}\left(t\right) = \mathcal{H}\left\{x\left(t+2\right)\right\}\]
       
       However, this can be very dangerous. For instance, if we let $y\left(t\right) = \mathcal{H}\left\{x\right\}\left(t\right)$, then in general the following does not hold: 
       \[y\left(t+2\right) \neq \mathcal{H}\left\{x\left(t + 2\right)\right\}\left(t\right) = \mathcal{H}\left\{x\left(t+2\right)\right\}\]

       However, the following does hold in general: 
       \[y\left(t+2\right) = \mathcal{H}\left\{x\right\}\left(t+2\right) = \mathcal{H}\left\{x\left(t\right)\right\}\left(t+2\right)\]
       
       The difference is rather subtle. When we compute $\mathcal{H}\left\{x\left(t+2\right)\right\}$, we feed another function to the operator, thus possibly getting another function as result. When we compute $\mathcal{H}\left\{x\left(t\right)\right\}\left(t+2\right)$, we just evaluate the resulting function at another point. This difference is much clearer when keeping in mind that the evaluation of the function $y$ is at the end of the line, not inside the operator.
   }
}

\parag{Definition: Linear system}{
    A CT system is \important{linear} if, for all $a_1, a_2 \in \mathbb{R}$ and $x_1\left(t\right)$, $x_2\left(t\right)$: 
    \[\mathcal{H}\left\{a_1 x_1\left(t\right) + a_2 x_2\left(t\right)\right\} = a_1 \mathcal{H}\left\{x_1\left(t\right)\right\} + a_2 \mathcal{H}\left\{x_2\left(t\right)\right\}\]
    
    Similarly, a DT system is linear if, for all $a_1, a_2 \in \mathbb{R}$ and $x_1\left[n\right]$, $x_2\left[n\right]$: 
    \[\mathcal{H}\left\{a_1 x_1\left[n\right] + a_2 x_2\left[n\right]\right\} = a_1 \mathcal{H}\left\{x_1\left[n\right]\right\} + a_2 \mathcal{H}\left\{x_2\left[n\right]\right\}\]
}

\parag{Definition: Time invariant system}{
    A CT system is time invariant if, for all $\tau \in \mathbb{R}$ and $x\left(t\right)$: 
    \[y\left(t\right) = \mathcal{H}\left\{x\left(t\right)\right\} \implies y\left(t - \tau\right) = \mathcal{H}\left\{x\left(t - \tau\right)\right\}\]
    
    Similarly, a DT system is time invariant if, for all $n_0 \in \mathbb{N}$ and $x\left[n\right]$: 
    \[y\left[n\right] = \mathcal{H}\left\{x\left[n\right]\right\} \implies y\left[n - n_0\right] = \mathcal{H}\left\{x\left[n - n_0\right]\right\}\]
    
    \subparag{Personal remark}{
        This is a very good example where these notations are awful. Let's try an example with better notations, considering the following system: 
        \[\mathcal{H}\left\{x\right\}\left(t\right) = x\left(t^2\right) = y\left(t\right)\]
        
        Now, we want to verify if, for any arbitrary $\tau \in \mathbb{R}$, $y\left(t - \tau\right) \over{=}{?} \mathcal{H}\left\{x\left(t - \tau\right)\right\}\left(t\right)$. The left hand side is given by: 
        \[y\left(t - \tau\right) = x\left(\left(t - \tau\right)^2\right)\]
        
        To understand better the right hand side, let's remove the shortcut we took, writing $z\left(t\right) = x\left(t - \tau\right)$. This gives us that: 
        \[\mathcal{H}\left\{x\left(t - \tau\right)\right\}\left(t\right) = \mathcal{H}\left\{z\right\}\left(t\right) = z\left(t^2\right) = x\left(t^2 - \tau\right)\]

        Clearly, both hand sides are not equal, meaning that our system is not time invariant.
    }
}

\parag{Definition: Memoryless system}{
    A system is \important{memoryless} if the current system output only depends on the current system input.

    In other words, for CT systems, for all signals $x\left(t\right)$ and $\tau \in \mathbb{R}$, we need that, if $y\left(t\right) = \mathcal{H}\left\{x\left(t\right)\right\}$, then $y\left(\tau\right)$ depends only on $x\left(\tau\right)$.

    Similarly, for a DT system, for all signals $x\left[n\right]$ and all $n_0 \in \mathbb{N}$, we need that, if $y\left[n\right] = \mathcal{H}\left\{x\left[n\right]\right\}$ then $y\left[n_0\right]$ depends only on $x\left[n_0\right]$.
}

\parag{Definition: Causal system}{
    A system is \important{causal} if  the output signal depends on current and past inputs only, not on future inputs.

    In other words, for CT systems, for all signals $x\left(t\right)$ and $\tau \in \mathbb{R}$, we need that, if $y\left(t\right) = \mathcal{H}\left\{x\left(t\right)\right\}$, then $y\left(\tau\right)$ depends only on $x\left(t\right)$ for $-\infty\leq t \leq \tau$.

    Similarly, for a DT system, for all signals $x\left[n\right]$ and all $n_0 \in \mathbb{N}$, we need that, if $y\left[n\right] = \mathcal{H}\left\{x\left[n\right]\right\}$ then $y\left[n_0\right]$ depends only on $x\left[n\right]$ for $-\infty \leq n \leq n_0$.

    \subparag{Remark}{
        If a system is memoryless, then it is causal.
    }
    
    \subparag{Example}{
        Let's consider the following system: 
        \[y\left[n\right] = \sum_{k=n-1}^{n+1} x\left[k\right]\]
        
        This is like a non-normalised moving average. It is not causal since $y\left[n\right]$ depends on $y\left[n+1\right]$.
    }
}

\parag{Definition: Invertible system}{
    A system is \important{invertible} if distinct inputs lead to distinct outputs.

    In other words, for CT signals: 
    \[\mathcal{H}\left\{x_1\left(t\right)\right\} = \mathcal{H}\left\{x_2\left(t\right)\right\} \implies x_1\left(t\right) = x_2\left(t\right)\]
    
    Similarly, for DT signals: 
    \[\mathcal{H}\left\{x_1\left[n\right]\right\} = \mathcal{H}\left\{x_2\left[n\right]\right\} \implies x_1\left[n\right] = x_2\left[n\right]\]
}

\parag{Definition: Stable system}{
    A system is \important{stable} if bounded inputs lead to bounded outputs. This is thus known as BIBO stability.

    In other words, for CT signals, there must exist $B, B_0 \in\mathbb{R}$ such that, for all $t \in \mathbb{R}$: 
    \[\left|x\left(t\right)\right| \leq B \implies \left|\mathcal{H}\left\{x\left(t\right)\right\}\right| \leq B_0\]

    Similarly, for DT signals there must exist $B, B_0 \in\mathbb{R}$ such that, for all $n \in \mathbb{N}$: 
    \[\left|x\left[n\right]\right| \leq B \implies \left|\mathcal{H}\left\{x\left[n\right]\right\}\right| \leq B_0\]
}


\end{document}
