% !TeX program = lualatex
% Using VimTeX, you need to reload the plugin (\lx) after having saved the document in order to use LuaLaTeX (thanks to the line above)

\documentclass[a4paper]{article}

% Expanded on 2023-03-03 at 13:29:23.

\usepackage{../../style}

\title{Signals and systems}
\author{Joachim Favre}
\date{Vendredi 03 mars 2023}

\begin{document}
\maketitle

\lecture{2}{2023-03-03}{Convolutions}{
\begin{itemize}[left=0pt]
    \item Definition of LTI systems.
    \item Definition of the unit impulse and the unit step for DT and CT signals.
    \item Definition of the impulse response of DT and CT systems, and proof that they completely define the system thanks to DT and CT convolutions.
    \item Explanation (and proof of some) properties of DT and CT convolutions.
\end{itemize}

}

\section{LTI systems}
\subsection{Fundamentals}

\begin{parag}{Definition: LTI system}
    A \important{LTI system} (short for linear time-invariant system) is a system which is both linear and time invariant \textit{(I bet you did not see that coming!)}.
\end{parag}

\begin{parag}{Example}
    Let us say that we know that some system is LTI, $y_1\left[n\right] = \mathcal{H}\left\{x_1\left[n\right]\right\}$. 

    Now, if we get $x_2\left[n\right] = 0.5x_1\left[n\right] + 2x_1\left[n-1\right]$, we can compute $y_2\left[n\right] = \mathcal{H}\left\{x_2\left[n\right]\right\}$ using things we already computed for $y_1\left[n\right]$: 
    \[y_2\left[n\right] = \mathcal{H}\left\{x_2\left[n\right]\right\} = \mathcal{H}\left\{0.5 x_1\left[n\right] + 2x_1\left[n-1\right]\right\}\]

    Now, using linearity and then time invariance: 
    \[y_2\left[n\right] = 0.5 \mathcal{H}\left\{x_1\left[n\right]\right\} + 2 \mathcal{H}\left\{x_1\left[n-1\right]\right\} = 0.5 y_1\left[n\right] + 2 y_1 \left[n-1\right]\]
    

    \begin{subparag}{Remark}
        This shows how powerful a LTI system is. In fact, because of this, we will mainly focus our analysis on them in this course. 
    \end{subparag}
\end{parag}

\begin{parag}{Definition: Unit impulse}
    The \important{unit impulse} (or the Kronecker-delta function) is a signal defined as:
    \begin{functionbypart}{\delta \left[n\right]}
        1, & \text{if } n = 0 \\
        0, & \text{otherwise}
    \end{functionbypart} 

    \begin{subparag}{Property}
        We can write any signal as a linear sum of this unit impulse: 
        \[x\left[n\right] = \sum_{k=-\infty}^{\infty} x\left[k\right] \delta \left[n-k\right]\]
    \end{subparag}
\end{parag}

\begin{parag}{Definition: Unit step}
    The \important{unit step} is a signal defined as: 
    \[u\left[n\right] = \sum_{k=-\infty}^{n} \delta \left[k\right]\]
    
    \begin{subparag}{Intuition}
        This is basically a function which is 0 for all $n < 0$, and 1 for all $n \geq 0$.

        \imagehere[0.5]{DicreteUnitStep.png}
    \end{subparag}

    \begin{subparag}{Property}
        We have the following property: 
        \[\delta \left[n\right] = u\left[n\right] - u\left[n-1\right]\]
    \end{subparag}
\end{parag}


\begin{parag}{Definition: Impulse response}
    Let $\mathcal{H}$ be any system.

    The \important{impulse response} of the system $\mathcal{H}$ is:
    \[h\left[n\right] = \mathcal{H}\left\{\delta \left[n\right]\right\}\]

    \begin{subparag}{Remark}
        The impulse response completely characterises an LTI system.

        As we saw, we can write:
        \[x\left[n\right] = \sum_{k=-\infty}^{\infty} x\left[k\right] \delta \left[n-k\right]\]

         So, we get that: 
         \autoeq{y\left[n\right] = \mathcal{H}\left\{x\left[n\right]\right\} = \mathcal{H}\left\{\sum_{k=-\infty}^{\infty} x\left[k\right] \delta \left[n - k\right]\right\} = \sum_{k=-\infty}^{\infty} x\left[k\right] \mathcal{H}\left\{\delta \left[n-k\right]\right\} = \sum_{k=-\infty}^{\infty} x\left[k\right] h\left[n-k\right]}

         This sum is very important, so we name it \important{convolution sum}, and we write it using a convolution:
         \[y\left[n\right] = \sum_{k=-\infty}^{\infty} x\left[k\right] h\left[n-k\right] = \left(x * h\right)\left[n\right]\]
    \end{subparag}
\end{parag}

\begin{parag}{Definition: Dirac Delta}
    We would like to define something similar to the unit impulse, but for the continuous case. 

    Informally, we define the \important{Dirac delta} (or unit impulse) as the following signal: 
    \begin{functionbypart}{\delta\left(t\right)}
        \infty, & t = 0 \\
        0, & \text{otherwise}
    \end{functionbypart}

    Under the following constraint:
    \[\int_{-\infty}^{\infty} \delta\left(t\right)dt = 1\]
    
    We write this using distribution theory notations:
    \imagehere[0.5]{DiracDelta.png}

    \begin{subparag}{As a limit}
        A more formal way to define this Dirac delta is as a limit. Let us consider the following function:
        \begin{functionbypart}{\delta_{\Delta}\left(t\right)}
            \frac{1}{\Delta} & \text{if } 0 \leq t \leq \Delta \\
            0 & \text{otherwise}
        \end{functionbypart}

        Then, we can see $\delta\left(t\right)$ to be: 
        \[\delta\left(t\right) = \lim_{\Delta \to 0^+} \delta_{\Delta}\left(t\right)\]
    \end{subparag}

    \begin{subparag}{Remark}
        Just as for the discrete case, for any ``nicely behaved'' $x\left(t\right)$: 
        \[x\left(t\right) = \int_{-\infty}^{\infty} x\left(\tau\right) \delta\left(t - \tau\right) d\tau\]
    \end{subparag}

    
\end{parag}

\begin{parag}{Example}
    Let us consider $\delta\left(\frac{1}{2} t\right)$. This still has first property, so we only need to consider the second one. Doing the change of variable $\frac{1}{2}t = s$: 
    \[\int_{-\infty}^{\infty} \delta\left(\frac{1}{2}t\right)dt = \int_{-\infty}^{\infty} \delta\left(s\right) 2ds = 2 \int_{-\infty}^{\infty} \delta\left(t\right)dt = 2\]
    
    We thus see that: 
    \[\delta\left(\frac{1}{2} t\right) = 2\delta\left(t\right)\]
\end{parag}

\begin{parag}{Definition: Unit step}
    We define the \important{unit step} signal to be:
    \begin{functionbypart}{u\left(t\right)}
        0, & t < 0 \\
        1, & t \geq 0
    \end{functionbypart} 

    \begin{subparag}{Remark}
        In general, we don't really care what value it has at $t = 0$.
    \end{subparag}

    \begin{subparag}{Property}
        We can link the unit impulse to the Dirac delta: 
        \[u\left(t\right) = \int_{-\infty}^{t} \delta\left(\tau\right) d\tau\]
        Also: 
        \[\delta\left(t\right) = \frac{du\left(t\right)}{dt}\]
        
        Note that we are using the generalised version of the derivative in distribution theory, for it to exist and make sense.
    \end{subparag}
\end{parag}

\begin{parag}{Definition: Impulse response}
    Let $\mathcal{H}$ be any system. We define the \important{impulse response} of this system to be: 
    \[h\left(t\right) = \mathcal{H}\left\{\delta\left(t\right)\right\}\]
    
    \begin{subparag}{Remark}
        Just as for the discrete case, the impulse response completely characterises an LTI system. 

        Indeed, we get that: 
        \autoeq{y\left(t\right) = \mathcal{H}\left\{x\left(t\right)\right\} = \mathcal{H}\left\{\int_{-\infty}^{\infty} x\left(\tau\right)\delta\left(t - \tau\right)d\tau\right\} = \int_{-\infty}^{\infty} \mathcal{H}\left\{x\left(t\right)\delta\left(t - \tau\right)\right\}d\tau = \int_{-\infty}^{\infty} x\left(\tau\right) \mathcal{H}\left\{\delta\left(t - \tau\right)\right\}d\tau
        = \int_{-\infty}^{\infty} x\left(\tau\right) h\left(t-\tau\right)d\tau}

        This integral is very fundamental, so we name it the \important{convolution integral}, and write it using a convolution: 
        \[y\left(t\right) = \int_{-\infty}^{\infty} x\left(\tau\right) h\left(t-\tau\right)d\tau = \left(x * h\right)\left(t\right)\]
    \end{subparag}
\end{parag}

\begin{parag}{Example}
    Let's say we have the following LTI system: 
    \[y\left(t\right) = \int_{-\infty}^{t} x\left(\tau\right)d\tau\]
    
    Its impulse response is given by: 
    \[h\left(t\right) = \int_{-\infty}^{t} \delta\left(\tau\right)d\tau = u\left(t\right)\]
    
    Now, let's consider the following signal, for some $a > 0$: 
    \[x\left(t\right) = e^{-at} u\left(t\right)\]
    
    We want to find $y\left(t\right)$. When $t \leq 0$, we get: 
    \[y\left(t\right) = \int_{-\infty}^{t} e^{-a\tau}u\left(\tau\right)d\tau = \int_{-\infty}^{t} 0 d\tau = 0\]

    And, if $t > 0$: 
    \[y\left(t\right) = \int_{0}^{t} e^{-a\tau} d\tau = \frac{-1}{a} e^{-a \tau} \eval_{0}^{\tau} = \frac{1}{a}\left(1 - e^{-at}\right)\]
    
    So, this gives us that: 
    \[y\left(t\right) = \frac{1}{a}\left(1 - e^{-at}\right)u\left(t\right)\]

    We could also have used a convolution. Both ways are equivalent.
\end{parag}

\begin{parag}{Definition: Convolution}
    We defined a discrete-time convolution as: 
    \[\left(x*h\right)\left[n\right] = \sum_{k=-\infty}^{\infty} x\left[k\right] h\left[n-k\right]\]
    
    Also, we defined the continuous-time one as: 
    \[\left(x*h\right)\left(t\right) = \int_{-\infty}^{\infty} x\left(\tau\right)h\left(t - \tau\right) d\tau\]
\end{parag}

\begin{parag}{Definition: Absolutely summable}
    A DT signal is \important{absolutely summable} if: 
    \[\sum_{n=-\infty}^{\infty} \left|x\left[n\right]\right| < \infty\]

    In other words, this sum must converge to a real and finite number.
\end{parag}

\begin{parag}{Definition: Absolutely integrable}
    A CT signal is \important{absolutely integrable} if: 
    \[\int_{-\infty}^{\infty} \left|x\left(t\right)\right|dt < \infty\]

    In other words, this integral must converge to a real and finite number.
\end{parag}

\begin{parag}{Properties}
    A discrete-time convolution has the following properties:
    \begin{itemize}
        \item Identity: $\displaystyle \left(x*\delta\right)\left[n\right] = x\left[n\right]$
        \item Time invariance: $\displaystyle \left(x * \delta_{n_0}\right)\left[n\right] = x\left[n - n_0\right]$ where $\delta_{n_0}\left[n\right] = \delta \left[n - n_0\right]$
        \item Commutative: $\displaystyle \left(x*h\right)\left[n\right] = \left(h * x\right)\left[n\right]$
        \item Distributive: $\displaystyle \left(x*\left(h_1 + h_2\right)\right)\left[n\right] = \left(x*h_1\right)\left[n\right] + \left(x *h_2\right)\left[n\right]$
        \item Associative: $\displaystyle \left(\left(x *h_1\right)*h_2\right)\left[n\right] = \left(x * \left(h_1 * h_2\right)\right)\left[n\right]$ if $x\left[n\right], h_1\left[n\right], h_2\left[n\right]$ are absolutely hummable \textit{(typo I decided to keep)}.
    \end{itemize}
    
    Similarly, for continuous-time convolutions:
    \begin{itemize}
        \item Identity: $\displaystyle \left(x*\delta\right)\left(t\right) = x\left(t\right)$
        \item Time invariance: $\displaystyle \left(x * \delta_{t_0}\right)\left(t\right) = x\left(t - t_0\right)$ where $\delta_{t_0}\left(t\right) = \delta \left(t - t_0\right)$
        \item Commutative: $\displaystyle \left(x*h\right)\left(t\right) = \left(h * x\right)\left(t\right)$
        \item Distributive: $\displaystyle \left(x*\left(h_1 + h_2\right)\right)\left(t\right) = \left(x*h_1\right)\left(t\right) + \left(x *h_2\right)\left(t\right)$
        \item Associative: $\displaystyle \left(\left(x *h_1\right)*h_2\right)\left(t\right) = \left(x * \left(h_1 * h_2\right)\right)\left(t\right)$ if $x\left(t\right), h_1\left(t\right), h_2\left(t\right)$ are absolutely integrable.
    \end{itemize}

    \begin{subparag}{Proof: DT commutativity}
        Let us prove that DT convolutions are commutative. Using the change of variables $i = n -k$:
        \[\left(x*h\right)\left[n\right] = \sum_{k=-\infty}^{\infty} x\left[k\right]h\left[n-k\right] =  \sum_{i=-\infty}^{\infty} x\left[n-i\right]h\left[i\right] = \left(h*x\right)\left[n\right]\]

        \qed
    \end{subparag}

    \begin{subparag}{Proof: DT associativity}
        Let us prove that DT convolutions are associative. Using the change of variable $i = m-k$:
        \autoeq{\left(\left(x * h_1\right) * h_2\right)\left[n\right] = \sum_{m=-\infty}^{\infty} \left(x*h_1\right)\left[n\right] h_2\left[n-m\right] = \sum_{m=-\infty}^{\infty} \left(\sum_{k=-\infty}^{\infty} x\left[k\right]h_1\left[m-k\right]\right) h_2\left[n-m\right] = \sum_{k=-\infty}^{\infty} x\left[k\right]\left(\sum_{m=-\infty}^{\infty} h_1\left[m-k\right] h_2\left[n-m\right]\right) = \sum_{k=-\infty}^{\infty} x\left[k\right] \left(\sum_{i=-\infty}^{\infty} h_1\left[i\right]h_2\left[n - i - k\right]\right) = \sum_{k=-\infty}^{\infty} x\left[k\right] \left(h_1 * h_2\right)\left[n-k\right] = \left(x* \left(h_1 * h_2\right)\right)\left[n\right]}

        \qed
    \end{subparag}
\end{parag}


\end{document}
