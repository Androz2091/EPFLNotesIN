% !TeX program = lualatex
% Using VimTeX, you need to reload the plugin (\lx) after having saved the document in order to use LuaLaTeX (thanks to the line above)

\documentclass[a4paper]{article}

% Expanded on 2023-03-11 at 23:45:13.

\usepackage{../../style}

\title{Signals and systems}
\author{Joachim Favre}
\date{Samedi 11 mars 2023}

\begin{document}
\maketitle

\lecture{3}{2023-03-10}{Lots of stuff}{
\begin{itemize}[left=0pt]
    \item Explanation of the flip-and-drag method.
    \item Proof of the link between the impulse response and the properties of memorylessness, causality, invertibility and stability.
    \item Definition of series and parallel composition of systems.
    \item Proof of the link between the impulse response and the combination of systems.
    \item Definition of differential and difference equations.
\end{itemize}

}

\parag{Flip-and-drag method}{
    To compute a convolution, we can use a graphical method name the flip-and-drag method. Let us consider again the definition of DT convolutions: 
    \[\left(x * h\right)\left[n\right] = \sum_{k=-\infty}^{\infty} x\left[k\right] h\left[n-k\right]\]
    
    So, we can plot the input as a function of $k$. Then, we flip $h\left[k\right]$ and shift it $n$ units to the right. We just compute the product of the two functions, and add everything up to get the value for the current $n$. Then we drag the function and do it for another $n$.

    The case for continuous time is completely similar, excepted that we drag over continuous steps instead of discrete ones, and then just compute the of the product of functions.
}

\subsection{LTI system properties}

\parag{Observation}{
    We have defined a lot of properties systems can have. However, we know that LTI systems are completely expressed by their impulse response, so we can deduce the property from their impulse response.
}

\parag{Memorylessness}{
    An LTI system $\mathcal{H}$ with impulse response $h\left(t\right)$ or $h\left[n\right]$ is memoryless if and only if there exists $a \in \mathbb{R}$ such that:
    \[h\left(t\right) = a \delta\left(t\right), \mathspace h\left[n\right] = a \delta \left[n\right]\]
    
    \subparag{Proof $\implies$}{
        Since $\mathcal{H}$ is memoryless and time invariant by hypothesis, this necessarily means that $y\left(t\right) = \mathcal{H}\left\{x\right\}\left(t\right)$ only depends on $x\left(t\right)$, and thus that there exists a $f$ such that: 
        \[y\left(t\right) = \mathcal{H}\left\{x\right\}\left(t\right) = f\left(x\left(t\right)\right)\]

        Now, using the fact that $\mathcal{H}$ is linear:
        \autoeq{\mathcal{H}\left\{\alpha x_1\left(t\right) + \beta x_2\left(t\right)\right\}\left(t\right) = \alpha \mathcal{H}\left\{x_1\right\}\left(t\right) + \beta \mathcal{H}\left\{x_2\right\}\left(t\right)  \iff f\left(\alpha x_1\left(t\right) + \beta x_2\left(t\right)\right) = \alpha f\left(x_1\left(t\right)\right) + \beta g\left(x_2\left(t\right)\right)}
        
        This means that $f$ is a $\mathbb{R}^1 \mapsto \mathbb{R}^1$ linear map, and thus that it must be of the form $f\left(x\right) = ax$ for $a \in \mathbb{R}^{1 \times 1} = \mathbb{R}$ by a Linear Algebra theorem. We have shown so far that, for our LTI system to be memoryless, we need:
        \[y\left(t\right) = ax\left(t\right)\]
        
        Now, we know that we can express this using a convolution: 
        \[y\left(t\right) = \left(x * h\right)\left(t\right) = \int_{-\infty}^{\infty} x\left(\tau\right)h\left(t - \tau\right) d\tau\]
        
        However, for the two to be equal, this means that $h\left(t\right)$ must be zero for any $t \neq 0$. Thus, we must have the form: 
        \[h\left(t\right) = a \delta\left(t\right)\]

        The case for DT is similar.
    }
}

\parag{Causality}{
    An LTI system with impulse response $h\left(t\right)$ is causal if and only if:
    \[h\left(t\right) = 0, \mathspace \text{ for } t <0\]

    Similarly, for DT systems:
    \[h\left[n\right] = 0, \mathspace \text{ for } n <0\]
    
    \subparag{Proof}{
        The argument is very similar to memorylessness. Also, the case for DT is very similar.
    }
}

\parag{Invertibility}{
    Any LTI system with an impulse response $h\left(t\right)$ is invertible if and only if there exists a $g\left(t\right)$ such that: 
    \[\left(g * h\right)\left(t\right) = \delta\left(t\right)\]

    Similarly, for the DT case: 
    \[\left(g * h\right)\left[n\right] = \delta \left[n\right]\]

    \subparag{Proof}{
        There must exist an inverse $g\left(t\right)$: if distinct inputs lead to distinct outputs, then we can find an inverse.

        The case for DT systems in completely analogous.
    }
}

\parag{Stability}{
    Any LTI system with impulse response $h\left(t\right)$ is stable if and only if: 
    \[\int_{-\infty}^{\infty} \left|h\left(t\right)\right|dt < \infty\]
    
    Similarly for the DT case: 
    \[\sum_{k=-\infty}^{\infty} \left|h\left[k\right]\right| < \infty\]
    
    \subparag{Proof $\implies$}{
        We will only consider the DT case; the case for CT is completely similar.

        We do the proof by the contrapositive: we suppose that $\sum_{k=-\infty}^{\infty} \left|h\left[k\right]\right| = \infty$, and we want to show that $h\left[k\right]$ is not stable. Thus, we consider the following signal: 
        \begin{functionbypart}{x\left[n\right]}
            0, & h\left[-n\right] = 0 \\
            \frac{\left(h\left[-n\right]\right)^*}{\left|h\left[-n\right]\right|}, & \text{otherwise}
        \end{functionbypart}
        where $z^*$ is the complex conjugate of $z$. 

        We notice that $\left|x\left[n\right]\right| \leq 1$ for all $n$, showing it is bounded. However: 
        \[y\left[0\right] = \sum_{k=-\infty}^{\infty} h\left[k\right]x\left[-k\right] = \sum_{k=-\infty}^{\infty} \left|h\left[k\right]\right| = \infty\]
        by our assumption. We found a bounded input which led to an unbounded input, showing that the system is not stable; as required.
    }
    
    \subparag{Proof $\impliedby$}{

        Let's suppose that we have some signal such that $\left|x\left[n\right]\right| < B$ for all $n$ and that our system follows our property: 
        \[\sum_{k=-\infty}^{\infty} \left|h\left[k\right]\right| = C \in\mathbb{R}\]
        
        This implies that: 
        \autoeq{\left|y\left[k\right]\right| = \left|\sum_{k=-\infty}^{\infty} h\left[k\right]x\left[k-n\right]\right| \leq \sum_{k=-\infty}^{\infty} \left|h\left[k\right]\right| \left|x\left[k-n\right]\right| \leq B \sum_{k=-\infty}^{\infty} \left|h\left[k\right]\right| \leq B C}
        which is finite, as required.

        \qed
    }

}


\subsection{System composition} 

\parag{Goal}{
    In real life, it is typical for different systems to be built by composing other systems together.

    In this course, we will focus on three kind of compositions: parallel, series and feedback. This last one will only be considered later.
}

\parag{Definition: Series composition}{
    Two systems $\mathcal{H}_1$ and $\mathcal{H}_2$ are composed in \important{series} (or in cascade) if they are taken one after the other: 
    \[y\left(t\right) = \mathcal{H}_2\left\{\mathcal{H}_1\left\{x\left(t\right)\right\}\right\}, \mathspace y\left[n\right] = \mathcal{H}_2\left\{\mathcal{H}_1\left\{x\left[n\right]\right\}\right\}\]

    \imagehere[0.5]{SystemsInSeries.png}
}

\parag{Definition: Parallel composition}{
    Two systems $\mathcal{H}_1$ and $\mathcal{H}_2$ are composed in \important{parallel} if they are computed in parallel, and then summed up: 
    \[y\left(t\right) = \mathcal{H}_1\left\{x\left(t\right)\right\} + \mathcal{H}_2\left\{x\left(t\right)\right\}, \mathspace y\left[n\right] = \mathcal{H}_1\left\{x\left[n\right]\right\} + \mathcal{H}_2\left\{x\left[n\right]\right\}\]
    
    \imagehere[0.5]{SystemsInParallel.png}
}

\parag{LTI systems in series}{
    Let $\mathcal{H}_1$ and $\mathcal{H}_2$ be two LTI systems with impulse response $h_1\left(t\right)$ and $h_2\left(t\right)$.

    If they composed in series, then the result is also LTI, and it has an impulse response of:
    \[h\left(t\right) = \left(h_2 * h_1\right)\left(t\right) = \left(h_1 * h_2\right)\left(t\right)\]

    The case for DT systems is completely analogous.

    \subparag{Proof}{
        It is possible rather easily to show that the composition is LTI. To derive the impulse response, we just use the associative property of convolutions: 
        \[y\left(t\right) = \mathcal{H}_2\left\{\mathcal{H}_1\left\{x\left(t\right)\right\}\right\} = \left(h_2 *\left(h_1 * x\right)\right)\left(t\right) = \left(\left(h_2 * h_1\right)* x\right)\left(t\right)\]
        
        We thus get that $\mathcal{H}$ has the following impulse response: 
        \[h\left(t\right) = \left(h_2 * h_1\right)\left(t\right) = \left(h_1 * h_2\right)\left(t\right)\]
        
        We also note that the composition order does not matter for LTI systems (though this is not the case for non-LTI systems). 
    }
}

\parag{LTI systems in parallel}{
    Let $\mathcal{H}_1$ and $\mathcal{H}_2$ be two LTI systems with impulse response $h_1\left(t\right)$ and $h_2\left(t\right)$.

    If they composed in parallel, then the result is also LTI, and it has an impulse response of:
    \[h\left(t\right) = h_1\left(t\right) + h_2\left(t\right)\]

    The case for DT systems is completely analogous.

    \subparag{Proof}{
        It is pretty easy to show that the two systems in parallel is also LTI. To find its impulse response, we can use the distributive property of convolutions: 
        \autoeq{y\left(t\right) = \mathcal{H}\left\{x\left(t\right)\right\} = \mathcal{H}_1\left\{x\left(t\right)\right\} + \mathcal{H}_2\left\{x\left(t\right)\right\} = \left(h_1 * x\right)\left(t\right) + \left(h_2 * x\right)\left(t\right) = \left(\left(h_1 + h_2\right) * x\right)\left(t\right)}

        This thus gives us that $\mathcal{H}$ has the following impulse response: 
        \[h\left(t\right) = h_1\left(t\right) + h_2\left(t\right)\]
    }
}


\subsection{Differential and difference equations}
\parag{Remark}{
    Many physical systems can be represented using differential equations (for continous-time systems) and discrete-time systems (for difference equations).

    We will focus on causal LTI systems in this section.
}

\parag{Definition: Linear constant-coefficient differential equation}{
    A \important{linear constant-coefficient differential equation} as the following form: 
    \[\sum_{k=0}^{N} a_k \frac{d^{k}y\left(t\right)}{dt^{k}} = \sum_{k=0}^{M} b_k \frac{d^{k}x\left(t\right)}{dt^{k}} \]
    
    \subparag{Observation}{
        If $N=0$, we have an explicit expression for $y\left(t\right)$. However, when $N \geq 1$, we need some more information to completely describe our system.
    }
}

\parag{Definition: Linear constant-coefficient difference equation}{
    A linear constant-coefficient difference equation has the form: 
    \[\sum_{k=0}^{N} a_k y\left[n-k\right] = \sum_{k=0}^{M} b_k x\left[n-k\right]\]
    
    This is also known as a recursive equation.

    \subparag{Observation}{
        Just as for differential equations, if $N \geq 1$ we need auxiliary conditions.
    }
}

\parag{Definition: Condition of initial rest}{
    A system follows the \important{condition of initial rest} if, for any $t_0 \in \mathbb{R}$ and any signal $x$ such that $x\left(t\right) = 0$ for all $t < t_0$, then: 
    \[y\left(t\right) = 0, \mathspace \forall t < t_0\]

    The case for DT systems is very similar.
}

\parag{Theorem}{
    An LTI system $\mathcal{H}$ is causal if and only if it satisfies the condition of initial rest.

    \subparag{Proof $\implies$}{
        Let's suppose that $\mathcal{H}$ is causal and $x\left[n\right] = 0$ for $n < n_0$.

        This implies that, for any $n < n_0$:
        \autoeq{y\left[n\right] = \sum_{k=-\infty}^{\infty} x\left[k\right] h\left[n-k\right] = \sum_{k=n_0}^{\infty} x\left[k\right]h\left[n-k\right] = \sum_{k=n_0}^{n} x\left[k\right]h\left[n-k\right] = 0}
        using first that $x\left[k\right] = 0$ for $k < n_0$, then that $h\left[k\right] = 0$ for $k < 0$ and finally that $n < n_0$.
        
    }

    \subparag{Proof $\impliedby$}{
        Let's suppose that $\mathcal{H}$ satisfies the condition of initial rest.

        We can consider $n_0 = 0$ and $x\left[n\right] = \delta \left[n\right]$, which are indeed such that: 
        \[\delta \left[n\right] = 0, \mathspace \forall n < 0\]
        
        By the condition of initial rest, this implies that $y\left[n\right] = \mathcal{H}\left\{\delta\right\}\left[n\right] = h\left[n\right]$ is such that:
        \[h\left[n\right] = 0, \mathspace \forall n < 0\]

        However, as we have seen, this implies that the system is causal.

        \qed
    }
}

\parag{Example}{
    Let us consider the following linear constant-coefficient differential equation: 
    \[\frac{dy\left(t\right)}{dt} + 2y\left(t\right) = x\left(t\right)\]
    where $x\left(t\right) = Ke^{3t}u\left(t\right)$.

    We know that the general solution is of the form $y\left(t\right) = y_p\left(t\right) + y_h\left(t\right)$, where $y_p\left(t\right)$ is a particular solution and $y_h\left(t\right)$ is the general solution to the homogeneous equation.

    The homogeneous equation is: 
    \[\frac{dy\left(t\right)}{dt} + 2y\left(t\right) = 0\]
    
    We can take the \textit{Ansatz} $y_h\left(t\right) = Ae^{s t}$, giving us $s = -2$ and thus $y_h\left(t\right) = Ae^{-2t}$.

    Now, for the particular solution, we can let the \textit{Ansatz} $y_p\left(t\right) = Ye^{3t}$ for $t > 0$. This yields that $Y = \frac{K}{5}$ and thus that $y_p\left(t\right) = \frac{K}{5} e^{3t}$ for $t > 0$.

    We finally get that: 
    \[y\left(t\right) = A e^{-2t} + \frac{K}{5} e^{3t}, \mathspace t > 0\]
    
    However, we would need more information in order to determine the value of $A$. Let us suppose that our system is also causal. By the condition of initial rest, since $x\left(t\right) = 0$ for all $t \leq 0$, we need $y\left(t\right) = 0$ for all $t \leq 0$ and, in particular:
    \[y\left(0\right) = 0 \iff A = -\frac{K}{5}\]

    Still using the initial rest condition, we get: 
    \[y\left(t\right) = \frac{K}{5}\left[e^{3t} - e^{-2t}\right]u\left(t\right)\]
    
    \subparag{Observation}{
        Solving differential equations that way is very tedious, so we will develop tools to help us.
    }
    
}




 
\end{document}
