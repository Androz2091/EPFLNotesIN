% !TeX program = lualatex
% Using VimTeX, you need to reload the plugin (\lx) after having saved the document in order to use Kuala Tex (thanks to the line above)

\documentclass[a4paper]{article}

% Expanded on 2023-03-31 at 13:26:26.

\usepackage{../../style}

\title{Signals and systems}
\author{Joachim Favre}
\date{Vendredi 31 mars 2023}

\begin{document}
\maketitle

\lecture{6}{2023-03-31}{Filters}{
\begin{itemize}[left=0pt]
    \item Intuitive interpretation of frequency responses. 
    \item Application of frequency responses to system composition.
    \item Definition of low-pass, band-pass and high-pass filters.
\end{itemize}

}

\subsection{Frequency responses}
\begin{parag}{Derivation: Continuous time}
    Let's suppose that we have the following signal: 
    \[x\left(t\right) = e^{j \omega_0 t}\]
    
    We know that we can write: 
    \[y\left(t\right) = \left(x * h\right)\left(t\right) = \int_{-\infty}^{\infty} e^{j \omega_0 \left(t- \tau\right)} h\left(\tau\right)d \tau = e^{j \omega_0 t} \int_{-\infty}^{\infty} e^{-j \omega_0 \tau} h\left(\tau\right) d\tau = e^{j\omega_0 t} H\left(\omega_0\right)\]
    where $H\left(\omega_0\right)$ is the frequency response.

    We can notice that any complex exponential is only multiplied by the frequency response when passed into a system.
\end{parag}

\begin{parag}{Interpretation}
    We can write our result as:
    \[y\left(t\right) = H\left(\omega_0\right) e^{j \omega_0 t} = \left|H\left(\omega_0\right)\right| e^{j \arg\left(H\left(\omega_0\right)\right)} e^{j \omega_0 t} = \left|H\left(\omega_0\right)\right| \exp\left(j \omega_0 \left(t + \frac{\arg\left(H\left(\omega_0\right)\right)}{\omega_0}\right)\right)\]
    
    Thus, our system scales the magnitude of our input complex exponential, and shifts it in time by $\frac{\arg\left(H\left(\omega_0\right)\right)}{\omega_0}$.
\end{parag}

\begin{parag}{Example}
    Let us consider an LTI system described by: 
    \[y\left(t\right) = x\left(t - 3\right)\]
    
    Its impulse and frequency response are: 
    \[h\left(t\right) = \delta\left(t - 3\right) \implies H\left(\omega\right) = e^{-3 \omega j}\]

    Thus, we indeed see that any signal will be of shifted $\frac{\arg\left(H\left(\omega_0\right)\right)}{\omega_0} = -3$.
\end{parag}

\begin{parag}{Derivation: Discrete-time}
    Let us consider the following signal: 
    \[x\left[n\right] = e^{j \omega_0 n}\]
    
    We have that: 
    \[y\left[n\right] = \left(x * h\right)\left[n\right] = \sum_{k=-\infty}^{\infty} e^{j \omega_0 \left(n-k\right)} h\left[k\right] = e^{j \omega_0 n} \sum_{k=-\infty}^{\infty} e^{-j \omega_0 k} h\left[k\right] = e^{j \omega_0 n} H\left(e^{j \omega_0}\right)\]
    where $H\left(e^{j \omega_0}\right)$ is the frequency response.
\end{parag}

\begin{parag}{Interpretation: DT}
    Just like for CT, we can notice that we can write: 
    \[\left|H\left(e^{j \omega}\right)\right| \exp\left(j \omega_0 \left(t + \frac{\arg\left(H\left(e^{j \omega}\right)\right)}{\omega_0}\right)\right)\]

    We can thus have the same interpretation as for DT.
\end{parag}

\begin{parag}{Theorem}
    The frequency response of any DT system is $2\pi$-periodic.

    \begin{subparag}{Proof}
        This is a direct implication from the fact that the DTFT is $2\pi$-periodic.
    \end{subparag}
\end{parag}

\begin{parag}{Theorem}
    When the impulse response is real-valued, the frequency response satisfies: 
    \[H\left(\omega\right) = H^*\left(-\omega\right), \mathspace H\left(e^{j\omega}\right) = H^*\left(e^{j\omega}\right)\]

    \begin{subparag}{Proof}
        This is also something which comes from the properties of DT and CT Fourier Transforms.
    \end{subparag}
    
\end{parag}

\begin{parag}{Example}
    Let us consider some LTI system with real impulse response $h\left(t\right)$, and the following signal: 
    \[x\left(t\right) = \cos\left(\omega_0 t\right)\]
    
    We want to derive the output in terms of the frequency response. First, let us write our signal using complex exponentials: 
    \[x\left(t\right) = \frac{1}{2} e^{j \omega_0 t} + \frac{1}{2} e^{-j \omega_0 t}\]
    
    Now, by construction of the frequency response: 
    \autoeq{y\left(t\right) = \frac{1}{2} H\left(\omega_0\right) e^{j \omega_0 t} + \frac{1}{2} H\left(-\omega_0\right) e^{-j \omega_0 t} = \frac{H\left(\omega_0\right) e^{j \omega_0 t} + H^*\left(\omega_0\right) e^{-j \omega_0 t}}{2} = \frac{\left|H\left(\omega_0\right)\right|}{2} \left(e^{j \omega_0 t + j \arg\left(H\left(\omega_0\right)\right)} + e^{-j \omega_0 t - j \arg\left(H\left(\omega_0\right)\right)}\right) = \left|H\left(\omega_0\right)\right| \cos\left(\omega_0 t + \arg\left(H\left(\omega_0\right)\right)\right)}
    
    Thus, the system scales our cosine, and shift it in time by $\frac{\arg\left(H\left(\omega_0\right)\right)}{\omega_0}$, just like for complex exponentials.
\end{parag}

\begin{parag}{Parallel system composition}
    We know that, in the time domain, the impulse response of systems composed in parallel is: 
    \[h_1\left(t\right) + h_2\left(t\right)\]
    
    Thus, in the frequency domain, it is: 
    \[H_1\left(\omega\right) + H_2\left(\omega\right)\]
\end{parag}

\begin{parag}{Series system composition}
    In the time domain, the impulse response of two systems being composed in series is: 
    \[\left(h_1 * h_2\right)\left(t\right)\]
    
    Thus, in the frequency domain, the frequency response is: 
    \[H_1\left(\omega\right) H_2\left(\omega\right)\]
\end{parag}

\begin{parag}{Example 1}
    Let us consider the following system composition:
    \imagehere[0.7]{SystemCompositionExample.png}

    We want to characterise this system using its frequency response, $D\left(e^{j\omega}\right)$.

    Looking at the second half of the system, we see that: 
    \[Y\left(e^{j\omega}\right) = G\left(e^{j\omega}\right) W\left(e^{j \omega}\right) + H\left(e^{j\omega}\right) V\left(e^{j \omega}\right)\]
    
    Moreover, we can see that $V\left(e^{j\omega}\right) = W\left(e^{j\omega}\right)$ by symmetry, and: 
    \autoeq{W\left(e^{j\omega}\right) = X\left(e^{j\omega}\right) + G\left(e^{j\omega}\right)X\left(e^{j\omega}\right) + H\left(e^{j\omega}\right)X\left(e^{j\omega}\right) = X\left(e^{j\omega}\right)\left(1 + G\left(e^{j\omega}\right) + H\left(e^{j\omega}\right)\right)}
    
    Putting everything together, we get: 
    \[Y\left(e^{j\omega}\right) = \left(G\left(e^{j\omega}\right) + H\left(e^{j\omega}\right)\right)\left(1 + G\left(e^{j\omega}\right) + H\left(e^{j\omega}\right)\right) X\left(e^{j\omega}\right)\]
    
    We can thus identify that: 
    \[D\left(e^{j\omega}\right) = \left(G\left(e^{j\omega}\right) + H\left(e^{j\omega}\right)\right) \left(1 + G\left(e^{j\omega}\right) + H\left(e^{j\omega}\right)\right)\]
\end{parag}

\begin{parag}{Example 2}
    We generalise the composition from the previous example. For instance, for $k = 3$:
    \imagehere[0.9]{SystemCompositionExample-2.png}

    We can notice that we can write this recursively: 
    \autoeq{D_k = \left(D_{k-1} + 1\right)\left(G + H\right) = \left[\left(D_{k-2} + 1\right)\left(G + 1\right) + 1\right]\left(G + H\right) = \ldots = \left(G+H\right)^k + \ldots + \left(G + h\right)}
    
    We can confirm that this is exactly what we found before, when $k = 2$.
\end{parag}

\subsection{Frequency selective filters}
\begin{parag}{Goal}
    Frequency selective filters are a type of LTI systems which keep some frequencies but remove others.

    This is for instance very useful in audio-processing or in communication systems to isolate the communication frequency.

    To see how they behave, we will analyse their frequency response; its magnitude and argument more precisely, since it allows us to know which frequencies it removes (where its magnitude is close to 0) and how it shifts them.

    \begin{subparag}{Example}
        A simple low-pass filter can be an RC circuit. 
    \end{subparag}
    
\end{parag}

\begin{parag}{Ideal low-pass filter}
    And ideal low-pass filter is a system which only keeps low frequencies, meaning that it is described by the following frequency response:
    \imagehere[0.8]{LowPassFilter.png}

    $\omega_c$ is named the cut-off frequency.

    They are ideal since they have sharp edges, and zero phase. However, we could also have a non-zero argument, making less ideal (even though this would still be very ideal).

    \begin{subparag}{Time domain}
        We can notice that the impulse response of a low-pass filter is a $\sinc$ function. We clearly see that interpreting such filters is much easier in the frequency domain: it would not have been easy to know that the sinc function is a low-pass filter without Fourier analysis.
    \end{subparag}
\end{parag}

\begin{parag}{Ideal band-pass filter}
    An ideal band-pass filter only keeps frequencies in a given range, meaning it has a frequency response looking like:
    \imagehere[0.8]{BandPassFilter.png}
\end{parag}

\begin{parag}{Ideal high-pass filter}
    A high-pass filter lets high frequencies pass, but stops small ones:
    \imagehere[0.8]{HighPassFilter.png}
\end{parag}

\begin{parag}{Example: Moving average}
    Let us consider the following system, which takes a moving average: 
    \[y\left[n\right] = \frac{1}{N + M + 1} \sum_{k=-N}^{M} x\left[n-k\right]\]
    
    The frequency response is: 
    \[H\left(e^{j\omega}\right) = \frac{1}{N+M+1} \sum_{k=-N}^{M} e^{-j \omega k} = \frac{1}{N+M+1} e^{j\omega\left(N - \frac{M}{2}\right)} \frac{\sin\left(\omega \frac{M+N}{2}\right)}{\sin\left(\frac{\omega}{2}\right)}\]
    
    Its magnitude can be represented as:
    \imagehere[0.6]{MovingAverageFrequencyResponse.png}

    We thus see that it is some kind of low-pass filter. This makes sense since, intuitively, when taking a moving average, we remove the noisy fluctuation, meaning the ones with very high frequency.
\end{parag}

\end{document}
