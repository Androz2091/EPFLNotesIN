% !TeX program = lualatex
% Using VimTeX, you need to reload the plugin (\lx) after having saved the document in order to use LuaLaTeX (thanks to the line above)

\documentclass[a4paper]{article}

% Expanded on 2023-05-05 at 15:54:54.

\usepackage{../../style}

\title{Signals and systems}
\author{Joachim Favre}
\date{Vendredi 05 mai 2023}

\begin{document}
\maketitle

\lecture{7}{2023-05-05}{The best lecture so far}{
\begin{itemize}[left=0pt]
    \item Definition of sampling interval and sampling frequency.
    \item Explanation and proof of the bandlimited interpolation sampling strategy.
    \item Explanation and proof of the sampling theorem.
    \item Definition of the Nyquist rate.
\end{itemize}

}

\section{Sampling}

\begin{parag}{Definition: Sampling}
    Sampling is the process of turning a continuous signal to a discrete one.
    
    \begin{subparag}{Example}
        This is for instance very important when we turn an analogue picture to a digital one, or when we are recording sound. In general, this leads to loss of information, but our goal is to minimise.
    \end{subparag}

    \begin{subparag}{Problem}
        The problem can be represented as the following system, where we have an analog-digital and a digital-analog converter:
        \imagehere[1]{SamplingSystem.png}

        At first, we will simply consider that the system $\mathcal{H}$ is an identity one.
    \end{subparag}
\end{parag}

\begin{parag}{Main idea}
    Let $T \in \mathbb{R}^*_+$ and some continuous signal $x\left(t\right)$. The main idea is to pick samples at distance $T$ one from another: 
    \[x\left[n\right] = x\left(nT\right)\]
    
    $T$ is named the \important{sampling interval}.

    \begin{subparag}{Problem}
        The main problem we have now is that it seems very hard to reconstruct $x\left(t\right)$ from $x\left[n\right]$: there is an infinite amount of functions which could have been mapped to $x\left[n\right]$. We will thus instead try to pick the simplest function which is coherent.
    \end{subparag}
\end{parag}

\begin{parag}{Example}
    Let us consider the following continuous signal, which we want to sample with a sampling interval $T = 10$: 
    \[x\left(t\right) = \cos\left(\frac{\pi}{5}t\right)\]
    
    Then, we get that: 
    \[x\left[n\right] = x\left(nT\right) = \cos\left(\frac{\pi}{5} nT\right) = \cos\left(2\pi\right) = 1\]
    
    Now, to reconstruct this is very hard. By simplicity, we will just try to find the lowest frequency sinusoid that fits the samples, giving: 
    \[x_r\left(t\right) = 1\]
    
    This is not correct because we did not pick a sampling interval too small. This problem is known as \important{aliasing}.
\end{parag}

\begin{parag}{Definition: Sampling frequency}
    Let $T$ be some sampling interval. Then, the \important{sampling frequency} is: 
    \[\omega_s = \frac{2\pi}{T}\]
\end{parag}


\begin{parag}{Complex exponential}
    For simplicity, let us try to sample the following continuous signal, with sampling interval $T$: 
    \[x\left(t\right) = \exp\left(j \omega_0 t\right)\]
    
    We see that: 
    \[x\left[n\right] = x\left(nT\right) = \exp\left(j \omega_0 n T\right) = \exp\left(j \left(\omega_0 T\right)n\right) = \exp\left(j \hat{\omega}_0n\right)\]
    where we define $\hat{\omega}_0 = \omega_0 T + 2\pi \ell $ such that $\ell \in \mathbb{Z}$ and $-\pi < \hat{\omega}_0 \leq \pi$.

    We now want to reconstruct $x_r\left[n\right]$. We know that we started with an exponential, so we want to end with one too. The thing is, because of the periodicity of complex exponentials, our $x\left[n\right]$ is consistent with sampling any: 
    \[\exp\left(j \left(\frac{\hat{\omega}_0 + 2\pi \ell }{T}\right)t\right), \mathspace \ell \in \mathbb{Z}\]
    
    Again by simplicity, we reconstruct to the lowest frequency, with $\ell = 0$: 
    \[x_r\left(t\right) = \exp\left(j \frac{\hat{\omega}_0}{T}t\right)\]
    
    We notice that $x_r\left(t\right) = x\left(t\right)$ if and only if: 
    \[\frac{\hat{\omega}_0}{T} = \omega_0 \iff \left|\omega_0 T\right| \leq \pi \iff 2 \left|\omega_0\right| \leq \frac{2\pi}{T} = \omega_s\]

    Indeed, in order to have $\hat{\omega}_0 = \omega_0 T$, we need $-\pi < \omega_0 T \leq \pi$ (the $\ell $ in the definition of $\hat{\omega}_0$ must be 0), which indeed implies that $\left|\omega_0 T\right| \leq \pi$.

    Requiring $2 \left|\omega_0\right| \leq \omega_s$ is a condition on the sampling frequency we want to be able to reconstruct without loss which we will want to generalise.
\end{parag}

\begin{parag}{Example}
    Let us consider the following continuous signal:
    \[x\left(t\right) = e^{j \frac{\pi}{5}t}\]
    
    We notice that, for it to be reconstructed correctly, we need: 
    \[\omega_s = \frac{2\pi}{T} > 2\left|\frac{\pi}{5}\right| \iff T < 5 \]

    For instance, if $T_1 = 4$, we get that everything works correctly: 
    \[x_1\left[n\right] = e^{j \frac{\pi}{5}T_1 n} = e^{j \frac{4}{5}\pi n} \implies x_r\left(t\right) = e^{j\frac{4}{5}\pi \frac{t}{T_1}} = e^{j \frac{\pi}{5}t} = x\left(t\right)\]
    
    However, if $T_2 = 9$, then things no longer work: 
    \[x_2\left[n\right] = e^{j\frac{\pi}{5}T_2 n} = e^{j\frac{9}{5}\pi n} = e^{-j \frac{\pi}{5}n} \implies x_r\left(t\right) = e^{-j \frac{\pi}{5} \frac{t}{T_2}} = e^{-j \frac{\pi}{45}t} \neq x\left(t\right)\]
\end{parag}

\begin{parag}{Linear combination}
    Let us consider the following continuous signal: 
    \[x\left(t\right) = A_0 e^{j \omega_0 t} + A_1 e^{j\omega_1 t}\]

    Then, we see that: 
    \[x\left[n\right] = A_0 e^{j \omega_0 T n} + A_1 e^{j \omega_1 T n} = A_0 e^{j \frac{\omega_0}{\omega_s} 2\pi n} + A_1 e^{j \frac{\omega_1}{\omega_s} 2\pi n}\]
    
    Since we want to reconstruct both using a linear combination of exponentials with smallest frequency, we need both $\left|\frac{\omega_0}{\omega_s} 2\pi\right| < \pi$ and $\frac{\omega_1}{\omega_s} 2\pi < \pi$, which is equivalent to: 
    \[\omega_s > \max\left\{2\left|\omega_0\right|, 2\left|\omega_1\right|\right\}\]

    \begin{subparag}{Remark}
        Now that we are able to sample linear combination of complex exponential, it seems like we should be able to generalise our result to signals which Fourier transform exists (since a Fourier transform is a continuous linear combination of complex exponentials).
    \end{subparag}
    
\end{parag}

\begin{parag}{Definition: Band-limited signal}
    A band-limited signal is a continuous signal which Fourier transform exists, and for which there exists an $\omega_M \in \mathbb{R}$ such that:
    \[x\left(t\right) = \frac{1}{2\pi} \int_{-\infty}^{\infty} X\left(\omega\right) e^{j \omega t} d\omega = \frac{1}{2\pi} \int_{-\omega_M}^{\omega_M} X\left(\omega\right) e^{j \omega t} d\omega\]
    
    In other words, $X\left(\omega\right) = 0$ for all $\left|\omega\right| > \omega_M$.
\end{parag}

\begin{parag}{Definition: Impulse train}
    An \important{impulse train} signal is a signal of the form: 
    \[p\left(t\right) = \sum_{n=-\infty}^{\infty} \delta\left(t - nT\right)\]
    for some $T \in \mathbb{R}$.

    We notice that, the bigger $T$, the farther the $\delta$ will be one-another.

    \begin{subparag}{Fourier transform}
        The Fourier transform of the impulse train is: 
        \[P\left(\omega\right) = \frac{2\pi}{T} \sum_{k=-\infty}^{\infty} \delta\left(\omega - \frac{2\pi k}{T}\right)\]

        We notice that, the bigger $T$, the closer the $\delta$ will be one-another.
    \end{subparag}
\end{parag}

\begin{parag}{Bandlimited interpolation}
    Let $x\left(t\right)$ be some signal we want to sample, and $T$ be the sampling interval.

    We take our sampled signal as: 
    \[x_p\left(t\right) = x\left(t\right)p\left(t\right) = x\left(t\right)\sum_{n=-\infty}^{\infty} \delta\left(t - nT\right) = \sum_{n=-\infty}^{\infty} x\left(t\right)\delta\left(t - nT\right) = \sum_{n=-\infty}^{\infty} x\left(nT\right) \delta\left(t - nT\right)\]
    
    This looks like: 
    \imagehere[0.7]{SampledSignalArrows.png}

    We notice that $x_p\left(t\right)$ is equivalent to $x\left[n\right] = x\left(nT\right)$; they both have the exact same amount of information since there is a bijection between the two. We will thus analyse $x\left[n\right]$ by analysing $x_p\left(t\right)$.

    Let us consider the Fourier transform of $x_p\left(t\right)$: 
    \autoeq{X_p\left(\omega\right) = \frac{1}{2\pi}\left(X\left(\omega\right) * P\left(\omega\right)\right) = \frac{1}{2\pi} X\left(\omega\right) * \left(\frac{2\pi}{T} \sum_{k=-\infty}^{\infty} \delta\left(\omega - \frac{2\pi k}{T}\right)\right) = \frac{1}{T} \sum_{k=-\infty}^{\infty} \left(X\left(\omega\right) * \delta\left(\omega - \omega_s k\right)\right) = \frac{1}{T}\sum_{k=-\infty}^{\infty} X\left(\omega - \omega_s k\right)}

    For instance, the magnitude of our Fourier transform may look like:
    \imagehere{SamplingAliasing.png}

    We are thus creating many copies of our original Fourier transform, and at some places they will overlap and we will thus loose some information; this creates aliasing. To avoid loosing any information, we need those to be spaced out enough.

    So, if $x\left(t\right)$ is a bandlimited signal, meaning that there exists a $\omega_M$ such that $X\left(\omega\right) = 0$ for all $\left|\omega\right| > \omega_M$, we only need to take $\omega_s > 2 \omega_M$.
    \imagehere{SamplingNoAliasing.png}

    In that case, to reconstruct the signal, we can simply apply a low-pass filter, $H\left(\omega\right)$: 
    \begin{functionbypart}{H\left(\omega\right)}
        T, & \left|\omega\right| < \omega_c \\
        0, & \left|\omega\right| > \omega_c
    \end{functionbypart}
    where the cut-off frequency can be anywhere $\omega_M < \omega_c < \omega_s - \omega_M$
    
    In that case, we simply have $X_r\left(\omega\right) = H\left(\omega\right) H_p\left(\omega\right)$, which is such that $X_r\left(\omega\right) = X\left(\omega\right)$ by construction. 

    In the time domain, our low-pass filter looks like: 
    \[h\left(t\right) =  \frac{\omega_c T \sin\left(\omega_c t\right)}{\pi \omega_c t}\]
    
    We can apply it to recover our original signal (if we indeed have $\omega_s > 2 \omega_M$): 
    \autoeq{x_r\left(t\right) = \left(h* x_p\right)\left(t\right) = \left(h * \sum_{n=-\infty}^{\infty} x\left(nT\right)\delta\left(t - nT\right)\right) = \sum_{n=-\infty}^{\infty} x\left(nT\right)h\left(t - nT\right) = \sum_{n=-\infty}^{\infty} x\left(nT\right) \frac{\omega_c T}{\pi} \frac{\sin\left(\omega_c\left(t - nT\right)\right)}{\omega_c\left(t - nT\right)}}
    
    This is named bandlimited interpolation, and this is another instance of a problem which is definitely easier to think of in the frequency domain than in the time domain.
\end{parag}

\begin{parag}{Sampling theorem}
    Let $x\left(t\right)$ be a bandlimited signal, meaning $X\left(\omega\right) = 0$ for $\left|\omega\right| > \omega_M$.

    It is uniquely determined by its samples $x\left(nT\right)$ if: 
    \[\omega_s > 2 \omega_M\]
    where $\omega_s = \frac{2\pi}{T}$.

    \begin{subparag}{Remark}
        $\omega_s = 2 \omega_M$ is named the \important{Nyquist rate}.
    \end{subparag}

    \begin{subparag}{Formulas}
        We found that that we can sample:
        \[x\left[n\right] = x\left(nT\right)\]

        If we indeed have $\omega_s > 2 \omega_M$, then we can recover our signal by using bandlimited interpolation: 
        \[x\left(t\right) = \sum_{n=-\infty}^{\infty} x\left[n\right] \frac{\omega_c T}{\pi} \frac{\sin\left(\omega_c\left(t - nT\right)\right)}{\omega_c\left(t - nT\right)}\]
    \end{subparag}
    
\end{parag}

\begin{parag}{Example 1}
    Let us consider the following signal again: 
    \[x\left(t\right) = \cos\left(\frac{\pi}{5}t\right)\]
    
    We noticed that sampling it with a sampling interval $T = 10$ was not correct. Let's analyse this with our new tools.

    We know that its Fourier transform is: 
    \[X\left(\omega\right) = \pi\left(\delta\left(\omega - \frac{\pi}{5}\right) + \delta\left(\omega + \frac{\pi}{5}\right)\right)\]
    
    This is a bandlimited signal with $\omega_M = \frac{\pi}{5}$. We thus need: 
    \[\omega_s = \frac{2\pi}{T} > 2 \omega_M = \frac{2\pi}{5} \implies T < 5\]
    
    Indeed, $T = 10$ could not work. Let's now see what happens in the frequency domain. We developed that the sample signal in the frequency domain is of the form: 
    \[X_p\left(\omega\right) = \frac{1}{T} \sum_{k=-\infty}^{\infty} X\left(\omega - \omega_s k\right)\]
    
    This looks like:
    \imagehere[0.7]{SamplingAliasingExample.png}

    We see that, indeed, we have some clutter, we don't have just $X\left(\omega\right)$ in $\left]-\omega_M, \omega_M\right[ $ (in fact, the black arrows from $X\left(\omega\right)$ are outside this interval, and all we have is an arrow from $X\left(\omega - \omega_s\right)$ in blue and one from $X\left(\omega + \omega_s\right)$ in green). In fact, applying a low-pass filter, we get: 
    \[X_r\left(\omega\right) = \frac{2\pi}{T} \delta\left(t\right) \implies x_r\left(t\right) = 1 \neq x\left(t\right)\]
    exactly as what we had got before.
\end{parag}

\begin{parag}{Example 2}
    Let us consider the exact same signal, but with a sampling interval $T =4$: 
    \[x\left(t\right) = \cos\left(\frac{\pi}{5}t\right)\]
    
    We indeed have: 
    \[\omega_s = \frac{2\pi}{T} = \frac{\pi}{2} > \frac{2\pi}{5} = 2 \omega_M\]
    
    Now, in the frequency domain, we have:
    \imagehere[0.7]{SamplingNoAliasingExample.png}

    Applying a low-pass filter, we indeed get the Fourier transform $X\left(\omega\right)$, which yields that $x_r\left(t\right) = x\left(t\right)$.    
\end{parag}

\end{document}
