% !TeX program = lualatex
% Using VimTeX, you need to reload the plugin (\lx) after having saved the document in order to use LuaLaTeX (thanks to the line above)

\documentclass[a4paper]{article}

% Expanded on 2023-05-26 at 13:17:45.

\usepackage{../../style}

\title{Signals and systems}
\author{Joachim Favre}
\date{Vendredi 26 mai 2023}

\begin{document}
\maketitle

\lecture{10}{2023-05-26}{A lecture that will evoke some memories}{
\begin{itemize}[left=0pt]
    \item Definition of the $Z$-transform, and of the transfer function of DT signals.
    \item Proof of the link between $Z$-transforms and Fourier transforms.
    \item Definition of the ROC for $Z$-transforms.
    \item Application of $Z$-transforms to difference equations.
\end{itemize}

}

\section{$Z$-transform}
\subsection{General properties}

\parag{Derivation of the transfer function}{
    Let's say that we have an LTI system $\mathcal{H}$ with impulse response $h$. We also let $z_0 = A e^{j \omega_0}$ to be an arbitrary complex number. If we input $x\left[n\right] = z_0^n$ to our system, we get: 
    \[y\left[n\right] = \left(h * x\right)\left[n\right]= \sum_{k=-\infty}^{\infty} z_0^{n-k} h\left[k\right] = z_0^n \sum_{k=-\infty}^{\infty} z_0^{-k} h\left[k\right]\]
    
    If the sum converges, it does not depend on $n$. Thus, we can write: 
    \[y\left[n\right] = H\left(z_0\right) z_0^n = H\left(z_0\right)x\left[n\right]\]
    where $H$ is named the \important{transfer function}, it is the $Z$-transform of $h\left[n\right]$.

    We had already done a very similar argument for Fourier transforms and Laplace transforms.

    \subparag{Fourier}{
        For Fourier transforms, we noticed that: 
        \[\mathcal{H}\left\{e^{j \omega_0 n}\right\} = H\left(e^{j \omega_0}\right)e^{j \omega_0 n}\]
        where $H\left(e^{j \omega}\right)$ is the frequency response.

        Thus, we get that: 
        \[H\left(e^{j\omega_0}\right) = H\left(z = E^{j \omega_0}\right)\]
        
        However, this time, we are allowed to have a constant $A$ in front of the complex exponential; we don't need the circle to be a unit one. 
    }
}

\parag{Definition: $Z$-transform}{
    Let $x\left[n\right]$ be a DT signal. Its $Z$-transform is: 
    \[X\left(z\right) = \sum_{n=-\infty}^{\infty} x\left[n\right] z^{-n}\]
    for every $z$ such that the sum converges.

    \subparag{Pairs}{
        We often write the $\left(x\left[n\right], X\left(z\right)\right)$ pair as: 
        \[x\left[n\right] \ztransformpair X\left(z\right) \text{ and } R\]
        where $R$ is the region of convergence, the set of values where the sum converges.
    }
}

\parag{Link with DTFT}{
    Let us consider the $Z$-transform, writing $z$ in its polar form: 
    \[X\left(z\right) = \sum_{n=-\infty}^{\infty} x\left[n\right] z^{-n} = \sum_{n=\infty}^{\infty} x\left[n\right] \left(\left|z\right| e^{j \arg\left(z\right)}\right)^{-n} = \sum_{n=-\infty}^{\infty} \left(x\left[n\right] \left|z\right|^{-n}\right) e^{-j \arg\left(z\right) n}\]
    
    We notice that the $Z$-transform of $x\left[n\right]$ at $z$ is the Fourier transform of $x\left[n\right] \left|z\right|^{-n}$ at $\omega = \arg\left(z\right)$. In other words, the $Z$-transform generalises the discrete-time Fourier transform.
}

\parag{Inverse $Z$-transform}{
    It is possible to inverse the $Z$-transform using a formula. However, it is rather complicated and we will, as usual, instead mostly use tables.
}


\parag{Example: Region of convergence}{
    Let us consider the two following signals: 
    \begin{functionbypart}{h_c\left[n\right] = a^n u\left[n\right]}
        a^n, & n \geq 0 \\
        0, & n < 0
    \end{functionbypart}

    \begin{functionbypart}{h_a\left[n\right] = -a^n u\left[-n - 1\right]}
        0, & n \geq 0 \\
        -a^n, & n < 0 \\
    \end{functionbypart}

    The first signal is causal, whereas the second is anti-causal, hence their name.

    Let us compute the $Z$-transform of the first signal: 
    \[H_c\left(z\right) = \sum_{k=0}^{\infty} z^{-k}a^k = \sum_{k=0}^{\infty} \left(az^{-1}\right)^k = \frac{1}{1 - az^{-1}}\]
    which converges if and only if $\left|az^{-1}\right| < 1 \iff \left|z\right| > \left|a\right|$.

    Now, let us compute the $Z$-transform of the second signal: 
    \[H_a\left(z\right) = -\sum_{k=-\infty}^{-1} z^{-k} a^k = -\sum_{m=1}^{\infty} z^m a^{-m} = -\sum_{m=1}^{\infty} \left(za^{-1}\right)^m = 1 - \frac{1}{1 - za^{-1}} = \frac{1}{1 - az^{-1}}\]
    which converges if and only if $\left|za^{-1}\right| < 1 \iff \left|z\right| < \left|a\right|$.

    We notice that $H_c = H_a$, which is really strange since we have an inversion formula. However, the difference is made on the regions of convergence.
}

\parag{Definition: ROC}{
    The \important{region of convergence} (ROC) of some $Z$-transform is the set of inputs where its sum converges.
}

\parag{Observation}{
    We notice that we can write the $Z$-transform as: 
    \[X\left(z\right) = \sum_{n=-\infty}^{\infty} x\left[n\right] \left|z\right|^{-n} e^{-n\arg\left(z\right)j}\]
    
    Since a complex exponential is bounded, the convergence of the $Z$-transform only depends on $\left|z\right|$. This implies that the ROCs are disks or annulus (possibly with infinite radius) centred at 0.
}

\parag{Theorem}{
    Let $x\left[n\right]$ be a DT signal and $X\left(z\right)$ its $Z$-transform. 

    $x\left[n\right]$ is stable if and only if its region of convergence contains the unit circle.

    \subparag{Proof}{
        We have seen that the $Z$-transform of $x\left[n\right]$ at $z = A e^{j \omega_0}$ is the DT Fourier transform of $x\left[n\right]\left|z\right|^{-n}$ at $\omega = \arg\left(z\right)$. This means that the Fourier transform of $x\left[n\right]$ exists if and only if $\left|z\right| = 1$, the unit circle, belongs to the ROC.

        Moreover, we know that a signal has a Fourier transform if and only if it is stable. This indeed means that a system is stable if and only if the ROC of its transfer function contains the unit circle.

        \qed
    }
}

\parag{Rational function}{
    In many cases, $H\left(z\right)$ will be a rational function: 
    \[H\left(z\right) = \frac{P\left(z\right)}{Q\left(z\right)}\]
    where $P$ and $Q$ are polynomials in $z$.

    We define the roots of $Q$ to be \important{poles}, and the roots of $P$ to be \important{zeroes}.

    \subparag{Observation}{
        A circle containing a pole can never belong to the ROC. The ROC of rational function must thus either be bounded by poles or extend to infinity.
    }
}

\parag{Definition: Zero-pole plot}{
    Just as for the CT case, we can draw the \important{zero-pole plot} of a rational transfer function $H\left(z\right)$ on the complex plane. Its zeroes are represented by circles and its poles by crosses.
}


\parag{Example}{
    Let us consider the following expression: 
    \[H\left(z\right) = \frac{2}{\left(1 - \frac{1}{2}z^{-1}\right)\left(1 - \frac{3}{2}z^{-1}\right)}\]

    We can write it into the two following forms, the first one being better for inverting it and the second one being better for drawing the zero-pole plot: 
    \[H\left(z\right) = \frac{3}{\left(1 - \frac{1}{2}z^{-1}\right)} - \frac{1}{\left(1 - \frac{3}{2}z^{-1}\right)} = \frac{2z^2}{\left(z - \frac{1}{2}\right)\left(z - \frac{3}{2}\right)}\]

    We notice that we have 3 regions of convergence: 
    \[\left|z\right| < \frac{1}{2}, \mathspace \frac{1}{2} < \left|z\right| < \frac{3}{2}, \mathspace \frac{3}{2} < \left|z\right|\]

    For each ROC, we can find a coherent inverse $Z$-transform. We moreover notice that we have already seen this form of $Z$-transform before defining ROCs, allowing us to easily invert each term.

    In the first ROC, if $\left|z\right| < \frac{1}{2}$, we get: 
    \[h\left[n\right] = \left[\left(\frac{1}{2}\right)^n - 3 \left(\frac{3}{2}\right)^n\right]u\left[-n-1\right]\]
    since we require $\left|z\right| < \frac{1}{2}$ for the first term and $\left|z\right| < \frac{3}{2}$ for the second one.
    
    In the second ROC, if $\frac{1}{2} < \left|z\right| < \frac{3}{2}$, we have: 
    \[h\left[n\right] = -\left(\frac{1}{2}\right)^n u\left[n\right] - 3\left(\frac{3}{2}\right)^n u\left[-n - 1\right]\]
    since we require $\left|z\right| > \frac{1}{2}$ for the first term and $\left|z\right| < \frac{3}{2}$ for the second one.
    
    In the third ROC, if $\frac{3}{2} < \left|z\right|$, we get: 
    \[\left[3\left(\frac{3}{2}\right)^n - \left(\frac{1}{2}\right)^n\right]u\left[n\right]\]
    since we require $\left|z\right| > \frac{1}{2}$ for the first term and $\left|z\right| > \frac{3}{2}$ for the second one.
}

\parag{$Z$-transform pairs}{
    As usual, we can make the following table of $Z$-transform pairs:
    \begin{center}
    \begin{tabular}{|rcll|}
        \hline
        $\displaystyle \delta \left[n\right]$ & $\ztransformpair$ & $\displaystyle 1$ & \\
        $\displaystyle \delta \left[n - m\right]$ & $\ztransformpair$ & $\displaystyle z^{-m}$, & $\displaystyle \begin{cases} z \neq 0, & \text{if } m > 0 \\ z \neq \infty, & \text{if } m < 0 \end{cases}$ \\
        \hline
        $\displaystyle \alpha^n u\left[n\right]$ & $\ztransformpair$ & $\displaystyle \frac{1}{1 - a z^{-1}}$, & $\left|z\right| > \left|\alpha\right|$ \\
        $\displaystyle -\alpha^n u\left[-n - 1\right]$ & $\ztransformpair$ & $\displaystyle \frac{1}{1 - a z^{-1}}$, & $\left|z\right| < \left|\alpha\right|$ \\
        $\displaystyle n\alpha^n u\left[n\right]$ & $\ztransformpair$ & $\displaystyle \frac{\alpha z^{-1}}{\left(1 - a z^{-1}\right)^2}$, & $\left|z\right| > \left|\alpha\right|$ \\
        $\displaystyle -n\alpha^n u\left[-n - 1\right]$ & $\ztransformpair$ & $\displaystyle \frac{\alpha z^{-1}}{\left(1 - a z^{-1}\right)^2}$, & $\left|z\right| < \left|\alpha\right|$ \\
        \hline
        $\displaystyle r^n \cos\left(\omega_0 n\right) u\left[n\right]$ & $\ztransformpair$ & $\displaystyle \frac{1 - r \cos\left(\omega_0\right)z^{-1}}{1 - 2r \cos\left(\omega_0\right)z^{-1} + r^2 z^{-2}}$, & $\left|z\right| > r$ \\
        $\displaystyle r^n \sin\left(\omega_0 n\right) u\left[n\right]$ & $\ztransformpair$ & $\displaystyle \frac{r \sin\left(\omega_0\right)z^{-1}}{1 - 2r \cos\left(\omega_0\right)z^{-1} + r^2 z^{-2}}$, & $\left|z\right| > r$ \\
        \hline
    \end{tabular}
    \end{center}
}

\parag{$Z$-transform properties}{
    We also have the following properties:
    \begin{center}
    \begin{tabular}{|rcll|}
        \hline
        $\displaystyle a x_1\left[n\right] + b x_2\left[n\right]$ & $\ztransformpair$ & $\displaystyle a X_1\left(z\right) + b X_2\left(z\right)$, & $\displaystyle \hat{R} \supseteq R_1 \cap R_2$ \\
        $\displaystyle x\left[n - n_0\right]$ & $\ztransformpair$ & $\displaystyle z^{-n_0}X\left(z\right)$, & $\displaystyle \hat{R} \in \left\{R, R \cup \left\{0\right\}, R \setminus \left\{0\right\}\right\}$ \\
        $\displaystyle z_0^n x\left[n\right]$ & $\ztransformpair$ & $\displaystyle X\left(\frac{z}{z_0}\right)$, & $\displaystyle z \in \hat{R} \iff z z_0 \in R$ \\
        $\displaystyle x_{\left(k\right)}\left[n\right] = \begin{cases} x\left[r\right], & n = rk \\ 0, & n \neq rk \end{cases}$ & $\ztransformpair$ & $\displaystyle X\left(z^k\right)$, & $\displaystyle z \in \hat{R} \iff z^{\frac{1}{k}} \in R$ \\
        \hline
        $\displaystyle \left(x_1 * x_2\right)\left[n\right]$ & $\ztransformpair$ & $\displaystyle X_1\left(z\right)X_2\left(z\right)$, & $\displaystyle \hat{R} \supseteq R_1 \cap R_2$ \\
        \hline
        $\displaystyle n x\left[n\right]$ & $\ztransformpair$ & $\displaystyle -z \frac{dX\left(z\right)}{dz}$, & $\displaystyle \hat{R} = R$ \\
        \hline
        $\displaystyle \sum_{k=-\infty}^{n} x\left[k\right]$ & $\ztransformpair$ & $\displaystyle \frac{1}{1 - z^{-1}} X\left(z\right)$, & $\displaystyle \hat{R} \supseteq R \cap \left\{\left|z\right| > 1\right\}$ \\
        \hline
        $\displaystyle x^*\left[n\right]$ & $\ztransformpair$ & $\displaystyle X^*\left(z^*\right)$, & $\displaystyle \hat{R} = R$ \\
        \hline
    \end{tabular}
    \end{center}
    
    \subparag{Remark}{
        We notice that the properties are very close to the ones of the DT Fourier transform. This makes sense since the $Z$-transform generalises the DT Fourier transform.
    }
}

\parag{Definition: Right-sided signal}{
    Let $x\left[n\right]$ be a DT signal.

    $x\left[n\right]$ is \important{right-sided} if there exists some $N \in \mathbb{N}$ such that, for all $n < N$: 
    \[x\left[n\right] = 0\]
    
    \subparag{Remark}{
        Just as for CT right-sideness, any causal signal is right-sided, but the reciprocal is not true.
    }
    
}

\parag{Definition: Left-sided signal}{
    Let $x\left[n\right]$ be a DT signal.

    $x\left[n\right]$ is \important{left-sided} if there exists some $N \in \mathbb{N}$ such that, for all $n > N$: 
    \[x\left[n\right] = 0\]

    \subparag{Remark}{
        Any anti-causal signal is left-sided, but the reciprocal is not true.
    }
}

\parag{Observation}{
    Any finite duration signal is both left-sided and right-sided.
}
    
\parag{Definition: Two-sided signal}{
    Let $x\left[n\right]$ be a DT signal.
    
    $x\left[n\right]$ is \important{both-sided} if it is neither right-sided nor left-sided.
}

\parag{Theorem}{
    Let $x\left[n\right]$ be some DT signal, which transfer function $X\left(z\right)$ exists.

    If $x\left[n\right]$ is right-sided, then the ROC of $X\left(z\right)$ contains a neighbourhood of $\left|z\right| = \infty$. If moreover $x\left[n\right]$ is causal, then the ROC of $X\left(z\right)$ contains $\left|z\right| = \infty$ (meaning that all the limits exist).

    \subparag{Remark}{
        The reciprocals are false in general. However, they are true for rational functions $X\left(z\right)$.
    }


    \subparag{Intuition}{
        By hypothesis, there exists some $N \in \mathbb{N}$ such that, for all $n < N$: 
        \[x\left[n\right] = 0\]
        
        In particular, this means that we can write its transfer function $X\left(z\right)$ as: 
        \[X\left(z\right) = \sum_{n=N}^{\infty} x\left[n\right] z^{-n}\]

        Let's suppose first that $N < 0$. We have:
        \[X\left(z\right) = \sum_{n=N}^{-1} x\left[n\right] z^{-n} + \sum_{n=0}^{\infty} x\left[n\right] z^{-n}\]

        The first sum is finite, so it converges for all $z \in \mathbb{C}$, where $\left|z\right| \neq \infty$. The second sum gets smaller as $\left|z\right|$ grows bigger; meaning that, if a circle belongs to the ROC, all circles with greater radius are also in the ROC. Putting both together, the ROC must contain a negihbourhood of all $\left|z\right| = \infty$, but not necessarily $\left|z\right| = \infty$.

        However, if $N \geq 0$, then we only have the second sum. This yields that $X\left(z\right)$ exists for any $\left|z\right| = \infty$.
    }
}


\parag{Theorem}{
    Let $x\left[n\right]$ be some DT signal, which transfer function $X\left(z\right)$ exists.

    If $x\left[n\right]$ is left-sided, then the ROC of $X\left(z\right)$ contains a neighbourhood of 0. If moreover $x\left[n\right]$ is anti-causal, then the ROC of $X\left(z\right)$ contains 0.

    \subparag{Remark}{
        The reciprocals are false in general. However, they are true for rational functions $X\left(z\right)$.
    }
}

\parag{Example: Finite duration}{
    Using the properties, we notice that a signal is finite duration if and only if its region of convergence is the whole complex plane, except possibly for $\left|z\right| = 0$ and $\left|z\right| = \infty$.

    For instance, if $x\left[n\right] = \delta \left[n -m \right]$ for some $m \in \mathbb{Z}$: 
    \[X\left(z\right) = z^{-m}\]
    
    Everything it converges for all $z \in \mathbb{C}$ except for $z = 0$ if $m > 0$ and except for $z = \infty$ if $m < 0$. 
}


\subsection{Application to LTI systems}

\parag{LTI systems}{
    Let $\mathcal{H}$ be an LTI system with impulse response $h\left[n\right]$. We know that we can write: 
    \[y\left[n\right] = \mathcal{H}\left\{x\right\}\left[n\right] = \left(x * h\right)\left[n\right]\]
    

    From the convolution property, we know that the $Z$-transform of $\left(x * h\right)\left[n\right]$ is: 
    \[H\left(z\right)X\left(z\right)\]
    
    Thus, we have, as usual: 
    \[Y\left(z\right) = H\left(z\right) X\left(z\right)\]
    where $R_Y \supseteq R_X \cap R_H$.
}

\parag{Property: Stability}{
    Let $\mathcal{H}$ be a LTI system with transfer function $H\left(z\right)$.

    $\mathcal{H}$ is stable if and only if the ROC of $H\left(z\right)$ includes the unit circle.

    \subparag{Proof}{
        This directly comes from our earlier observations.
    }
}


\parag{Property: Causality}{
    Let $\mathcal{H}$ be a LTI system with transfer function $H\left(z\right)$.

    If $\mathcal{H}$ is causal, then the ROC of $H\left(z\right)$ contains $\left|z\right| = \infty$.

    \subparag{Proof}{
        If $\mathcal{H}$ is causal, then its impulse response $h\left[n\right]$ is right-sided and $H\left(z\right)$ must exist for all $\left|z\right| = \infty$.
    }
}

\parag{Property: Rational function causality}{
    Let $\mathcal{H}$ be a LTI system with a \textit{rational} transfer function $H\left(z\right) = \frac{P\left(z\right)}{Q\left(z\right)}$.

    $\mathcal{H}$ is causal if and only if the ROC of $H\left(z\right)$ contains $\left|z\right| = \infty$.
}

\parag{Difference equation}{
    We know the following property of $Z$-transforms: 
    \[x\left[n - n_0\right] \ztransformpair z^{-n_0}X\left(z\right)\]
    
    We can use it to solve difference equations for LTI systems. DTFT required that the system was stable, but $Z$-transforms do not need this hypothesis.
}

\parag{Example}{
    Let's suppose that a LTI system satisfies the following difference equation: 
    \[y\left[n\right] - \frac{1}{2}y\left[n-1\right] = x\left[n\right] + \frac{2}{3}x\left[n-1\right] - \frac{3}{5}x\left[n-2\right]\]
    
    Applying a $Z$-transform on both sides, this yields: 
    \autoeq{Y\left(z\right)\left(1 - \frac{1}{2}z^{-1}\right) = X\left(z\right)\left(1 + \frac{2}{3}z^{-1} - \frac{3}{5} z^{-2}\right) \iff H\left(z\right) = \frac{1 + \frac{2}{3} z^{-1} - \frac{3}{5}z^{-2}}{1 - \frac{1}{2} z^{-1}} = \frac{z^2 + \frac{2}{3}z - \frac{3}{5}}{\left(z - \frac{1}{2}\right)z}}

    We have two different regions of convergence: 
    \[0 < \left|z\right| < \frac{1}{2}, \mathspace \frac{1}{2} < \left|z\right|\]
    
    We notice that $H\left(z\right)$ can be written as: 
    \[H\left(z\right) = \left(1 + \frac{2}{3}z^{-1} - \frac{3}{5}z^{-2}\right) \cdot  \frac{1}{1 - \frac{1}{2}^{-1}}\]
    
    The first factor can be applied as the time-shift property, and the second one is the $Z$-transform of a one-sided exponential. This yields that, if $0 < \left|z\right| < \frac{1}{2}$: 
    \[h\left[n\right] = -\left(\frac{1}{2}\right)^n u\left[-n-1\right] - \frac{2}{3} \left(\frac{1}{2}\right)^{n-1} u\left[-n\right] + \frac{3}{5} \left(\frac{1}{2}\right)^{n-2} u\left[-n + 1\right]\]

    We can deduce that this $h\left[n\right]$ is a left-sided signal which is neither causal, anticausal nor stable from the ROC.

    If however $\frac{1}{2} < \left|z\right|$, we have:
    \[h\left[n\right] = \left(\frac{1}{2}\right)^n u\left[n\right] + \frac{2}{3} \left(\frac{1}{2}\right)^{n-1} u\left[n-1\right] - \frac{3}{5}\left(\frac{1}{2}\right)^{n-2} u\left[n-2\right]\]
    which we can deduce to be a causal and stable signal from the ROC.
}

\end{document}
