% !TeX program = lualatex
% Using VimTeX, you need to reload the plugin (\lx) after having saved the document in order to use LuaLaTeX (thanks to the line above)

\documentclass[a4paper]{article}

% Expanded on 2023-02-27 at 15:13:38.

\usepackage{../../style}

\title{Theory of computation}
\author{Joachim Favre}
\date{Lundi 27 février 2023}

\begin{document}
\maketitle

\lecture{2}{2023-02-27}{Killing cats}{
\begin{itemize}[left=0pt]
    \item Definition of regular languages.
    \item Proof that the complement of a regular language, and the union and intersection of any two regular languages, are also regular, and explanation of the DFA constructions.
    \item Definition of NFAs.
    \item Proof that any NFA is equivalent to a DFA, and explanation of the construction.
    \item Proof that the concatenation of any two regular languages is also regular, and explanation of its NFA construction.
\end{itemize}
}

\parag{Definition: Regular language}{
    $A \subseteq \Sigma^*$ is named a \important{regular language} if there exists at least a DFA $M$ such that $A = L\left(M\right)$ (meaning that it is the language of some DFA).

    \subparag{Example}{
        For instance, the following language is a regular language, as we have already seen a DFA for it: 
        \[L = \left\{w \in \left\{0, 1\right\}^* \suchthat w \text{ contains an even number of 1's}\right\}\]

        Formally, we would need to prove the correctness of the DFA.
    }
}

\parag{Definition: Complement}{
    The \important{complement} of some language $L$ is defined as: 
    \[\bar{L} = \left\{w \in \Sigma^* \suchthat w \not\in L\right\}\]
}

\parag{Theorem: Complement}{
    Let $L$ be some regular language and $M = \left(Q, \Sigma, \delta, q_0, F\right)$ be a machine recognising it.

    Then, the complement of $L$, $\bar{L}$, is regular and is the language of the following machine: 
    \[M' = \left(Q, \Sigma, \delta, q_0, \bar{F} = Q \setminus F\right)\]
    
    \subparag{Remark}{
        We just swap the accepting state and the non-accepting state. This naturally means that strings that were accepted are no longer, and inversely.
    }
}

\parag{Definition: Union}{
    The \important{union} of two languages $L_1$ and $L_2$ is: 
    \[L_1 \cup L_2 = \left\{w \in \Sigma^* \suchthat w \in L_1 \text{ or } w \in L_2\right\}\]
}

\parag{Theorem: Union}{
    Let $L_1$ and $L_2$ be the languages of machines $M_1 = \left(Q_1, \Sigma, \delta_1, q_1, F_1\right)$ and $M_2 = \left(Q_2, \Sigma, \delta_2, q_2, F_2\right)$, respectively.

    Then, the union of $L_1$ and $L_2$, $L_1 \cup L_2$ is regular, and it is accepted by the machine $M = \left(Q, \Sigma, \delta, q_0, F\right)$, where: 
    \[Q = Q_1 \times Q_2, \mathspace \delta\left(\left(r_1, r_2\right), a\right) = \left(\delta_1\left(r_1, a\right), \delta_2\left(r_2, a\right)\right)\]
    \[q_0 = \left(q_1, q_2\right), \mathspace F = \left\{\left(r_1, r_2\right) \suchthat r_1 \in F_1 \text{ or } r_2 \in F_2\right\}\]

    \subparag{Intuition}{
        We are basically simulating the two machines at the same time, keeping track of the two states as a tuple. We then consider that we end up in an accepting state if either machine ended up in one.
    }

    \subparag{Example}{
        Let us consider the following two machines:
        \imagehere[0.8]{DFAExample-UnionNotYetUnionised.png}

        Then, their union is:
        \imagehere[0.6]{DFAExample-Union.png}
    }
}

\parag{Definition: Intersection}{
    The \important{intersection} of two languages $L_1$ and $L_2$ is: 
    \[L_1 \cap L_2 = \left\{w \in \Sigma^* \suchthat w \in L_1 \text{ and } w \in L_2\right\}\]
}

\parag{Theorem: Intersection}{
    Let $L_1$ and $L_2$ be the languages of machines $M_1 = \left(Q_1, \Sigma, \delta_1, q_1, F_1\right)$ and $M_2 = \left(Q_2, \Sigma, \delta_2, q_2, F_2\right)$, respectively.

    Then, the intersection of $L_1$ and $L_2$, $L_1 \cap L_2$ is regular, and it is accepted by the machine $M = \left(Q, \Sigma, \delta, q_0, F\right)$, where: 
    \[Q = Q_1 \times Q_2, \mathspace \delta\left(\left(r_1, r_2\right), a\right) = \left(\delta_1\left(r_1, a\right), \delta_2\left(r_2, a\right)\right) \]
    \[q_0 = \left(q_1, q_2\right), \mathspace F = \left\{\left(r_1, r_2\right) \suchthat r_1 \in F_1 \text{ \textcolor{red}{and} } r_2 \in F_2\right\}\]

    \subparag{Intuition}{
        This is the \textit{exact} same construction as for intersection, except that we consider the accepting states to be the ones where both DFA ended up in one.
    }
}

\parag{Definition: Concatenation}{
    The \important{concatenation} of two languages $L_1$ and $L_2$ is: 
    \[L_1 \circ L_2 = \left\{w \in\Sigma^* \suchthat w = w_1 . w_2, w_1 \in L_1 \text{ and } w_2 \in L_2\right\}\]
}

\parag{Observation}{
    It is very hard to know if concatenating two regular languages always yield a regular language. The main problem is that we do not know when to switch from the first machine to the second one: a string can be obtained through concatenation in many ways ($101 = 1 \circ 01 = 10 \circ 1 = 101 \circ \epsilon$).

    We will see that the concatenation of two languages is indeed regular but, for this, we need to do some non-determinisim.
}

\subsection{Non-deterministic finite automaton}
\parag{Definition: Power set}{
    Let $Q$ be some set. We note $2^Q$ the power set of $Q$, meaning the set of all the subsets of $Q$.

    \subparag{Example}{
        For instance, let us consider $Q = \left\{q_1, q_2\right\}$. Then: 
        \[2^Q = \left\{\o, \left\{q_1\right\}, \left\{q_2\right\}, \left\{q_1, q_2\right\}\right\}\]
    }
}

\parag{Definition: NFA}{
    A \important{non-deterministic finite automaton} (NFA) $M$ is a 5-tuple $\left(Q, \Sigma, \delta, q_0, F\right)$, where:
    \begin{itemize}
        \item $Q$ is a finite set called the states.
        \item $\Sigma$ is a finite set called the alphabet.
        \item $\delta: Q \times \left(\Sigma \cup \left\{\epsilon\right\}\right) \mapsto 2^Q$ is the transition function.
        \item $q_0 \in Q$ is the start state.
        \item $F \subseteq Q$ is the set of accept states, where we allow $F = \o$.
    \end{itemize}

    A NFA has multiple ways to compute the final state for some string. It accepts the word if there exists a set of choices, that make the word to be accepted. In other words, if we run a parallel computer that follows all possibilities in parallel, we consider the word to be accepted if any of them are in an accepting state at the end.
    
    \subparag{Comparison with DFA}{
        A NFA has two main difference to DFA. 

        First, it has the ability to transition to more than one state on a given symbol, or no symbol at all: the codomain of $\delta$ allows any subset of $Q$, meaning that it could map to $\o$ or $\left\{q_1, q_2\right\}$. Note that if at any point we reach a state where we cannot transition out (because $\delta\left(\sigma, q_i\right) = \o$), then the path ends and does not continue further. This path is not considered accepted even though it died on an accepting state.

        Second, it has the possibility to take a step in-between reading symbols, thanks to $\epsilon$ transition. In other words, if $\delta\left(\epsilon, q_i\right) = \left\{q_j\right\}$, then the machine can decide to transition form $q_i$ to $q_j$ even without reading any character.
    }
}

\parag{Example 1}{
    Let's consider the state diagram of the following NFA:
    \imagehere[0.6]{NFA-BasicExample.png}

    As we can see, there are indeed some states which can transition to more than one state on a given symbol, some which have no transition, and some which have an $\epsilon$-transition.

    A good way to see the computation of an NFA is a tree where any level represents all the possible' states reachable at that point. In this case, if we read input $010110$, then we get:
    \imagehere[0.5]{NFA-BasicExample-Tree.png}

    Note that, as mentioned before, a thread which cannot take transition is \textit{killed} (the fact that the professor represented the different parallel threads as cats makes this really violent). Also, we can see that the $\epsilon$ transitions can be taken in-between reading symbols, this was mentioned above as well.

    \imagehere[0.05]{cat.png}

    Since, in this example, we have some threads which end up in an accepting state at the end of the string, then it is accepted.

    \subparag{Observation}{
        Note that, to show an input is accepted, we don't need to draw the whole tree, only the branch going to an accepting state. This is named guess and verify.
    }
    
    \subparag{Remark}{
        In fact, we can convince ourselves that this NFA recognises the following language: 
        \[L = \left\{w \in \left\{0, 1\right\}^* \suchthat w \text{ contains 11 or 101 as a substring}\right\}\]
    }

    \subparag{Formal definition}{
        The formal definition of our machine is $M = \left(Q, \Sigma, \delta, q_1, F\right)$, where: 
        \[Q = \left\{q_1, q_2, q_3, q_4\right\}, \mathspace \Sigma = \left\{0, 1\right\}, \mathspace F = \left\{q_4\right\}\]
        
        Also, the function $\delta$ is defined as:
        \begin{center}
        \begin{tabular}{c|ccc}
            $\delta$ & 0 & 1 & $\epsilon$ \\
            \hline
            $q_1$ & $\left\{q_1\right\}$ & $\left\{q_1, q_2\right\}$ & $\o$ \\
            $q_2$ & $q_3$ & $\o$ & $\left\{q_3\right\}$ \\
            $q_3$ & $\o$ & $\left\{q_4\right\}$ & $\o$ \\
            $q_4$ & $\left\{q_4\right\}$ & $\left\{q_4\right\}$ & $\o$ \\
        \end{tabular}
        \end{center}

        Just as for DFA, this formal definition and the state diagram above are completely equivalent.
    }
    
}

\parag{Example 2}{
    Let's consider the following NFA:
    \imagehere[0.6]{NFA-Example2.png}

    We notice that, for instance, ``1000'' is not accepted since the only thread which reaches $q_4$ dies before the end of the string.

    We can convince ourselves that it recognises the following language: 
    \[L = \left\{w \in \left\{0, 1\right\}^* \suchthat w \text{ contains a 1 in the third position from the end}\right\}\]
}

\parag{Example 3}{
    Let us now consider an NFA over a unary language:
    \imagehere[0.5]{NFA-example3.png}

    The beginning with the two $\epsilon$-transitions, the only non-deterministic part of the machine, allows to have a non-deterministic choice of starting state. 

    This machine is the union of two regular languages: $w \in \left\{0\right\}^*$ has a length which is a multiple of 2, and $w \in \left\{0\right\}^*$ has a length which is a multiple of 3. This construction is more efficient in size, and it gives: 
    \[L = \left\{w \in \left\{0\right\}^* \suchthat w \text{ has a length which is multiple of 2 or 3}\right\}\]
}

\parag{Remark}{
    An NFA may be much smaller than its deterministic counterpart, or its functioning may be easier to understand. In fact, they seem really more powerful than DFA, but this is a wrong intuition as we will see with the following theorem, for which we will need some lemma beforehand.
}

\parag{Theorem}{
    Every non-deterministic finite automaton has an equivalent deterministic finite automaton.

    \subparag{Proof}{
        We will only prove the case without $\epsilon$-transitions since it slightly complicates the initial states. However, the proof with them is very similar, and we will show in the second series of exercises that any NFA with $\epsilon$-transitions can be translated to any NFA without such transitions.

        Let $N = \left(Q_N, \Sigma, \delta_N, q_0, F_n\right)$ be an arbitrary NFA. We want to construct a DFA $M$ such that: 
        \[L\left(N\right) = L\left(M\right)\]
        
        The idea is to consider that a state of the DFA represents the set of states from the NFA which could be reached by one of the threads. In other words, the transition function maps a set of states into another set of all the possible states the threads could reach in one step. Mathematically, we let $M = \left(2^Q, \Sigma, \widetilde{\delta}, \left\{q_0\right\}, \widetilde{F}\right)$, where: 
        \[\widetilde{\delta}\left(A, a\right) = \bigcup_{q \in A} \delta\left(q, a\right)\]


        Also, we let $\widetilde{F}$ to contain all the sets of states which contain an accepting state: 
        \[\widetilde{F} = \left\{A \in 2^Q \suchthat A \cap F \neq \o\right\}\]
        
        We now want to show that $M$'s state after $x \in \Sigma^*$ is $\widetilde{\delta}\left(\left\{q_0\right\}, x\right) = A$ if and only if $A$ is $N$'s possible states after reading $x$. We will prove this by induction.

        Let's begin with the base case, taking $x = \epsilon$. We see that $\widetilde{\delta}\left(\left\{q_0\right\}, \epsilon\right) = \left\{q_0\right\}$ by definition of the starting state, which is indeed $N$'s possible states after $\epsilon$. Note that this is where appears our hypothesis that our machine does not have an $\epsilon$-transition, since, else, we may not have the equality hereinabove and it thus complicates the proof. 

        Let us now consider the inductive step. As usual, we assume that the claim holds for $x$, and we want to prove it for $x.a$. We see that: 
        \[\widetilde{\delta}\left(\left\{q_0\right\}, x.a\right) = \widetilde{\delta}\left(\widetilde{\delta}\left(\left\{q_0\right\}, x\right), a\right) \over{=}{\text{déf}} \bigcup_{q \in \widetilde{\delta}\left(\left\{q_0\right\}, x\right)} \delta\left(q, a\right) \]
        where $\widetilde{\delta}\left(\left\{q_0\right\}, x\right)$ is $N$'s possible states after reading all the characters of $x$, as usual. However, we are taking the union of all the states we can reach from those those possible states, keeping our inductive property.
        
        In other words, since we indeed had all the possible states of $N$ for $x$, then we still have all the possible states of $N$ after $x.a$, as required.

        \qed
    }
  
    \subparag{Example}{
        Let's consider this construction for the following NFA:
        \imagehere[0.7]{NFAToDFA-OriginalNFA.png}

        We can construct a new DFA which considers all the $8$ subsets of $\left\{p, q, r\right\}$ as states, and we consider an accepting state to be any set containing $r$:
        \imagehere{NFAToDFA-ResultingDFA.png}

        For instance, if we are in any of the state $q$ or $r$ and we read a 1, we can go to any of $p$, $q$ and $r$. This indeed means that $\delta\left(\left\{p, q\right\}, 1\right) = \left\{p, q, r\right\}$.

        We can see that the this DFA can never reach some of its states, such as $\o$.
    }
}

\parag{Corollary}{
    A language is regular if and only if some NFA recognises it.
}


\parag{Theorem: Concatenation}{
    Let $L_1$ and $L_2$ be regular languages.

    Then, the concatenation of $L_1$ and $L_2$, $L_1 \circ L_2 = L$, is regular.

    \subparag{Proof}{
        Concatenation is very easy with non-determinism. Indeed, we just need to link the accepting states of the first machine to the initial state of the second machine using $\epsilon$-transition. In other words, we allow to switch to the second machine as soon as we recognise a word, but also to stay on the first one.
        \imagehere{NFAConcatenation.png}

        However, by the corollary above, since we have a NFA which language is the concatenation of $L_1$ and $L_2$ is indeed regular.

        \qed
    }
}


\end{document}
