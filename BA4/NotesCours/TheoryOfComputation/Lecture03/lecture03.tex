% !TeX program = lualatex
% Using VimTeX, you need to reload the plugin (\lx) after having saved the document in order to use LuaLaTeX (thanks to the line above)

\documentclass[a4paper]{article}

% Expanded on 2023-03-06 at 15:15:22.

\usepackage{../../style}

\title{Theory of computation}
\author{Joachim Favre}
\date{Lundi 06 mars 2023}

\begin{document}
\maketitle

\lecture{3}{2023-03-06}{''Pigeon collisions''}{
\begin{itemize}[left=0pt]
    \item Explanation and proof of the pumping lemma.
    \item Examples of application of the pumping lemma to show that some languages are not regular.
\end{itemize}
}

\subsection{Non-regular languages}

\begin{parag}{Existence of non-regular languages}
    We have seen that many languages are regular: using a finite automaton, we can verify a number has a given substring, check if it is divisible modulo a given number, skip some prefix and suffix, do complements, unions, intersections, and so on.
    
    We now wonder if all languages are regular. In fact, sadly, no. Let us consider the following language: 
    \[B = \left\{0^n 1^n \suchthat n \geq 0\right\} = \left\{\epsilon, 01, 0011, 000111, \ldots\right\}\]
    
    The problem is that we need some kind of memory depending on the size of the input: we need to recall how many zeroes we had when we are counting the ones. 

    \begin{subparag}{Proof}
        Let us prove that $B$ is indeed not regular.

        Let's suppose for contradiction that $B$ is recognised by a DFA with $p$ states. Note that $p$ has to be finite.

        Now, let us consider the countable infinite set $\left\{0^1, 0^2, \ldots\right\}$. By the pigeon hole principle, since there is a finite number of states $p$ but an infinite number of $0^k$, there must exist $i, j$ such that $0^i$ and $0^j$ end up in the same state. We have a pigeon collision.
        \imagehere[0.4]{PigeonCollision.png}

        However, this means that if $0^i 1^i$ is accepted, then $0^j 1^i$ is also accepted. However, $0^j 1^i$ should not be accepted, which is our contradiction.

        \qed
    \end{subparag}
\end{parag}

\begin{parag}{Remark}
    Note that, to show a language is not regular, we cannot just say ``it needs memory depending on the size of the input'' blindly. Let us consider the following language: 
    \[D = \left\{w \suchthat w \text{ has an equal number of occurrences of 01 and 10 as substrings}\right\}\]
    
    In fact, we can show that it is equivalent to: 
    \[D = \left\{w \suchthat w \text{ begins and ends with the same character, or } \text{length}\left(w\right) \leq 1\right\}\]

    Thus, $D$ is regular.
\end{parag}

\begin{parag}{Pumping lemma}
    If $A$ is a regular language, then there is a number $p$ (named the pumping length) such that, for every string $s$ in $A$ of length at least $p$, there exists a division of $s$ into three pieces, $s = xyz$, such that:
    \begin{enumerate}
        \item For any $i \geq 0$, $x y^i z \in A$.
        \item $\left|y\right| \geq 1$.
        \item $\left|xy\right| \leq p$.
    \end{enumerate}
    
    \begin{subparag}{Intuition}
        It is much better to remember the proof of this lemma, instead of the lemma itself.

        In fact, it just generalises the argument we had before. Let $p = \left|Q\right|$. By the pigeon hole principle, if we have a string of length at least $p$, we need to have a loop: let $y$ be a string which loops from a state $q_i$ to the same state $q_i$. Then, we also need $x$ to be a string bringing from the initial state to $q_i$, and $z$ to be a string bringing from $q_i$ to an accepting state. We can then go from the initial step to $q_i$, loop any number of times we want (even 0 times) and go to an accepting state.

        \imagehere[0.5]{PumpingLemmaIntuition.png}

        $\left|xy\right| \leq p$ just means that the state collision (which leads to the loop) has to take place before $p$.
    \end{subparag}

    \begin{subparag}{Proof}
        Let M be an arbitrary DFA accepting $A$, and let $p = \left|Q\right|$ be the number of states in $M$. Also, let $s$ be any arbitrary string of length $\left|s\right| \geq p$.

        By the pigeonhole principle, since we have $p$ states and we visit $p+1$ states when reading a string of length $p$, this means that we must always have some collision for any string which length is less than or equal to $p$. Thus, when reading the $p$ first characters of $s$, there must be a loop: there must be some $x, y, z$ such that $\left|xy\right| \leq p$ (the loop must happen in the first $p$ characters of $s$), $\left|y\right| \geq 1$ (because of the loop) and:
        \[\delta\left(q_0, x\right) = q, \mathspace \delta\left(q_0, xy\right) = q, \mathspace \delta\left(q, z\right) = r \in F\]
        
        In other words, $x$ takes us to some state $q$; then, $y$ loops us back to this same state; and finally $z$ brings us to an accepted state. 

        However, this definitely means that: 
        \[\delta\left(q_0, xy^i z\right) = r, \mathspace \forall i \geq 0\]

        \qed
    \end{subparag}
\end{parag}

\begin{parag}{Example 1}
    Let us consider the following language: 
    \[F = \left\{ww \suchthat w \in \left\{0, 1\right\}^*\right\}\]
    
    In other words, we only want words which left part is equal to its right part.

    Let's suppose for contradiction that this language is regular. By the pumping lemma, it must have some pumping length $p$. Now, any long string can be pumped. Let's pick $s = 0^p 1^p 0^p 1^p \in F$ (note that the choice of which string to pick is the whole game when doing such proofs).

    By the pumping lemma, we know that there exists some $x, y, z$ such that:
    \[s = xyz, \mathspace \left|xy\right| \leq p, \mathspace \left|y\right| \geq 1, \mathspace xy^i z \in F,\ \forall i \geq 0\]

    However, since $\left|xy\right| \leq p$, this necessarily means that $xy = 0^j$ for some $j$ by construction. Moreover, since $\left|y\right| \geq 1$, we get that $y = 0^k$ for some $k \geq 1$.

    According to the pumping lemma, we must have $x y^2 z \in F$. However: 
    \[xy^2 z = 0^{p+k} 1^p 0^p 1^p \not \in F\]
    since the left part of this new string does not match the right one.
    
    This is our contradiction.
\end{parag}

\begin{parag}{Example 2}
    Let us consider the following language: 
    \[E = \left\{0^m 1^n \suchthat m > n\right\}\]
 
    We want to show that $E$ is not regular. Let us suppose for contradiction that $E$ is regular, with $p$ as pumping length.

    Let us consider the following string: 
    \[s = 0^{p+1} 1^{p} \in E\]
    
    Since we know that $\left|xy\right| \leq p$ and $\left|y\right| \geq 1$, we necessarily have that $y = 0^k$ for some $k \geq 1$. However, we see that: 
    \[x y^0 z = xz = 0^{p+1-k} 1^p \not \in E\]
    since $p + 1 - k \leq p$. By the pumping lemma, we should have $xz \in E$, this is our contradiction.
\end{parag}

\begin{parag}{Example 3}
    Let us consider the following language: 
    \[C = \left\{w \suchthat w \text{ has an equal number of 0s and 1s}\right\}\]
    
    We again want to show that it is not regular. Thus, let's suppose for contradiction that $C$ is regular, and that $p$ is its pumping length.

    Let us consider the following string:
    \[s = 0^p 1^p \in C\]

    Since we know that $\left|xy\right| \leq p$ and $\left|y\right| \geq 1$, we necessarily have that $y = 0^k$ for some $k \geq 1$. Moreover: 
    \[x y^2 z = 0^{p+k} 1^p \not \in C\]
    
    However, since the pumping lemma tells us that $x y^2 z \in C$, we got our contradiction.

    \begin{subparag}{Other proof}
        We can use an easier approach than the pumping lemma. 

        Let us consider again a language we have proven not to be regular at the start of this lecture: 
        \[B = \left\{0^n 1^n \suchthat n \geq 0\right\} = \left\{\epsilon, 01, 0011, 000111, \ldots\right\}\]
        
        However, we can see that: 
        \[B = C \cap \left\{0^m 1^n \suchthat m, n \geq 0\right\}\]
        
        Let us now suppose for contradiction that $C$ is regular. However, we know that $\left\{0^m 1^n \suchthat m, n \geq 0\right\}$ is a regular language, and the intersection of two regular language is also regular. This thus means that $B$ is regular, which is a contradiction.

        This shows that the pumping lemma, though very powerful, is not the always the only way to go.
    \end{subparag}
\end{parag}




\end{document}
