% !TeX program = lualatex
% Using VimTeX, you need to reload the plugin (\lx) after having saved the document in order to use LuaLaTeX (thanks to the line above)

\documentclass[a4paper]{article}

% Expanded on 2023-04-24 at 16:18:36.

\usepackage{../../style}

\title{Theory of computation}
\author{Joachim Favre}
\date{Lundi 24 avril 2023}

\begin{document}
\maketitle

\lecture{8}{2023-04-24}{Turning satisfying assignments to graphs}{
\begin{itemize}[left=0pt]
    \item Proof that SAT, GI and INDSET are in NP.
    \item Explanation of why NP is called nondeterministic P.
    \item Definition of polynomial-time reductions, and proof of some of their properties.
    \item Definition of NP-hard and NP-complete problems.
    \item Proof that INDSET is NP-complete.
\end{itemize}

}

\begin{parag}{Proposition: SAT}
    SAT is in NP.

    \begin{subparag}{Proof}
        We construct a verifier for SAT: given some input $\left\langle \phi, C \right\rangle$, we interpret $C$ as a truth assignment to the variables of $\phi$, and check if our sentence indeed reduces to true. We can thus indeed write our language as: 
        \[\text{SAT} = \left\{\left\langle \phi \right\rangle \suchthat \exists C \text{ such that the above verifier accepts $\left\langle \phi, C \right\rangle$}\right\}\]
    \end{subparag}
\end{parag}

\begin{parag}{Proposition: GI}
    GI is in NP.
    
    \begin{subparag}{Proof}
        We construct a verifier for GI: given some input $\left\langle G_1, G_2, C \right\rangle$, we interpret $C$ as function $V\left(G_1\right) \mapsto V\left(G_2\right)$ and check that it is both a bijection and preserves adjacency. This is a poly-time verifier, indeed showing this is in NP.
    \end{subparag}
\end{parag}

\begin{parag}{Proposition: INDSET}
    INDSET is in NP.

    \begin{subparag}{Proof}
        We construct a verifier for the language INDSET: on input $\left\langle G, k, C \right\rangle$, we interpret $C$ as a subset $C \subseteq V\left(G\right)$, check that $\left|C\right| = k$ and check that, for all $u, v \in C$, we have $\left\{u, v\right\} \not \in E\left(G\right)$.

        Since this verifier runs in polynomial time, we have indeed shown INDSET is in NP.
    \end{subparag}
\end{parag}

\begin{parag}{Definition: Non-deterministic Turing Machines}
    In a regular Turing machine, the transition function is $\delta : Q \times \Gamma \mapsto Q \times \Gamma \times \left\{L, R\right\}$. In a Nondeterministic Turing Machine (NTM), we instead let it to be: 
    \[\delta: Q \times \Gamma \mapsto \mathcal{P}\left(Q \times \Gamma \times \left\{L, R\right\}\right)\]
    
    We thus allow several transitions for any given state and tape symbol, just as for NFAs.

    \begin{subparag}{Observation}
        Those are not really computable efficiently, we would require an exponential amount of parallel threads.
    \end{subparag}

    \begin{subparag}{Multithreading}
        Note that NTMs are weaker than multithreaded computers: threads cannot communicate between each other. This is more some kind of ``quantum parallel worlds''.
    \end{subparag}
\end{parag}

\begin{parag}{Definition: Nondeterministic decider}
    A \important{nondeterministic decider} for language $L$ is an NTM $N$ such that for each $x \in \Sigma^*$, every computation of $N$ on $x$ halts. Moreover, if $x \in L$, then some threads of $N$ on $x$ accepts and, if $x \not \in L$, then every computation of $N$ on $x$ rejects.
\end{parag}

\begin{parag}{Definition: Polynomial time NTM}
    An NTM is a \important{polynomial time NTM} if the running time of its longest computation on $x$.
\end{parag}

\begin{parag}{Theorem}
    Let $L \subseteq \Sigma^*$ be an arbitrary language.

    $L$ has a nondeterministic poly-time decider if and only if $L$ has a poly-time verifier.

    \begin{subparag}{Proof $\implies$}
        We know by hypothesis that there exists some nondeterministic polynomial-time decider $N$ for $L$

        We construct a verifier $V$ for $L$. It interprets the certificate as a set of choices for which thread to choose. In other words, on input $\left\langle x, C \right\rangle$, it simulates $N\left(x\right)$ with nondeterministic choices given by $C$. We know that there exists a path in our decision tree which has a polynomial length by definition of polynomial time NTM. Thus, we know this path can be encoded by a certificate with polynomial length.
    \end{subparag}
    
    \begin{subparag}{Proof $\impliedby$}
        We know by hypothesis that there exists a verifier $V$ for $L$.

        We construct a nondeterministic poly-timer decider. Using non-determinism, we can try all certificates in parallel. In other words, at each step, we split into three threads: one which appends a 0 to the current certificate, one that appends a 1, and one that uses the current certificate (which runs $V$ on $\left\langle x, C \right\rangle$). If there exists a certificate which works, we will find it in polynomial time. This gives us an algorithm which runs in polynomial time for each thread, which is thus a polynomial-time NTM.
        
        \qed
    \end{subparag}
\end{parag}

\begin{parag}{Definition: NTIME complexity class}
    We define the NTIME complexity class as: 
    \[\text{NTIME}\left(t\left(n\right)\right) = \left\{L \suchthat L \text{ has a nondeterministic $O\left(t\left(n\right)\right)$ time decider}\right\}\]
\end{parag}


\begin{parag}{Equivalent definition: NP Class}
    \important{NP} can be equivalently defined as: 
    \[\text{NP} = \bigcup_{k=1}^{\infty} \text{NTIME}\left(n^k\right)\]

    \begin{subparag}{Remark}
        This explains why NP is called nondeterministic P: a language in NP has a nondeterministic polynomial-time decider.
    \end{subparag}
    
\end{parag}

\subsection{Polynomial-time reductions}
\begin{parag}{Definition: Poly-time computable function}
    A function $f: \Sigma^* \mapsto \Sigma^*$ is a \important{poly-time computable function} if there exists some poly-time TM $M$ such that, on every input $w$, it halts with just $f\left(w\right)$ on its tape.
\end{parag}

\begin{parag}{Definition: Poly-time reduction}
    A language $A$ is \important{poly-time mapping reducible} to language $B$, written $A \leq_P B$, if there exists a poly-time computable function $f: \Sigma^* \mapsto \Sigma^*$ such that, for every $w \in \Sigma^*$: 
    \[w \in A \iff f\left(w\right) \in B\]
\end{parag}

\begin{parag}{Theorem}
    If $A \leq_P B$ and $B$ is in P, then $A$ is in P.

    \begin{subparag}{Proof}
        We assume by hypothesis that $M$ is an $O\left(n^p\right)$-time decider for $B$, and $f$ is an $O\left(n^q\right)$-time reduction from $A$ to $B$.

        We can then construct a TM $N$ as:
        \imagehere[0.7]{ReductionClassP.png}

        In other words, on input $w$, it computes $y = f\left(w\right)$, runs $M$ on $y$ and outputs whatever $M$ outputs. $f$ outputs something which is at most its running time, so $\left|y\right| = O\left(\left|w\right|^q\right)$. Thus, our machine runs in $O\left(\left|w\right|^{pq}\right)$.

        We have indeed construct a poly-time TM for our problem $A$, showing it is in P.

        \qed
    \end{subparag}

    \begin{subparag}{Intuition}
        This is another instance of the intuition behind reductions: if $A$ reduces to $B$ and $B$ is easy, then $A$ is easy.
    \end{subparag}
    
\end{parag}

\begin{parag}{Corollary}
    If $A \leq_P B$ and $A$ is not in P, then $B$ is not in P.
\end{parag}

\begin{parag}{Theorem: Tranistivity}
    If $A \leq_P B$ and $B \leq_P C$, then $A \leq_P C$.

    \begin{subparag}{Proof}
        The proof is completely analogous to the one we have just done: we can show that running $f_{AB}$ and then $f_{BC}$ runs in $O\left(\left|w\right|^{pq}\right)$.
    \end{subparag}
\end{parag}

\begin{parag}{Definition: NP-hardness}
    A language $L$ is said to be \important{NP-hard} if every language $L'$ in NP is such that: 
    \[L' \leq_P L\]
\end{parag}


\begin{parag}{Definition: NP-completeness}
    A language $L$ is said to be \important{NP-complete} if $L$ is in NP and $L$ is NP-hard.
    
    \begin{subparag}{Observation}
        If any of the NP-complete language has a polynomial time decider, then every language in NP have a polynomial time decider and thus $P = NP$. Thus, NP-complete problems are the hardest NP problems. In other words, if we show that a problem is NP-complete, we showed that, even if we are not able to find an efficient algorithm, neither 40 years of research could.
    \end{subparag}
\end{parag}

\begin{parag}{Cook-Levin Theorem}
    SAT is NP-complete.
\end{parag}

\begin{parag}{Remark}
    In practice, from now on, to show that a language is NP-complete we will always follow the same recipe. 

    We first need to show that $L$ belongs to NP, and thus we need to give a poly-time verifier for $L$. Then, for NP-hardness, we only need to show $H \leq_P L$ for some NP-complete language $H$. Indeed, since we know that $\forall L' \in NP$ we have $L' \leq _P H$ and we show $H \leq_P L$, we indeed get that $L' \leq_P L$.
\end{parag}

\begin{parag}{Proposition: INDSET}
    We can make the following reduction: 
    \[\text{SAT} \leq_P \text{INDSET}\]
    
    Since INDSET is in NP, this implies that INDSET is NP-complete.

    \begin{subparag}{Proof}
        We are given as input a CNF formula $\phi$, and we want to turn it into a graph for INDSET in polynomial time.

        The idea is to construct a vertex per literal per clause, grouping the literals within a clause. We then link with an edge every vertex within a group (in brown) and every conflicting literals (so $x$ and $\bar{x}$; in blue). For instance, we could have:
        \imagehere{ReductionSatIndset.png}

        Our reduction outputs $f\left(\phi\right) = \left(G, m\right)$, where $m$ is the number of clauses ($m = 4$ in our example, since we have $K_1, \ldots, K_4$). This can definitely be done in polynomial time, so we now want to show that:
        \[\phi \in \text{SAT} \iff f\left(\phi\right) \in \text{INDSET}\]

        Let's suppose that $\phi \in \text{SAT}$, and thus that there exists some satisfying assignment $C$ of $\phi$. In $C$, there is at least a true literal for each clause. For each group, we can take one of those literals, and construct an independent set of size $m$. Note that, because of the construction of $f\left(\phi\right)$, two vertices in this independent set cannot be linked by an edge. Indeed, we took a single variable per clause (therefore there cannot be any brown edge), and we cannot have both $x_i$ and $\bar{x}_i$ true (therefore there cannot be a blue edge either). We have thus indeed shown that $\phi \in \text{SAT} \implies f\left(\phi\right) \in \text{INDSET}$

        Let's suppose that $f\left(\phi\right) \in \text{INDSET}$, meaning that there exists an independent set $C$ in $G$ of size $\left|C\right| = m$. Again, by construction, $C$ contains a vertex from each group. We can thus set the corresponding literal to true, obtaining a satisfiable assignment for $\phi$. Indeed, this allows to have one literal set to true by group, and no contradiction (it is impossible to need $x_i = \bar{x}_i = T$, since there would be a blue edge between our elements and thus it would not be an independent set to start with). We have also shown that $f\left(\phi\right) \in \text{INDSET} \implies \phi \in \text{SAT}$, finishing our proof.

        \qed
    \end{subparag}
    
    \begin{subparag}{Remark}
        This proof is really beautiful: we managed to make a link between independent sets and satisfiable CNF formulas.
    \end{subparag}
\end{parag}

\end{document}
