% !TeX program = lualatex
% Using VimTeX, you need to reload the plugin (\lx) after having saved the document in order to use LuaLaTeX (thanks to the line above)

\documentclass[a4paper]{article}

% Expanded on 2023-11-19 at 11:29:14.

\usepackage{../../style}

\title{Algebra}
\author{Joachim Favre}
\date{Dimanche 19 novembre 2023}

\begin{document}
\maketitle

\lecture{8}{2023-11-13}{Moving onto rings}{
\begin{itemize}[left=0pt]
    \item Definition of commutative ring.
    \item Definition of zero divisor, and proof that multiplicative inverses are not divisors.
    \item Definition of fields and integral domains, and proof that the first one implies the second one.
    \item Definition of ideal.
    \item Proof of a proposition allowing to generate more ideals from known ideals.
\end{itemize}

}

\section{Rings and fields}

\begin{parag}{Definition: Ring}
    A ring is a set $A$ with 2 operations $+$ and $\cdot $, satisfying:
    \begin{enumerate}
        \item $\left(A, +\right)$ is an abelian group of neutral element $0 \in A$.
        \item $\cdot $ is associative: 
        \[a\cdot \left(b\cdot c\right) = \left(a\cdot b\right)\cdot c, \mathspace \forall a,b,c \in A\]
        \item $\cdot $ has a neutral element $1 \in A$, such that:
        \[1\cdot a = a\cdot 1 = a, \mathspace \forall a \in A\]
        \item $\cdot $ distributes on $+$ on the left: 
        \[a\cdot \left(b+c\right) =  ab + ac, \mathspace \forall a, b, c \in A\]

        And it distributes on the right: 
        \[\left(a + b\right)\cdot c = ac + bc, \mathspace \forall a, b, c \in A\]
    \end{enumerate}
    
    \begin{subparag}{Remark}
        We don't require $\cdot $ to have inverses, nor to be commutative.
    \end{subparag}
\end{parag}

\begin{parag}{Definition: Commutative ring}
    Let $\left(A, +, \cdot \right)$ be a ring.

    We say that this is a \important{commutative ring} if the multiplication is commutative, $ab = ba$ for any $a, b \in A$.

    \begin{subparag}{Remark}
        We will only consider commutative rings in this course.
    \end{subparag}
\end{parag}


\begin{parag}{Example 1}
    $\left(\mathbb{Z}, +, \cdot \right)$ is a commutative ring.

    We notice that some elements don't have a multiplicative inverse. For instance, $2 \in \mathbb{Z}$ but $\frac{1}{2} \not \in \mathbb{Z}$.
\end{parag}

\begin{parag}{Notation}
    Let $k \in \mathbb{R}$ and $S$ be some set. We define the following notation: 
    \[S \left[k\right] = \left\{a + b k \suchthat a, b \in S\right\}\]
    
    \begin{subparag}{Observation}
        We notice that, for any $k \in \mathbb{R}$ and set $S$, then:
        \[0, 1 \in S\left[k\right]\]

        Indeed, we can just set $b = 0$, and $a = 0$ or $a = 1$.
    \end{subparag}
\end{parag}


\begin{parag}{Example 2}
    Let us consider: 
    \[A = \mathbb{Z} \left[2\right] = \left\{a + b \sqrt{2} \suchthat a, b \in \mathbb{Z}\right\}\]
    
    This is a ring. Indeed, $0, 1 \in A$. Moreover, addition is closed in $A$: 
    \[\left(a + b\sqrt{2}\right) + \left(c + d\sqrt{2}\right) = \left(a+c\right) + \left(b+d\right)\sqrt{2} \in A\]

    Any element $a + b\sqrt{2} \in A$ has an additive inverse, $-a - b\sqrt{2} \in A$.

    Multiplication is also closed in $A$: 
    \[\left(a + b\sqrt{2}\right)\left(c + d\sqrt{2}\right) = \left(ac + 2bd\right) + \sqrt{2}\left(bc + ad\right) \in A\]
    
    The other properties come from the fact that we are using the regular additions and multiplications.

    However, we notice that there are no multiplicative inverses in $A$ in general: 
    \[\frac{1}{a + b\sqrt{2}} = \frac{a - b\sqrt{2}}{a^2 - 2b^2} = \underbrace{\frac{a}{a^2 - 2b^2}}_{\not \in \mathbb{Z} \text{ in general}} - \underbrace{\frac{b}{a^2 - 2b^2}}_{\not\in\mathbb{Z} \text{ in general}}\sqrt{2} \not \in A \text{ in general}\]
    in general.
\end{parag}

\begin{parag}{Example 3}
    Let us consider: 
    \[B = \mathbb{Q}\left[\sqrt{2}\right] = \left\{a + b\sqrt{2} \suchthat a, b \in \mathbb{Q}\right\}\]
    
    It is possible to show that $B$ is a ring such that every non-zero element has a multiplicative inverse (i.e, a field as we will define later).
\end{parag}

\begin{parag}{Remark}
    Let $\left(A, +, \cdot \right)$ be a commutative ring, and let $a \in A$ (which is not necessarily an integer). We notice that given $n \in \mathbb{Z}$, it makes sense to consider $n\cdot a$.

    If $n > 0$, we can write: 
    \[na = a + a + \ldots + a \in A\]
    
    If $n < 0$, then: 
    \[na = -\left(-n\right)a \in A\]

    And, if $n = 0$, then $na = 0$.

    \begin{subparag}{Implication}
        Many formulas thus hold in commutative rings. For instance, we can show that, for any $a, b \in A$: 
        \[\left(a+ b\right)^n = \sum_{k=0}^{n} \binom{n}{k} a^k b^{n-k}\]
    \end{subparag}
\end{parag}

\begin{parag}{Definition: Zero divisor}
    Let $\left(A, +, \cdot \right)$ be a commutative ring, and $a \in A$.

    $a$ is a \important{zero divisor} if there exists a $x \in A \setminus \left\{0\right\}$ such that: 
    \[ax = 0\]
    
    \begin{subparag}{Trivial zero divisor}
        For instance, $0 \in A$ is always a zero divisor: 
        \[0\cdot x = \left(0 + 0\right)\cdot x = 0\cdot x + 0\cdot x \implies 0\cdot x = 0\]

        This is called the \important{trivial} zero divisor.
    \end{subparag}
\end{parag}

\begin{parag}{Example 1}
    The following sets with addition and multiplication are rings without non-trivial zero divisors: 
    \[\mathbb{Z}, \mathspace \mathbb{R}, \mathspace \mathbb{C}\]
\end{parag}

\begin{parag}{Example 2}
    We want to find a ring with non-trivial zero divisors. 

    We can consider the set of equivalence classes modulo $n$: 
    \[A = \mathbb{Z}/n\mathbb{Z} = \left\{\left[0\right], \ldots, \left[n-1\right]\right\}\]

    This is indeed a ring with addition and multiplication modulo $n$: we do have that $\left(\mathbb{Z}/n\mathbb{Z}, +\right) \simeq \left(C_n, +\right)$ is an abelian group, $\cdot $ is associative and we have the neutral multiplicative element $\left[1\right]$.

    Let $a \in A \setminus \left\{\left[0\right]\right\}$. $a$ is a nontrivial zero divisor if and only if $\gcd\left(a, n\right) > 1$.

    \begin{subparag}{Proof $\implies$}
        We do this proof by the contrapositive. We thus suppose by hypothesis that $d = \gcd\left(a, n\right) = 1$. We want to show that $a$ is not a zero divisor. By Bézout's theorem, there exists $x, y \in \mathbb{Z}$ such that: 
        \[ax + ny = 1 \iff \congruent{ax}{1}{n} \iff \left[a\right]\left[x\right] = \left[1\right] \iff \left[x\right] = \left[a\right]^{-1}\]
        
        Now, this yields that: 
        \[\left[b\right]\left[a\right] = \left[0\right] \implies \left[b\right]\underbrace{\left[a\right]\left[x\right]}_{= \left[1\right]} = \left[0\right] \left[x\right] \implies \left[b\right] = 0\]
    \end{subparag}

    \begin{subparag}{Proof $\impliedby$}
         We suppose by hypothesis that $d = \text{gcd}\left(a, n\right) > 1$. We indeed have that $\left[a\right]$ is a nontrivial zero divisor: 
    \[\left[a\right] \cdot \left[\frac{n}{d}\right] = \left[\frac{an}{d}\right] = \left[\frac{a}{d}\right] \underbrace{\left[n\right]}_{= \left[0\right]} = \left[0\right]\]
    \end{subparag}
    
    \begin{subparag}{Observation}
        An element $\left[a\right] \in \mathbb{Z}/n\mathbb{Z}$ is thus either a zero divisor, or invertible.

        This tells us that we have $n - \phi\left(n\right)$ zero divisors in $\mathbb{Z}/n\mathbb{Z}$.
    \end{subparag}
\end{parag}

\begin{parag}{Definition: Integral domain}
    Let $\left(A, +, \cdot \right)$ be a commutative ring.

    If the only zero divisor of $A$ is the trivial zero divisor, then $A$ is called a \important{integral domain}.
\end{parag}

\begin{parag}{Definition: Field}
    Let $\left(A, +, \cdot \right)$ be a commutative ring.

    It is called a \important{field} if all non-zero elements have multiplicative inverses. In other words, for any $a \in A \setminus \left\{0\right\}$, there exists a $b \in A$ such that $ab = 1$.
\end{parag}

\begin{parag}{Proposition}
    Let $n \in \mathbb{N}_{\geq 2}$. We have that:
    \begin{enumerate}
        \item $\left(\mathbb{Z}/n\mathbb{Z}, +, \cdot \right)$ has non-trivial zero divisors if and only if $n \not \in \mathbb{P}$ is not a prime.
        \item $\left(\mathbb{Z}/n\mathbb{Z}, +, \cdot \right)$ is a field if and only if $n \in \mathbb{P}$ is a prime.
    \end{enumerate}

    \begin{subparag}{Proof 1}
        We see that the following propositions are equivalent:
        \begin{itemize}
            \item There exists a nontrivial zero divisor, $a \in \mathbb{Z}/n\mathbb{Z} \setminus \left\{\left[0\right]\right\}$.
            \item There exists some $a$ such that $1 \leq a \leq n-1$ and $\gcd\left(a, n\right) > 1$, by our previous example.
            \item There exists some $a$ such that $1 \leq a \leq n-1$, and $n$ is divisible by $a$.
            \item $n$ is not a prime.
        \end{itemize}
    \end{subparag} 

    \begin{subparag}{Proof 2}
        We see that the following propositions are equivalent:
        \begin{itemize}
            \item $n$ is a prime.
            \item For any $\left[b\right] \in \mathbb{Z}$, we have $\gcd\left(b, n\right) = 1$.
            \item Any $\left[b\right] \in \mathbb{Z}$ has a multiplicative inverse.
            \item $\mathbb{Z}/n\mathbb{Z}$ is a field, by definition. 
        \end{itemize}

        \qed
    \end{subparag}
\end{parag}

\begin{parag}{Example 1}
    Let us consider: 
    \[\mathbb{Z}/5\mathbb{Z} = \left\{\left[0\right], \left[1\right], \left[2\right], \left[3\right], \left[4\right]\right\}\]
    
    Since $5$ is prime, this is a field. Thus, all elements different from $\left[0\right]$ have a multiplicative inverse: 
    \[\left[1\right]^{-1} = \left[1\right], \mathspace \left[2\right]^{-1} = \left[3\right], \mathspace \left[3\right]^{-1} = \left[2\right], \mathspace \left[4\right]^{-1} = \left[4\right]\]
\end{parag}

\begin{parag}{Example 2}
    Let us consider $\mathbb{Z}/6\mathbb{Z}$. We notice that $\left[2\right]\left[3\right] = \left[0\right]$, and thus it is not an integral domain.
\end{parag}

\begin{parag}{Lemma}
    Let $\left(A, +, \cdot \right)$ be a commutative ring, and $a \in A$.

    If $a$ has a multiplicative inverse, then $a$ is not a zero divisor.

    \begin{subparag}{Proof}
        We know by hypothesis that there exists some $a^{-1} \in A$ such that $aa^{-1} = 1$.

        Now, let's suppose that $ab = 0$. This yields that: 
        \[0 = a^{-1} 0 = a^{-1} ab = b\]
        
        Thus, $a$ is not a zero-divisor.

        \qed
    \end{subparag}

    \begin{subparag}{Converse}
        The converse is wrong. For instance, in $\mathbb{Z}$, $2$ is neither a zero-divisor nor inversible.
    \end{subparag}
\end{parag}


\begin{parag}{Proposition}
    Let $\left(A, +, \cdot \right)$ be a commutative ring.

    If it is a field, then it is an integral domain.

    \begin{subparag}{Proof}
        We know that all non-zero element of a field are invertible, so they aren't zero divisors. This shows that $A$ is an integral domain.
    \end{subparag}

    \begin{subparag}{Converse}
        The converse is false: $\mathbb{Z}$ is an integral domain, but not a field.

        However, as we showed earlier, $A = \mathbb{Z}/n\mathbb{Z}$ is an integral domain if and only if it is a field.
    \end{subparag}
\end{parag}

\begin{parag}{Observation}
    We have thus found the following inclusions:
    \[\text{Fields} \subset \text{Integral domains} \subset \text{Commutative rings}\]
\end{parag}

\subsection{Ideals in a ring}

\begin{parag}{Definition: Ideal}
    Let $A$ be a commutative ring, and $I \subset A$.

    $I$ is called an \important{ideal} if it has the following properties:
    \begin{enumerate}
        \item $\left(I, +\right)$ is a subgroup of $\left(A, +\right)$.
        \item For any $x \in A$ and $a \in I$, then $xa \in I$.
    \end{enumerate}
\end{parag}

\begin{parag}{Proposition: Trivial ideal}
    Let $\left(A, +, \cdot \right)$ be a commutative ring.

    Then, $\left\{0\right\}$ is an ideal, named the \important{trivial ideal}.
\end{parag}

\begin{parag}{Proposition: Non-proper ideal}
    Let $\left(A, +, \cdot \right)$ be a commutative ring.

    Then, $A$ is an ideal, named the \important{non-proper ideal}.
\end{parag}

\begin{parag}{Example}
    Let us consider the ring $A = \mathbb{Z}$, and let $d \in \mathbb{Z}$. 

    Then, the following is an ideal: 
    \[d\mathbb{Z} = \left\{dk \suchthat k \in \mathbb{Z}\right\} \subset \mathbb{Z}\]
    
    $\left(d\mathbb{Z}, +\right)$ is indeed a subgroup since this is closed under addition, it contains $0$, and it contains additive the inverses. It moreover also follows the multiplicative property: 
    \[da \cdot  x \in d\mathbb{Z}, \mathspace \forall x \in \mathbb{Z}\]
\end{parag}

\begin{parag}{Properties}
    Let $A$ be a commutative ring, and let $I, J \subset A$ be ideals.

    Then:
    \begin{enumerate}
        \item If $1 \in I$, then $I = A$.
        \item $I \cap J$ is an ideal.
        \item $I + J \over{=}{def} \left\{x + y \suchthat x \in I, Y \in J\right\}$ is an ideal.
        \item $\displaystyle  I\cdot J \over{=}{def} \left\{\sum_{i=1}^{k} x_i y_i \suchthat x_i \in I, y_i \in J, k \in \mathbb{N}\right\} \subset A$ is an ideal.
    \end{enumerate}
    
    \begin{subparag}{Proof 1}
        Let $a \in A$ be arbitrary. Since $1 \in I$ by hypothesis, we know by the multiplicative property that: 
        \[1\cdot a \in I \implies a \in I\]
        
        Thus, $A \subset I$. Since we always have $I \subset A$, this yields that $A = I$.
    \end{subparag}

    \begin{subparag}{Proof 2}
        Let $I, J$ be ideals in $A$, and let $x, y \in I \cap J$.

        We first notice that $\left(I + J, +\right)$ is indeed an additive subgroup:
        \begin{enumerate}
            \item Since $\left(I, +\right)$ and $\left(J, +\right)$ are additive subgroups, we know that $x+y \in I$ and $x + y \in J$. This implies by set theory that $x + y \in  I \cap J$.
            \item By a similar reasoning, we know that $-x \in I$ and $-x \in J$. This indeed means that $-x \in I \cap J$.
            \item By the exact same reasoning, we get $0 \in I \cap J$. 
        \end{enumerate}
        
        We still need to show the multiplicative property. Let $a \in A$. We know that $ax \in I$ and $ax \in J$ since they are ideals. This indeed means that $ax \in I \cap J$.
    \end{subparag}

    \begin{subparag}{Proof 3}
        Let $I, J$ be ideals in $A$, let $x_1 + x_2, y_1 + y_2 \in I + J$. 

        This closed under addition: 
        \[x_1 + x_2 + y_1 + y_2 = \underbrace{x_1 + y_1}_{\in I} + \underbrace{x_2 + y_2}_{\in J}\]

        We can check the other properties to show that $I + J$ is an additive subgroup.

        Now, let's consider the multiplicative property. Let $a \in A$ . Then: 
        \[a\left(x + y\right) = \underbrace{ax}_{\in I} + \underbrace{ay}_{\in J} \in I + J\]
    \end{subparag}

    \begin{subparag}{Proof 4}
        Let $I, J$ be ideals in $A$.

        We directly see that $I\cdot J$ is closed with respect to addition. We can check all other properties to see that this is an additive subgroup.

        Now, let's check the multiplicative property. Let $a \in A$. We have: 
        \[a \sum_{i=1}^{k} x_i y_i = \sum_{i=1}^{k} \underbrace{\left(a x_i\right)}_{\in I}y_i = \sum_{i=1}^{k} z_i y_i \in I J\]
        
        \qed
    \end{subparag}
\end{parag}

\begin{parag}{Property}
    Let $A$ be a commutative ring, and let $I, J \subset A$ be ideals.

    Then, $I \cup J$ is not necessarily an ideal.

    \begin{subparag}{Proof}
        We want to show that $I \cup J$ is not an additive subgroup in general. We do this using a counter-example.

        Let $I = 3\mathbb{Z} \subset \mathbb{Z}$ and $J = 5\mathbb{Z} \subset \mathbb{Z}$. Then, $I \cup J$ is the multiples of 3 and the ones of 5: 
        \[I \cup J = \left\{0, \pm 3, \pm 5, \pm 6, \pm 9, \pm 10, \ldots\right\}\]
        
        However, $3 + 5 = 8 \not \in I \cup J$. It is thus not an additive subgroup, and not an ideal.
    \end{subparag}
\end{parag}

\begin{parag}{Example}
    Let $A = \mathbb{Z}$, $I = 6\mathbb{Z}$ and $J = 10\mathbb{Z}$. Then, the following are ideals:
    \begin{enumerate}
        \item $\displaystyle I \cap J = \left\{z \in \mathbb{Z} \suchthat z = 6n \text{ and } z = 10m; n, m \in \mathbb{Z}\right\} = \lcm\left(6, 10\right)\mathbb{Z} = 30 \mathbb{Z}$.
        \item $\displaystyle I + J = \left\{6n + 10m \suchthat n, m \in \mathbb{Z}\right\} = \text{gcd}\left(6, 10\right)\mathbb{Z} = 2\mathbb{Z}$, by Bézout's theorem.
        \item $\displaystyle IJ = \left\{\sum_{i=1}^{k} 6x_i\cdot 10y_i \suchthat x_i, y_i \in \mathbb{Z}\right\} = \left\{60 \sum_{i=1}^{k} x_i y_i \suchthat x_i, y_i \in \mathbb{Z}\right\} = 60 \mathbb{Z}$
    \end{enumerate}

    \begin{subparag}{Generalisation}
        We can generalise our result. Let $n, m \in \mathbb{Z}^*$. The ideals $I = n\mathbb{Z}$ and $J = m\mathbb{Z}$ generate:
        \begin{enumerate}
            \item $I \cap J = \text{lcm}\left(n, m\right)\mathbb{Z}$
            \item $I + J = \text{gcd}\left(n, m\right)\mathbb{Z}$
            \item $IJ = nm\mathbb{Z}$
        \end{enumerate}
    \end{subparag}
\end{parag}

\begin{parag}{Remark}
    Let $I, J \subset A$ be two ideals. Then: 
    \[IJ \subset I \cap J \subset I \subset I + J\]
    
    By symmetry, we naturally also have 
    \[IJ \subset I \cap J \subset J \subset I + J\]
\end{parag}

\begin{parag}{Example}
    Let us consider the set of polynomials in one variable with real coefficients: 
    \[A = \mathbb{R}\left[x\right] = \left\{a_0 + a_1 x + \ldots + a_n x^n \suchthat a_0, \ldots, a_n \in \mathbb{R}, n \in \mathbb{N}\right\}\]
   
    This is indeed a commutative ring: it closed under addition and multiplication, which are commutative and associative, and we have additive inverses.

    We can consider the set of polynomials divisible by $\left(x+5\right)$: 
    \[I = \left\{\left(x+5\right)f\left(x\right) \suchthat f\left(x\right) \in \mathbb{R}\left[x\right]\right\}\]

    And, similarly, the set of polynomials divisible by $\left(x^2 + 2\right)$: 
    \[J = \left\{\left(x^2 + 2\right) f\left(x\right) \suchthat f\left(x\right) \in \mathbb{R}\left[x\right]\right\}\]
    
    It is possible to verify that $I$ and $J$ are indeed ideals. Then, the following are ideals:
    \begin{enumerate}
        \item $I \cap J = \left\{\left(x + 5\right)\left(x^2 + 2\right)f\left(x\right) \suchthat f\left(x\right) \in \mathbb{R}\left[x\right]\right\}$
        \item $IJ$ is expressed as:
        \autoeq{IJ = \left\{\sum_{i=1}^{k} \left(x+5\right) f_i\left(x\right)\left(x^2 + 2\right)g_i\left(x\right) \suchthat f_i\left(x\right), g_i\left(x\right) \in \mathbb{R}\left[x\right]\right\} = \left\{\left(x+5\right)\left(x^2 + 2\right)f\left(x\right) \suchthat f\left(x\right) \in \mathbb{R}\left[x\right]\right\}}

        \item $I + J = \left\{\left(x + 5\right)f\left(x\right) + \left(x^2 + 2\right)g\left(x\right) \suchthat f\left(x\right), g\left(x\right) \in \mathbb{R}\left[x\right]\right\}$. It seems like this ideal contains many element, so we want to see if it contains 1, to see if it is not proper. To compensate with the $x^2$, we can let $f\left(x\right)$ be degree one and $g\left(x\right)$ be degree zero. We can thus solve the following equation: 
        \[\left(x+5\right)\left(ax + b\right) + \left(x^2 + 2\right)c = 1\]
        
        Doing so, we get that: 
        \[\left(x + 5\right)\left(x-5\right)\frac{-1}{27} - \left(x^2 + 2\right)\left(\frac{-1}{27}\right) =  1 \in \mathbb{R}\left[x\right]\]
        
        Thus, $I + J = \mathbb{R}\left[x\right]$.
    \end{enumerate}

    In some form, it would make sense to say that $\text{gcd}\left(x+5, x^2 + 2\right) = 1$.
\end{parag}


\end{document}
