% !TeX program = lualatex
% Using VimTeX, you need to reload the plugin (\lx) after having saved the document in order to use LuaLaTeX (thanks to the line above)

\documentclass[a4paper]{article}

% Expanded on 2023-11-24 at 00:59:39.

\usepackage{../../style}

\title{Algebra}
\author{Joachim Favre}
\date{Vendredi 24 novembre 2023}

\begin{document}
\maketitle


\lecture{9}{2023-11-20}{\emoji{sparkles} Prof. de Lataillade \emoji{sparkles}}{
\begin{itemize}[left=0pt]
    \item Definition of principal ideal.
    \item Definition of congruence relation.
    \item Proof that ideals generate congruence relations and inversely.
    \item Definition of principal ideal domains.
    \item Definition of ring homomorphism.
    \item Definition of the characteristic of a ring.
\end{itemize}

}

\begin{parag}{Definition: Ideal generated by a set}
    Let $\left(A, +,\cdot \right)$ be a commutative ring, and $S \subset A$ be a subset.

    The minimal ideal $I$ containing $S$, written $I =\left(S\right)$, is the \important{ideal generated by the set $S$}. It can be written as: 
    \[\left(S\right) = \left\{\sum_{i}^{} a_i s_i \suchthat a_i \in A, s_i \in S\right\}\]

    \begin{subparag}{Remark}
        We can also sometimes write: 
        \[\left(S\right) = \left\langle S \right\rangle\]
    \end{subparag}
\end{parag}

\begin{parag}{Definition: Principal ideal}
    Let $\left(A, +, \cdot \right)$ be a commutative ring, and $I \subset A$ be an ideal.

    If $I = \left(x\right)$ is generated by a single element, it is called \important{principal}. It can be written as: 
    \[I = \left\{xa \suchthat a \in A\right\}\]
\end{parag}

\begin{parag}{Example 1}
    Let $\left(A, +, \cdot \right)$ be a commutative ring. We consider the ideals $\left\{0\right\} \subset A$ and $A \subset A$. They are both principal:
    \[\left\{0\right\} = \left(0\right), \mathspace A = \left(1\right)\]
\end{parag}

\begin{parag}{Example 2}
    Let $n \in \mathbb{N}^*$. We consider the ideal $n \mathbb{Z} \subset \mathbb{Z}$. It is principal: 
    \[n\mathbb{Z} = \left(n\right)\]
\end{parag}

\begin{parag}{Theorem}
    Let $\left(A, +, \cdot \right)$ be a commutative ring.

    $\left(A, +, \cdot \right)$ is a field if and only if $0$ and $A$ are the only ideals in $A$.

    \begin{subparag}{Proof $\implies$}
        We suppose by hypothesis that $\left(A, +, \cdot \right)$ is a field. We consider an arbitrary non-trivial ideal $I \neq \left\{0\right\}$. Our goal is to show that it is non-proper.

        We pick some arbitrary element $a \in I \setminus \left\{0\right\}$. Since $a \neq 0$ and $A$ is a field, it means that there exists a multiplicative inverse $a^{-1} \in A$. We thus get that $a^{-1} a = 1 \in I$, by the multiplicative property of ideals. However, this implies that $I = A$.

        This indeed shows that any non-trivial ideal is non-proper.
    \end{subparag}

    \begin{subparag}{Proof $\impliedby$}
        We suppose by hypothesis that $\left\{0\right\}$ and $A$ are the only ideals in $A$. Let $a \in A \setminus \left\{0\right\}$ be arbitrary. We want to show that $a$ has a multiplicative inverse.

        We consider the following ideal:
        \[I = \left(a\right) = \left\{x a \suchthat x \in A\right\}\]

        Since $a \neq 0$, we know $I \neq \left\{0\right\}$. This means that, by our hypothesis, $I = A$. But then, since $1 \in A$, this means that there exists some $y \in A$ such that $xy = 1$ by definition of $I$. $y$ is an inverse of $x$, ending our proof.

        \qed
    \end{subparag}
\end{parag}

\subsection{Quotient rings over an ideal}

\begin{parag}{Goal}
    We are interested by ideals because they play the role of normal subgroups for fields: we can use them to get a quotient ring.
\end{parag}


\begin{parag}{Definition: Equivalence relation}
    Let $E$ be a \textit{set}, and let $\sim$ be a relation on $E$.

    It is named an \important{equivalence relation} if it follows the following properties for any $a, b, c \in E$:
    \begin{enumerate}
        \item (Reflexivity) $a \sim a$.
        \item (Symmetry) If $a \sim b$, then $b \sim a$.
        \item (Transitivity) If $a\sim b$ and $b \sim c$, then $a \sim c$.
    \end{enumerate}
    
    \begin{subparag}{Intuition}
        Equivalence relations basically generalise the notion of equality.
    \end{subparag}
\end{parag}

\begin{parag}{Definition: Congruence relation}
    Let $\left(A, + , \cdot \right)$ be a \textit{commutative ring}, and let $\sim$ be an equivalence relation on $A$.

    It is named a \important{congruence} if, for any $a, b, c, d \in A$ such that $a \sim b$ and $c \sim d$, then: 
    \[a + c \sim b + d \mathspace \text{and} \mathspace ac \sim bd\]
\end{parag}

\begin{parag}{Theorem}
    Let $\left(A, +, \cdot \right)$ be a commutative ring.

    We can construct congruences from ideals, and ideals from congruences:
    \begin{enumerate}
        \item Let $I \subset A$ be an arbitrary ideal. Then, the relation $a \sim b \iff b -a \in I$ is a congruence relation.
        \item Let $\sim$ be an arbitrary congruence relation in $A$. Then, $I = \left\{a \in A \suchthat a \sim 0\right\}$ is an ideal in $A$.
    \end{enumerate}
    
    \begin{subparag}{Proof 1}
        It is possible to show that $a \sim b$ is indeed an equivalence relation.

        Moreover, we can check that this is a congruence. Indeed, if $b-a \in I$ and $d - c \in I$, then, since $I$ is an additive subgroup: 
        \[I \ni b-a + d -c = \left(b+d\right) - \left(a + c\right) \implies b+d \sim a + c\]
        
        We can use a similar argument to show that $\left(b-a\right)\left(d - c\right) \in I$.
    \end{subparag}

    \begin{subparag}{Proof 2}
        Let $a, b \in A$. 

        Our relation indeed makes an additive subgroup since, supposing that $a \sim 0$ and $b\sim0$, then $a + b \sim 0 $. We also have the additive identity $0 \sim 0$, and the additive inverse $-a \sim 0$.

        Moreover, this follows the multiplicative property. Indeed, if $a \sim 0$ and $x \in A$, then: 
        \[x \sim x \implies ax \sim 0x \implies ax \sim 0 \]

        \qed
    \end{subparag}
\end{parag}

\begin{parag}{Example}
    Let us consider the congruences modulo $n$ in $\mathbb{Z}$: 
    \[a \sim b \iff \exists k \in \mathbb{Z},\ b-a = kn\]
    
    The generated ideal is: 
    \[I = \left\{a \in \mathbb{Z} \suchthat a \sim 0\right\} = n\mathbb{Z} = \left(n\right)\]
\end{parag}

\begin{parag}{Theorem}
    Let $A$ be a commutative ring, and $\sim$ be a congruence relation in $A$ such that $1 \not\sim 0$.

    Then, the set of congruence classes is a commutative ring: 
    \[\congclasses{A} \over{=}{def} A / \left\{x \in A \suchthat x \sim 0\right\}\]

    The elements are congruence classes $\bar{a} = \left\{x \in A \suchthat x \sim a\right\}$, and we define: 
    \[\bar{a} + \bar{b} = \bar{a + b}, \mathspace \bar{a}\cdot \bar{b} = \bar{a\cdot b}\]

    \begin{subparag}{Proof}
        We consider an arbitrary congruence class: 
        \[\bar{a} = \left\{x \in A \suchthat x \sim a\right\}\]
        
        The operations are well defined since, for $a_1 \sim a_2$ and $b_1 \sim b_2$, we have $a_1 + b_1 \sim a_2 + b_2$ and $a_1 b_1 \sim a_2 b_2$.

        We notice that, importantly, $1 \not\sim 0$. This means that $\bar{1} \neq \bar{0}$, which is important since we require for rings that the additive and multiplicative identities are different.

        \qed
    \end{subparag}

    \begin{subparag}{Implication}
        An ideal allows to create a congruence relation, which in turns allows to create commutative rings. We can thus write $A / I$ to mean $\congclasses{A}$ where $\sim$ is the congruence relation generated by $I$.

        Ideals allow to create commutative rings, just like normal subgroups allow to create groups.
    \end{subparag}
\end{parag}

\begin{parag}{Example 1}
    Let us consider $\congclasses{\mathbb{Z}}$ for the following relation:
    \[a \sim b \iff \exists k \in \mathbb{Z},\ b-a = kn\]

    Then, we have that: 
    \[\congclasses{\mathbb{Z}} = \mathbb{Z}/n\mathbb{Z} = \left\{\left[0\right], \ldots, \left[n-1\right]\right\}\]
\end{parag}

\begin{parag}{Example 2}
    We consider the ring of polynomials with real coefficients, $A = \mathbb{R}\left[x\right]$, and the ideal generated by $x^2 - 4$: 
    \[I = \left\langle \left(x^2 - 4\right) \right\rangle\]
    
    We can consider the commutative ring $B = \mathbb{R}\left[x\right] / I$. In it: 
    \[\bar{\left(x + 2\right)}\cdot \bar{\left(x + 1\right)} = \bar{x^2 + 3x + 2} = \bar{x^2 + 3x + 2 - \left(x^2 - 4\right)} = \bar{3x + 6}\] 
    \[\bar{x}\cdot \bar{x} = \bar{x^2} = \bar{x^2 - \left(x^2 - 4\right)} = \bar{4}\]
    
    We also notice that $x+2$ and $x-2$ are non-trivial zero-divisors in $B$: 
    \[\bar{\left(x+2\right)}\cdot \bar{\left(x-2\right)} = \bar{x^2 - 4} = \bar{0}\]
    
    $B$ is thus not an integral domain.

    We notice that, in $B$, we have both any $x \in \mathbb{R}$ and any polynomial of first degree that all yield different congruence classes. Now, any polynomial with higher degree can be turned into a polynomial of first degree or lower by using polynomial division by $x^2 - 4$. Therefore: 
    \[B = \left\{\bar{ax + b} \suchthat a, b \in \mathbb{R}\right\}\]
\end{parag}

\begin{parag}{Definition: Principal ideal domain}
    Let $\left(A, +, \cdot \right)$ be an integral domain.

    If all its ideals are principal, it is called \important{a principal ideal domain} (PID).

    \begin{subparag}{Remark}
        This must not be mistaken with simple groups, which are a similar but different notion for groups.
    \end{subparag}

    \begin{subparag}{Summary}
        To sum up, a PID is a commutative ring, without nontrivial zero divisors, and such that every ideal is generated by a single element.
    \end{subparag}
\end{parag}


\begin{parag}{Proposition}
    Let $\left(A, +, \cdot \right)$ be a commutative ring.

    If it is a field, then it is a PID.

    \begin{subparag}{Proof}
        We know a field has only two ideals, which are both principal: $A = \left(1\right)$ and $\left\{0\right\} = \left(0\right)$.

        \qed
    \end{subparag}
\end{parag}

\begin{parag}{Proposition}
    $\mathbb{Z}$ is a PID.

    \begin{subparag}{Remark}
        We know that $\mathbb{Z}$ is not a field, showing that PIDs are not necessarily fields. The converse of the previous proposition is thus wrong.
    \end{subparag}

    \begin{subparag}{Proof}
        Let $I$ be an arbitrary ideal. We split our proof in two cases.

        If $I = \left\{0\right\}$, then $I = \left(0\right)$. This is indeed principal.

        Let's now suppose that $I \neq \left\{0\right\}$. Thus, we know that there exists a $a \in I \setminus \left\{0\right\}$. We know that this implies that $-a \in I$, and thus that $\left|a\right| \in I$. It thus makes sense to consider the smallest positive element of $I$, which we write $d \in I$.

        Let $n \in I$ be arbitrary. By Euclidean division, we know we can write $n = kd + r$, where $0 \leq r \leq d - 1$. By ideal properties, we know that $r = n - kd \in I$ since $n \in I$ and $kd = d + \ldots + d \in I$. However, since $d$ is the smallest positive element of the ideal and $0 \leq r \leq d -1$, we get that $r = 0$. This tells us that any element $n$ can be written as $n = kd$ for some $k \in \mathbb{Z}$. In other words: 
        \[I = \left(d\right)\]

        \qed
    \end{subparag}
\end{parag}

\begin{parag}{Example}
    Let $a_1, \ldots, a_n \in \mathbb{Z}$. We consider the ideal generated by those elements, $J = \left(a_1, \ldots, a_n\right) \subset \mathbb{Z}$. By our proposition, we know that $J$ is principal. In other words, it is generated by a single element.

    It is possible to show that $J = \left(k\right)$ where $k = \gcd\left(a_1, \ldots, a_n\right)$, by using induction on $n$ and Bézout's theorem.
\end{parag}

\subsection{Ring homomorphisms}

\begin{parag}{Definition: Ring homomorphism}
    Let $\left(A, +_A, \cdot_A \right)$ and $\left(B, +_B, \cdot_B \right)$ be commutative rings, and $f: A \mapsto B$ be a map.

    $f$ is said to be a \important{ring homomorphism} if:
    \begin{enumerate}
        \item $f\left(a +_A b\right) = f\left(a\right) +_B f\left(b\right)$
        \item $f\left(a\cdot_A b\right) = f\left(a\right)\cdot_B f\left(b\right)$
        \item $f\left(1_A\right) = 1_B$
    \end{enumerate}

    \begin{subparag}{Property}
        We notice that we always  have that: 
        \[f\left(0\right)= f\left(0 + 0\right) = f\left(0\right) + f\left(0\right) \implies f\left(0_A\right) = 0_B\]

        Moreover: 
        \autoeq{f\left(a\right) - f\left(a\right) = 0 = f\left(0\right) = f\left(a - a\right) = f\left(a\right) + f\left(-a\right) \implies f\left(-a\right) = -f\left(a\right)}
    \end{subparag}
\end{parag}

\begin{parag}{Definition: Ring isomorphism}
    Let $A$ and $B$ be commutative rings, and let $f: A \mapsto B$ be a ring homomorphism.

    If $f$ is bijective, we call it a \important{ring isomorphism}. We then say that $A$ and $B$ are \important{isomorphic}.
\end{parag}

\begin{parag}{Property}
    Let $k \in \mathbb{Z}$, and $f: A \mapsto B$ be a ring homomorphism.

    Then: 
    \[f\left(k\cdot 1_A\right) = k\cdot 1_B\]
    
    \begin{subparag}{Proof}
        Let's first suppose that $k = 0$. Then: 
        \[f\left(0\right) = 0\]
        
        Now, let's suppose that $k > 0$: 
        \[f\left(k\cdot 1_A\right) = f\left(1_A + \ldots + 1_A\right) = f\left(1_A\right) + \ldots + f\left(1_A\right) = 1_B + \ldots + 1_B = k 1_B\]
        
        Finally, if $k < 0$: 
        \[f\left(k \cdot 1_A\right) = -f\left(\left|k\right| \cdot 1_A\right) = -\left|k\right| 1_B = k 1_B\]
    \end{subparag}
\end{parag}

\begin{parag}{Definition: Subring}
    Let $\left(B, +, \cdot \right)$ be a commutative ring, and $C \subset B$ be a subset.

    We say that $C$ is a subring if it is a ring with the same additive identity, multiplicative identity, addition and multiplication (meaning that those operations are closed inside $C$).

    \begin{subparag}{Remark}
        As we will see, this is a very strong definition; there will typically not be many subrings.
    \end{subparag}
\end{parag}

\begin{parag}{Proposition}
    Let $f: A \mapsto B$ be a ring homomorphism.

    Then, $\ker\left(f\right) \subset A$ is an ideal and $\im\left(f\right) \subset B$ is a subring.
\end{parag}

\begin{parag}{Example}
    Let $C \subset \mathbb{Z}$ be an arbitrary subring. 

    Then, we need to have: 
    \[0 \in C, \mathspace 1 \in C\]

    This moreover yields that: 
    \[-1 \in C\]
    
    Moreover, we need to have: 
    \[n = 1 + 1 + \ldots + 1 \in C\]
    
    This means that any element $x \in \mathbb{Z}$ is such that $x \in C$, i.e. $\mathbb{Z} \subset C$. Since we always have $C \subset \mathbb{Z}$, we get that the only subring of $\mathbb{Z}$ is $C = \mathbb{Z}$.
\end{parag}

\begin{parag}{Proposition}
    Let $n, m \in \mathbb{N}_{\geq 2}$, and let $f: \mathbb{Z}/n\mathbb{Z} \mapsto \mathbb{Z}/m\mathbb{Z}$ be a ring homomorphism. Then:
    \begin{enumerate}
        \item $\im\left(f\right) = \mathbb{Z}/m\mathbb{Z}$, which is the unique subring of $\mathbb{Z}/m\mathbb{Z}$.
        \item $n$ divides $m$.
        \item This $f$ is unique.
    \end{enumerate}

    \begin{subparag}{Proof 1}
        We know that $\im\left(f\right)$ is a subring in $\mathbb{Z}/m\mathbb{Z}$. Moreover, we know that $\left[1\right]_m \in \im\left(f\right)$. But then, this means that: 
        \[\left[k\right]_m = \left[1\right]_m + \ldots + \left[1\right]_m \in \im\left(f\right)\]
        
        Thus, this means that $\im\left(f\right) = \mathbb{Z}/m\mathbb{Z}$, which is the only subring by a similar argument as the previous example.
    \end{subparag}
    
    \begin{subparag}{Proof 2}
        The existence of this ring homomorphism necessarily implies that: 
        \[\left[0\right]_m = f\left(\left[0\right]_n\right) = f\left(\left[n\right]_n\right) = f\left(n \left[1\right]_n\right) = n\left[1\right]_m = \left[n\right]_m \]
        
        This means that $m$ must divide $n$.
    \end{subparag}

    \begin{subparag}{Proof 3}
        Finally, we know that $f\left(\left[1\right]_n\right) = \left[1\right]_m$. Therefore:
        \[f\left(\left[k\right]_n\right) = f\left(k\left[1\right]_n\right) = k\left[1\right]_m = \left[k\right]_m\]
        which is forced. This $f$ is thus indeed unique.

        \qed
    \end{subparag}
\end{parag}

\begin{parag}{Example 1}
    Let us consider a group homomorphism $f: \mathbb{Z}/10\mathbb{Z} \mapsto \mathbb{Z}/5\mathbb{Z}$.

    We necessarily have that: 
    \[f\left(\left[0\right]\right) = \left[0\right], \mathspace f\left(\left[1\right]\right) = \left[1\right], \mathspace \ldots, \mathspace f\left(\left[4\right]\right) = \left[4\right], \mathspace f\left(\left[5\right]\right) = \left[0\right], \mathspace \ldots, \mathspace f\left(\left[9\right]\right) = \left[4\right]\]
    
    Then, we have that: 
    \[\ker\left(f\right) = \left\{\left[0\right], \left[5\right]\right\} = \left(\left[5\right]\right), \mathspace \im\left(f\right) = \mathbb{Z}/5\mathbb{Z}\]
\end{parag}

\begin{parag}{Example 2}
    There is no ring homomorphism $f: \mathbb{Z}/6\mathbb{Z} \mapsto \mathbb{Z}/12\mathbb{Z}$.
\end{parag}

\begin{parag}{Example 3}
    We can construct a unique ring homomorphism $f: \mathbb{Z} \mapsto \mathbb{Z}/6\mathbb{Z}$ such that: 
    \[\ker\left(f\right) = \left\{0, \pm6, \pm12, \ldots\right\} = \left(6\right), \mathspace \im\left(f\right) = \mathbb{Z}/6\mathbb{Z}\]
\end{parag}


\subsection{Characteristic of a ring}

\begin{parag}{Proposition}
    Let $\left(A, +, \cdot \right)$ be a commutative ring.

    There exists a unique ring homomorphism $\tau: \mathbb{Z} \mapsto A$.

    \begin{subparag}{Proof}
        We know that we need: 
        \[\tau\left(0\right) = 0, \mathspace \tau\left(1\right) = 1_A\]
        
        Now, we know that, for any $n \in \mathbb{Z}$: 
        \[\tau\left(n\right) = \tau\left(n\cdot 1\right) = n 1_A \in A\]
        
        Therefore, $\tau\left(n\right) = n 1_A \in A$ is uniquely determined. We moreover see that this is indeed a ring homomorphism since: 
        \[\tau\left(nk\right) = nk 1_A = n 1_A \cdot  k 1_A= \tau\left(n\right)\tau\left(k\right)\]

        \qed
    \end{subparag}

    \begin{subparag}{Remark}
        By construction of $\tau$, the kernel is generated by a single element. However, $\ker\left(\tau\right) \neq \left(1\right)$, because we know that $\tau\left(1\right) = 1 \neq 0$. 

        This tells us that $\ker\left(\tau\right) = \left(0\right)$, or $\ker\left(\tau\right) = \left(d\right)$ for some $d \geq 2$.
    \end{subparag}
\end{parag}

\begin{parag}{Definition: Characteristic of a ring}
    Let $A$ be a commutative ring, and $\tau : \mathbb{Z} \mapsto A$ be the unique homomorphism. Let $d \in \mathbb{N}_0 \setminus \left\{1\right\}$ be the number such that $\ker\left(\tau\right) = \left(d\right)$.

    The \important{characteristic} of $A$ is defined as: 
    \[c\left(A\right) = c_A = d\]

    \begin{subparag}{Remark}
        By our remark in the previous paragraph, we always have $d \neq 1$.
    \end{subparag}
\end{parag}

\begin{parag}{Example 1}
    We want to find the characteristic of $\mathbb{R}$.

    The unique homomorphism from $\mathbb{Z}$ is the following:
    \[\begin{split}
    \tau: \mathbb{Z} &\longmapsto \mathbb{R} \\
    n &\longmapsto n
    \end{split}\]
    
    $\ker\left(\tau\right) = \left\{0\right\} = \left(0\right)$, telling us that $c\left(\mathbb{R}\right) = 0$. 
\end{parag}

\begin{parag}{Example 2}
    We want to find the characteristic of $\mathbb{Z}/n\mathbb{Z}$, for $n \geq 2$.

    The unique homomorphism from $\mathbb{Z}$ is: 
    \[\begin{split}
    \tau: \mathbb{Z} &\longmapsto \mathbb{Z}/n\mathbb{Z} \\
    k &\longmapsto \left[k\right]_n
    \end{split}\]
    
    We notice that $\ker\left(\tau\right) = \left(n\right)$, and thus $c\left(\mathbb{Z}/n\mathbb{Z}\right) = n$.
\end{parag}

\begin{parag}{Definition: Direct product}
    Let $A, B$ be commutative rings.

    The \important{direct product} of $A$ and $B$, is a ring over the Cartesian product of $A$ and $B$ :
    \[A \times B = \left\{\left(a, b\right) \suchthat a \in A, b \in B\right\}\]
    
    The addition and multiplication are done componentwise: 
    \[\left(a_1, b_1\right) + \left(a_2, b_2\right)= \left(a_1 + a_2, b_1 + b_2\right), \mathspace \left(a_1, b_1\right)\cdot \left(a_2, b_2\right) =\left(a_1\cdot a_2, b_1\cdot b_2\right)\]
    
    Finally, the additive identity is $\left(0_A, 0_B\right)$ and the multiplicative identity is $\left(1_A, 1_B\right)$.
\end{parag}

\begin{parag}{Example}
    We want to find the characteristic of $A = \mathbb{Z}/n\mathbb{Z} \times \mathbb{Z}/m\mathbb{Z}$.

    The unique homomorphism from $\mathbb{Z}$ is:
    \[\begin{split}
    \tau: \mathbb{Z} &\longmapsto \mathbb{Z}/n\mathbb{Z} \times \mathbb{Z}/m\mathbb{Z} \\
    k &\longmapsto \left(\left[k\right]_n, \left[k\right]_m\right)
    \end{split}\]
    
    Thus, the kernel is generated by the smallest $k$ such that: 
    \[\tau\left(k\right) = \left(\left[0\right]_n, \left[0\right]_m\right) \iff \congruent{k}{0}{n} \text{ and } \congruent{k}{0}{m}\]

    Since $k$ is the smallest number that has this property, $k = \lcm\left(m, n\right)$. This tells us that $c_A = \lcm\left(m, n\right)$.
\end{parag}

\begin{parag}{Generalisation}
    Let $A, B$ be rings such that $c_A \neq 0$ and $c_B \neq 0$.

    Then $c_{A \times B} = \lcm\left(c_A, c_B\right)$.

    \begin{subparag}{Remark}
        This generalises the previous example.
    \end{subparag}
\end{parag}

\begin{parag}{Example 1}
    Let $B = \mathbb{Z} \times \mathbb{Z}/n\mathbb{Z}$. We want to find its characteristic. The unique ring homomorphism from $\mathbb{Z}$ is given by: 
    \[\tau\left(k\right) = \left(k, \left[k\right]_n\right)\]
    
    The unique case where $\tau\left(k\right) = \left(0, \left[0\right]_n\right)$ is $k = 0$. Therefore, $\ker\left(\tau\right) = \left\{0\right\} = \left(0\right)$, and thus $c_B = 0$.
\end{parag}

\begin{parag}{Example 2}
    Let $D = \mathbb{Z}/n\mathbb{Z}\left[x\right]$ be the ring of polynomials in $x$ with coefficients in $\mathbb{Z}/n\mathbb{Z}$. We want to find its characteristic.

    The unique kernel homomorphism is: 
    \[\tau\left(k\right) = \left[k\right]_n\]
    
    Therefore, $\ker\left(\tau\right) = \left(n\right)$, telling us that $c_D = n$.
\end{parag}

\begin{parag}{Proposition}
    Let $A$ be an integral domain.

    Then, $c_A \in \mathbb{P} \cup \left\{0\right\}$ is either a prime or zero.

    \begin{subparag}{Remark}
        Since fields are integral domains, this property also holds for them.
    \end{subparag}

    \begin{subparag}{Proof}
        We do this proof by the contrapositive. We thus suppose that $c_A = m k$ for $m> 1, k > 1$. We notice that $\tau\left(m\right) \neq 0$ and $\tau\left(k\right) \neq 0$ since, by definition, $c_A$ is the \textit{smallest} positive integer that maps to 0. Then: 
        \[\underbrace{\tau\left(m\right)}_{\neq 0} \underbrace{\tau\left(k\right)}_{\neq 0} = \tau\left(m k\right) = 0\]
        
        This tells us that $\tau\left(k\right), \tau\left(m\right) \in A$ are non-trivial zero divisors.

        \qed
    \end{subparag}

    \begin{subparag}{Converse}
        The converse is wrong. Indeed, let $p \in \mathbb{P}$ be a prime number. We know that the characteristic of $A = \mathbb{Z}/p\mathbb{Z} \times \mathbb{Z}/p\mathbb{Z}$ is $c_A = \lcm\left(p, p\right) = p$.

        However, $A$ has nontrivial zero divisors, such as $\left(1, 0\right)$ and $\left(0, 1\right)$: 
        \[\left(0, 1\right)\cdot \left(1, 0\right) = \left(0, 0\right)\]
    \end{subparag}
\end{parag}

\end{document}

