J'ai fait ce document pour mon usage, mais je me suis dit que des notes dactylographiées pouvaient intéresser d'autres personnes. Faites attention au fait qu'il y a des erreurs, c'est impossible de ne pas en faire. Si vous en trouvez, n'hésitez pas à me les partager (les erreurs de grammaire et de vocabulaire sont naturellement aussi bienvenues). Vous pouvez me contacter à l'adresse e-mail suivante:
\begin{center}
    \texttt{joachim.favre@epfl.ch}
\end{center}

Si vous n'avez pas obtenu ce document par le biais de mon repo GitHub, vous serez peut-être intéressé par le fait que j'en ai un sur lequel je mets ces notes dactylographiées et leur code LaTeX. Voici le lien (je vous invite à lire le README pour comprendre comment télécharger les fichiers qui vous intéressent):
\begin{center}
    \url{https://github.com/JoachimFavre/EPFLNotesIN}
\end{center}

Notez que le contenu ne m'appartient pas. J'ai fait quelques modifications de structure, j'ai reformulé certains bouts, et j'ai ajouté quelques notes personnelles; mais les formulations et les explications viennent principalement de la personne qui nous a donné ce cours, et du livre dont elle s'est inspirée.

Je pense qu'il est intéressant de préciser que, pour avoir ces notes dactylographiées, j'ai pris mes notes en \LaTeX pendant le cours, puis j'ai fait quelques corrections. Je ne pense pas que mettre au propre des notes écrites à la main est faisable niveau quantité de travail. Pour prendre des notes en \LaTeX, je me suis inspiré du lien suivant, écrit par Gilles Castel. Si vous voulez plus de détails, n'hésitez pas à me contacter à mon adresse e-mail, mentionnée ci-dessus.
\begin{center}
    \url{https://castel.dev/post/lecture-notes-1/}
\end{center}

Je tiens aussi à préciser que les mots ``trivial'' et ``simple'' n'ont, dans ce cours, pas la définition que vous trouvez dans un dictionnaire. Nous sommes à l'EPFL, rien de ce que nous faisons n'est trivial. Quelque chose de trivial, c'est quelque chose que quelqu'un pris de manière aléatoire dans la rue serait capable de faire. Dans notre contexte, comprenez plutôt ces mots comme ``plus simple que le reste''. Aussi, ce n'est pas grave si vous prenez du temps à comprendre quelque chose qui est dit trivial (surtout que j'adore utiliser ce mot partout hihi).

Puisque vous lisez ces lignes, je vais me permettre de vous donner un petit conseil. Le sommeil est un outil bien plus puissant que ce que vous pouvez imaginer, donc ne négligez pas une bonne nuit de sommeil au profit de vos révisions (particulièrement la veille d'un examen). J'espère que vous vous amuserez pendant vos examens.
